\documentclass [spanish,12pt]{article}
\usepackage[activeacute]{babel}
\usepackage{array}
\usepackage{delarray}
\usepackage{hhline}
%\usepackage{shapepar}
%\usepackage{picinpar}
\usepackage{enumerate}
\usepackage{endnotes}
\usepackage{multicol}
\setlength{\topmargin}{ -0.4in} \setlength{\oddsidemargin}{0.1in}
\setlength{\textheight}{8.9in} \setlength{\textwidth}{6.5in}
\newcommand{\noin}{\noindent}
\newcommand{\spi}{\vskip .05 in}
\pagestyle{plain}
%  \usepackage{graphics}
  \usepackage{amsmath}
  \usepackage{amssymb}
%  \usepackage[all]{xy}
%\usepackage{graphicx}
%\usepackage[dvips]{color}
%\usepackage{graphicx,color}
\usepackage[dvips]{graphicx}
\usepackage{epsfig}
\usepackage{graphicx,color}
\include{psfig}
\renewcommand{\thefootnote}{\fnsymbol{footnote}}

%\newtheorem{ejemplo}[teorema]{Ejemplo}

\begin{document}

\section{Matem\'aticas Avanzadas}



\subsection{Fonts y s\'{\i}mbolos en F\'ormulas}

\subsubsection{Nombres de los comandos de fonts matem\'aticos}

\subsubsection{S\'{\i}mbolos matem\'aticos}

\subsection{S\'{\i}mbolos de composici\'on, par\'entesis y operadores}

\subsubsection{Signos de integrales m\'ultiples}

\subsubsection{Flechas sobre y debajo}

\subsubsection{Puntos}

\subsubsection{Acentos en matem\'aticas}

\subsubsection{Acentos arriba}

\subsubsection{Acentos de puntos}

\subsubsection{Ra\'{\i}ces}

\subsubsection{F\'ormulas en cajas}

\subsubsection{Flechas}

\subsubsection{Overset, underset, sideset}

\subsubsection{El comando smash}

\subsubsection{El comando text}

\subsubsection{Nombres de operadores}

\subsubsection{Mod y sus parientes}

\subsubsection{Fracciones y construcciones relacionadas}

\subsubsection{Fracciones Continuas}

\subsubsection{Par\'entesis grandes}

\subsection{Ambientes de matrices y de diagramas conmutativos} 

\subsubsection{El ambiente casos}

\subsubsection{Los ambientes para matrices}


\end{document}
