\documentclass [spanish,12pt]{article}
\usepackage[activeacute]{babel}
\usepackage{array}
\usepackage{delarray}
\usepackage{hhline}
%\usepackage{shapepar}
%\usepackage{picinpar}
\usepackage{enumerate}
\usepackage{endnotes}
\usepackage{multicol}
\setlength{\topmargin}{ -0.4in} \setlength{\oddsidemargin}{0.1in}
\setlength{\textheight}{8.9in} \setlength{\textwidth}{6.5in}
\newcommand{\noin}{\noindent}
\newcommand{\spi}{\vskip .05 in}
\pagestyle{plain}
%  \usepackage{graphics}
  \usepackage{amsmath}
  \usepackage{amssymb}
%  \usepackage[all]{xy}
%\usepackage{graphicx}
%\usepackage[dvips]{color}
%\usepackage{graphicx,color}
\usepackage[dvips]{graphicx}
\usepackage{epsfig}
\usepackage{graphicx,color}
\include{psfig}
\renewcommand{\thefootnote}{\fnsymbol{footnote}}

%\newtheorem{ejemplo}[teorema]{Ejemplo}

\begin{document}

\section{Matem\'aticas Avanzadas}



\subsection{Fonts y s\'{\i}mbolos en F\'ormulas}

\subsubsection{Nombres de los comandos de fonts matem\'aticos}

\subsubsection{S\'{\i}mbolos matem\'aticos}

\subsection{S\'{\i}mbolos de composici\'on, par\'entesis y operadores}

\subsubsection{Signos de integrales m\'ultiples}

\subsubsection{Flechas sobre y debajo}

\subsubsection{Puntos}

\subsubsection{Acentos en matem\'aticas}

\subsubsection{Acentos arriba}

\subsubsection{Acentos de puntos}

\subsubsection{Ra\'{\i}ces}

\subsubsection{F\'ormulas en cajas}

\subsubsection{Flechas}

\subsubsection{Overset, underset, sideset}

\subsubsection{El comando smash}
Sin el comando -smash-.
$ x_j = (|/\sqrt{\lambda_j})x_j' $
\\ \\
Con el comando -smash-.
$ x_j = (|/\sqrt{\smash[b]{\lambda_j}})x_j' $


\subsubsection{El comando text}

\subsubsection{Nombres de operadores}

\subsubsection{Mod y sus parientes}
$ u \equiv v+1 \pmod{n^2} $
\\
$ u \equiv v+1 \mod{n^2} $
\\
$ u \equiv v+1 \pod{n^2} $

\subsubsection{Fracciones y construcciones relacionadas}

\subsubsection{Fracciones Continuas}

\begin{equation}
\cfrac{1}{\sqrt{2}+
\cfrac{1}{\sqrt{3}+
\cfrac{1}{\sqrt{4}+
\cfrac{1}{\sqrt{5}}}}}
\end{equation}


\subsubsection{Par\'entesis grandes}

\subsection{Ambientes de matrices y de diagramas conmutativos} 

Ambiente0:\\
$ \sum_{n<k, \ n \mbox{impar}}nE_n $
\\ \\
Ambiente0-1:\\
$ \sum_{n<k, \ n \mbox{impar}}nE_n $

\subsubsection{El ambiente casos}
Ecuaciones:\\
\begin{equation*}
\begin{split}
(a+b)^n = & \sum_{k=0}^n \binom{n}{k}a^kb^{n-k} = a^n + \binom{n}{1}a^{n-1} \\
& + \binom{n}{2}a^{n-2}b^2+\cdots \\
& + \cdots + \\
& + \binom{n}{k}a^{n-k}b^k \cdots \\
& + \binom{n}{n-1}ab^n
\end{split}
\end{equation*}


\subsubsection{Los ambientes para matrices}
Matriz simple:\\
$$
\begin{matrix}
0 & 1 \\
1 & 0
\end{matrix}
$$
\\
Matriz normal:\\
$$
\begin{pmatrix}
0 & 1 \\
1 & 0
\end{pmatrix}
$$
\\
Matriz cuadrada:\\
$$
\begin{bmatrix}
0 & 1 \\
1 & 0
\end{bmatrix}
$$
\\
Matriz barra:\\
$$
\begin{vmatrix}
0 & 1 \\
1 & 0
\end{vmatrix}
$$
\\
Matriz doble barra:\\
$$
\begin{Vmatrix}
0 & 1 \\
1 & 0
\end{Vmatrix}
$$



\end{document}
