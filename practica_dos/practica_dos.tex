\documentclass[spanish,12pt]{article}
%\usepackage{utf8}
%\usepackage[activeacute]{babel}
%\usepackage{ulem}
%\usepackage{shapepar}
%\usepackage{picinpar}
\usepackage{titling}
\usepackage{enumerate}
%\usepackage{endnotes}
\usepackage{multicol}
\setlength{\topmargin}{ -0.4in} \setlength{\oddsidemargin}{0.1in}
\setlength{\textheight}{8.9in} \setlength{\textwidth}{6.5in}
\newcommand{\noin}{\noindent}
\newcommand{\spi}{\vskip .05 in}
\pagestyle{plain}
%  \usepackage{graphics}
  \usepackage{amsmath}
  \usepackage{amssymb}
%  \usepackage[all]{xy}
%\usepackage{graphicx}
%\usepackage[dvips]{color}
%\usepackage{graphicx,color}
\usepackage[dvips]{graphicx}
\usepackage{epsfig}
\usepackage{graphicx,color}
\include{psfig}
\renewcommand{\thefootnote}{\fnsymbol{footnote}}

%\newtheorem{ejemplo}[teorema]{Ejemplo}

\date{\today}

\begin{document}

\begin{center}
\section*{Herramientas b\'asicas de Formato}
Emanuel GP
\end{center}

\begin{flushright}
\thedate
\end{flushright}

\begin{multicols}{2}[\section{Descenso}]{
El siguiente m'etodo de demostraci'on fue utilizado frecuentemente 
por {\bf Pierre Fermat}\footnote{Matem\'atico Frances} (1601-1665) por lo que a 'el se le atribuye. Por lo general, 
es usado para demostrar que algo no sucede. 
Por ejemplo, {\bf Fermat}\footnote{Mismo Matem\'atico} lo utiliz'o
para mostrar que no hay soluciones enteras de la ecuaci'on $x^4 + y^4=z^2$, 
con $xyz\neq 0$.


La base te'orica de su m'etodo es que no hay una colecci'on 
infinita de enteros positivos que sea decreciente, esto es, no podemos 
encontrar una infinidad de enteros positivos que cumplan 
$n_1 > n_2 > n_3>\cdots$.

\noindent Hay dos maneras de usar esta idea para demostrar afirmaciones. 
La primera es tener una afirmaci'on $\mathcal{P}(n_1)$ que se supone v'alida. 
Si de 'esta se puede encontrar un entero positivo $n_2 <n_1$ tal que  
$\mathcal{P}(n_2)$ es v'alida y, a su vez, si de 'esta se encuentra un 
entero positivo $n_3< n_2$ tal que  $\mathcal{P}(n_3)$ es v'alida, y as'i 
sucesivamente, entonces una infinidad de enteros positivos se genera de tal 
forma  que cumple que $n_1>n_2>n_3>\cdots$, pero esto es imposible, por lo 
que  $\mathcal{P}(n_1)$ no es verdadera. Veamos un ejemplo para ilustrar este 
m'etodo.}

El n'umero $\sqrt{2}$ no es un n'umero racional\footnote{Fracci\'on}.

\vspace{.2in}

\noindent Supongamos que $\sqrt 2$ es un n'umero racional, entonces 
$\sqrt 2=\frac{m_1}{n_1}$, con $m_1$ y $n_1$ n'umeros enteros positivos.
Como $\sqrt{2}+1=\frac{1}{\sqrt{2}-1}$, tenemos que

\begin{eqnarray*}		  
 \sqrt{2}+1 &=& \frac{1}{\frac{m_1}{n_1}-1}\\=\frac{n_1}{m_1-n_1},\\
 \quad\text{por lo que}\quad   \sqrt{2} &=& \frac{n_1}{m_1-n_1}-1=\frac{2n_1-m_1}{m_1-n_1}.
\end{eqnarray*}

\noindent Como  $1< \sqrt{2}<2$, sustituyendo el supuesto valor racional de $\sqrt{2}$ se tiene que  $1< \frac{m_1}{n_1}<2$, de donde
$n_1 < m_1 < 2n_1$. De aqu'i tenemos que, $2n_1-m_1 >0$ y $m_1- n_1 >0$. Luego, si definimos $m_2=2n_1-m_1$ y $n_2=m_1-n_1$, tenemos que $m_2< m_1$ y   $n_2<n_1$, ya que $n_1<m_1$ y $m_1<2n_1$, respectivamente. Luego, $\sqrt{2}=\frac{m_1}{n_1}=\frac{m_2}{n_2}$, con $m_2<m_1$ y $n_2<n_1$. Siguiendo este proceso, podemos generar una infinidad de enteros positivos\footnote{Naturales} $m_i$ y $n_i$ que cumplen que
$$ \sqrt{2} =\frac{m_1}{n_1}=\frac{m_2}{n_2}=\frac{m_3}{n_3}=\cdots,$$
con $m_1 >m_2>m_3>\cdots$ y $n_1 >n_2>n_3>\cdots$,
pero esto no es posible. Por lo tanto, $\sqrt{2}$ no es un n'umero racional.

\end{multicols}

\vspace{.2in}


\end{document}
