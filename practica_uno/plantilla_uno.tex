\documentclass [12pt]{article}
\setlength{\topmargin}{ -0.4in} \setlength{\oddsidemargin}{0.1in}
\setlength{\textheight}{8.9in} \setlength{\textwidth}{6.5in}
\newcommand{\noin}{\noindent}
\newcommand{\spi}{\vskip .05 in}
\pagestyle{plain}
%  \usepackage{graphics}
  \usepackage{amsmath}
  \usepackage{amssymb}
%  \usepackage[all]{xy}
%\usepackage{graphicx}
%\usepackage[dvips]{color}
%\usepackage{graphicx,color}
\usepackage[dvips]{graphicx}
\usepackage{epsfig}
\usepackage{graphicx,color}
\include{psfig}

\begin{document}


\begin{center}
          {\bf\Large $27^{ava}$ Olimpiada Mexicana de Matem\'aticas}
          \vspace{.5cm}

          Concurso Estatal de la Olimpiada de Matem\'aticas de Morelos

	    \vspace{.3cm}
          Tercera Etapa, Junio 28 de 2013\\
          Examen 1
\end{center}
	

\vspace{.5in}

\begin{itemize}

%%%%%%%%1
\item En cada casilla de un tablero de tama\~no $2013 \times 2013$ se coloca
alg\'un n\'umero de la lista $1$, $2$, $3$, $\dots$, $2013$, de tal manera que en cada rengl\'on y en 
cada columna todos esos n\'umeros aparecen. Si una vez colocados los n\'umeros en las casillas, 
el tablero es sim\'etrico con respecto a una de las diagonales, muestra que en esta diagonal todos los 
n\'umeros $1$, $2$, $3$, $\dots$, $2013$ aparecen.

\vspace{.6in}


\item  Encuentra todos los n\'umeros de tres d\'{\i}gitos $\overline{xyz}$, donde
$x$, $y$, $z$ son d\'{\i}gitos, tales que 
$$\overline{xyz}=x+y+z+xy+yz+zx+xyz.$$
$$ x^2 + y^2 + z^2 = (xyz)^2 $$
\begin{center}
	Clase de LaTeX
	$$ x^2 + y^2 = 3 \  \mbox{implica que } \  z = 2 $$ \\
%	$$ x = \frac{-b\pm{\sqrt{b^2-4ac}}}{2a} = \frac{-3\pm{\sqrt{3^2-4(2)(5)}}}{2(2)} = %\frac{-3\pm{\sqrt{9-40}}}{4} = \frac{-3\pm{\sqrt{-31}}}{4} = \frac{-3\pm{i\sqrt{31}}}{4} $$
\end{center}

\begin{eqnarray*}
		a &=& 2 \\
		b &=& 3 \\
		c &=& 5\\ \\
		x &=& \frac{-b\pm{\sqrt{b^2-4ac}}}{2a} \\
		  &=& \frac{-3\pm{\sqrt{3^2-4(2)(5)}}}{2(2)} \\
		  &=& \frac{-3\pm{\sqrt{9-40}}}{4} \\
		  &=& \frac{-3\pm{\sqrt{-31}}}{4} \\
		  &=& \frac{-3\pm{i\sqrt{31}}}{4}
\end{eqnarray*}

\begin{eqnarray*}
	5x + 3y &=& 2 \\
	6x + 4y &=& 8 
\end{eqnarray*}
$$
\left|
\begin{array}{ c c c}
	1 & 2 & 3 \\
	4 & 5 & 6 \\
	7 & 8 & 9 \\
\end{array}
\right|
$$
\vspace{.6in}

\item  Sea $ABC$ un triangulo is\'osceles con $AB=AC$  y sea $D$ el punto  medio
de $BC$. Sea $E$ el pie de la perpendicular desde $D$ en el lado $AB$ y sea $F$ el punto medio de $DE$.
Muestra que $AF$ es perpendicular a $CE$.

\end{itemize}


\end{document}
