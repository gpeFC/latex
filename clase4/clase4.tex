\documentclass [spanish,12pt]{article}
\usepackage[activeacute]{babel}
\usepackage{array}
\usepackage{delarray}
\usepackage{hhline}
%\usepackage{shapepar}
%\usepackage{picinpar}
\usepackage{enumerate}
\usepackage{endnotes}
\usepackage{multicol}
\setlength{\topmargin}{ -0.4in} \setlength{\oddsidemargin}{0.1in}
\setlength{\textheight}{8.9in} \setlength{\textwidth}{6.5in}
\newcommand{\noin}{\noindent}
\newcommand{\spi}{\vskip .05 in}
\pagestyle{plain}
%  \usepackage{graphics}
  \usepackage{amsmath}
  \usepackage{amssymb}
%  \usepackage[all]{xy}
%\usepackage{graphicx}
%\usepackage[dvips]{color}
%\usepackage{graphicx,color}
\usepackage[dvips]{graphicx}
\usepackage{epsfig}
\usepackage{graphicx,color}
\include{psfig}
\renewcommand{\thefootnote}{\fnsymbol{footnote}}

%\newtheorem{ejemplo}[teorema]{Ejemplo}

\begin{document}

\section{Material de Tabulaciones}

\subsection{Ejemplos de comandos de preambulo}

\begin{center}

%>{\large}c significa que lo haga largo y lo centre
%>{\bfseries}l que lo ponga en negrito y a la izquierda
%>{\itshape}c que lo haga en italico y centrado
\begin{tabular}{| >{\large}c | >{\bfseries}l | >{\itshape} c |}
\hline A & B & C \\
\hline 100 & 10 & 1 \\
\hline
\end{tabular}

\end{center}


\vspace{.1in}

\begin{center}

%define la long entre la letra y el arreglo, tiene una altura extra al renglon
\setlength{\extrarowheight}{4pt}
\begin{tabular}{| >{\large}c | >{\bfseries}l | >{\itshape} c |}
\hline A & B & C \\
\hline 100 & 10 & 1 \\
\hline
\end{tabular}

\end{center}

\vspace{.1in}

\begin{center}
%p{1.2cm} le da distancia a la celda la p los manda hacia arriba
\begin{tabular}{|p{2cm}|p{1cm}|p{1cm}|}
\hline 1 1 1 1 1 1 1 1 1 1 1 1  &
       2 2 2 2 2 2 2 2 2        & 3 3 3 3 \\
\hline
\end{tabular} 

\end{center}

\vspace{.1in}

\begin{center}
%m{1.2cm} le da distancia a la celda la p los manda hacia en medio
\begin{tabular}{|m{1.2cm}|m{1cm}|m{1cm}|}
\hline 1 1 1 1 1 1 1 1 1 1 1 1  &
       2 2 2 2 2 2 2 2 2        & 3 3 3 3 \\
\hline
\end{tabular} 

\end{center}

\vspace{.1in}

\begin{center}
%la b los manda hacia abajo
\begin{tabular}{|b{1.2cm}|b{1.2cm}|b{1cm}|}
\hline 1 1 1 1 1 1 1 1 1 1 1 1  &
       2 2 2 2 2 2 2 2 2        & 3 3 3 3 \\
\hline
\end{tabular} 
\end{center}

\vspace{.1in}

\begin{center} 
\begin{tabular}
%{|>{\setlength{\parindent}{5mm}}p{2cm} la primer columna tendra sangria de 5mm y lo mandara hacia arriba
{|>{\setlength{\parindent}{5mm}}p{2cm}|p{2cm}|}
\hline 1 2 3 4 5 6 7 8 9 0 1 2 3 4 5 6 7 8 9 0 &
1 2 3 4 5 6 7 8 9 0 1 2 3 4 5 6 7 8 9 0 \\
\hline
\end{tabular} 
\end{center}

\vspace{.1in}

 
\setlength{\extrarowheight}{2pt}
%arreglo con simbolos matematicos
%\[ este simbolo es igual a && para centrar cosas matematicas y se cierra \]
%|l| primer columna a la izq
%>{$}l significa que no aparecera en forma matematica  y a la izq
\[ \begin{array}{|l| >{$}l < {$}|}
\hline 
10!^{10!}  & a big number \\
10^{-999}  & a small number \\ 
\hline \end{array} \]

\vspace{.1in}

%fbox 
\begin{tabular}{r@{\hspace{5mm}}l}
\fbox{LEFT BOX} & \fbox{RIGHT BOX}
\end{tabular}

%\hspace es el espacio entre los dos cuadritos

\par \vspace{\baselineskip} \par
\begin{tabular}{r!{\hspace{5mm}}l}
\fbox{LEFT BOX} & \fbox{RIGHT BOX}
\end{tabular}

\subsubsection{Ancho variable de las lineas verticales}

\begin{center}
%tabula,centra signo de admiracion para la sig instrcuccion y sirve para lo que vas a escriir %despues

%la instruccionen es pra hacer una linea gruesa
\begin{tabular}{|c!{\vrule width 3pt}c|c|}
\hline
A  &  B  &  C \\ \hline
100 & 10 & 1 \\ \hline
\end{tabular}

\end{center}

\subsubsection{Controlando la separaci\'on horizontal entre columnas}


\begin{tabular}{ll} %%%parte que varia, alineada left left
BOXES & BOXES \\
BOXES & BOXES \\
\end{tabular}


\begin{tabular}{@{}ll@{}} %%%parte que varia aqui le deja un espacion extra con @ {} y hay no %pone nada
BOXES & BOXES \\
BOXES & BOXES \\
\end{tabular}


\begin{tabular}{@{}l@{}l@{}} %%%parte que varia espacio entre left left y como no pone nada lo %junta, tambien si pone phantom borra el espacio
BOXES & BOXES \\
BOXES & BOXES \\
\end{tabular}

\vspace{.1in}

\begin{tabular}{|l|l|} %%%parte que varia
\hline
BOXES & BOXES \\ \hline
BOXES & BOXES \\ \hline
\end{tabular}

\vspace{.1in}

\begin{tabular}{|@{}l|l@{}|} %%%parte que varia
\hline
BOXES & BOXES \\ \hline
BOXES & BOXES \\ \hline
\end{tabular}

\vspace{.1in}

\begin{tabular}{|@{}l@{}|l@{}|} %%%parte que varia
\hline
BOXES & BOXES \\ \hline
BOXES & BOXES \\ \hline
\end{tabular}

\vspace{.1in}

\begin{tabular}{|@{}l@{}|@{}l@{}|} %%%parte que varia
\hline
BOXES & BOXES \\ \hline
BOXES & BOXES \\ \hline
\end{tabular}

\vspace{.2in}

\begin{tabular}{||l||l||} %%%parte que varia
\hline \hline
BOXES & BOXES \\ \hline \hline
BOXES & BOXES \\ \hline \hline
\end{tabular}

\vspace{.1in}

\begin{tabular}{||@{}l||l@{}||} %%%parte que varia
\hline \hline
BOXES & BOXES \\ \hline \hline
BOXES & BOXES \\ \hline \hline
\end{tabular}

\vspace{.1in}

\begin{tabular}{||@{}l@{}||l@{}||} %%%parte que varia
\hline \hline
BOXES & BOXES \\ \hline \hline
BOXES & BOXES \\ \hline \hline
\end{tabular}

\vspace{.1in}

\begin{tabular}{||@{}l@{}||@{}l@{}||} %%%parte que varia
\hline \hline
BOXES & BOXES \\ \hline \hline
BOXES & BOXES \\ \hline \hline
\end{tabular}

\vspace{.1in}

\begin{tabular}{|@{}|@{}l@{}|@{}|@{}l@{}|@{}|} %%%parte que varia
\hline \hline
BOXES & BOXES \\ \hline \hline
BOXES & BOXES \\ \hline \hline
\end{tabular}


\subsection{Especificando limites en un arreglo}
%matriz normal ({cc}) que este centrada y los parentesis de lado
\[
\begin{array} \{{cc}\}
A & B \\
C & D 
\end{array}
\]

\vspace{.1in}

%podemos cambiar el environment de array para modificarlos
%y define un nuevo tipo de columna L
\newcolumntype{L}{>{$}l<{$}} %%se olvida de lo matematico que esta dentro
%este newenviroment es para una funcion con dos casos, {lL} significa que una columna es a la %izq y la otra es L que no sea matematico el .} que sigue es para cerrar ese tipo de fnc
\newenvironment{Cases}{\begin{array}\{ {lL}.}{\end{array}}
%%es para escribir matrices, |{*{20}{c}}| es para el tamaño el *20
\newenvironment{Matrix}{\begin{array}|{*{20}{c}}|}{\end{array}}

\newenvironment{Pmatrix}{\begin{array}({*{20}{c}})}{\end{array}}


%% como L no es matematico y quieres poner cosas matematicas dentro usas los $$
%% 2mm es el espacio que deja entre las columnas
\[ 
|x| = \begin{Cases}
x, & if $x \geq 0$; \\[2mm] 
-x, & otherwise. \\
\end{Cases}
\]


\vspace{.1in}

\[ a = {\begin{Matrix}
x - \lambda & 1 & 0 \\
0 & x -\lambda & 1 \\
0 & 0 & x -\lambda \\
\end{Matrix}
}^2
\]

\vspace{.1in}

\[ A = \begin{Pmatrix}
x - \lambda & 1 & 0 \\
0 & x -\lambda & 1 \\
0 & 0 & x -\lambda \\
\end{Pmatrix}
\]

\vspace{.1in}

%[t] es para un arreglo de una sola columna centrado al top
%[c] es para "" center
%[b] es para "" boton
%manda todo el array no solo los numeros

\[ \begin{array}[t]({c})1\\2\\3 \end{array}
 \begin{array}[c]({c})1\\2\\3 \end{array}
 \begin{array}[b]({c})1\\2\\3 \end{array}
\]

\vspace{.1in}
%este es a la izq centrado al top, center, boton, solo manda el espacio
\[  \left(
\begin{array}[t]{c}1\\2\\3\end{array}
\right)
\left(
\begin{array}[c]{c}1\\2\\3\end{array}
\right)
\left(
\begin{array}[b]{c}1\\2\\3\end{array}
\right)
\]

\subsection{Combinando lineas verticales y horizontales}

Con la instrucci'on $\backslash$hhline se tienen los argumentos:

\begin{itemize}     %\backslash escribe esto \
\item[=] Genera una $\backslash$hline doble el ancho de la columna
\item[-] Genera una $\backslash$hline sencilla el ancho de la columna
\item[$\thicksim$] Genera una columna sin $\backslash$hline, es decir, un espacio del 
ancho de la columna.
\item[$|$] Genera una $\backslash$vline que corta atraves de una 
$\backslash$hline sencilla o doble
\item[:] Genera una $\backslash$vline que es rota por por una 
$\backslash$hline doble.
\item[$\sharp$] Genera un segmento $\backslash$hline doble entre dos 
$\backslash$vlines
\item[t] La parte superior de un segmento $\backslash$hline doble
\item[b] La parte inferior de un segmento $\backslash$hline doble
\end{itemize}

\setlength{\arrayrulewidth}{.8pt} %%que tan gruesa es la linea
\begin{tabular}{||cc||c|c||}
%%primero lo cierra, luego dibuja | luego pone t que es doble, luego : rompe la |
\hhline{|t:==:t:==:t|} 
a & b & c & d \\ \hhline{|:==:|~|~||}
1 & 2 & 3 & 4 \\ \hhline{#==#~|=#}
i & j & k & l \\ \hhline{||--||--||}
w & x & y & z \\ \hhline{|b:==:b:==:b|}
\end{tabular}


\end{document}
