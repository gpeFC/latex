\documentclass [12pt]{article}
\setlength{\topmargin}{ -0.4in} \setlength{\oddsidemargin}{0.1in}
\setlength{\textheight}{8.9in} \setlength{\textwidth}{6.5in}
\newcommand{\noin}{\noindent}
\newcommand{\spi}{\vskip .05 in}
\pagestyle{plain}
%  \usepackage{graphics}
  \usepackage{amsmath}
  \usepackage{amssymb}
%  \usepackage[all]{xy}
%\usepackage{graphicx}
%\usepackage[dvips]{color}
%\usepackage{graphicx,color}
\usepackage[dvips]{graphicx}
\usepackage{epsfig}
\usepackage{graphicx,color}
\include{psfig}

\begin{document}


\begin{center}
          {\bf\Large $27^{ava}$ Olimpiada Mexicana de Matem\'aticas}
          \vspace{.5cm}

          Concurso Estatal de la Olimpiada de Matem\'aticas de Morelos

	    \vspace{.3cm}
          Tercera Etapa, Junio 28 de 2013\\
          Examen 1
\end{center}
	

\vspace{.5in}


%%%%%%%%1
\noindent {\bf Problema 1.} En cada casilla de un tablero de tama\~no $2013 \times 2013$ se coloca
alg\'un n\'umero de la lista $1$, $2$, $3$, $\dots$, $2013$, de tal manera que en cada rengl\'on y en 
cada columna todos esos n\'umeros aparecen. Si una vez colocados los n\'umeros en las casillas, 
el tablero es sim\'etrico con respecto a una de las diagonales, muestra que en esta diagonal todos los 
n\'umeros $1$, $2$, $3$, $\dots$, $2013$ aparecen.

\vspace{.6in}


\noindent {\bf Problema 2.} Encuentra todos los n\'umeros de tres d\'{\i}gitos $\overline{xyz}$, donde
$x$, $y$, $z$ son d\'{\i}gitos, tales que 
$$\overline{xyz}=x+y+z+xy+yz+zx+xyz.$$

\vspace{.6in}

\noindent {\bf Problema 3.} Sea $ABC$ un triangulo is\'osceles con $AB=AC$  y sea $D$ el punto  medio
de $BC$. Sea $E$ el pie de la perpendicular desde $D$ en el lado $AB$ y sea $F$ el punto medio de $DE$.
Muestra que $AF$ es perpendicular a $CE$.


\end{document}
