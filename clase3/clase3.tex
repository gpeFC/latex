%FORMATO BASE PARA UNA letter
%<alt>126 = ~
%abrir comillas = <alt>96 <alt>96
%\usepackage[T1]{fontenc}

\documentclass[spanish,11pt]{book}
\usepackage[activeacute]{babel}
\usepackage{makeidx}
%\usepackage{showidx}
\usepackage{amssymb}
\usepackage{pst-all}
\usepackage{ifthen}
\usepackage{amsmath}
%\usepackage{psfig}
\usepackage{pstricks}
\usepackage{epic}
\usepackage{eepic}
\usepackage{tabularx}
\usepackage{hhline}
\usepackage{calc}
\usepackage{float}
\usepackage{fancyhdr}
\usepackage{afterpage}
\usepackage{pstcol,pst-plot}
\usepackage{pst-eucl}
\usepackage{pst-poly}
\usepackage{relsize}
\usepackage{times}
\usepackage{graphicx,color}
\usepackage{url}
\usepackage{pst-xkey}
\usepackage{stmaryrd}
\usepackage{psfrag}

\spanishdecimal{.}

\renewcommand{\familydefault}{cmss}

%%	CAMBIO A MEXICANO

\addto\captionsspanish{\renewcommand{\contentsname}{Contenido}}
\addto\captionsspanish{\renewcommand{\listfigurename}{Lista de Figuras}}
\addto\captionsspanish{\renewcommand{\listtablename}{Lista de Tablas}}
\addto\captionsspanish{\renewcommand{\indexname}{\'Indice}}


%%	HEADERS
\pagestyle{fancyplain}
\renewcommand{\chaptermark}[1]{\markboth{#1}{}}
\renewcommand{\sectionmark}[1]{\markright{\thesection\, #1}}
\lhead[\fancyplain{}{\bfseries\thepage}]%
      {\fancyplain{}{\bfseries\rightmark}}
\rhead[\fancyplain{}{\bfseries\leftmark}]%
      {\fancyplain{}{\bfseries\thepage}}
\cfoot[]{}


\hyphenation{con-si-de-re-mos cir-cun-fe-ren-cia cir-cun-fe-ren-cias
con-si-de-re co-rres-pon-dien-tes co-rres-pon-dien-te con-gruen-tes cons-tru-ya cons-tru-yen
con-si-de-rar con-cu-rren-tes cons-tru-ya-mos cir-cuns-cri-tas cir-cuns-cri-ta
coin-ci-de cons-tru-ye ca-rac-te-ri-zan}
\hyphenation{dia-go-nal dia-go-na-les de-si-gual-da-des des-cri-tas
di-fe-ren-tes}
\hyphenation{eu-cli-dea-na equi-va-len-te}
\hyphenation{fi-gu-ras}
\hyphenation{ge-ne-ra-li-dad}
\hyphenation{in-me-dia-to ins-cri-to}
\hyphenation{ma-yor ma-ne-ra}
\hyphenation{or-to-cen-tro o-pues-tos}
\hyphenation{po-si-ti-vos po-si-ti-vo pro-por-cio-na-li-dad 
pa-ra-le-lo-gra-mo pers-pec-ti-va pa-ra-le-la per-pen-di-cu-lar pro-ble-mas 
per-pen-di-cu-la-res}
\hyphenation{re-fe-ren-cia re-fle-xio-nes res-pec-ti-va-men-te rec-tas 
res-pec-to}
\hyphenation{si-guien-te si-mi-la-res su-pon-ga-mos si-guien-tes su-ce-der
sa-tis-fa-cen}
\hyphenation{te-ne-mos tie-nen te-ner}
\hyphenation{uti-li-ce usa-re-mos}
\hyphenation{va-ria-bles}

%Contadores de las listas
\newcounter{num}
\newcounter{numi}
\newcounter{numii}

\newcommand{\Hrule}{\rule{\linewidth}{1mm}}
\newcommand{\rr}{\mathbb{R}}
\newcommand{\zz}{\mathbb{Z}}
\newcommand{\ii}{\mathbb{I}}
\newcommand{\cc}{\mathbb{C}}
\newcommand{\es}{\mathbb{L}}
\newcommand{\nn}{\mathbb{N}}
\newcommand{\qq}{\mathbb{Q}}

\newcommand{\FONTONCE}[1]{\fontsize{11pt}{11pt}\selectfont #1}



\makeindex


\newcommand{\veii}{\vspace{.5cm}}
\newcommand{\vei}{\vspace{.15cm}}
\newcommand{\ven}{\vspace{-.35cm}}
\newcommand{\ve}{\vspace{.3cm}}
\newcommand{\veg}{\vspace{.4cm}}
\newcommand{\vs}{\vspace{3.5cm}}
\newcommand{\vsa}{\vspace{7cm}}
\newcommand{\fullref}[1]{\ref{#1} en la p\'agina~\pageref{#1}}

\parindent=0in

%%%%%Tamanio de Pagina
%\setlength{\textwidth}{6in}
%\setlength{\textheight}{8.5in}
%\setlength{\topmargin}{.5in}
%\setlength{\headsep}{.2in}
%\setlength{\oddsidemargin}{0.5in}
%\setlength{\evensidemargin}{0.5in}

%Contadores de ejercicios
\newcounter{prob}
\setcounter{prob}{0}
\newcounter{sol}
\setcounter{sol}{0}
\newcounter{refe}
\setcounter{refe}{0}
\newcounter{sugrefe}
\setcounter{sugrefe}{0}
\newcounter{ejcprel}
\setcounter{ejcprel}{0}
\newcounter{sucprel}
\setcounter{sucprel}{0}
\newcounter{surefcprel}
\setcounter{surefcprel}{0}
\newcounter{refejcprel}
\setcounter{refejcprel}{0}
\newcounter{ejercprogresiones}
\setcounter{ejercprogresiones}{0}
\newcounter{sugcprogresiones}
\setcounter{sugcprogresiones}{0}
\newcounter{sugrefecprogresiones}
\setcounter{sugrefecprogresiones}{0}
\newcounter{refejercprogresiones}
\setcounter{refejercprogresiones}{0}
\newcounter{ejercinduccion}
\setcounter{ejercinduccion}{0}
\newcounter{sugcinduccion}
\setcounter{sugcinduccion}{0}
\newcounter{sugrefecinduccion}
\setcounter{sugrefecinduccion}{0}
\newcounter{refejercinduccion}
\setcounter{refejercinduccion}{0}
\newcounter{ejercpolcuadcub}
\setcounter{ejercpolcuadcub}{0}
\newcounter{refejercpolcuadcub}
\setcounter{refejercpolcuadcub}{0}
\newcounter{sugcpolcuadcub}
\setcounter{sugcpolcuadcub}{0}
\newcounter{sugrefecpolcuadcub}
\setcounter{sugrefecpolcuadcub}{0}
\newcounter{ejerccomplejos}
\setcounter{ejerccomplejos}{0}
\newcounter{refejerccomplejos}
\setcounter{refejerccomplejos}{0}
\newcounter{sugccomplejos}
\setcounter{sugccomplejos}{0}
\newcounter{sugrefeccomplejos}
\setcounter{sugrefeccomplejos}{0}
\newcounter{ejercfunciones}
\setcounter{ejercfunciones}{0}
\newcounter{refejercfunciones}
\setcounter{refejercfunciones}{0}
\newcounter{sugcfunciones}
\setcounter{sugcfunciones}{0}
\newcounter{sugrefecfunciones}
\setcounter{sugrefecfunciones}{0}
\newcounter{ejercsucesiones}
\setcounter{ejercsucesiones}{0}
\newcounter{refejercsucesiones}
\setcounter{refejercsucesiones}{0}
\newcounter{sugcsucesiones}
\setcounter{sugcsucesiones}{0}
\newcounter{sugrefecsucesiones}
\setcounter{sugrefecsucesiones}{0}
\newcounter{ejercpolinomios}
\setcounter{ejercpolinomios}{0}
\newcounter{refejercpolinomios}
\setcounter{refejercpolinomios}{0}
\newcounter{sugcpolinomios}
\setcounter{sugcpolinomios}{0}
\newcounter{sugrefecpolinomios}
\setcounter{sugrefecpolinomios}{0}




\newcommand{\ejercpreliminares}[1]{
	\stepcounter{ejcprel}
	\refstepcounter{refejcprel}
	\ifthenelse{\equal{\xxeleccion}{1}}{
		\noindent\textbf{Ejercicio~\thechapter.\theejcprel}
		\def\@currentlabel{\therefejcprel{}}
		{\it #1 \par}
		\bigskip
	}{}
}

\newcommand{\solcpreliminares}[1]{
	   \stepcounter{sucprel}
	   \refstepcounter{surefcprel}
           \ifthenelse{\equal{\xxeleccion}{11}}{
		\noindent\textbf{Soluci\'on~1.\thesucprel}
		\def\@currentlabel{\thesurefcprel{}}
		#1\par
		\bigskip
	}{}
}


\newcommand{\ejercprogresiones}[1]{
	\stepcounter{ejercprogresiones}
	\refstepcounter{refejercprogresiones}
	\ifthenelse{\equal{\xxeleccion}{2}}{
		\noindent\textbf{Ejercicio~\thechapter.\theejercprogresiones}
		\def\@currentlabel{\therefejercprogresiones{}}
		{\it #1\par}
		\bigskip
	}{}
}

\newcommand{\solcprogresiones}[1]{
	   \stepcounter{sugcprogresiones}
	   \refstepcounter{sugrefecprogresiones}
           \ifthenelse{\equal{\xxeleccion}{22}}{
		\noindent\textbf{Soluci\'on~2.\thesugcprogresiones}
		\def\@currentlabel{\thesugrefecprogresiones{}}
		#1\par
		\bigskip
	}{}
}

\newcommand{\ejercinduccion}[1]{
	\stepcounter{ejercinduccion}
	\refstepcounter{refejercinduccion}
	\ifthenelse{\equal{\xxeleccion}{3}}{
		\noindent\textbf{Ejercicio~\thechapter.\theejercinduccion}
		\def\@currentlabel{\therefejercinduccion{}}
		{\it #1\par}
		\bigskip
	}{}
}

\newcommand{\solcinduccion}[1]{
	   \stepcounter{sugcinduccion}
	   \refstepcounter{sugrefecinduccion}
           \ifthenelse{\equal{\xxeleccion}{33}}{
		\noindent\textbf{Soluci\'on~3.\thesugcinduccion}
		\def\@currentlabel{\thesugrefecinduccion{}}
		#1\par
		\bigskip
	}{}
}

\newcommand{\ejerccomplejos}[1]{
	\stepcounter{ejerccomplejos}
	\refstepcounter{refejerccomplejos}
	\ifthenelse{\equal{\xxeleccion}{5}}{
		\noindent\textbf{Ejercicio~\thechapter.\theejerccomplejos}
		\def\@currentlabel{\therefejerccomplejos{}}
		{\it #1\par}
		\bigskip
	}{}
}

\newcommand{\solccomplejos}[1]{
	   \stepcounter{sugccomplejos}
	   \refstepcounter{sugrefeccomplejos}
           \ifthenelse{\equal{\xxeleccion}{55}}{
		\noindent\textbf{Soluci\'on~5.\thesugccomplejos}
		\def\@currentlabel{\thesugrefeccomplejos{}}
		#1\par
		\bigskip
	}{}
}

\newcommand{\ejercpolcuadcub}[1]{
	\stepcounter{ejercpolcuadcub}
	\refstepcounter{refejercpolcuadcub}
	\ifthenelse{\equal{\xxeleccion}{4}}{
		\noindent\textbf{Ejercicio~\thechapter.\theejercpolcuadcub}
		\def\@currentlabel{\therefejercpolcuadcub{}}
		{\it #1\par}
		\bigskip
	}{}
}

\newcommand{\solcpolcuadcub}[1]{
	   \stepcounter{sugcpolcuadcub}
	   \refstepcounter{sugrefecpolcuadcub}
           \ifthenelse{\equal{\xxeleccion}{44}}{
		\noindent\textbf{Soluci\'on~4.\thesugcpolcuadcub}
		\def\@currentlabel{\thesugrefecpolcuadcub{}}
		#1\par
		\bigskip
	}{}
}

\newcommand{\ejercfunciones}[1]{
	\stepcounter{ejercfunciones}
	\refstepcounter{refejercfunciones}
	\ifthenelse{\equal{\xxeleccion}{6}}{
		\noindent\textbf{Ejercicio~\thechapter.\theejercfunciones}
		\def\@currentlabel{\therefejercfunciones{}}
		{\it #1\par}
		\bigskip
	}{}
}

\newcommand{\solcfunciones}[1]{
	   \stepcounter{sugcfunciones}
	   \refstepcounter{sugrefecfunciones}
           \ifthenelse{\equal{\xxeleccion}{66}}{
		\noindent\textbf{Soluci\'on~6.\thesugcfunciones}
		\def\@currentlabel{\thesugrefecfunciones{}}
		#1\par
		\bigskip
	}{}
}


\newcommand{\ejercsucesiones}[1]{
	\stepcounter{ejercsucesiones}
	\refstepcounter{refejercsucesiones}
	\ifthenelse{\equal{\xxeleccion}{7}}{
		\noindent\textbf{Ejercicio~\thechapter.\theejercsucesiones}
		\def\@currentlabel{\therefejercsucesiones{}}
		{\it #1\par}
		\bigskip
	}{}
}

\newcommand{\solcsucesiones}[1]{
	   \stepcounter{sugcsucesiones}
	   \refstepcounter{sugrefecsucesiones}
           \ifthenelse{\equal{\xxeleccion}{77}}{
		\noindent\textbf{Soluci\'on~7.\thesugcsucesiones}
		\def\@currentlabel{\thesugrefecsucesiones{}}
		#1\par
		\bigskip
	}{}
}


\newcommand{\ejercpolinomios}[1]{
	\stepcounter{ejercpolinomios}
	\refstepcounter{refejercpolinomios}
	\ifthenelse{\equal{\xxeleccion}{8}}{
		\noindent\textbf{Ejercicio~\thechapter.\theejercpolinomios}
		\def\@currentlabel{\therefejercpolinomios{}}
		{\it #1\par}
		\bigskip
	}{}
}

\newcommand{\solcpolinomios}[1]{
	   \stepcounter{sugcpolinomios}
	   \refstepcounter{sugrefecpolinomios}
           \ifthenelse{\equal{\xxeleccion}{88}}{
		\noindent\textbf{Soluci\'on~8.\thesugcpolinomios}
		\def\@currentlabel{\thesugrefecpolinomios{}}
		#1\par
		\bigskip
	}{}
}

\newcommand{\problema}[1]{
	\stepcounter{prob}
	\refstepcounter{refe}
	\ifthenelse{\equal{\xxeleccion}{9}}{
		\noindent\textbf{Problema~\thechapter.\theprob}
		\def\@currentlabel{\therefe{}}
		{\it #1\par}
		\bigskip
	}{}
}

\newcommand{\solucion}[1]{
	   \stepcounter{sol}
	   \refstepcounter{sugrefe}
           \ifthenelse{\equal{\xxeleccion}{99}}{
		\noindent\textbf{Soluci\'on~9.\thesol}
		\def\@currentlabel{\thesugrefe{}}
		#1\par
		\bigskip
	}{}
}

\newcommand{\demostracion}[1]{
		\noindent\textbf{Demostraci\'on.}
		#1 \par
		\bigskip
}

\newcommand{\pdemostracion}[1]{
		\noindent\textbf{Primera demostraci\'on.}
		#1 \par
		\bigskip
}

\newcommand{\sdemostracion}[1]{
		\noindent\textbf{Segunda demostraci\'on.}
		#1 \par
		\bigskip
}

\newcommand{\solu}[1]{
		\noindent\textbf{Soluci\'on.}
		#1 \par
		\bigskip
}


\newcommand{\psolucion}[1]{
		\noindent\textbf{Primera Soluci\'on.}
		#1 \par
		\bigskip
}

\newcommand{\ssolucion}[1]{
		\noindent\textbf{Segunda Soluci\'on.}
		#1 \par
		\bigskip
}


\newcommand{\tsolucion}[1]{
		\noindent\textbf{Tercera Soluci\'on.}
		#1 \par
		\bigskip
}

\newcommand{\csolucion}[1]{
		\noindent\textbf{Cuarta Soluci\'on.}
		#1 \par
		\bigskip
}



\newcommand{\titulo}[1]{
\def\xxtitulo{#1}
}


\psset{unit=.4cm}
\psset{linewidth=1pt}
\SpecialCoor


\newtheorem{teorema}{Teorema}[section]
\newtheorem{lema}[teorema]{Lema}
\newtheorem{axioma}[teorema]{Axioma}
\newtheorem{axiomas}[teorema]{Axiomas}
\newtheorem{ejemplo}[teorema]{Ejemplo}
\newtheorem{propiedad}[teorema]{Propiedad}
\newtheorem{proposicion}[teorema]{Proposici\'on}
\newtheorem{ejemplos}[teorema]{Ejemplos}
\newtheorem{ejercicios}[teorema]{Ejercicios}
\newtheorem{soluciones}[teorema]{Soluciones}
\newtheorem{sugerencias}[teorema]{Sugerencias}
\newtheorem{ejercicio}[teorema]{Ejercicio}
\newtheorem{corolario}[teorema]{Corolario}
\newtheorem{definicion}[teorema]{Definici\'on}
\newtheorem{observacion}[teorema]{Observaci\'on}
\newtheorem{observaciones}[teorema]{Observaciones}
\newtheorem{propiedades}[teorema]{Propiedades}
\newtheorem{algoritmo}[teorema]{Algoritmo}


%%%%%%%%%%%%%%%%%%%%%%%%%INICIO %%%%%%%%%%%%%%%%%%%%%%%%%%%%%%%

\begin{document}


\titulo{
 \thispagestyle{empty}
 \vfill
 \noindent \Hrule
 \begin{flushright}
  {\bf\Huge Clase 3\\[1mm]
}
 \end{flushright}
  \Hrule
 \vfill
 \begin{flushright}
  {\bf\Large 
%  Radmila Bulajich Manfrino\\[1.2mm]
%  Jos'e Antonio G'omez Ortega\\[2.5mm]
  Rogelio Valdez Delgado}
 \end{flushright}

\cleardoublepage

}



\frontmatter
\xxtitulo
\setcounter{page}{9}

\thispagestyle{empty}
        \markboth{}{}
	 \Hrule
        \begin{flushright}	
           \huge\bf  Introducci'on
        \end{flushright}
       \Hrule
        \bigskip
	\addcontentsline{toc}{chapter}{\protect\numberline{}{\bf Introducci\'on}}
	\markboth{}{Introducci\'on}
	'Algebra se ha convertido en un 'area fundamental en las olimpiadas de 
mate\-m'a\-ticas. Son frecuentes los problemas de este tema que aparecen en los concursos,
y son tambi'en  frecuentes los problemas de otras 'areas que hacen uso del 'algebra para 
su soluci'on. En este libro queremos se\~nalar las principales herramientas de 'algebra 
que un alumno deber'a asimilar paso a paso en su preparaci'on para los concursos y olimpiadas 
de matem'aticas. 

Algunos de los t'opicos que tratamos en el libro son parte de los temarios de matem'aticas 
del bachillerato, pero hay otros que son de nivel universitario. Esto permite que el libro pueda 
ser utilizado como un texto de consulta para los alumnos del primer a\~no de la 
universidad  que gusten de enfrentar  problemas de 'algebra y tengan inter'es en aprender t'ecnicas para resolverlos.

El libro se ha dividido en diez cap'itulos. Los primeros cuatro corresponden a temas del 
bachillerato y son b'asicos para los alumnos que se entrenan para las olimpiadas de matem'aticas  
a nivel estatal y nacional. Los siguientes cuatro cap'itulos  usualmente se estudian 
en cursos del primer a\~no de una carrera universitaria, pero se han convertido en t'opicos y 
herramientas  que los alumnos que participan en 
competencias internacionales deben conocer y dominar.  Los 'ultimos dos cap'itulos contienen problemas y soluciones del material tratado a lo largo del libro.  



El primer cap'itulo,  cubre material b'asico de 'algebra, como son  sistemas n'umericos,  valor 
absoluto, productos notables, factorizaci'on, entre otros. Buscamos que el lector  adquiera 
destreza en la manipulaci'on de ecuaciones y f'ormulas algebraicas para llevarlas a formas 
equivalentes m'as f'aciles de entender y trabajar.

En el cap'itulo 2 se presenta el estudio de las sumas finitas de n'umeros, como por ejemplo, la 
suma de los cuadrados de los primeros $n$ naturales. Se analizan  sumas
telesc'opicas, progresiones aritm'eticas y geom'etricas, as'i como  varias de sus 
propiedades.

El cap'itulo 3  trata sobre la t'ecnica matem'atica para demostrar 
proposiciones, conocida como 
el principio de inducci'on matem'atica, adem'as se ejemplifica su uso con 
varios problemas. Tambi'en  se presentan  formulaciones equivalentes del 
principio de inducci'on.

\newpage

Para completar la primera parte del libro, en el cap'itulo 4 se estudian
polinomios cuadr'aticos y c'ubicos, haciendo 'enfasis en el estudio del 
discriminante de los cuadr'aticos y de las f'ormulas de Vieta para 
estos dos tipos de polinomios.

%Para completar esta primera parte del libro, en el cap'itulo 4 se estudian
%los polinomios cuadr'aticos y c'ubicos, haciendo 'enfasis en el estudio del 
%discriminante de un polinomio cuadr'atico y de las f'ormulas de Vieta para 
%estos dos tipos de polinomios.

La segunda parte del texto, inicia en el cap'itulo 5 donde se estudian los n'umeros complejos, 
sus propiedades y algunas aplicaciones. Todo esto siempre ejemplificado con problemas de las olimpiadas
de matem'aticas. Se incluye tambi'en una demostraci'on elemental del teorema fundamental del 'algebra.

En el cap'itulo 6 se estudian las propiedades principales de las funciones. Tambi'en se presenta una 
introducci'on a las ecuaciones funcionales, sus propiedades y  se dan algunas 
recomendaciones para resolver problemas donde aparecen ecua\-ciones de este tipo.

El cap'itulo 7 habla  de la noci'on de sucesi'on y serie. Se estudian  sucesiones especiales como las acotadas, 
las  periodicas, las mon'otonas, las recursivas entre otras.
Se introduce tambi'en el concepto de  convergencia para  sucesiones y series.

En el cap'itulo 8 se generaliza el estudio de los polinomios que se trat'o en la primera parte. 
Se presenta la teor'ia de polinomios de grado arbitrario y di\-versas t'ecnicas para analizar 
propiedades de los mismos. Al final del 
cap'itulo se  introducen los polinomios de varias variables.

La mayor'ia de las secciones de   estos primeros ocho  cap'itulos tienen al final una 
lista de ejercicios para el lector, seleccionados y  adecuados para practicar los temas que se abordan en ellas. 
La dificultad de los ejercicios var'ia desde ser una aplicaci'on directa de un resultado 
visto en la secci'on hasta ser un problema de un concurso, que con la t'ecnica tratada es factible resolver.  

El cap'itulo 9 es una recopilaci'on de problemas, cada uno de ellos  cercano a uno o m'as de los 
temas tratados en el libro. Estos problemas tienen un \linebreak grado de dificultad mayor a los ejercicios.  
La mayor'ia de ellos han aparecido en alg'un concurso u olimpiada de matem'aticas. En la soluci'on  de cada problema 
est'a impl'icito el conocimiento y destreza que se debe adquirir para la  manipulaci'on de expresiones algebraicas.  


Finalmente,  el cap'itulo 10 contiene las soluciones de todos los ejercicios y   
problemas planteados en el libro.

El lector podr'a notar que al final del t'itulo de algunas secciones aparece el s'imbolo $\star$, esto indica  
que el nivel de  tal secci'on  es m'as d'ificil.  En una primera lectura, el lector puede 
omitir estas secciones, sin embargo recomendamos que las tenga presentes por las t'ecnicas que se utilizan en ellas. 


Agradecemos infinitamente a Leonardo Ignacio Mart'inez Sandoval por sus \linebreak siempre 'utiles  comentarios y 
sugerencias, los cuales contribuyeron al mejoramiento del material presentado en este  libro.   

\vei

Radmila Bulajich \hspace{.75in} Jos'e Antonio G'omez \hspace{.75in}
Rogelio Valdez



	
	\newpage
\tableofcontents
\mainmatter
 \chapter{Preliminares}
\label{preliminares}
\setcounter{ejcprel}{0}
\setcounter{refejcprel}{0}
\def\xxeleccion{1}


\section{N'umeros}
\label{numeros}

\noindent Consideramos que el lector est'a familiarizado con el conjunto de 
n'umeros que se utilizan para contar. A este conjunto se le conoce como el 
conjunto de  n'umeros naturales y se denota por
$\nn$, es decir, \index{N'umeros naturales}
$$
     \nn = \{1,2,3,\dots\}.
$$
En este conjunto estamos acostumbrados a realizar dos operaciones, la suma y la multiplicaci'on, 
entendiendo con esto que si sumamos o multiplicamos dos n'u\-meros del conjunto obtenemos otro n'umero natural. 
A estas operaciones las conocemos como la suma  (o adici'on) y la multiplicaci'on (o producto). En \linebreak algunos libros el 0 se considera tambi'en como un n'umero natural, sin embargo, en este libro no, pero convenimos que 0 es tal que $n+0=n$, para todo n'umero natural $n$. 

Ahora, supongamos que deseamos resolver la ecuaci'on $x+a = 0$, con $a\in \nn$, es decir, encontrar una $x$ para la cual la igualdad anterior se cumpla.  
Esta ecuaci'on no tiene soluci'on en el conjunto de los n'umeros naturales $\nn$, por lo cual necesitamos definir un conjunto de n'umeros que incluya al 
conjunto de n'umeros $\nn$ y a sus negativos.  Es decir, necesitamos extender el conjunto de los n'umeros
$\nn$ para que este tipo de ecuaciones tengan soluci'on en el nuevo conjunto.   A este  conjunto lo llamamos el conjunto de los  n'umeros enteros y lo  denotamos por $\zz$, es decir,\index{N'umeros enteros}
$$
     \zz = \{\dots, -3 ,-2, -1, 0, 1, 2, 3,\dots\}.
$$
En este conjunto tambi'en hay dos operaciones, la suma y la
multiplicaci'on,  que satisfacen las siguientes propiedades.


\begin{propiedades}
\begin{description}
\item[$(a)$] La suma y la multiplicaci'on de n'umeros enteros son operaciones 
conmutativas. 
Esto es, si $a,\,b\in \zz$, entonces
$$
       a+b= b+a\quad \text{y}\quad ab = b a.
$$
\item[$(b)$] La suma y el producto de n'umeros enteros son operaciones 
asociativas. Esto es, 
si $a,\,b\,
\text{y}\; c\in \zz$, entonces
$$
       (a+b)+c= a+(b+c) \quad \text{y}\quad (ab)c= a(bc).
$$
\item[$(c)$] Existe en $\zz$ un elemento neutro para la suma, el n'umero 0. 
Es decir,
si $a\,\in \zz$, entonces
$$
       a+0= 0+a=a.
$$
\item[$(d)$] Existe en $\zz$  un elemento neutro para la multiplicaci'on, el 
n'umero 1.
Es decir,
si $a\,\in \zz$, entonces
$$
       a1= 1a=a.
$$
\item[$(e)$] Para cada $a\,\in\zz$ existe  su inverso aditivo que se
denota por $-a$. Esto es,
$$
       a+(-a)= (-a)+a=0.
$$
\item[$(f)$] En $\zz$, el producto se  distribuye con respecto a la suma.  
Es decir, si
$a$, $b$ y $c \in \zz$, entonces
\begin{equation*}
       a(b+c) =   ab+ac.
\end{equation*}
\end{description}
\end{propiedades}


\vei

Notemos que la existencia del inverso aditivo nos permite resolver cualquier 
ecuaci'on del tipo mencionado, es decir, $x+a= b$, donde $a$ y $b$ son 
n'umeros enteros. Sin embargo,
no existe necesariamente un n'umero entero $x$ que resuelva la ecuaci'on 
$qx=p$, con
$p$ y $q$ n'umeros enteros, por lo que nuevamente surge la necesidad de 
extender el  conjunto de n'umeros.  Consideramos ahora el  conjunto de los 
n'umeros 
racionales, que denotamos  como $\qq$, es decir,  \index{N'umeros racionales}
$$
     \qq = \left \{\frac{p}{q} \ | \ p\in \zz \ \text{y} \ q\in 
\zz\backslash \{0\}  \right \}.
$$
En general, para trabajar con los n'umeros racionales $\frac{p}{q}$ pedimos 
que $p$ y $q$ no tengan factores primos comunes, es decir, que sean primos 
relativos, esto lo denotamos como $(p,q)=1$.
En el conjunto de n'umeros racionales  tambi'en existen las operaciones de 
suma y  producto, las cuales cumplen las mismas propie\-dades que los n'umeros 
enteros.  Adem'as, en el producto existe otra  propiedad: la existencia del inverso multiplicativo.

\begin{propiedad}
Si $\frac{p}{q}\in \qq$, con $p\neq 0$ y $(p,q)=1$, entonces existe un 
'unico n'umero, $\frac{q}{p}\in \qq$, llamado el inverso multiplicativo 
de $\frac{p}{q}$ tal que
$$
               \frac{p}{q}\cdot \frac{q}{p} = 1.
$$
\end{propiedad}
\index{N'umeros racionales}


\noindent Con esta nueva propiedad tenemos garant'ia de poder resolver cualquier ecuaci'on de la forma $q x = p$. Sin embargo, existen n'umeros que no podemos escribir como cociente de dos n'umeros enteros, por ejemplo, si queremos resolver la ecuaci'on  $x^2-2=0$, 'esta no tiene soluci'on en el conjunto de los 
n'umeros $\qq$. Las soluciones de la ecuaci'on son 
$x=\pm\sqrt{2}$ y mostramos que $\sqrt{2}$ no est'a en $\qq$.

\begin{proposicion}
El n'umero $\sqrt{2}$ no es un n'umero racional.
\end{proposicion}

\demostracion{Supongamos lo contrario, es decir, que $\sqrt{2}$ es un n'umero racional, entonces lo podemos escribir como $\sqrt{2}=\frac{p}{q}$, donde $p$ y $q$ no tienen factores comunes. Elevando al cuadrado de ambos lados tenemos que
$2=\frac{p^2}{q^2}$, es decir, $2q^2= p^2$. Esto quiere decir, que $p^2$ es un 
n'umero par, pero entonces el mismo $p$ es par.  Pero si $p$ es par, digamos de la forma $p=2m$,  entonces $2q^2= (2m)^2= 4m^2$. Dividiendo entre 2 ambos lados de  la ecuaci'on tenemos que $q^2=2m^2$, esto es, $q^2$ es par y entonces $q$ es tambi'en par. As'i, $p$ y $q$ son pares,  contradiciendo el hecho de que $p$ y $q$ no tienen factores comunes.  Por lo tanto, $\sqrt{2}$ no es un n'umero 
racional.
}

Podemos dar una representaci'on geom'etrica de los n'umeros racionales como puntos sobre una recta, que llamamos la 
{\bf recta num'erica}\index{Recta!num'erica}. Una recta la podemos recorrer en dos sentidos, a uno de ellos le llamamos  
{\bf sentido positivo} y al otro {\bf sentido negativo}. Una vez convenido cual es el sentido positivo decimos que tenemos 
una {\bf recta orientada}\index{Recta!orientada}. Por ejemplo, podemos convenir que el sentido positivo es el que va de 
izquierda a derecha. Si consideramos dos puntos $O$ y $U$ en la recta, le daremos la misma orientaci'on al segmento que a 
la recta que lo contiene. Esto es, tomando el punto $O$ como el 0 y 
$U$  a la derecha de 'el, decimos que el segmento $OU$ se recorre en el sentido positivo. Si el punto $U$ representa al 1, 
llamamos a $OU$ un segmento orientado y unitario. As'i, podemos ir colocando, hacia la derecha, todos los n'umeros enteros 
positivos a lo largo de la recta separados, cada dos consecutivos,  una distancia $OU$. Para representar los n'umeros enteros 
negativos basta que hagamos lo mismo pero ahora iniciando en $O$ y recorriendo la recta en sentido negativo.

El n'umero racional de la forma $\frac{p}{q}$, lo definimos como el segmento orientado
$\frac{p}{q}  OU$
que es el segmento que se obtiene al sumar $p$ veces la $q$-'esima parte del segmento $OU$.  Con m'as precisi'on, hacemos lo siguiente:

$(a)$ Dividimos en $q$ partes iguales el segmento $OU$. Para esto, trazamos una
recta auxiliar por $O$  y sobre ella tomamos $q$ puntos $W_1$, $\dots$, $W_q$, 
donde dos consecutivos est'an separados una distancia $OW_1$. Ahora, se une 
$W_q$ con $U$ y por cada uno de los puntos $W_j$ se traza una recta paralela a
$UW_q$, los puntos de corte de las paralelas con $OU$ ser'an los puntos de 
divisi'on de $OU$ en $q$ partes iguales. Si $V$ es el punto de corte de la
paralela $UW_q$ por $W_1$, se tendr'a que $V$ es el punto que representa al 
n'umero $\frac{1}{q}$ (note que $OV$ tiene la misma orientaci'on de $OU$).  
Consideramos  tambi'en $V^\prime$ el punto sim'etrico, con respecto a $O$, de 
$V$.  En la siguiente figura, hemos tomado $q=4$

\centerline{
\psset{unit=1.5cm}
\begin{pspicture}(0,-2)(0,2.5)
       \psline(-4,0)(4,0)
        \psline(-2,-2)(2,2)
       	\psline(.3,.28)(.5,0)
        \psline(.58,.58)(1,0)
        \psline(.88,.88)(1.5,0)
         \psline(1.18,1.18)(2,0)
	\psline(-.5,0)(-.3,-.28)
       	\psline(-1,0)(-.58,-.58)
        \psline(-1.5,0)(-.88,-.88)
        \psline(-2,0)(-1.18,-1.18)
        \psdot(2,0)
        \psdot(4,0)
        \psdot(.5,0)
       \psdot(-.5,0)	
       \psdot(1,0)
        \psdot(1.5,0)
       \psdot(.58,.58)
       \psdot(.88,.88)
       \psdot(1.18,1.18)
       \psdot(-2,0)
       \psdot(-1.18,-1.18)
        \psdot(.3,.3)
        \psdot(-4,0)
         \psdot(-1,0)
        \psdot(-1.5,0)
        \psdot(-1,0)
         \psdot(-.3,-.3)
        \psdot(-.58,-.6)
       \psdot(-.88,-.9)
        \psdot(-1.5,0)
        \rput(0,-.3){$O$}
        \rput(0,.2){$0$}
	\rput(2,-.2){$U$}
	\rput(2.1,.2){$1$}
        \rput(4,.2){$2$}
       \rput(.5,-.2){$V$}
       \rput(-.5,.2){$V^\prime$}
       \rput(.2,.5){$W_1$}
       \rput(1.05,1.4){$W_q$}
     \end{pspicture}
}


$(b)$ Si $p$ es un n'umero entero no negativo, tomamos
$$
         OP=\underbrace{OV+OV+\cdots+OV}_{p\,\text{ veces}} = p \cdot OV.
$$
El segmento $OP$ es, por definici'on, $\frac{p}{q} OU$.  En la siguiente 
figura marcamos al punto $P$, con $p=6$ y $q=4$

\centerline{
\psset{unit=1.5cm}
\begin{pspicture}(0,-1)(0,1)
       \psline(-4,0)(4,0)
        \psdot(2,0)
        \psdot(4,0)
        \psdot(.5,0)	
       \psdot(1,0)
        \psdot(1.5,0)
       \psdot(-2,0)
       \psdot(2.5,0)
        \psdot(3,0)
       \psdot(0,0)
        \rput(0,-.3){$O$}
        \rput(0,.3){$0$}
	\rput(2,-.3){$U$}
        \rput(-2,-.3){$U^\prime$}
	\rput(2,.3){$1$}
        \rput(4,.3){$2$}
       \rput(3,-.3){$P$}
     \end{pspicture}
}

$(c)$ Si $p$ es negativo, sea $p^\prime$ el n'umero entero positivo tal que $p=-p^\prime$. Tomamos entonces
$$
         OP=\underbrace{OV^\prime+OV^\prime+\cdots+OV^\prime}_{p^\prime\,\text{ veces}} = p^\prime OV^\prime=(-p^\prime) OV=p \cdot OV.
$$
El segmento $OP$ es, por definici'on, $\frac{p}{q} OU$.  Como $OU$ es el segmento unitario, a este punto lo denotamos simplemente como  $\frac{p}{q}$.

\vei

Con esta representaci'on de los n'umeros racionales, tenemos que todo n'umero racional est'a representado por un punto en la recta num'erica, pero hay puntos en la recta num'erica que no representan ning'un  n'umero racional.  Por ejemplo, determinemos en la recta num'erica el n'umero $\sqrt{2}$, que ya vimos que no es un n'umero racional.  Si tomamos un tri'angulo rect'angulo con catetos de longitud 1, entonces, por el teorema de Pit'agoras, la hipotenusa de este tri'angulo mide $\sqrt{2}$.  Si tomamos un comp'as y trazamos una circunferencia de radio $\sqrt{2}$ y centro en $0$, el punto donde la circunferencia intersecta a la parte 
positiva de la recta num'erica es el punto en la recta num'erica que 
corresponde a $\sqrt{2}$.

\centerline{
\psset{unit=1.5cm}
\begin{pspicture}(0,-.5)(3,2.5)
       \psline(-1,0)(4,0)
        \psdot(2,0)
        \psdot(2,2)
        \psline(2,0)(2,2)(0,0)	
       \psarc(0,0){2.83}{0}{45}
        \rput(0,-.3){$O$}
        \rput(0,.3){$0$}
	\rput(2,-.3){$U$}
        \rput(1,-.2){$1$}
	\rput(2.3,1){$1$}
        \rput(.8,1.1){$\sqrt{2}$}
       \rput(2.83,-.2){$\sqrt{2}$}
     \end{pspicture}
}
\noindent Un punto de la recta num'erica que no corresponda 
a un n'umero racional representar'a a un n'umero irracional y al conjunto de 
los n'umeros irracionales lo denotamos por $\ii$.
\index{N'umeros irracionales}

A la uni'on de estos dos conjuntos, le llamamos el conjunto de los n'umeros 
reales y lo denotamos como $\rr$, es decir, 
$\rr=\qq\cup\ii$.\index{N'umeros reales}

\noindent El conjunto de los n'umeros $\rr$ contiene al conjunto de los n'umeros naturales, al  de los n'umeros enteros y al de los n'umeros racionales.  De hecho, tenemos las siguientes contenciones $\nn\subset\zz\subset\qq\subset\rr$.

\vei

Dados dos puntos en la recta num'erica que sabemos representan a dos n'umeros reales, podemos localizar al punto que es la suma de ellos  de la siguiente manera:  si $P$ y $Q$ son dos puntos sobre la recta y $O$ es el origen,  la suma ser'a la suma de  los segmentos dirigidos $OP$ y $OQ$, como se muestra en la siguiente figura.


\centerline{
\psset{unit=1.5cm}
\begin{pspicture}(0,-1)(-1,2)
       \psline(-4,0)(2,0)
        \psdot(-2,0)
        \psdot(-3,0)	
        \psdot(0,0)
        \psdot(1,0)
        \psline(-3,.7)(-2.9,.8)(-2.6,.8)(-2.5,.9)(-2.4,.8)(-2.1,.8)(-2,.7)		
      \psline(-3,.5)(-2.9,.6)(-1.6,.6)(-1.5,.7)(-1.4,.6)(-.1,.6)(0,.5)
	\psline(-3,1.3)(-2.9,1.4)(-1.1,1.4)(-1,1.5)(-.9,1.4)(.9,1.4)(1,1.3)
      \psline(0,.5)(.1,.6)(.4,.6)(.5,.7)(.6,.6)(.9,.6)(1,.5)		
        \rput(-3,-.3){$O$}
        \rput(-3,.3){$0$}
	\rput(-2,-.3){$P$}
        \rput(0,-.3){$Q$}
        	\rput(-2.5,1.15){$OP$}
	\rput(.5,.9){$OP$}
        	\rput(-1.5,.9){$OQ$}
	\rput(1,-.3){$P+Q$}
	\rput(-1,1.7){$OP+OQ$}
     \end{pspicture}
}

Asimismo, podemos encontrar el punto que representa el producto de dos puntos $P$ y $Q$  sobre la recta num'erica como sigue.  Consideremos una recta auxiliar que ser'a una  copia de la recta real con el mismo origen $O$.  Localizamos en la recta auxiliar la unidad $U$ y el punto $Q$. Por $Q$ trazamos la recta paralela a $UP$ la cual intersecta a la recta real en $R$.


Como los tri'angulos  $ORQ$ y $OPU$ son semejantes tenemos que
$\frac{OR}{OP}=\frac{OQ}{OU}$ por lo que $OR\cdot OU=OP\cdot OQ$,
de donde $OR$ representa al producto de $P$ y $Q$.

\centerline{
\psset{unit=1.5cm}
\begin{pspicture}(0,-1.3)(2,2.5)
       \psline(-1.5,0)(3,0)
       \psline(-1,-1)(2,2)       
       \psline(.58,.58)(1,0)
       \psline(1.18,1.18)(2,0)
       \psdot(1,0)
       \psdot(.58,.58)
       \psdot(2,0)
       \psdot(1.18,1.18)
       \rput(1,1.35){$Q$}
       \rput(0,-.2){$O$}
       \rput(2,-.2){$R$}
       \rput(.5,.8){$U$}
       \rput(1,-.2){$P$}
     \end{pspicture}
}


Con esto podemos localizar la suma y el producto de cualesquiera dos n'u\-meros  
reales sobre la recta num'erica, sin importar si son n'umeros racionales o 
irracionales.  

Al igual que en el conjunto de los n'umeros enteros,  
las operaciones en el conjunto de n'umeros reales cumplen  todas las  
propiedades mencionadas.

\begin{propiedades}
\begin{description}
\item[$(a)$] La suma de dos n'umeros reales es un n'umero real.
\item[$(b)$] La suma de dos n'umeros reales es conmutativa.
\item[$(c)$] La suma es asociativa.
\item[$(d)$] El n'umero $0$ es el neutro aditivo. Es decir, $x+0=x$, para todo $x \in \rr$.
\item[$(e)$] Todo n'umero real $x$ tiene un inverso aditivo. Es decir, existe un n'umero real que se denota por $-x$ y cumple que  $x+(-x)= 0$.
\item[$(f)$] El producto de dos n'umeros reales es un n'umero real.
\item[$(g)$] El producto  es conmutativo.
\item[$(h)$] El producto  es asociativo.
\item[$(i)$] El n'umero $1$ es el neutro multiplicativo. Es decir, $x\cdot 1=x$, para todo $x \in \rr$.
\item[$(j)$] Todo n'umero real $x$ distinto de $0$, tiene inverso multiplicativo.  Es decir, existe  un n'umero real que se denota por $x^{-1}$, tal que
$x\cdot x^{-1} =1$.
\item[$(k)$] El producto distribuye a la suma, es decir, si
$x$, $y$, $z\in \rr$, entonces
$$
      x \left(y+z\right ) = x\cdot y+ x\cdot z.
$$
\end{description}
\end{propiedades}



\noindent En los n'umeros enteros tenemos un orden. Con esto queremos se\~nalar que  dados dos n'umeros enteros $a$ y $b$ podemos decir cual es el mayor de ellos.  Decimos que $a$ {\bf es mayor que} $b$ si $a-b$ es un n'umero natural, en s'imbolos tenemos\index{N'umeros!mayor que}
$$
       a > b \;\;\text{si y s'olo si}\;\; a-b \in\nn.
$$
Esto es equivalente a decir que $a-b>0$.

\noindent En general, la notaci'on $a>b$ es equivalente a $b<a$.  La expresi'on $a\geq b$ significa que $a>b$ o $a=b$. An'alogamente, $a\leq b$ significa que $a < b$ o $a=b$.


\begin{propiedades}
Si $a$ es un n'umero entero, se cumple una y solamente una de las condiciones siguientes:
\begin{description}
\item[$(a)$] $a>0$,
\item[$(b)$] $a=0$,
\item[$(c)$] $a<0$.
\end{description}
\end{propiedades}

En los n'umeros racionales y en los n'umeros reales tambi'en hay un orden.   El orden en los n\'umeros reales
nos permitir\'a comparar dos n\'umeros y
decidir cual de ellos es mayor o bien si son iguales. A fin de evitar
justificaciones tediosas, asumiremos que en los n\'umeros reales hay un
conjunto $P$ que llamamos el conjunto de n\'umeros positivos, y
simb\'olicamente escribimos $x>0$, para decir que un n\'umero $x$
est\'a en $P$. En la representaci'on geom'etrica de los n'umeros reales, el conjunto $P$ en la 
recta num'erica es, de las dos partes en que $O$ ha dividido a la recta, la parte que contiene a $U$ (el $1$). Resaltamos que se cumplen las   propiedades siguientes.

\ve

\begin{propiedad}
Cada n\'umero real $x$ tiene una y s\'olo una de
las siguientes caracter\'{\i}sticas:
\label{tricotomia}
\begin{description}
\item[$ (a)$] $x=0$.
\item[$(b)$] $x \in P$ (esto es $x>0$).
\item[$(c)$]$-x\in P$ (esto es $-x>0$).
\end{description}
\end{propiedad}

\begin{propiedad} 
Sean $x$, $y$ n'umeros reales.
\begin{description}
\item[$(a)$] Si $x$, $y \in P$, entonces $x+y\in P$
\noindent (en s\'{\i}mbolos $x>0$, $y>0\Rightarrow x+y>0$).
\label{propsuma}
\item[$(b)$] Si $x$, $y\in P$, entonces $xy\in P$
(en s\'{\i}mbolos $x>0$, $y>0\Rightarrow xy>0$).
\label{propproducto}
\end{description}
\end{propiedad}

\noindent Ahora podemos definir la relaci\'on $x$ {\bf es mayor que}
\index{N'umeros!mayor que} $y$,
si $x-y\in P$ (en s\'{\i}mbolos $x>y$). An\'alogamente, $x$ {\bf es menor
que}\index{N'umeros!menor que} $y$, si $y-x\in P$ (en s\'{\i}mbolos $x<y$).
Observemos que
$x<y$ es equivalente a $y>x$. Definimos tambi\'en, $x$ {\bf es menor o igual
que}\index{N'umeros!menor o igual que} $y$, si $x<y$ o $x=y$,
(en s\'{\i}mbolos $x\leq y$).

\noindent Denotamos al conjunto $P$ de
n\'umeros reales positivos por $\rr^{+}$.

\begin{ejemplo}  Sean $x$, $y$, $z$ n'umeros reales.
\begin{description}
\item[$(a)$] Si $x<y$, entonces $x+z<y+z$.
\item[$(b)$] Si $x<y$ y $z>0$, entonces $xz<yz$.
\label{ordenenlosreales}
\end{description}
\end{ejemplo}

\noindent En efecto, para mostrar $(a)$ tenemos que $x+z<y+z$ si y s'olo si
$(y+z)-(x+z)>0$ si y s'olo si $y-x>0$ si y s'olo si $x<y$. Para ver $(b)$, tenemos
que $x<y$ implica $y-x>0$ y como $z>0$, resulta que $(y-x)z>0$, luego
$yz-xz>0$ y entonces $xz<yz$.

\ve 


%%%1
\ejercpreliminares{Muestre las siguientes afirmaciones:
\label{maspormas}
\begin{description}
\item[$ (i)$] Si $a<0$, $b<0$, entonces $ab>0$.

\ven

\item[$ (ii)$] Si $a<0$, $b>0$, entonces $ab<0$.

\ven

\item[$ (iii)$] Si $a<b$, $b<c$, entonces $a<c$.

\ven

\item[$ (iv)$] Si $a<b$, $c<d$, entonces $a+c<b+d$.

\ven


\item[$ (v)$] Si $a>0$, entonces $a^{-1}>0$.

\ven

\item[$ (vi)$]  Si $a<0$, entonces $a^{-1} <0$.
\end{description}
}

\solcpreliminares{$(i)$ Si $a<0$, entonces $-a>0$. Use
tambi\'en que $(-a)(-b)=ab$. 
$(ii)$ $(-a)b>0$. 
$(iii)$ $a<b\Leftrightarrow b-a>0$, use ahora la propiedad \ref{propsuma}. 
$(iv)$ Use la propiedad \ref{propsuma}. 
$(v)$ $aa^{-1}=1>0$. 
$(vi)$ Si $a<0$, entonces $-a>0$.
}


%%%%%%% 2
\ejercpreliminares{Sean $a$, $b$ n\'{u}meros reales.  Muestre que, si 
$a+b$, $a^2+b$ y $a+b^2$ son n'umeros racionales y $a+b\neq 1$, entonces 
$a$ y $b$ son n'umeros racionales.
}

\solcpreliminares{Observe que si $a^2+b-(a+b^2)\in \qq$, entonces
$(a-b)(a+b-1)\in \qq$ y como $a+b-1\in \qq\setminus\{0\}$, entonces 
$(a-b)\in \qq$. Luego, si $a+b\in\qq$ y $a-b\in \qq$, entonces $2a$ y 
$2b$ est'an en $\qq$. Por lo tanto, $a$ y $b$ son n'umeros racionales.
}

%%%%%%%%% 3
\ejercpreliminares{Sean $a, b$ n\'{u}meros reales tales que 
$a^2+b^2$, $a^3+b^3$ y $a^4+b^4$ son  n'umeros racionales.  Muestre que 
$a+b$, $ab$ son tambi'en  n'umeros racionales.

}

\solcpreliminares{Si $a=0$ o $ b=0$ el resultado es claro. 
Suponga entonces que $ab\neq 0$. Como $(a^2+b^2)^2-(a^4+b^4)=2a^2b^2$, 
se tiene que $a^2b^2\in\qq$.  Note que $a^6+b^6=(a^2+b^2)^3-
3a^2b^2(a^2+b^2)\in\qq$, por lo que $(a^3+b^3)^2-(a^6+b^6)=2a^3b^3\in\qq$. 
Luego, 
$$
  ab= \frac{a^3b^3}{a^2b^2}\in \qq\quad \text{y}\quad 
 a+b= \frac{a^3+b^3}{a^2+b^2-ab}\in\qq.
$$
}

%%%%%%%%% 4
\ejercpreliminares{$(i)$\, Demuestre que si $p$ es un n'umero primo, 
entonces $\sqrt{p}$ es un n'umero irracional.

$(ii)$\, Demuestre que si $m$ es un n'umero entero positivo que no 
es cuadrado perfecto, entonces $\sqrt{m}$ es un n'umero irracional.
}

\solcpreliminares{$(i)$ Suponga que $\sqrt{p}$ no es un n'umero irracional, 
es decir, \linebreak $\sqrt{p}=\frac{m}{n}$, donde $m$, $n$ son n'umeros enteros 
con $(m,n)=1$, es decir, $m$ y $n$ primos relativos. Elevando al cuadrado, 
se tiene $p n^2=m^2$, esto es, $p$ divide a $m^2$, entonces $p$ divide a $m$.  
Por lo que  $m=ps$ y  $pn^2 = p^2 s^2$ implican que   $n^2 = p s^2$, lo cual 
garantiza que $p$ divide a $n^2$ y entonces divide a  $n$. Luego, $p$ divide 
a $m$ y a $n$ contradiciendo el hecho de que  $m$ y $n$ son primos relativos. 

$(ii)$ Suponga que $\sqrt{m}$ no es un n'umero irracional, 
es decir, $\sqrt{m}=\frac{r}{s}$, donde $r$, $s$ son n'umeros enteros 
con $(r,s)=1$. Elevando al cuadrado 
se tiene $m s^2=r^2$. Como $m$ no es un cuadrado perfecto, tiene un factor 
de la forma $p^{\alpha}$, donde $p$ es un n'umero primo y $\alpha$ es un 
entero positivo impar. Entonces, $p^{\alpha}$ divide a $r^2$ lo que implica que 
el primo $p$ aparece un n'umero par de veces en la descomposici'on de factores 
de $r^2$. Como $r$ y $s$ son primos relativos, $p$ no divide a $s$, de donde
$p$ aparece un n'umero impar de veces como factor de $m s^2$, lo cual es una 
contradicci'on.
}

%%%%5
\ejercpreliminares{Demuestre que existen una infinidad de parejas de n'umeros 
irracionales $a$, $b$ tales que $a+b=ab$ y adem'as este n'umero es entero.
}

\solcpreliminares{Si $a+b=ab=n$, entonces $b=n-a$ y $n=a(n-a)$.  
La 'ultima ecuaci'on es equivalente a $a^2-na+n=0$ y resolviendo se obtiene que
$$
 a=\frac{n\pm\sqrt{n^2-4n}}{2},
\quad\text{de donde }\quad b=\frac{n\mp\sqrt{n^2-4n}}{2}.
$$
Para $n\geq 5$, se tiene que 
$ (n-3)^2 <n^2-4n <(n-2)^2,$
por lo que $\sqrt{n^2-4n}$ es un  n'umero irracional, y entonces $a$ y 
$b$ son n'umeros irracionales.
}

%%%%% 6
\ejercpreliminares{Si los coeficientes de 
$$
   a x^2+b x+c =0
$$
son n'umeros enteros impares, entonces las ra'ices de la ecuaci'on 
no pueden ser n'umeros racionales.
}

\solcpreliminares{Suponga que $\frac{m}{n}$ es  ra'iz,  con $(m,n)=1$. 
Entonces $m$ y $n$ no pueden ser  ambos pares.  Por otro lado, como 
$a \left (\frac{m}{n}\right )^2+b \left (\frac{m}{n}\right )+c =0$,
se tiene que $ a m^2+b mn+c n^2 =0$. El lado derecho de la  'ultima 
ecuaci'on  es par y el izquierdo siempre es impar. Si  $m$ y $n$ son 
impares, los tres sumandos del lado izquierdo son impares. Ahora bien,  
si uno de ellos es par y el otro impar, entonces dos sumandos son pares, 
el tercero impar y la suma es impar nuevamente. Esta contradicci'on 
implica que la ecuaci'on no puede tener ra'ices racionales. 

\vei 

\ssolucion{El discriminante $b^2 - 4 ac$ deber'a ser un cuadrado 
perfecto. Pero como $a$, $b$ y $c$ son impares, se puede mostrar que
$b^2 - 4 ac\equiv 5$ $\mod 8$. Sin embargo, los cuadrados de 
n'umeros impares s'olo dejan residuo 1 m'odulo 8.}
}

%%%%%%%%%% 7
\ejercpreliminares{Muestre que para n'umeros reales positivos 
$a$ y $b$, con $\sqrt{b} < a$, se tiene que
$$
\sqrt{a+\sqrt{b}}=\sqrt{\frac{a+\sqrt{a^{2}-b}}{2}}+
\sqrt{\frac{a-\sqrt{a^{2}-b}}{2}}.
$$
}

\solcpreliminares{Sea $u=a+\sqrt{b}$ y $v=a-\sqrt{b}$, entonces
\begin{eqnarray*}
 \sqrt{a+\sqrt{b}}& = & \sqrt{u}=\frac{\sqrt{u}+\sqrt{v}}{2} + 
\frac{\sqrt{u}-\sqrt{v}}{2}\\
& = & \sqrt {\frac{(\sqrt{u}+\sqrt{v})^2}{4}} + 
\sqrt {\frac{(\sqrt{u}-\sqrt{v})^2}{4}}\\
& = & \sqrt {\frac{\frac{u+v}{2}+\sqrt{uv}}{2}} 
+\sqrt {\frac{\frac{u+v}{2}-\sqrt{uv}}{2}}\\
& = & \sqrt {\frac{\frac{a+\sqrt{b}+a-\sqrt{b}}{2}+
\sqrt{a^2-b}}{2}} +\sqrt {\frac{\frac{a+\sqrt{b}+a-
\sqrt{b}}{2}-\sqrt{a^2-b}}{2}}\\
& = & \sqrt {\frac{a+\sqrt{a^2-b}}{2}} 
+\sqrt {\frac{a-\sqrt{a^2-b}}{2}},
\end{eqnarray*}
como se quer'ia probar. 
}

%%%%%%% 8
\ejercpreliminares{Para n\'{u}meros positivos $a$ y $b$ 
encuentre el valor de:

\vei

\noindent $(i)$ $\sqrt{a\sqrt{a\sqrt{a\sqrt{a \dots}}}}.$ 
\qquad \qquad \qquad 
$(ii)$ $\sqrt{a\sqrt{b\sqrt{a\sqrt{b\dots}}}}.$
}

\solcpreliminares{$(i)$ Sea $x=\sqrt{a\sqrt{a\sqrt{a\sqrt{a \dots}}}}$, 
entonces $x^2=a\sqrt{a\sqrt{a\sqrt{a\sqrt{a \dots}}}}$, de donde $x^2=ax$. 
Factorizando, $x(x-a)=0$.  Por lo tanto, como $a$ es positivo la soluci'on 
es $x=a$.

\vei

\ssolucion{Podemos dar otra soluci'on utilizando series.  Tenemos que 
$$
     x=a^{\frac{1}{2}}a^{\frac{1}{4}}a^{\frac{1}{8}}\ldots=a^{\frac{1}{2}+\frac{1}{4}+\frac{1}{8}
+\cdots}= a,
$$
ya que $\sum_{j=1}^{\infty} \frac{1}{2^j}=1$, ver la secci'on 
\ref{seriesdepotencia}.
}

$(ii)$ Sea $x=\sqrt{a\sqrt{b\sqrt{a\sqrt{b\dots}}}}$, entonces 
$x^2=a\sqrt{b\sqrt{a\sqrt{b\sqrt{a\dots}}}}$, de donde 
$x^4=a^2 b x$. Como  $x\neq 0$, $x^3=a^2 b$. Entonces $x=\sqrt[3]{a^2b}$.

\vei

\ssolucion{Podemos tambi'en hacer otra soluci'on utilizando series.  
Tenemos que 
$$
     x=a^{\frac{1}{2}+\frac{1}{8}+\frac{1}{32}+\cdots}\, 
b^{\frac{1}{4}+\frac{1}{16}+\frac{1}{64}+\cdots}=a^{\frac{2}{3}}\,b^{\frac{1}{3}},
$$
ya que $\sum_{j=1}^{\infty} \frac{1}{2^{2j}}=\frac{1}{3}$ y 
$\sum_{j=0}^{\infty} \frac{1}{2^{2j+1}}=\frac{2}{3}$, ver la secci'on 
\ref{seriesdepotencia}. 
}
}

%%%%% 9
\ejercpreliminares{(Rumania, 2001) Sean  $x$, $y$ y $z$ n'umeros 
reales distintos de cero tales que
$xy$, $yz$ y $zx$ son n'umeros racionales. Muestre que:

$(i)$ $x^2+y^2+z^2$ es un n'umero racional.

$(ii)$ Si $x^3+y^3+z^3$ es un n'umero racional distinto de cero, 
entonces 
$x$, $y$ y $z$ son n'umeros racionales.
}

\solcpreliminares{$(i)$ Si $xy$, $yz$  y $zx$ est'an en $\qq$, 
entonces $\frac{(xy)(zx)}{yz}=x^2\in\qq$.  An'alogamente, 
$y^2$, $z^2$ $\in \qq$. Por lo tanto, $x^2+y^2+z^2\in\qq$.

$(ii)$ Por  $(i)$ se tiene que  $(x^2)^2+(xy)y^2+(xz) 
z^2=x(x^3+y^3+z^3)\in \qq$, luego  $x\in \qq$.  
An'alogamente, $y$, $z$ $\in \qq$. 
}

%%%%%%% 10
\ejercpreliminares{(Rumania, 2011) Sean $a$, $b$ n\'{u}meros reales 
positivos y distintos, tales que $a-\sqrt{ab}$ y $b-\sqrt{ab}$ 
son ambos n'umeros racionales. Muestre que $a$ y $b$ son  
n'umeros racionales.
}

\solcpreliminares{Como $a-\sqrt{ab}=a\left( 1-\frac{\sqrt{b}}{\sqrt{a}}\right)$,
bastar'a ver que  $1-\frac{\sqrt{b}}{\sqrt{a}}$ es un  n'umero 
racional distinto de cero para asegurar que $a$ es un  n'umero racional. 

\noindent Pero $ \frac{b-\sqrt{ab}}{a-\sqrt{ab}}=
\frac{\sqrt{b}(\sqrt{b}-\sqrt{a})}{\sqrt{a}(\sqrt{a}-\sqrt{b})}= 
-\frac{\sqrt{b}}{\sqrt{a}}$ 
es un  n'umero  racional diferente de $-1$ (ya que $a\neq b$), 
luego $1-\frac{\sqrt{b}}{\sqrt{a}}$ es un  n'umero racional 
distinto de 0. An'alogamente, $b$ es un  n'umero racional.
}











\index{Sistema decimal}
\noindent El sistema decimal es un sistema posicional en el que
cada d'igito toma un valor de acuerdo a su posici'on con relaci'on al punto
decimal. Esto es, el d'igito se multiplica por una potencia de 10.  
Para el d'igito de las unidades, o sea, el
d'igito que est'a inmediatamente a la izquierda del punto decimal, lo  tenemos que multiplicar por $10^n$, con $n=0$.  El d'igito de las decenas lo multiplicamos por $10^1=10$. El exponente
aumenta de uno en uno conforme nos movemos a la izquierda y disminuye de uno en uno conforme nos movemos a la derecha. Por ejemplo,
$$
    87325.31= 8\cdot 10^4+7 \cdot 10^3+3\cdot 10^2+2\cdot 10^1+5\cdot 10^0+3\cdot 10^{-1}+1\cdot 10^{-2}.
$$

\vei

En general, todo n'umero real puede escribirse como una expansi'on decimal 
infinita de la siguiente manera
$$
                 b_m\dots b_1b_0.a_1a_2a_3\dots,
$$
donde los $b_i$ y los $a_i$ est'an en $\{0,1,\dots,9\}$. Los puntos suspensivos
de la derecha significan que despu'es del punto decimal podemos tener una 
infinidad de d'igitos, as'i el n'umero $ b_m\dots b_1b_0.a_1a_2a_3\dots$, 
representa al n'umero real 
$$
b_m\cdot 10^m+\cdots+b_1 \cdot 10^1+b_0\cdot 10^0+a_1\cdot 
10^{-1}+a_2\cdot 10^{-2}+\cdots.
$$
Por ejemplo,
$$
\begin{array}{lcccccl}
    \frac{1}{3} & = & 0.3333\dots, & & \frac{3}{7} & = & 
0.428571428571\dots,\\[.2cm]
    \frac{1}{2} & = & 0.50000\dots, & & \sqrt{2} & = & 1.4142135\dots.
\end{array}
$$
\noindent Con esta notaci'on podemos tambi'en distinguir entre los n'umeros racionales y los irracionales. Los n'umeros racionales son aquellos para los cuales la expansi'on decimal es finita o bien infinita pero en alg'un momento se hace peri'odica, como por ejemplo en $\frac{34}{275}=0.123636 \dots$, que se hace peri'odica de periodo 2 a partir del tercer d'igito. En cambio, para los n'umeros irracionales, la expansi'on decimal es infinita, pero no s'olo eso, sino que adem'as nunca se hace peri'odica.

\vei

\noindent Al igual  que en la representaci'on decimal en base 10, podemos representar a los n'umeros enteros en cualquier base. Si $m$ es un n'umero entero positivo, para encontrar su representaci'on en base $b$ lo
escribimos como suma de potencias de $b$, es decir, $m=a_r b^r+\cdots+a_1 b+a_0$. Los n'umeros
enteros que aparecen como coeficientes de las potencias de $b$ en la
representaci'on deben  ser menores que $b$.

\begin{observacion}
Cuando escribimos un n'umero en una base distinta de 10, ponemos como sub'indice
la base en la que est'a escrito el n'umero, por ejemplo, $1204_7$ significa que el
n'umero 1204 es un n'umero en base 7.
\end{observacion}


Veamos el siguiente ejemplo.
\begin{ejemplo}
?`En qu'e base el n'umero $221$ es un factor de $1215$?
\end{ejemplo}

El n'umero 1215 en base $a$ se escribe como $a^3 + 2a^2 + a + 5$ y el
n'umero 221 en base $a$ es $ 2a^2 + 2a + 1$.  Ahora bien, si dividimos 
$a^3 + 2a^2 + a + 5$ entre $ 2a^2 + 2a + 1$ obtenemos que
$$
   a^3 + 2a^2 + a + 5 = (2a^2 + 2a + 1)\left (\frac{1}{2}a+\frac{1}{2}\right)+
                        \left(-\frac{1}{2}a + \frac{9}{2}\right).
$$
Como $1215_a$ tiene que ser un m'ultiplo de $221_a$, el residuo
$\left(-\frac{1}{2}a + \frac{9}{2}\right)$ tiene que ser 0 y
$\left(\frac{1}{2}a + \frac{1}{2}\right )$ tiene que ser un n'umero entero.  Por lo
tanto, $a=9$.


\ve 
\ve



%%%%%%%%%%11

\ejercpreliminares{Escriba en la forma $\frac{m}{n}$, con $n$ y $m$ 
n'umeros enteros positivos, a los siguiente n'umeros reales:

$(i)$  $0.11111\dots$.  

$(ii)$ $1.14141414\dots$.
}

\solcpreliminares{Para resolver $(i)$, defina $x=0.111\dots$, 
entonces $10x=1.11\dots$. Restando la primera ecuaci'on de la 
segunda se tiene que $9x=1$, luego,  $x=\frac{1}{9}$.

\noindent $(ii)$ Sea $x=1.141414\ldots$, entonces 
$100x=114.141414\ldots$. Restando la pri\-mera ecuaci'on de la 
segunda se tiene que $99x=113$, de donde $x=\frac{113}{99}$. }

\ejercpreliminares{$(i)$\; Muestre que $121_b$ es un cuadrado 
perfecto en cualquier base $b\geq 2$.

$(ii)$\; Determine el menor valor de $b$ para el cual  $232_b$ 
es un cuadrado perfecto.
}

\solcpreliminares{$(i)$\; Primero observe que 
$121_b=(1\times b^2)+ (2\times b) +1 =(b+1)^2$ entonces 
$121_b$ es un cuadrado perfecto en cualquier base $b\geq 2$.

$(ii)$\; Como $232_b=2b^2+3b+2$ debe ser cuadrado y como 3 es 
uno de sus d'igitos, $b\geq 4$.

Para $b=4$, $232_4=46$, para $b=5$, $232_5=67$, para $b=6$, 
$232_6=92$ y para $b=7$, $232_7=121$. Luego, $b=7$ es el menor 
entero positivo tal que $232_b$ es un cuadrado perfecto. 
}

%%%%%% 12
%\ejercpreliminares{Sea $b \geq 2$ un entero positivo.
%(a) Muestre que para que un entero $N$, escrito en base $b$, sea igual a la suma del cuadrado %de sus d'igitos, es necesario que  $N = 1$ o que 
%$N$ tenga solamente dos d'igitos.
%(b) Give a complete list of all integers not exceeding 50 that, relative to
%some base $b$, are equal to the sum of the squares of their digits.
%(c) Show that for any base $b$ the number of two-digit integers that are
%equal to the sum of the squares of their digits is even.
%(d) Show that for any odd base $b$ there is an integer other than 1 that is
%equal to the sum of the squares of its digits.
%}

%%%%%%%% 13
\ejercpreliminares{(IMO, 1970) Sean $a$, $b$ y $n$ n'umeros enteros mayores 
que 1. Sean $A_{n-1}$ y $A_n$ dos n'umeros escritos en el sistema num'erico en 
base  $a$ y, $B_{n-1}$ y $B_n$ dos n'umeros escritos en el sistema n'umerico 
en base  $b$. Estos n'umeros 
est'an relacionados de la siguiente forma,
\begin{eqnarray*}
        A_n = x_nx_{n-1} \dots x_0, & & A_{n-1} = x_{n-1}x_{n-2}\dots x_0,\\
        B_n = x_nx_{n-1} \dots x_0, & & B_{n-1} = x_{n-1}x_{n-2}\dots x_0,
\end{eqnarray*}
con 
$x_n\neq 0$ y $x_{n-1}\neq 0$.  Muestre que  $a > b$ si y s'olo si
$$
                  \frac{A_{n-1}}{A_{n}} <\frac{B_{n-1}}{B_{n}}.
$$
}

\solcpreliminares{Suponga que $a>b$. Entonces para todos los 
enteros $0\leq k\leq n$, $x_nx_ka^nb^k\geq x_nx_kb^na^k$, con 
igualdad solamente cuando $k=n$ o $x_k=0$. En particular, se 
tiene una desigualdad estricta para $k=n-1$. En resumen, esto 
se convierte en
$$
              x_n a^n\sum_{k=0}^n x_kb^k > x_nb^n \sum_{k=0}^n x_k a^k
$$
o 
$$
             \frac{ x_n a^n}{A_n}> \frac{x_nb^n}{B_n}.
$$
Esto implica que
$$
             \frac{ A_{n-1}}{A_n}= 1-\frac{ x_n a^n}{A_n} < 
1-\frac{ x_n b^n}{B_n} =\frac{B_{n-1}}{B_n}.
$$ 
Por otro lado, si $a=b$, entonces evidentemente 
$\frac{ A_{n-1}}{A_n}= \frac{ B_{n-1}}{B_n}$ y si 
$a<b$, por lo que se demostr'o antes, se tiene que, 
$ \frac{ A_{n-1}}{A_n}> \frac{ B_{n-1}}{B_n}$. Por lo 
tanto, $\frac{A_{n-1}}{A_{n}} <\frac{B_{n-1}}{B_{n}}$ si y 
s'olo si $a>b$.
}





\section{Valor absoluto}
\label{valorabsoluto}

\noindent Definimos el {\bf  valor absoluto}\index{Valor absoluto} de un n'umero real $x$ como
\begin{equation}
\label{ecuac1.3.1}
     |x|=\left \{\begin{array}{lc}
                 x, & \makebox[1cm]{si} x \geq 0,\\[2mm]
                 -x, & \makebox[1cm]{si} x < 0.
              \end{array}\right.
\end{equation}


\noindent Para $k$ un n'umero real no negativo, la identidad $|x|=k$ s'olo la satisfacen los n'umeros $x=k$ y $x=-k$.

\noindent La desigualdad $|x|\leq k$ es equivalente a
$-k\leq x\leq k$, lo cual podemos ver de la siguiente manera. Si $x\geq 0$, entonces
$0\leq x=|x|\leq k$. Por otro lado, si $x\leq 0$, entonces $-x=|x|\leq k$, de donde $x\geq -k$.  Como consecuencia de lo anterior observemos que $x\leq |x|$.  En la figura siguiente se muestran los valores de $x$ que
satisfacen la desigualdad,  'estos son los que se encuentran entre $-k$ y $k$, incluy'endolos.  Al conjunto $[-k,k]=\{x\in\rr\, |\, -k\leq x\leq k\}$ le llamamos un {\bf intervalo cerrado}, ya que contiene a $k$ y $-k$.\index{Intervalo!cerrado} A $-k$ y $k$ les llamamos los {\bf puntos extremos} del intervalo. \index{Puntos extremos}

\centerline{
 \begin{pspicture}(6,0)(7,4)
 \psset{unit=1cm}
	\psline(1,1)(2,1)
	\psline(4,1)(5,1)
	\psset{linewidth=2.5pt}
	\psline(2,1)(4,1)
	\rput{0}(2,.5){\large $-k$}
	\rput{0}(4,.5){\large $k$}
	\rput{0}(3,.5){ $O$}
	\rput{0}(3,1){\large $|$}
	\rput{0}(2,1){\large [}
	\rput{0}(4,1){\large ]}
 \end{pspicture}
}

\noindent An'alogamente, la desigualdad $|x|\geq k$ es equivalente a
$ x\geq k$ o $ -x\geq k$.
En la figura siguiente los valores de $x$
 que satisfacen
las desigualdades son los que se encuentran antes,  o son iguales,  a $-k$ o despu'es, o son iguales, a $k$. El conjunto $(-k,k)=\{x\in\rr\, |\, -k < x < k\}$ le llamamos un {\bf intervalo abierto}, ya que no contiene a $k$ y $-k$, es decir, un intervalo abierto es aquel que no contiene sus puntos extremos.\index{Intervalo!abierto}  Con esta definici'on vemos que el conjunto de las $x$ que cumplen que $|x|\geq k$, son los valores de $x \notin (-k,k)$. 


\centerline{
 \begin{pspicture}(6,0)(7,4)
 \psset{unit=1cm}
 \psline(2,1)(4,1)
 \psset{linewidth=2.5pt}
 \psline(1,1)(2,1)
 \psline(4,1)(5,1)
 \rput{0}(2,.5){\large $-k$}
 \rput{0}(4,.5){\large $k$}
 \rput{0}(3,.5){ $O$}
 \rput{0}(3,1){$|$}
 \rput{0}(2,1){\large ]}
 \rput{0}(4,1){\large [}
 \end{pspicture}
}

\index{Plano Cartesiano}

\begin{ejemplo}
\noindent  Encontremos, en el plano cartesiano\footnote{El {\bf plano 
cartesiano} se define como $\rr^2=\rr \times \rr=\{(x,y) \,|\,x\in\rr, y \in\rr\}$.}, el 'area
encerrada por la gr'afica de la relaci'on $|x|+|y|=1$.
\end{ejemplo}
\noindent Para $|x|+|y|=1$ tenemos que considerar cuatro casos:
\begin{description}
\item[$(a)$] $x\geq 0$ y $y\geq 0$ lo que implica que $x+y=1$, es decir, $y=1-x$.
\item[$(b)$] $x\geq 0$ y $y < 0$ lo que implica que $x-y=1$, es decir, $y=x-1$.
\item[$(c)$] $x < 0$ y $y \geq 0$ lo que implica que $-x+y=1$, es decir, $y=x+1$.
\item[$(d)$] $x< 0$ y $y < 0$ lo que implica que $-x-y=1$, es decir, $y=-x-1$.
\end{description}

\noindent Podemos ahora dibujar la gr\'afica.

\centerline{
      \psset{unit=1cm}
      \begin{pspicture}(0,0)(4,4.5)
     % \psgrid(4,4)
       \psline(0,2.5)(3,2.5)
       \psline(1.5,1)(1.5,4)	
       \psline(0,2)(2,4)
	\psline(1,4)(3,2)
 	\psline(3,3)(1,1)
	\psline(2,1)(0,3)
	 \rput{0}(-.4,2.7){\scriptsize{$(-1,0)$}}
       \rput{0}(2.1,3.6){\scriptsize{$(0,1)$}}
       \rput{0}(3.3,2.7){\scriptsize{$(1,0)$}}
       \rput{0}(2.2,1.4){\scriptsize{$(0,-1)$}}	
       \end{pspicture}
}

\noindent El 'area encerrada por las cuatro rectas est\'a formada por
cuatro tri\'angulos rect\'angulos is'osceles, que tienen cada uno, dos lados iguales a $1$. Como el \'area de
cada uno de estos tri\'angulos es $\frac{1\times 1}{2}=\frac{1}{2}$,  el \'area del cuadrado es $4\left (\frac{1}{2}\right )=2$.

\begin{ejemplo}
Resolvamos la ecuaci'on $|2x-4|=|x+5|$.
\end{ejemplo}
\noindent Tenemos que
\begin{equation*}
    |2x-4|=\left \{\begin{array}{lr}
    		 2x-4, & \makebox[1cm]{si} x\geq 2,\\[2mm]	
                 -2x+4, &\makebox[1cm]{si}  x< 2.
                \end{array}\right.
\end{equation*}
Adem'as, tenemos que
\begin{equation*}
    |x+5|=\left \{\begin{array}{lr}
                 x+5, &\makebox[1cm]{si}  x\geq -5,\\[2mm]
                 -x-5, & \makebox[1cm]{si} x< -5.
                \end{array}\right.
\end{equation*}
Si $x\geq 2$, entonces $2x-4=x+5$, es decir, $x=9$. Si $x<-5$, entonces
$-2x+4=-x-5$, de donde $x=9$, lo cual es imposible ya que $x<-5$. El 'ultimo caso que nos falta considerar es
 $-5\leq x<2$, entonces la ecuaci'on que tenemos que resolver es $-2x+4=x+5$,
despejando $x$, tenemos que $x=-\frac{1}{3}$. Por lo tanto, los n'umeros que resuelven la ecuaci'on son $x=9$ y $x=-\frac{1}{3}$.

\vei

\noindent Muchas veces es m'as f'acil resolver estas ecuaciones sin
utilizar la forma expl'icita del valor absoluto,
si observamos que $|a|=|b|$ si y s'olo si $a=\pm b$ y utilizamos las propiedades del valor absoluto.

\vei


\begin{observacion}
Si $x$ es un n'umero real cualquiera, entonces la
relaci'on entre la ra'iz cuadrada y el valor absoluto est'a dada por
$\sqrt{x^2}=|x|$,
la identidad se sigue de que  $|x|^2=x^2$ y $|x|\geq 0$.	
\end{observacion}

\index{Valor absoluto!propiedades del}

\begin{propiedades}
Si $x$ y $y$ son n'umeros reales, se cumple lo siguiente:
\begin{description}
\item[$(a)$] $|xy|=|x||y|$. De aqu'i se sigue  tambi'en que $\left |\frac{x}{y}\right |=
\frac{|x|}{|y|}$, si $y\neq 0$.
\item [$(b)$] $|x+y|\leq |x|+|y|$, donde la igualdad se da si y s'olo si $xy\geq 0$.
\end{description}
\label{desigualdadesabsolut}
\end{propiedades}


\demostracion{$(a)$ La demostraci'on es directa de $|xy|^2=(xy)^2=x^2y^2=|x|^2|y|^2$, 
y ahora sacando ra'iz obtenemos el resultado.

$(b)$ Como ambos lados de la desigualdad son n\'umeros positivos,
bastar\'a entonces verificar que
$\left|x+y\right| ^{2}\leq \left( \left| x\right| +\left| y\right|
\right)^{2}$.
\begin{eqnarray*}
\left| x+y\right| ^{2} & = & (x+y)^{2}=x^{2}+2xy+y^{2}=\left| x\right|
^{2}+2xy+\left| y\right| ^{2}\\
& \leq & \left| x\right| ^{2}+2\left| xy\right|
+\left| y\right| ^{2}
 =  \left| x\right| ^{2}+2\left| x\right| \left| y\right| +\left| y\right|
^{2}=\left( \left| x\right| +\left| y\right| \right) ^{2}.
\end{eqnarray*}
\noindent En las relaciones anteriores hay una sola desigualdad y \'esta es
inmediata ya que  $xy\leq \left| xy\right|$. Adem'as, obtenemos la igualdad si y s'olo si $xy=|xy|$ que sucede 'unicamente cuando  $xy\geq 0$.
}

\noindent  La desigualdad $(b)$ en \ref{desigualdadesabsolut} se puede extender en una forma general
como,
$$
|\pm x_1\pm x_2\pm\cdots\pm x_n|\leq |x_1|+|x_2|+\cdots+|x_n|,
$$
para n'umeros reales $x_1$, $x_2$, $\dots$, $x_n$.
La igualdad se tiene cuando todos los $\pm x_i$
tienen el mismo signo.
'Esta se demuestra de manera similar, 
o bien por inducci\'on\footnote{Ver secci'on \ref{cap3sec1}, para ver
demostraciones por inducci\'on.}.

\vei



%%% 13
\ejercpreliminares{Si $a$ y $b$ son n'umeros reales 
cualesquiera, demuestre que
$$
             ||a|-|b||\leq |a-b|.
$$
}

\solcpreliminares{Observe que $|a|=|a-b+b|\leq |a-b|+|b|$, 
despejando se tiene que $|a|-|b|\leq |a-b|.$  An'alogamente, 
siguiendo los mismos pasos, se tiene que $|b| -|a|\leq |b-a|$. 
De estas dos desigualdades se sigue que $||a|-|b||\leq |a-b|$.
}



%%% 14
\ejercpreliminares{En cada caso encuentre los n'umeros reales 
$x$ que satisfacen:

$(i)$\;$|x-1|- |x+1|=0$.

$(ii)$\; $|x-1||x+1|=1$.

$(iii)$\; $|x-1|+ |x+1|=2$.
}

\solcpreliminares{$(i)$\;  $|x-1|- |x+1|=0$ es equivalente a 
$|x-1|=|x+1|$. Elevando al cuadrado y resolviendo la ecuaci'on 
$(x-1)^2 = (x+1)^2$ tenemos que $4x =0$,
luego, la 'unica soluci'on es $x=0$.

$(ii)$\; $|x-1||x+1|=1$  es equivalente a $|x^2-1|=1$, de donde
%$$
%\begin{array}{lcccl}
%          x^2 -1 =1 & &\text{o} & & -(x^2 -1) =1\\
%         x^2  =2 & &\text{o} & & x^2  =0\\
%         x = \pm\sqrt{2} & &\text{o} & & x  =0,
%\end{array}
%$$
las soluciones son $x = \pm\sqrt{2}$  y $ x  =0$.

$(iii)$\;  Si $x>1$ se cumple que  $|x+1|=x+1 >2$, luego no hay soluci'on.

Si $x<-1$ se cumple que  $|x-1|=-x+1 >2$ y tampoco hay soluci'on.

Si $-1\leq x \leq 1$,  entonces $x-1 \leq 0\leq x+1$,   luego 
$$
      |x-1|+|x+1 |=(1-x)+(x+1)=2.
$$
Por lo que los 'unicos valores de  $x$ que cumplen  la igualdad son 
 $-1\leq x\leq 1$.
}

%%% 15
\ejercpreliminares{Encuentre las ternas $(x,y,z)$  de n'umeros 
reales que satisfacen 
\begin{eqnarray*}
              |x+y| &\geq & 1\\
             2xy -z^2 & \geq & 1\\
            z-|x+y|  & \geq & -1.
\end{eqnarray*}
}

\solcpreliminares{De la primera y  tercera desigualdades 
se tiene que \linebreak $z \geq |x+y| -1\geq 0$. Por lo que, $z^2\geq 
(|x+y|-1)^2$. Ahora,   $2xy \geq z^2+1\geq (|x+y|-1)^2 + 1\geq 0$, 
entonces
$$
  2xy \geq  x^2+2xy+y^2-2|x+y|+2 \geq  |x|^2+2xy+|y|^2 - 2|x|- 2 |y| +2,
$$
cancelando $0\geq  |x|^2+|y|^2 - 2|x|- 2 |y|+2 = (|x|-1)^2 +(|y|-1)^2.$
Por lo que $|x|=1$ y $|y|=1$.
Luego,  $x$ y $y$ tienen que ser   $-1$ o 1.  Pero como $xy\geq 0$,  
los dos tienen que tener el mismo signo. Para  $x=y=1$ o $x=y=-1$ se 
tiene, sustituyendo en las ecuaciones originales,   
que $2-z^2\geq 1$ y $z-2\geq -1$. Luego, $z^2\leq 1$ y $z\geq 1$. El 
'unico valor de $z$ que satisface las dos desigualdades es $z=1$. 
Por lo tanto, hay dos soluciones al problema $x=y=z=1$ y $x=y=-1$, $z=1$. 
}


%%%%%16
\ejercpreliminares{(OMM, 2004) ?`Cu\'al es la mayor cantidad de 
n'umeros enteros positivos que se pueden encontrar de manera que 
cualesquiera dos de ellos, $a$ y $b$ (con $a\neq b$), cumplan que:
$$|a-b|\geq \frac{ab}{100}?$$
}

\solcpreliminares{Suponga que
$a_{1}<a_{2}< \dots<a_{n}$ es una colecci\'{o}n con la mayor cantidad de
n'umeros enteros con la propiedad.  Es claro que $a_{i}\geq i$, para 
toda $i=1, \ldots, n$.

\noindent Si $a$ y $b$ son dos n'umeros enteros de la colecci\'{o}n 
con $a>b$, como $%
\left\vert a-b\right\vert =a-b\geq \frac{ab}{100}$, se tiene que 
$a \left(1-\frac{b}{100} \right) \geq b$, por lo que si $100-b>0$, 
entonces $a\geq \frac{100b}{100-b}$.

\noindent Note que no existen dos n'umeros enteros $a$ y $b$ en la 
colecci\'{o}n
mayores que $100$, en efecto si $a>b>100$, entonces $a-b=\left\vert
a-b\right\vert \geq \frac{ab}{100}>a$, lo cual es falso.

\noindent Tambi\'{e}n se tiene que para n'umeros enteros $a$ y $b$ 
menores que $100$,
se cumple que $\frac{100a}{100-a}\geq \frac{100b}{100-b}$ si y s'olo si 
$100a-ab\geq 100b-ab$ si y s'olo si $a\geq b$.

\vei 

\noindent Es claro que $\left\{ 1,2,3,4,5,6,7,8,9,10\right\} $ es una 
colecci\'{o}n con la propiedad.

\noindent Ahora, $a_{11}\geq \frac{100a_{10}}{100-a_{10}}\geq \frac{100\cdot
10}{100-10}=\frac{100}{9}>11$, lo que implica que $a_{11}\geq 12$.

\ve

$a_{12}\geq \frac{100a_{11}}{100-a_{11}}\geq \frac{100\cdot 12}{100-12}=%
\frac{1200}{88}>13$, de donde $a_{12}\geq 14$.

\ve

$a_{13}\geq \frac{100a_{12}}{100-a_{12}}\geq \frac{100\cdot 14}{100-14}=%
\frac{1400}{86}>16$, de donde $a_{13}\geq 17$.

\ve

$a_{14}\geq \frac{100a_{13}}{100-a_{13}}\geq \frac{100\cdot 17}{100-17}=%
\frac{1700}{83}>20$, de donde $a_{14}\geq 21$.

\ve

$a_{15}\geq \frac{100a_{14}}{100-a_{14}}\geq \frac{100\cdot 21}{100-21}=%
\frac{2100}{79}>26$, de donde $a_{15}\geq 27$.

\ve

$a_{16}\geq \frac{100a_{15}}{100-a_{15}}\geq \frac{100\cdot 27}{100-27}=%
\frac{2700}{73}>36$, de donde  $a_{16}\geq 37$.

\ve

$a_{17}\geq \frac{100a_{16}}{100-a_{16}}\geq \frac{100\cdot 37}{100-37}=%
\frac{3700}{63}>58$, de donde  $a_{17}\geq 59$.

\ve

$a_{18}\geq \frac{100a_{17}}{100-a_{17}}\geq \frac{100\cdot 59}{100-59}=%
\frac{5900}{41}>143$, de donde  $a_{18}\geq 144$.

\ve

\noindent Adem'as, como ya se ha observado que no hay dos  
n'umeros enteros de la colecci\'{o}n mayores que $100$, 
la mayor cantidad es $18$.
La colecci\'{o}n de $18$ n'umeros enteros siguiente 
$\left\{ 1,2,3,4,5,6,7,8,9,10,12,14,17,21,27,37,59,144\right\}$ 
cumple la condici\'{o}n.
}












\section{Parte entera y parte fraccionaria}
\label{parteentera}

Dado cualquier n'umero $x \in \rr$, algunas veces es 'util considerar
el n'umero entero m'ax $\{ k \in \zz \ | \ k \leq x\}$, es decir, 
el mayor entero menor o igual que $x$. A este n'umero lo denotamos por 
$\lfloor x \rfloor$ y se le conoce como 
{\bf la parte entera de $x$}.\index{N'umero!parte entera}

De la definici'on anterior tenemos las siguientes propiedades.

\begin{propiedades} Sean $x, y \in \rr$, $n \in \nn$ y $m \in \zz$. 
Entonces se tiene que:
\begin{description}\label{parteentera}
\item[$(a)$] $x-1<\lfloor x \rfloor \leq x < \lfloor x \rfloor +1$.
\item[$(b)$]  $x$ es entero si y s'olo si $\lfloor x \rfloor=x$.
\item[$(c)$] $\lfloor x +m \rfloor= \lfloor x \rfloor +m$.
\item[$(d)$] $\left\lfloor \frac{\lfloor x \rfloor}{n}\right\rfloor =
 \left\lfloor \frac{x}{n}\right\rfloor$.
\item[$(e)$]  $\lfloor x \rfloor + \lfloor y \rfloor \leq
\lfloor x +y \rfloor \leq \lfloor x \rfloor + \lfloor y \rfloor +1$.
\end{description}
\end{propiedades}

\demostracion{Las primeras tres propiedades son inmediatas.  

$(d)$ Al dividir $\lfloor x\rfloor$ entre $n$ tenemos que $\lfloor x\rfloor= 
a n +b$, para un n'umero entero $a$ y para un n'umero entero $b$ tal que 
$0\leq b<n$.

Por un lado, tenemos que 
 $\left\lfloor \frac{\lfloor x \rfloor}{n}\right\rfloor =
 \left\lfloor \frac{an+b}{n}\right\rfloor= a+\left\lfloor 
\frac{b}{n}\right\rfloor= a$. Por otro lado, como $x=\lfloor x\rfloor+c$, 
con $0 \leq c < 1$, tenemos que
$\left\lfloor \frac{ x}{n}\right\rfloor =
 \left\lfloor \frac{an+b+c}{n}\right\rfloor= 
a+\left\lfloor \frac{b+c}{n}\right\rfloor= a$, ya que $b+c <n-1+1=n$.
Luego, la igualdad es v'alida. 

$(e)$ Como  $x=\lfloor x\rfloor+a$  y $y=\lfloor y\rfloor+b$ con $0\leq a, b < 1$, entonces
 $\lfloor x +y \rfloor =\lfloor x\rfloor + \lfloor  y\rfloor+\lfloor a+b \rfloor$ por la propiedad $(c)$. Las desigualdades se siguen de observar que si $0\leq a, b <1$ entonces $0\leq \lfloor a+b\rfloor \leq 1$.
}


\begin{ejemplo}
Para todo n'umero real $x$ se cumple que 
$$\lfloor x \rfloor + \left\lfloor x + \frac{1}{2}\right\rfloor
- \lfloor 2x \rfloor=0.$$
\label{exismasunmediomenosdosexis}
\end{ejemplo}

Si se hace $n=\lfloor x \rfloor$, $x$ se puede expresar de la forma 
$x=n+a$ con $0 \leq a < 1$, luego se tiene que
\begin{eqnarray*}  
\lfloor x \rfloor + \left\lfloor x + \frac{1}{2}\right\rfloor
- \lfloor 2x \rfloor & = & n + \left\lfloor n+a + \frac{1}{2}
\right\rfloor - \lfloor 2(n+a) \rfloor\\
& = & n + n + 
\left\lfloor a + \frac{1}{2}\right\rfloor -2n - \left\lfloor 2a \right\rfloor\\
& = & \left\lfloor a + \frac{1}{2}\right\rfloor - \left\lfloor 2a \right\rfloor,
\end{eqnarray*}
donde la segunda igualdad se sigue por la propiedad $(c)$. Ahora, si 
$0 \leq a < \frac{1}{2}$, entonces $\left\lfloor a + \frac{1}{2}\right\rfloor=
\lfloor 2a \rfloor=0$, mientras que en el caso $\frac{1}{2} \leq a < 1$, se 
tiene que $\left\lfloor a + \frac{1}{2}\right\rfloor=
\lfloor 2a \rfloor=1$.


\begin{ejemplo}
Si $n$ y $m$ son enteros positivos sin factores
comunes, entonces
$$
\left\lfloor \dfrac{n}{m}\right\rfloor +\left \lfloor \dfrac{2n}{m}\right 
\rfloor + \left\lfloor\dfrac{3n}{m}\right \rfloor +\cdots +
\left\lfloor\dfrac{(m-1)n}{m}\right \rfloor =\dfrac{(m-1)(n-1)}{2}.
$$
\label{sumadepartesenteras}
\end{ejemplo}

Consideramos en el plano cartesiano la recta que pasa por el origen y el punto
$(m,n)$. Como $m$ y $n$ son primos relativos, sobre el segmento de recta que 
une los puntos $(0,0)$ y $(m,n)$ no hay otro punto de coordenadas enteras. 

\centerline{
	\psset{unit=.7cm}
	\begin{pspicture}(0,-1)(6,6.5)
	\psaxes[labels=none]{->}(8.5,5.5)
	\psline(1,1)(7,1)(7,4)(1,4)(1,1)
	\psline(0,0)(8,5)
	\psdot(8,5)
         \pscircle[fillstyle=solid,fillcolor=black](8,0){.1}
         \pscircle[fillstyle=solid,fillcolor=black](0,5){.1}
	\rput{0}(1,-.5){$1$}
	\rput{0}(-.5,1){$1$}
	\rput{0}(2,-.5){$2$}
         \rput{0}(-.5,2){$\vdots$}
	\rput{0}(3.5,-.5){$\cdots$}
	\rput{0}(7,-.5){$m-1$}
	\rput{0}(-1,4){$n-1$}
        \rput{0}(8.6,.5){$A=(m,0)$}
	\rput{0}(8,5.5){$B=(m,n)$}
	\rput{0}(-1.7,5){$C=(0,n)$}
	\rput{0}(-.3,-.3){$O$}
        \end{pspicture}
}

La ecuaci'on de la recta es $y=\frac{n}{m} x$ y pasa por los puntos 
$(j, \frac{n}{m} j)$, con $j=1,\ldots, (m-1)$, y adem'as $\frac{n}{m} j$ 
no es entero. 
El n'umero $\left \lfloor \frac{n}{m} j\right \rfloor$ es igual al n'umero de 
puntos de coordenadas enteras que est'an sobre la recta $x=j$ y, entre las 
rectas $y=\frac{n}{m} x$ y $y=1$ inclusive. La suma es igual al n'umero 
de puntos de coordenadas enteras  en el interior del tri'angulo $OAB$, 
por simetr'ia es 
igual a la mitad de los puntos de coordenadas enteras dentro del rect'angulo 
$OABC$. Como 
la cantidad de  puntos de coordenadas enteras dentro del rect'angulo 
es $(n-1)(m-1)$, tenemos que
$$
\left\lfloor \dfrac{n}{m}\right\rfloor +\left \lfloor \dfrac{2n}{m}\right 
\rfloor + \left\lfloor\dfrac{3n}{m}\right \rfloor +\cdots +
\left\lfloor\dfrac{(m-1)n}{m}\right \rfloor =\dfrac{(m-1)(n-1)}{2}.
$$

\vei

\begin{observacion}
Como el lado derecho de la 'ultima igualdad es sim'etrico en $m$ y $n$, 
entonces
$$\left\lfloor \dfrac{n}{m}\right\rfloor +\left \lfloor \dfrac{2n}{m}\right 
\rfloor + \cdots +
\left\lfloor\dfrac{(m-1)n}{m}\right \rfloor =
\left\lfloor \dfrac{m}{n}\right\rfloor +\left \lfloor \dfrac{2m}{n}\right 
\rfloor + \cdots +
\left\lfloor\dfrac{(n-1)m}{n}\right \rfloor.$$
\end{observacion}

\ve

Para un n'umero real $x$, consideremos tambi'en el n'umero
$\{x\}=x - \lfloor x \rfloor$, al cual  llamamos la 
{\bf parte fraccionaria de $x$}\index{N'umero!parte fraccionaria},
y  cumple las siguientes propiedades. 

\begin{propiedades} Sean $x, y \in \rr$ y $n \in \zz$. Entonces se tiene que:
\begin{description}
\item[$(a)$] $0 \leq \{x\} <1$.

\item[$(b)$] $x =  \lfloor x \rfloor  + \{x\}$.

\item[$(c)$] $\{x+ y\} \leq \{x\} + \{y\} \leq  \{x +y\} +1$.

\item[$(d)$] $\{ x +n \}= \{x\}$.

\end{description}
\end{propiedades}


\ve

%%% 18
\ejercpreliminares{Para cualesquiera n'umeros reales $a$, $b>0$, 
se tiene que
$$
 \lfloor 2a \rfloor + \lfloor 2b \rfloor \geq \lfloor a \rfloor 
+\lfloor b \rfloor +\lfloor a+b\rfloor.
$$
}

\solcpreliminares{Por el ejemplo \ref{exismasunmediomenosdosexis}, 
$\lfloor 2a\rfloor = \lfloor a\rfloor + \lfloor a+\frac{1}{2}\rfloor$ 
y $\lfloor 2b\rfloor = \lfloor b\rfloor + \lfloor b+\frac{1}{2}\rfloor$, 
luego la desigualdad a demostrar es equivalente a
$$
  \lfloor a\rfloor + \left\lfloor a+\frac{1}{2}\right\rfloor+ 
\lfloor b\rfloor + \left\lfloor b+\frac{1}{2}\right\rfloor \geq 
\lfloor a\rfloor + \lfloor b\rfloor+ \lfloor a+b\rfloor,
$$
de donde bastar'a mostrar que $ \left\lfloor a+\frac{1}{2}\right\rfloor+
\left\lfloor b+\frac{1}{2}\right\rfloor \geq  \lfloor a+b\rfloor$.

Sean $a = n+y$, $b = m+x$, con $n, m\in \zz$ y $0\leq x, y < 1$. Entonces 
$0\leq x+ y <2$ y $a+b=n+m+x+y$.  Se tienen dos casos:

$(i)$ Si $1\leq x + y < 2$, entonces  $\lfloor a+ b\rfloor= n+m+1$ 
y al menos uno de los n'umeros $x$ o $y$ es mayor o igual que 
$\frac{1}{2}$.  Suponga que $x\geq \frac{1}{2}$. Entonces  
$\lfloor b+\frac{1}{2}\rfloor= \lfloor m+x+\frac{1}{2}\rfloor = m+1$, 
por lo que $ \lfloor a+\frac{1}{2}\rfloor+\lfloor b+
\frac{1}{2}\rfloor\geq m+n+1=\lfloor a+b\rfloor$.  


$(ii)$ Si $0\leq x + y < 1$, entonces  $\lfloor a+ b\rfloor= n+m$ y  
$ \lfloor a+\frac{1}{2}\rfloor+\lfloor b+\frac{1}{2}\rfloor\geq m+n 
=\lfloor a+b\rfloor$.
}


%%%%%%% 19
\ejercpreliminares{Encuentre  los valores de $x$ que cumplen 
la siguiente ecuaci'on:

$(i)$\,   $\lfloor x \lfloor x \rfloor  \rfloor =1$.

$(ii)$\,   $||x|-\lfloor x \rfloor| = \lfloor |x|- \lfloor x\rfloor\rfloor$.
}

\solcpreliminares{$(i)$\,   Se tiene que $\lfloor x \lfloor x \rfloor  
\rfloor =1$ si y s'olo si
$1\leq x\lfloor x \rfloor < 2$.  Si $x=m+y$, con $m\in\zz$ y $0\leq y <1$, 
entonces $1\leq m^2+my < 2$. Observe que   $m=0$ es imposible, al igual que 
$m\geq 2$ o $m\leq -2$.  Luego, resta ver qu'e sucede si $m=1$ o $m=-1$. 

Si $m=1$, entonces $1\leq 1+y<2$, de donde $0\leq y <1$ y entonces cualquier 
$x$ en el intervalo $[1,2)$ cumple la ecuaci'on.  Si $m=-1$, entonces, como
\linebreak 
$1\leq m^2+my < 2$, se  tiene que  $1\leq 1- y<2$, de donde $0\leq - y < 1$ 
y entonces  $y=0$ y $x=-1$.  
Por lo tanto, los n'umeros que cumplen la ecuaci'on son $x=-1$ y  
$x\in [1,2)$.

\vei 

$(ii)$\,   Como $\lfloor x \rfloor \leq x \leq |x|$, se tiene que,
$ |x|- \lfloor x \rfloor \geq 0$, por lo que\linebreak  $||x|-\lfloor x\rfloor |= 
|x| -\lfloor x\rfloor$.  Por otro lado,  por la propiedad $(c)$ en 
\ref{parteentera} se\linebreak  tiene que, $\lfloor |x|-\lfloor x\rfloor\rfloor 
= \lfloor |x|\rfloor -\lfloor x\rfloor$. Utilizando las 'ultimas igualdades 
la ecuaci'on se convierte en $|x| - \lfloor x\rfloor = \lfloor |x|\rfloor - 
\lfloor x\rfloor$ que es equivalente a $|x| = \lfloor |x| \rfloor$, luego 
$|x|$ es un n'umero entero y los valores de $x$ que cumplen la ecuaci'on 
son todos los n'umeros enteros.
}



%%% 20
\ejercpreliminares{Encuentre las soluciones del sistema de ecuaciones
\begin{eqnarray*}
x+\lfloor y \rfloor + \{z\} & = & 1.1,\\
\lfloor x \rfloor + \{y\}+z & = & 2.2,\\
\{x\} + y + \lfloor z \rfloor & = & 3.3.
\end{eqnarray*}
}

\solcpreliminares{Sume las tres ecuaciones para obtener que 
$2x+2y+2z=6.6$, luego $x+y+z=3.3$. Reste a esta 'ultima igualdad las ecuaciones
originales, para obtener $\{y\} + \lfloor z \rfloor =2.2$, 
$\{x\} + \lfloor y \rfloor =1.1$,
$\{z\}+\lfloor x \rfloor =0$.
La primera ecuaci'on da $\lfloor z \rfloor = 2$, $\{y\}=0.2$, la segunda 
$\lfloor y \rfloor=1$, $\{x\}=0.1$ y, la tercera $\lfloor x \rfloor=0$ y
$\{z\}=0$. Por lo tanto, la soluci'on es $x=0.1$, $y=1.2$ y $z=2$.
}


%%%%%%% 21
\ejercpreliminares{(Canad'a, 1987) Para cada n'umero natural $n$, 
muestre que
$$
         \lfloor \sqrt{n} +\sqrt{n+1} \rfloor = \lfloor \sqrt{4n+1} \rfloor=\lfloor \sqrt{4n+2} \rfloor=\lfloor \sqrt{4n+3} \rfloor. 
$$
}

\solcpreliminares{Se tiene que  $\sqrt{n} +\sqrt{n+1} < \sqrt{4n+2} $ si y 
s'olo si $2n+1+\sqrt{4n^2+4n} < 4n+2$, que es equivalente a $\sqrt{4n^2+4n} < 
2n+1$. Elevando al cuadrado nuevamente, la 'ultima desigualdad es equivalente 
a $4n^2+4n < 4n^2+4n+1$.  Esto prueba que $\sqrt{n} +\sqrt{n+1} < \sqrt{4n+2}$,
entonces $\lfloor \sqrt{n} +\sqrt{n+1}\rfloor 
\leq \lfloor \sqrt{4n+2}\rfloor$.  

\vei

Suponga que, para alg'un n'umero entero positivo $n$, 
$\lfloor \sqrt{n} +\sqrt{n+1}\rfloor \neq \lfloor \sqrt{4n+2}\rfloor$. Sea 
$q= \lfloor \sqrt{4n+2}\rfloor$, entonces $\sqrt{n} +\sqrt{n+1}< q \leq 
\sqrt{4n+2}$.  Elevando al cuadrado, se  obtiene que $2n+1+ \sqrt{4n^2+4n} < 
q^2\leq 4n+2$ o lo que es equivalente $\sqrt{4n^2+4n} < q^2-2n - 1 \leq 2n+1$.
Elevando al cuadrado nuevamente se\linebreak obtiene que  
$4n^2+4n < (q^2-2n - 1)^2 
\leq 4n^2+4n+1= (2n+1)^2$. Como no \linebreak existe un cuadrado entre dos 
enteros 
consecutivos, se tiene que $q^2-2n - 1 = 2n+1$ o que
$q^2 = 4n+2$, que es equivalente a decir que $q^2\equiv 2 \mod 4$.  Pero esto 
'ultimo es una contradicci'on, ya que todo cuadrado es congruente a 0 o a 1 
m'odulo $4$. Por lo tanto,  se tiene la igualdad.

\vei

Muestre ahora que, $\lfloor \sqrt{4n+1} \rfloor=\lfloor \sqrt{4n+2} 
\rfloor=\lfloor \sqrt{4n+3} \rfloor$. 

Para la  primera igualdad, suponga que 
existe una $n$ tal que $m=\lfloor \sqrt{4n+1} \rfloor < m+1=
\lfloor \sqrt{4n+2} \rfloor$, luego $m \leq \sqrt{4n+1} < m+1 \leq 
\sqrt{4n+2}$, por lo que 
$m^2 \leq 4n+1 < (m+1)^2\leq 4n+2$. 

Entonces, como $4n+1$ y 
$4n+2$ son dos n'umeros enteros consecutivos y, como $(m+1)^2 > 4n+1$, se 
tiene que $(m+1)^2 = 4n+2$ y nuevamente se ha encontrado un cuadrado que 
tiene residuo $2$ al dividirlo entre $4$, lo cual es imposible. Para la 
segunda igualdad, proceda de la misma forma. 
}






\section{Productos notables}
\label{cap1sec2}

\index{Producto notable!dos variables}

\noindent El 'area de un cuadrado es el cuadrado de la  longitud de su
lado.  Si sus lados miden $a+b$ entonces el 'area es $(a+b)^2$, pero el 'area de este cuadrado la podemos dividir en cuatro rect'angulos como se muestra en la figura. 

\index{Binomio!cuadrado}

\centerline{
       \psset{unit=1cm}
	\begin{pspicture}(0,0)(6,4.5)
	\psline(1,1)(1,4)(4,4)(4,1)(1,1)
	\psline(1,2)(4,2)
	\psline(3,1)(3,4)
	\psline(1,.9)(1.1,.8)(1.9,.8)(2,.7)(2.1,.8)(2.9,.8)(3,.9)
	\psline(3,.9)(3.1,.8)(3.4,.8)(3.5,.7)(3.6,.8)(3.9,.8)(4,.9)
	\psline(.9,1)(.8,1.1)(.8,1.4)(.7,1.5)(.8,1.6)(.8,1.9)(.9,2)
	\psline(.9,2)(.8,2.1)(.8,2.9)(.7,3)(.8,3.1)(.8,3.9)(.9,4)
	\rput{0}(2,.4){\large $a$}
	\rput{0}(3.5,.4){\large $b$}
	\rput{0}(2,1.5){\large $ab$}
	\rput{0}(3.5,3){\large $ab$}
	\rput{0}(2,3){\large $a^2$}
	\rput{0}(3.5,1.5){\large $b^2$}
	\rput{0}(.4,1.5){\large $b$}
	\rput{0}(.4,3){\large $a$}
	\end{pspicture}
}

Luego, la suma de las 'areas de  los cuatro rect'angulos ser'a igual al 'area del cuadrado, es decir, 
\begin{equation}
\label{ecuac1.1.1}
                         (a+b)^2=a^2+ab+ab+b^2=a^2+2ab+b^2.
\end{equation}

\noindent Veamos ahora c'omo obtener geom'etricamente el cuadrado de la diferencia $a-b$, donde $b\leq a$.   El problema es ahora
encontrar el 'area de un cuadrado de lado $a-b$.  

\centerline{
        \psset{unit=1cm}
	\begin{pspicture}(0,0)(6,4.5)
	\psline(1,1)(1,4)(4,4)(4,1)(1,1)
\psline[fillstyle=solid,fillcolor=lightgray,linewidth=1pt](1,1)(1,2)(3,2)(3,4)(4,4)(4,1)(1,1)
	\psline[linewidth=1pt](1,2)(4,2)
	\psline[linewidth=1pt](3,1)(3,4)
	\psline(4.1,1)(4.2,1.1)(4.2,1.4)(4.3,1.5)(4.2,1.6)(4.2,1.9)(4.1,2)
	\psline(.9,1)(.8,1.1)(.8,2.4)(.7,2.5)(.8,2.6)(.8,3.9)(.9,4)
	\rput{0}(.4,2.5){\large $a$}
	\rput{0}(4.5,1.5){\large $b$}
	\rput{0}(2,1.5){$(a-b)b$}
	\rput{90}(3.5,3){$(a-b)b$}
	\rput{0}(2,3){\large $(a-b)^2$}
	\rput{0}(3.5,1.5){\large $b^2$}
	\end{pspicture}
}

\noindent En la figura observamos que
el 'area de un cuadrado de lado $a$ es igual a la suma de las 'areas de los cuadrados de lados $(a-b)$ y $b$, m'as el 'area de dos rect'angulos iguales de lados $b$ y $(a-b)$. Esto es, $a^2=(a-b)^2+b^2+(a-b)b+b(a-b)$, de donde 				      
\begin{equation}
\label{ecuac1.1.2}
(a-b)^2=a^2-2ab+b^2.
\end{equation}

\noindent Para encontrar el 'area de la parte sombreada de la siguiente figura, 

\centerline{
      \psset{unit=1cm}
	\begin{pspicture}(0,0)(6,4.5)
	\psline(1,1)(1,4)(4,4)(4,1)(1,1)
  \psline[fillstyle=solid,fillcolor=lightgray,linewidth=1pt](1,1)(3,1)(3,2)(4,2)
   (4,4)(1,4)(1,1)
	\psline[linewidth=1pt](1,2)(4,2)
	\psline[linewidth=1pt](3,1)(3,4)
	\psline(4.1,1)(4.2,1.1)(4.2,1.4)(4.3,1.5)(4.2,1.6)(4.2,1.9)(4.1,2)
	\psline(4.1,2)(4.2,2.1)(4.2,2.9)(4.3,3)(4.2,3.1)(4.2,3.9)(4.1,4)
	\psline(.9,1)(.8,1.1)(.8,2.4)(.7,2.5)(.8,2.6)(.8,3.9)(.9,4)
	\psline(1,.9)(1.1,.8)(1.9,.8)(2,.7)(2.1,.8)(2.9,.8)(3,.9)
	\psline(3,.9)(3.1,.8)(3.4,.8)(3.5,.7)(3.6,.8)(3.9,.8)(4,.9)
	\rput{0}(.4,2.5){\large $a$}
	\rput{0}(4.5,1.5){\large $b$}
	\rput{0}(5,3){\large $a - b$}
	\rput{0}(2,.4){\large $a - b$}
	\rput{0}(3.5,.4){\large $b$}
	\rput{0}(2,1.5){ $(a-b)b$}
	\rput{90}(3.5,3){$(a-b)b$}
	\rput{0}(2,3){\large $(a-b)^2$}
	\rput{0}(3.5,1.5){\large $b^2$}
	\end{pspicture}
}
\noindent observamos que la suma de las 'areas de los rect'angulos 
que la forman es \linebreak $a(a-b)+b(a-b)$ y si 
factorizamos esta  suma tenemos que
\begin{equation}
\label{ecuac1.1.3}
 a(a-b)+b(a-b)=(a+b)(a-b),
\end{equation}
pero es equivalente al 'area del cuadrado grande menos el 'area del cuadrado
chico, es decir, 
\begin{equation}
\label{ecuac1.1.4}
(a+b)(a-b)= a^2-b^2.
\end{equation}

\vei

\index{Producto notable!tres variables}
 Otro producto notable, pero ahora  de tres variables, est'a dado por 
\begin{equation}
    (a+b+c)^2 =  a^2 +b^2+c^2+2ab+2ac+2bc.
\label{productonotabledetres}
\end{equation}
La representaci'on geom'etrica de este producto est'a dada por la igualdad entre el  'area del cuadrado con lados de longitud $a+b+c$ y la suma de las 'areas de los nueve rect'angulos en que se ha dividido el cuadrado, esto es,	      
$$
         (a+b+c)^2 =  a^2 +b^2+c^2+ab+ac+ba+bc+ca+cb= a^2 +b^2+c^2+2ab+2ac+2bc.
$$


\centerline{
      \psset{unit=.7cm}
      \begin{pspicture}(0,-.7)(6,6.5)
	\psline(0,1)(5,1)(5,6)(0,6)(0,1)
	\psline(1,1)(1,6)
	\psline(3.3,1)(3.3,6)
	\psline(0,5)(5,5)
	\psline(0,2.7)(5,2.7)
        \psline(5.2,1)(5.3,1.1)(5.3,1.75)(5.4,1.85)(5.3,1.95)(5.3,2.6)(5.2,2.7)
	\psline(5.2,2.7)(5.3,2.8)(5.3,3.75)(5.4,3.85)(5.3,3.95)(5.3,4.9)(5.2,5)
	\psline(5.2,5)(5.3,5.1)(5.3,5.4)(5.4,5.5)(5.3,5.6)(5.3,5.9)(5.2,6)
	\psline(0,.9)(.1,.8)(.4,.8)(.5,.7)(.6,.8)(.9,.8)(1,.9)
	\psline(1,.9)(1.1,.8)(2.05,.8)(2.15,.7)(2.25,.8)(3.2,.8)(3.3,.9)
	\psline(3.3,.9)(3.4,.8)(4.05,.8)(4.15,.7)(4.25,.8)(4.9,.8)(5,.9)
	\rput{0}(.5,.3){$a$}
	\rput{0}(2.15,.3){$b$}
	\rput{0}(4.15,.3){$c$}	
        \rput{0}(5.8,1.85){$c$}
	\rput{0}(5.8,3.85){$b$}
	\rput{0}(5.8,5.5){$a$}
        \rput{0}(.5,1.85){$ac$}		
	\rput{0}(2.15,1.85){$bc$}		
	\rput{0}(4.15,1.85){$c^2$}	
         \rput{0}(.5,3.85){$ab$}		
	\rput{0}(2.15,3.85){$b^2$}		
	\rput{0}(4.15,3.85){$cb$}		
         \rput{0}(.5,5.5){$a^2$}		
	\rput{0}(2.15,5.5){$ba$}		
	\rput{0}(4.15,5.5){$ca$}		
	      \end{pspicture}    
}

\noindent A continuaci'on damos una serie de identidades, algunas de ellas muy conocidas y otras no tanto, 'utiles para resolver varios problemas.

\ve

%%%%  22
\ejercpreliminares{Para todos los n'umeros reales $x$, $y$, 
se tienen las siguientes identidades de segundo grado:

\noindent $(i)$ $x^{2}+y^{2}=(x+y)^{2}-2xy=(x-y)^{2}+2xy$.

\noindent $(ii)$ $(x+y)^{2}+(x-y)^{2}=2(x^{2}+y^{2})$.

\noindent $(iii)$ $(x+y)^{2}-(x-y)^{2}=4xy$.

\noindent $(iv)$ $x^{2}+y^{2}+xy=\dfrac{x^{2}+y^{2}+(x+y)^{2}}{2}$.

\noindent $(v)$ $x^{2}+y^{2}-xy=\dfrac{x^{2}+y^{2}+(x-y)^{2}}{2}$.

\noindent $(vi)$ Muestre que  $x^{2}+y^{2}+xy\geq 0$ y 
$x^{2}+y^{2}-xy\geq 0$.
}

\solcpreliminares{Para los primeros cinco incisos utilice las ecuaciones 
(\ref{ecuac1.1.1}), (\ref{ecuac1.1.2}) y (\ref{ecuac1.1.4}).  Para el inciso $(vi)$ utilice $(iv)$ y $(v)$.}

%%%%%%%% 23
\ejercpreliminares{Para todos los n'umeros reales $x$, $y$, $z$, se tiene: 

\noindent $(i)$ $x^{2}+y^{2}+z^{2}+xy+yz+zx=\dfrac{(x+y)^{2}+(y+z)^{2}+(z+x)^{2}}{2}$.

\noindent $(ii)$ $x^{2}+y^{2}+z^{2}-xy-yz-zx=\dfrac{(x-y)^{2}+(y-z)^{2}+(z-x)^{2}}{2}.$
\label{equisyeyzetaalcuadrado}

\noindent $(iii)$ Muestre que
$x^{2}+y^{2}+z^{2}+xy+yz+zx\geq 0$
y $x^{2}+y^{2}+x^{2}-xy-yz-zx\geq 0.$
}

\solcpreliminares{Para los incisos $(i)$ y $(ii)$ utilice las ecuaciones (\ref{ecuac1.1.1}) y (\ref{ecuac1.1.2}). Para demostrar $(iii)$ use $(i)$ y $(ii)$.

}

% %%%  24
\ejercpreliminares{Para todos los n'umeros reales $x$, $y$, $z$ se tienen las siguientes identidades:

\noindent $(i)$ $(xy+yz+zx)(x+y+z)=(x^{2}y+y^{2}z+z^{2}x)+(xy^{2}+yz^{2}+zx^{2})+3xyz$. 

\noindent $(ii)$
$(x+y)(y+z)(z+x)=(x^{2}y+y^{2}z+z^{2}x)+(xy^{2}+yz^{2}+zx^{2})+2xyz$. 

\noindent $(iii)$
$(xy+yz+zx)(x+y+z)=(x+y)(y+z)(z+x)+xyz$.
\label{ejerciciotresdocetres} 


\noindent $(iv)$
$(x-y)(y-z)(z-x)=(xy^{2}+yz^{2}+zx^{2})-(x^{2}y+y^{2}z+z^{2}x)$. 


\noindent $(v)$
$(x+y)(y+z)(z+x)-8xyz=2z(x-y)^{2}+(x+y)(x-z)(y-z)$. 

\noindent $(vi)$
$xy^{2}+yz^{2}+zx^{2}-3xyz=z(x-y)^{2}+y(x-z)(y-z)$. 
}

\solcpreliminares{Para demostrar los incisos $(i)$ y $(ii)$ realice las operaciones del lado izquierdo de la ecuaci'on y reacomode. 

\noindent Para demostrar los incisos $(iii)$, $(iv)$, $(v)$ y $(vi)$ realice las operaciones de ambos lados de la ecuaci'on y vea que son iguales.
}

%%%%%%%% 25
\ejercpreliminares{Para todos los n'umeros reales $x$, $y$, $z$ se tiene:

\noindent $(i)$\; $x^{2}+y^{2}+z^{2}+3(xy+yz+zx) =(x+y)(y+z)+(y+z)(z+x)+(z+x)(x+y)$.

\noindent $(ii)$\; $ xy+yz+zx-\left(x^{2}+y^{2}+z^{2}\right) =(x-y)(y-z)+(y-z)(z-x)$\\
$ \ \ +(z-x)(x-y).$
}

\solcpreliminares{Para demostrar los incisos $(i)$ y $(ii)$ realice las operaciones del lado derecho de las ecuaciones y simplifique.
}

%%%% 22
\ejercpreliminares{Para todos 
los n'umeros reales $x$, $y$, $z$ se tiene,
\begin{eqnarray*}
  (x-y)^{2}+(y-z)^{2}+(z-x)^{2} & = & 2\left[ (x-y)(x-z)\right. \\
                                       &  &\left . +(y-z)(y-x)+(z-x)(z-y)\right].
\end{eqnarray*}

}

\solcpreliminares{Utilice las ecuaciones (\ref{ecuac1.1.1}) y (\ref{ecuac1.1.2}), haga las operaciones de ambos lados de la ecuaci'on.
}




 


\section{Matrices y determinantes}
\label{cap1sec1}

\noindent Una {\bf matriz de $2\times 2$} es un arreglo\index{Matriz!$2\times 2$} 
$$
     \left (\begin{array}{cc}
                 a_{11} &  a_{12}\\[1mm]
                 a_{21} &  a_{22}\\[1mm]
              \end{array}\right )
$$
donde $a_{11}$, $a_{12}$, $a_{21}$ y $a_{22}$ son n'umeros reales o 
complejos\footnote{Los n'umeros complejos los trataremos en el cap'itulo   \ref{numeroscomplejos}.}. 
\noindent El {\bf determinante} de la matriz anterior, \index{Determinante!matriz $2\times 2$} que denotamos por
$$
     \left |\begin{array}{cc}
                 a_{11} &  a_{12}\\[1mm]
                 a_{21} &  a_{22}\\[1mm]
              \end{array}\right |
$$
es el n'umero real definido por 
$a_{11}a_{22}- a_{12}a_{21}.$

\vei

\noindent Una {\bf matriz de $3\times 3$} es un arreglo\index{Matriz!$3\times 3$} 
$$
     \left (\begin{array}{ccc}
                 a_{11} &  a_{12} &a_{13}\\[1mm]
                 a_{21} &  a_{22}&a_{23}\\[1mm]
		 a_{31} &  a_{32}&a_{33}
              \end{array}\right )
$$
donde, nuevamente, cada $a_{ij}$ es un n'umero. Los sub'indices nos indican 
la po\-sici'on del n'umero en el arreglo.  As'i, $a_{ij}$ se encuentra en el 
$i$-'esimo
rengl'on y la $j$-'esima columna.  Definimos el {\bf determinante de una matriz de 
$3\times 3$} por la regla\index{Determinante!matriz $3\times 3$}
$$
     \left |\begin{array}{ccc}
                 a_{11} &  a_{12} &a_{13}\\[1mm]
                 a_{21} &  a_{22}&a_{23}\\[1mm]
		 a_{31} &  a_{32}&a_{33}
              \end{array}\right |=
      a_{11} \left |\begin{array}{cc}
                  a_{22} &a_{23}\\[1mm]
                 a_{32} &  a_{33}
              \end{array}\right |-
      a_{12} \left |\begin{array}{cc}
                  a_{21} &a_{23}\\[1mm]
                 a_{31} &  a_{33}
              \end{array}\right | +
       a_{13} \left |\begin{array}{cc}
                  a_{21} &a_{22}\\[1mm]
                 a_{31} &  a_{32}
              \end{array}\right |. 	      	        	      	      
$$
Es decir, nos movemos a lo largo del primer rengl'on, multiplicando $a_{1j}$ por
el determinante de la matriz de $2\times 2$ obtenida al eliminar el primer
rengl'on y la $j$-'esima columna, y despu'es sumando todo esto, pero recordando
poner un signo negativo antes de $a_{12}$. Cabe aclarar que el resultado del
determinante no se altera si en lugar de escoger el primer rengl'on como primer
paso escogemos el segundo o el tercero. En caso de que escojamos el segundo
rengl'on iniciamos con un signo negativo y si escogemos el tercer rengl'on el
primer signo es positivo, es decir,
$$
     \left |\begin{array}{ccc}
                 a_{11} &  a_{12} &a_{13}\\[1mm]
                 a_{21} &  a_{22}&a_{23}\\[1mm]
		 a_{31} &  a_{32}&a_{33}
              \end{array}\right |=
      -a_{21} \left |\begin{array}{cc}
                  a_{12} &a_{13}\\[1mm]
                 a_{32} &  a_{33}
              \end{array}\right |+
      a_{22} \left |\begin{array}{cc}
                  a_{11} &a_{13}\\[1mm]
                 a_{31} &  a_{33}
              \end{array}\right | -
       a_{23} \left |\begin{array}{cc}
                  a_{11} &a_{12}\\[1mm]
                 a_{31} &  a_{32}
              \end{array}\right |. 	      	        	      	      
$$
Los signos se van alternando, siguiendo el siguiente diagrama
$$
     \left |\begin{array}{ccc}
                 + &  - & +\\[1mm]
                 - &  + & -\\[1mm]
		 + & -  &+
              \end{array}\right |. 	      	        	      	      
$$

\noindent Los determinantes cumplen varias propiedades, que son inmediatas de las definiciones, las m'as 'utiles son las siguientes.
\index{Determinante!propiedades del}
\begin{propiedades}
$(a)$\; Al intercambiar dos renglones consecutivos o dos columnas consecutivas, el signo del determinante cambia, por ejemplo,
$$
     \left |\begin{array}{ccc}
                 a_{11} &  a_{12} &a_{13}\\[1mm]
                 a_{21} &  a_{22}&a_{23}\\[1mm]
		 a_{31} &  a_{32}&a_{33}
              \end{array}\right |= -
     \left |\begin{array}{ccc}
                 a_{21} &  a_{22}&a_{23}\\[1mm]
                 a_{11} &  a_{12} &a_{13}\\[1mm]
		 a_{31} &  a_{32}&a_{33}
              \end{array}\right |.
$$	      

$(b)$\;  Se puede sacar un factor com'un a cualquier rengl'on o columna de una 
matriz y los determinantes  se relacionan de la siguiente manera, por ejemplo, 
$$
     \left |\begin{array}{ccc}
                \alpha a_{11} & \alpha a_{12} &\alpha a_{13}\\[1mm]
                 a_{21} &  a_{22}&a_{23}\\[1mm]
		 a_{31} &  a_{32}&a_{33}
              \end{array}\right |= \alpha
     \left |\begin{array}{ccc}
                 a_{11} &  a_{12}&a_{13}\\[1mm]
                 a_{21} &  a_{22} &a_{23}\\[1mm]
		 a_{31} &  a_{32}&a_{33}
              \end{array}\right |.	      
$$
$(c)$\; Si a un rengl'on (o columna) le sumamos otro rengl'on (o
 columna), el valor del determinante no cambia, por ejemplo, 
\label{sumarenglonesdet}
\begin{eqnarray*}
     \left |\begin{array}{ccc}
                a_{11} &  a_{12} & a_{13}\\[1mm]
                 a_{21} &  a_{22}&a_{23}\\[1mm]
		 a_{31} &  a_{32}&a_{33}
              \end{array}\right | & = &
     \left |\begin{array}{ccc}
                 a_{11}+a_{21}  &  a_{12}+a_{22}  & a_{13}+a_{23} \\[1mm]
                 a_{21} &  a_{22}&a_{23}\\[1mm]
		 a_{31} &  a_{32}&a_{33}
              \end{array}\right |\\[1mm]
                         &=&    	\left ( \text{o} \ \ \
\left |\begin{array}{ccc}
                 a_{11}+a_{12} &  a_{12} & a_{13}\\[1mm]
                 a_{21}+a_{22} &  a_{22}&a_{23}\\[1mm]
		 a_{31}+a_{32} &  a_{32}&a_{33}
              \end{array}\right |, \right ).
\end{eqnarray*}	      	
$(d)$\; Si una matriz tiene dos renglones  (o dos columnas) iguales el
determinante es cero.
\end{propiedades}

\begin{ejemplo}
\label{ejemplocubicas}
Usando determinantes podemos establecer la siguiente iden\-tidad
\begin{equation}
a^3+b^3+c^3-3abc=(a+b+c)(a^2+b^2+c^2-ab-bc-ca).
\label{abccubicas}
\end{equation}
\end{ejemplo}
\noindent Notemos que 
\begin{eqnarray}
     D & = &\left |\begin{array}{ccc}
                a &  b & c\\[1mm]
                c &  a & b\\[1mm]
		 b &  c& a
              \end{array}\right |  =  a
            \left |\begin{array}{cc}
                  a &b\\[1mm]
                 c &  a
              \end{array}\right |-
      b \left |\begin{array}{cc}
                  c & b\\[1mm]
                 b & a
              \end{array}\right | +
        c \left |\begin{array}{cc}
                  c & a\\[1mm]
                  b &  c
              \end{array}\right |\nonumber\\
	      & = & a^3-abc-abc+b^3+c^3-abc=a^3+b^3+c^3-3abc. 	      	      	      \label{ecuacioncubicacondeterminantes}	      
\end{eqnarray}
Por otro lado, sumando a la primera columna las otras dos, tenemos
\begin{eqnarray*}
     D & = & \left |\begin{array}{ccc}
                a+b+c &  b & c\\[1mm]
                a+b+c &  a & b\\[1mm]
		 a+b+c &  c& a
              \end{array}\right |  =  (a+b+c)
            \left |\begin{array}{ccc}
                1 &  b & c\\[1mm]
                1 &  a & b\\[1mm]
		1 &  c& a
              \end{array}\right | \\[2mm]
	         & = & (a+b+c)(a^2+b^2+c^2-ab-bc-ca).
        \end{eqnarray*}
Por las propiedades $(b)$ y  ($c$), los determinantes son iguales. 

\vei

\noindent Observemos que la expresi'on $a^2+b^2+c^2-ab-bc-ca$ se puede 
escribir como\footnote{Ver el ejercicio 1.\ref{equisyeyzetaalcuadrado}.} 
\begin{equation*}
      \frac{1}{2} \left [ (a-b)^2+(b-c)^2 +(c-a)^2\right ].
\label{factorizaciondeacubica}
\end{equation*}      
Con esto obtenemos otra versi'on de la identidad (\ref{abccubicas}), es decir,
\begin{equation}
a^3+b^3+c^3-3abc=\frac{1}{2}(a+b+c)\left [ (a-b)^2+(b-c)^2 +(c-a)^2\right ].
\label{a3masb3masc3matrices}	     
\end{equation}

\vei

\noindent Observemos que si en la identidad anterior se cumple la condici'on 
$a+b+c=0$ o  la condici'on $a=b=c$, entonces tenemos  la siguiente identidad
\begin{equation}
a^3+b^3+c^3=3abc.
\label{a3masb3masc3igual0}
\end{equation}
Rec'iprocamente, si sucede la identidad (\ref{a3masb3masc3igual0}), entonces 
debe de cumplirse que $a+b+c=0$ o bien $a=b=c$.

\ve
\ve

%%%27
\ejercpreliminares{Muestre que  $\sqrt[3]{2+\sqrt{5}}+\sqrt[3]{2-\sqrt{5}}$ 
es un n'umero racional.
}

\solcpreliminares{Sea  $x=\sqrt[3]{2+\sqrt{5}}+\sqrt[3]{2-\sqrt{5}}$ entonces 
$$
         x-\sqrt[3]{2+\sqrt{5}}-\sqrt[3]{2-\sqrt{5}}=0.
$$	 
Por la ecuaci'on (\ref{abccubicas}), si $a+b+c=0$, entonces 
$a^3+b^3+c^3=3abc$, luego
$$
  x^3-\left (2+\sqrt{5}\right )-\left (2-\sqrt{5}\right )=
3x\sqrt[3]{\left (2+\sqrt{5}\right )\left (2-\sqrt{5}\right)},
$$	 
simplificando se tiene que $x^3+3x-4=0.$	 
Claramente una ra'iz de la ecuaci'on es $x=1$ y las otras dos ra'ices satisfacen
la ecuaci'on $x^2+x+4=0$ que no tiene soluciones reales. Como 
$\sqrt[3]{2+\sqrt{5}}+\sqrt[3]{2-\sqrt{5}}$ es un n'umero real, se sigue que
$\sqrt[3]{2+\sqrt{5}}+\sqrt[3]{2-\sqrt{5}}=1$, 
el cual es un n'umero racional.
}


%%% 28
\ejercpreliminares{Factorice $(x-y)^3+(y-z)^3+(z-x)^3$.
}

\solcpreliminares{Observe que si $x+y+z=0$, entonces se sigue de la ecuaci'on
(\ref{abccubicas}) que $x^3+y^3+z^3=3xyz$. Como $(x-y)+(y-z)+(z-x)=0$,
se obtiene la factorizaci'on      
$$
         (x-y)^3+(y-z)^3+(z-x)^3=3(x-y)(y-z)(z-x).
$$	 
}

%%%29
\ejercpreliminares{Factorice $(x+2y-3z)^3+(y+2z-3x)^3+(z+2x-3y)^3$.
}

\solcpreliminares{Observe que $(x+2y-3z)+(y+2z-3x)+(z+2x-3y)=0$, entonces se sigue de la 
ecuaci'on (\ref{abccubicas}) que 
$(x+2y-3z)^3+(y+2z-3x)^3+(z+2x-3y)^3=3(x+2y-3z)(y+2z-3x)(z+2x-3y)$. 
}

%%%%%30
\ejercpreliminares{Muestre que si $x$, $y$, $z$ son n'umeros reales diferentes, entonces 
$$
\sqrt[3]{x-y}+\sqrt[3]{y-z}+\sqrt[3]{z-x}\neq 0.
$$
}

\solcpreliminares{Sean $a=\sqrt[3]{x-y}$, 
$b=\sqrt[3]{y-z}$, $c=\sqrt[3]{z-x}$, y suponga que $a+b+c=0$, luego, por la ecuaci'on (\ref{abccubicas}), 
$a^{3}+b^{3}+c^{3}=3abc$, pero entonces 
$0=(x-y)+(y-z)+(z-x)=a^{3}+b^{3}+c^{3}=3abc=3\sqrt[3]{x-y}\sqrt[3]{y-z}
\sqrt[3]{z-x}\neq 0$, lo cual es un absurdo.
}

%%%% 31
\ejercpreliminares{Sea $r$ un n\'{u}mero real tal que 
$\sqrt[3]{r}-\frac{1}{\sqrt[3]{r}}=1,$ encuentre los valores de $r-\frac{1}{r}$ y de $r^{3}-\frac{1}{r^{3}}.$
}

\solcpreliminares{Al tomar $a=\sqrt[3]{r}$, $b=-\frac{1}{\sqrt[3]{r}}$ y 
$c=-1$, se tiene $a+b+c=0$, luego, $r-\frac{1}{r}-1=3\sqrt[3]{r}\left( -\frac{1%
}{\sqrt[3]{r}}\right) \left( -1\right) =3$, por lo que $r-\frac{1}{r}=4.$ An'alogamente,
$r^{3}-\frac{1}{r^{3}}-4^{3}=3r\left( -\frac{1}{r}\right) \left(
-4\right) =12,$ por lo que $r^{3}-\frac{1}{r^{3}}=76.$
}

%%%%%%%%%32
\ejercpreliminares{Sean $a$, $b$, $c$ d\'{\i}gitos diferentes de
cero. Muestre que si los n'umeros enteros (escritos en notaci\'{o}n decimal) $abc$, 
$bca$ y $cab$ son divisibles entre $n$, entonces tambi\'{e}n 
$a^{3}+b^{3}+c^{3}-3abc$ es divisible entre $n$.
}

\solcpreliminares{Se sigue de  
$$a^{3}+b^{3}+c^{3}-3abc=%
\begin{vmatrix}
a & b & c \\ 
c & a & b \\ 
b & c & a%
\end{vmatrix}%
=%
\begin{vmatrix}
100b+10c+a & b & c \\ 
100a+10b+c & a & b \\ 
100c+10a+b & c & a%
\end{vmatrix}%
=%
\begin{vmatrix}
bca & b & c \\ 
abc & a & b \\ 
cab & c & a%
\end{vmatrix}.
$$
}

%%%%%%%%%%33
\ejercpreliminares{?`Cu\'{a}ntas parejas
ordenadas de n'umeros enteros $(m,n)$ hay que cumplan las siguientes condiciones: 
$mn\geq 0$ y $m^{3}+99mn+n^{3}=33^{3}$?
}

\solcpreliminares{Escriba  la ecuaci\'{o}n como 
$m^{3}+n^{3}+(-33)^{3}-3mn(-33)=0,$ y usando la ecuaci'on 
(\ref{a3masb3masc3matrices}), se tiene 
$$
(m+n-33)\left[(m-n)^2+(m+33)^2+(n+33)^2\right] =0.
$$ 
La ecuaci\'{o}n $m+n=33$ tiene 
$34$ soluciones con $mn\geq 0$ que son $(k,33-k)$, con $k=0,1, \dots, 33$, y el
segundo factor es $0$ solamente cuando $m=n=-33$, luego hay $35$ soluciones.
}

%%%%%%%%%% 34
\ejercpreliminares{Encuentre el lugar geom\'{e}trico de los
puntos $(x,y)$ tales que $x^{3}+y^{3}+3xy=1$.
}

\solcpreliminares{Al reescribir la ecuaci\'{o}n como
$x^{3}+y^{3}+(-1)^{3}-3xy(-1)=0$ y, utilizando la ecuaci'on 
(\ref{a3masb3masc3matrices}), se tiene 
$$
(x+y-1)\left[(x-y)^{2}+(y+1)^{2}+(-1-x)^{2}\right] =0.
$$ 
Luego, los puntos $(x,y)$ deben cumplir con $x+y=1$ o bien $x=y=-1.$
}

%%%%%%%% 35
\ejercpreliminares{Encuentre las soluciones reales $x$, $y$, $z$ de
la ecuaci\'{o}n,
$$
           x^{3}+y^{3}+z^{3}=(x+y+z)^{3}.
$$
}

\solcpreliminares{Sustituyendo la ecuaci'on de la hip'otesis en  la ecuaci'on (\ref{abccubicas}) se obtiene que
\begin{eqnarray*}
(x+y+z)^{3} -3xyz & = &x^{3}+y^{3}+z^{3}-3xyz\\
                               & = & (x+y+z)(x^2+y^2+z^2- xy-yz-zx)\\
                                & = & (x+y+z)((x+y+z)^2-3xy-3yz-3zx)\\
                                & = & (x+y+z)^3-3(x+y+z)(xy+yz+zx).
\end{eqnarray*}
Ahora es claro que $xyz=(x+y+z)(xy+yz+zx)$, de ah'i que $(x+y)(y+z)(z+x)=0$. O bien use que $(x+y+z)^3=x^3+y^3+z^3+3(x+y)(y+z)(z+x)$. 

Luego, las soluciones son
$(x,-x,z)$, $(x,y,-y)$, $(x,y,-x)$, con $x$, $y$, $z$ cualesquiera n'umeros reales.
}










\section{Desigualdades}
\label{sec-desigualdades}
\index{Desigualdades}
\noindent Iniciamos esta secci'on con una de las  desigualdades m'as importantes.
Para cualquier n'umero real $x$, tenemos que
\begin{equation}
\label{ecuac1.2.1}
                  x^2  \geq  0.
\end{equation}
Esto se sigue de la igualdad  $x^2=|x|^2 \geq 0$.

\noindent A partir de este resultado podemos deducir  que la suma de $n$  
n'umeros cuadrados es no negativa,
\begin{equation}
\label{ecuac1.2.2}
                      x_1^2+x_2^2+\cdots+ x_n^2  \geq  0
\end{equation}
y ser'a cero  si y solamente si todos los $x_i$ son cero.

\noindent Si en la ecuaci'on (\ref{ecuac1.2.1}) sustituimos $x=a-b$, donde
$a$ y $b$ son n'umeros reales no negativos, tenemos que
$$
     (a-b)^2\geq 0.
$$
Desarrollando el binomio, la desigualdad anterior toma la forma,
\begin{equation}
        a^2+b^2 \geq 2ab.
\label{a2masb2mayoroiguala2ab}
\end{equation}
Como 
$$
   a^2+b^2\geq 2ab \;\;\text{si y s'olo si}\;\; 2a^2+2b^2\geq a^2+2ab+b^2=
	 (a+b)^2,
$$ 
tenemos tambi'en la siguiente desigualdad 
\begin{equation}
\label{ecuac1.2.3}\sqrt{ \frac{a^2+b^2}{2}}  \geq  \frac{(a+b)}{2}.
\end{equation}				
En caso de que $a$ y $b$ sean positivos, la  desigualdad (\ref{a2masb2mayoroiguala2ab}) garantiza que 
\begin{equation}
\label{numeroyreciproco}
        \frac{a}{b}+\frac{b}{a}\geq 2.
\end{equation}
Si en la desigualdad anterior tomamos $b=1$, entonces tenemos que  
$a+\frac{1}{a}\geq 2$, es decir, la suma de $a>0$ y su rec'iproco es 
mayor o igual que 2, y es 2 si y s'olo si  $a=1$.

\vei

\noindent Remplazando $a$, $b$ por $\sqrt{a}$, $\sqrt{b}$ en (\ref{a2masb2mayoroiguala2ab}) obtenemos
\begin{equation}
\label{ecuac1.2.5}
   a+b\geq 2\sqrt{ab}\;\; \text{si y s'olo si}\;\;\frac{a+b}{2}\geq \sqrt{ab}.
\end{equation}
Multiplicando la 'ultima desigualdad por $\sqrt{ab}$ y reacomodando,  tenemos
\begin{equation}
\label{ecuac1.2.4e}
 \sqrt{ab} \geq \frac{2ab}{a+b}.
\end{equation}
Juntando  las desigualdades (\ref{ecuac1.2.3}), (\ref{ecuac1.2.5})   y (\ref{ecuac1.2.4e}),  hemos demostrado que
\begin{equation}
    \frac{2ab}{a+b}\leq \sqrt{ab}\leq \frac{a+b}{2}\leq \sqrt{\frac{a^2+b^2}{2}}.
\label{todaslasmedias}
\end{equation}
La primera expresi'on  se conoce como la {\bf media arm'onica}\index{Media!arm'onica} ($MH$), la segunda es la
{\bf media geom'etrica} \index{Media!geom'etrica} ($MG$), la tercera es la
{\bf media aritm'etica} \index{Media!aritm'etica} ($MA$) y la 'ultima es la
{\bf media cuadr'atica}\index{Media!cuadr'atica} ($MQ$). 

Estas desigualdades tambi'en  se pueden demostrar
geom'etricamente como sigue. Consideremos un semicircunferencia con centro $O$, radio $\frac{a+b}{2}$ y los tri'angulos
rect'an\-gulos $ABC$, $DBA$
y $DAC$, como se muestra en la figura

\centerline{
	\psset{unit=1cm}
	\begin{pspicture}(0,0)(6,4.5)
	\psset{linewidth=0.5pt}
	\psline(0,1)(6,1)
	\psline(0,.6)(.1,.4)(1.3,.4)(1.5,.6)
	\psline(1.5,.6)(1.6,.4)(5.9,.4)(6,.6)
	\psarc[linewidth=.5pt](3,1){3}{0}{180}
	\psline(1.5,1)(1.5,3.61)
	\psline(1.5,3.61)(3,1)
	\psline(0,1)(1.5,3.61)(6,1)
	\psline(1.5,1)(2.68,1.58)
          \psline(1.3,1)(1.3,1.2)(1.5,1.2)
	\psline(2.4,1.44)(2.3,1.63)(2.57,1.77)
	\rput{0}(.75,.1){$ a$}
	\rput{0}(4,.1){$b$}
	\rput{0}(2,1.4){$e$}
	\rput{0}(1.3,2){$h$}
	\rput{0}(2.5,2.5){$y$}
	\rput{0}(3.2,1.3){$z$}
	\rput{0}(1.5,3.9){$A$}
	\rput{0}(-.3,1){$ B$}
	\rput{0}(6.3,1){$ C$}
	\rput{0}(1.5,.8){$ D$}
	\rput{0}(2.9,1.7){$E$}
	\rput{0}(3,.8){$ O$}
	\end{pspicture}
}
\noindent Estos tri'angulos son semejantes por lo que tenemos que
\begin{eqnarray*}
	     \frac{AD}{DB} & = &\frac{DC}{DA}\\
             \frac{h}{a} & = &\frac{b}{h}\\
                    h^2  & = & ab,
\end{eqnarray*}
es decir, que la altura com'un de los tri'angulos es $h=\sqrt{ab}$, que claramente
es menor que el radio de la semicircunferencia.  Luego,
$\sqrt{ab}\leq \frac{a+b}{2}$.

\noindent Para demostrar la primera desigualdad de (\ref{todaslasmedias}), observemos que los tri'angulos
$DAE$ y $OAD$ son semejantes, entonces
\begin{eqnarray*}
             \frac{AD}{AE} & = &\frac{AO}{AD}\\
                    h^2  & = & y(y+z)\\
             \frac{2ab}{a+b} & = & y,
\end{eqnarray*}
es decir, $y$ representa la media arm'onica. Claramente tenemos que $y\leq h$,
luego $\frac{2ab}{a+b}\leq \sqrt{ab}$.

\noindent Para demostrar geom'etricamente, la 'ultima desigualdad de (\ref{todaslasmedias}), conside\-remos
la siguiente figura

\centerline{\psset{unit=.9cm}
	\begin{pspicture}(0,0)(5,4.7)
	\psset{linewidth=0.5pt}
	\psline(0,1)(6,1)
	\psarc[linewidth=.5pt](3,1){3}{0}{180}
	\psline(1.5,1)(1.5,3.61)
	\psline(1.5,1)(3,4)
	\psline(3,1)(3,4)
	\rput{0}(1.5,3.9){$ A$}
	\rput{0}(3,4.3){$ L$}
	\rput{0}(1.5,.7){$ D$}
	\rput{0}(3,.7){$ O$}
	\end{pspicture}
}

\noindent Tenemos que $OD=\frac{a+b}{2}-a=\frac{b-a}{2}$ y utilizando el teorema
de Pit'agoras tenemos que
$$     
 DL^2  =  OD^2 + OL^2=\Big(\frac{b-a}{2}\Big )^2 +\Big(\frac{a+b}{2}\Big )^2 =  \frac{a^2+b^2}{2},
$$			
es decir, $DL =\sqrt{\frac{a^2+b^2}{2}}$ que claramente es mayor que
$\frac{a+b}{2}$.

\ve

\noindent Utilizando el ejemplo \ref{ejemplocubicas} podemos dar una
demostraci'on de la desigualdad entre la media geom'etrica y la media aritm'etica
 para tres n'umeros reales no negativos. En efecto, por la identidad
$$
a^3+b^3+c^3-3abc=\frac{1}{2}(a+b+c)\left [ (a-b)^2+(b-c)^2 +(c-a)^2\right ],	
$$
es claro que si $a$, $b$ y $c$ son no negativos, entonces
$a^3+b^3+c^3-3abc\geq 0$, es decir, $a^3+b^3+c^3 \geq 3abc$.
Adem'as, tenemos la igualdad si $a+b+c=0$ o $(a-b)^2+(b-c)^2 +(c-a)^2=0$, esto es solamente cuando $a=b=c$.
 Ahora si $x$, $y$ y $z$ son n'umeros no negativos, definiendo $a=\sqrt[3]{x}$,
$b=\sqrt[3]{y}$ y $c=\sqrt[3]{z}$, tenemos que
\begin{equation}
    \frac{x+y+z}{3}\geq \sqrt[3]{xyz},
\label{mediaaritgeomtresvariables}
\end{equation}
con igualdad si y s'olo si $x=y=z$.


\begin{ejemplo}
Para todo n'umero real $x$, sucede que
$\frac{x^2+2}{\sqrt{x^2+1}}\geq 2$.
\end{ejemplo}

En efecto,
$$
    \frac{x^2+2}{\sqrt{x^2+1}}  =  \frac{x^2+1}{\sqrt{x^2+1}}+\frac{1}{\sqrt{x^2+1}}
                                =  \sqrt{x^2+1}+\frac{1}{\sqrt{x^2+1}} \geq 2.
$$
La desigualdad se sigue de aplicar la desigualdad (\ref{numeroyreciproco}).

\begin{ejemplo}
Si $a$, $b$, $c$ son n'umeros no negativos, entonces
$$
          (a+b)(b+c)(a+c)\geq 8abc.
$$
\end{ejemplo}
\noindent Como hemos visto, $\frac{(a+b)}{2}\geq \sqrt{ab}$, $\frac{(b+c)}{2}\geq \sqrt{bc}$\; y $\frac{(a+c)}{2}\geq \sqrt{ac}$, de donde
$$
    \left (\frac{a+b}{2}\right )\left (\frac{b+c}{2}\right )\left (\frac{a+c}{2}\right ) \geq\sqrt{a^2b^2c^2}=abc.
$$

\begin{ejemplo}
\label{ejemploreacomodo}
Si $x_1>x_2>x_3$ y $y_1>y_2>y_3$, ?`cu'al de las siguientes sumas es mayor?
\begin{eqnarray*}
     S & = & x_1y_1+x_2y_2+x_3y_3\\
		 S^\prime & = & x_1y_2+x_2y_1+x_3y_3.
\end{eqnarray*}		
\end{ejemplo}
Consideremos la diferencia,
\begin{eqnarray*}
     S^\prime- S & = & x_1y_2-x_1y_1+x_2y_1-x_2y_2\\
               	 & = & x_1(y_2-y_1)+x_2(y_1-y_2)\\
		  & = & -x_1(y_1-y_2)+x_2(y_1-y_2)\\
		 & = & (x_2-x_1)(y_1-y_2) < 0,
\end{eqnarray*}		
por lo tanto, $S^\prime < S$.

\noindent M'as generalmente, para cualquier permutaci'on $\{y^\prime_1,y^\prime_2,y^\prime_3\}$ de
$\{y_1,y_2,y_3\}$ tenemos que,
\begin{equation}
         S\geq x_1y^\prime_1+x_2y^\prime_2+x_3y^\prime_3,
\label{desigualdadreacomodo}
\end{equation} que se conoce como la 
{\bf desigualdad del reacomodo}\footnote{Para una versi'on general de la 
desigualdad del reacomodo, vea el ejemplo \ref{desigualdaddelreacomodo}.}. 
\index{Desigualdad!del reacomodo}

\vei		




%%%% 35
\ejercpreliminares{Sean $a$, $b$ n\'{u}meros reales con $0\leq a\leq b\leq 1$,
muestre que:

$(i)$ $0\leq \dfrac{b-a}{1-ab}\leq 1$.

$(ii)$ $0\leq \dfrac{a}{1+b}+\dfrac{b}{1+a}\leq 1$.
}

\solcpreliminares{$(i)$ Como $0\leq b\leq 1$ y $1+a>0$, pasa que $b(1+a)\leq 1+a$, luego 
$0\leq b-a\leq 1-ab$, por lo que $0\leq \dfrac{b-a}{1-ab}\leq 1.$

\noindent $(ii)$ La desigualdad de la izquierda es clara. Como $1+a\leq 1+b,$
se tiene que $\frac{1}{1+b}\leq \frac{1}{1+a},$ luego, $\dfrac{a}{1+b}+\dfrac{b}{1+a}\leq \dfrac{a}{1+a}+\dfrac{b}{1+a}=\dfrac{a+b}{1+a}\leq 1.$
}



%%%% 36
\ejercpreliminares{(Desigualdad de Nesbitt) Si $a$, $b$,  $c\geq 0$, muestre que  
$$
         \frac{a}{b+c}+\frac{b}{a+c}+\frac{c}{a+b}\geq \frac{3}{2}.
$$
\index{Desigualdad! de Nesbitt}
}

\solcpreliminares{Al hacer $X=\frac{a}{b+c}+\frac{b}{a+c}+\frac{c}{a+b}$ y
sumando y restando tres veces la unidad se tiene
\begin{align*}
    X &=\frac{a}{b+c}+\frac{b+c}{b+c}+\frac{b}{a+c}
        +\frac{a+c}{a+c}+\frac{c}{a+b}+\frac{a+b}{a+b}-3\\[.3cm]
			&=\frac{a+b+c}{b+c}+\frac{a+b+c}{a+c}+\frac{a+b+c}{a+b} -3\\
         &=(a+b+c)\left(\frac{1}{b+c}+\frac{1}{a+c}+\frac{1}{a+b}\right)-3\\
     &=\frac{1}{2}((a+b)+(b+c)+(a+c))\left(\frac{1}{b+c}+\frac{1}{a+c}+\frac{1}{a+b}\right)-3.
\end{align*} 
Ahora, por la desigualdad entre la media geom'etrica y la media aritm'etica, $x+y+z\geq 3\sqrt[3]{xyz}$ y $\frac{1}{x}+\frac{1}{y}+\frac{1}{z}\geq 3\sqrt[3]{\frac{1}{x}\frac{1}{y}\frac{1}{z}}$. Luego, $X\geq \frac{1}{2}\cdot 3\cdot 3-3=\frac{3}{2}$.  
}

%%%% 37
\ejercpreliminares{Si $a$, $b$, $c$ son las longitudes de los lados
de un tri\'{a}ngulo, muestre que
$$
\sqrt[3]{\dfrac{a^{3}+b^{3}+c^{3}+3abc}{2}}\geq \max \left\{ a,b,c\right\}. 
$$
}

\solcpreliminares{Sin p'erdida de generalidad, se puede suponer que $a\geq b\geq c$, la desigualdad es equivalente a 
$-a^{3}+b^{3}+c^{3}+3abc\geq 0$. Pero, por la ecuaci'on (\ref{a3masb3masc3matrices}), 
$-a^3+b^3+c^3+3abc=\frac{1}{2}(-a+b+c)\left[
(a+b)^{2}+(a+c)^{2}+(b-c)^{2}\right]\geq 0$, ya que, por la desigualdad del tri'angulo,  $a<b+c$.}


%%%% 38


\ejercpreliminares{Sean $p$ y $q$ n'umeros reales positivos con $\frac{1}{p}+\frac{1}{q}=1$. Muestre que:

\noindent $(i)$ $\dfrac{1}{3}\leq \dfrac{1}{p(p+1)}+\dfrac{1}{q(q+1)}\leq \dfrac{1}{2}$.\\

\noindent $(ii)$ $\dfrac{1}{p(p-1)}+\dfrac{1}{q(q-1)}\geq 1$. 
}

\solcpreliminares{Observe que $\frac{1}{p}+ \frac{1}{q}=1$ implica que $p+q=pq=s$.  Ahora bien, $(p+q)^2\geq 4pq$ implica que $s\geq 4$.

\noindent  Para probar $(i)$, vea que
 \begin{align*}
\frac{1}{p(p+1)}+\frac{1}{q(q+1)} & = \frac{1}{p} - \frac{1}{p+1}+\frac{1}{q} - \frac{1}{q+1}= 1 - \frac{p+q+2}{(p+1)(q+1)}\\
& = 1 - \frac{s+2}{2s+1}=\frac{s-1}{2s+1}.
\end{align*}
Luego, hay  que mostrar que
$$
    \frac{1}{3}\leq \frac{s-1}{2s+1}\leq \frac{1}{2},
$$
pero $2s+1\leq 3s-3 \Leftrightarrow 4 \leq s$ y
$2s-2\leq 2s+1 \Leftrightarrow -2 \leq 1$. 

\vei

\noindent  Para probar $(ii)$, vea que
 \begin{align*}
\frac{1}{p(p-1)}+\frac{1}{q(q-1)} &= \frac{1}{p-1} - \frac{1}{p}+\frac{1}{q-1} - \frac{1}{q}= \frac{p+q-2}{(p-1)(q-1)} - 1 \\
& = \frac{s-2}{s-s+1}-1=s-3\geq 1.
\end{align*}
}

%%%% 39

\ejercpreliminares{Encuentre el menor n\'{u}mero positivo $k$ tal que, para cualesquiera $0<a$, $b<1$, con $ab=k$, se cumpla que
$$
\frac{a}{b}+\frac{b}{a}+\frac{a}{1-b}+\frac{b}{1-a}\geq 4.
$$
}

\solcpreliminares{Note primero que,
$$
\frac{a}{b}+\frac{a}{1-b}=\frac{a}{b(1-b)}\geq 4a,
$$
\noindent ya que
$$b(1-b)\leq \left( \frac{b+(1-b)}{2}\right) ^{2}=\frac{1}{4}.
$$
\noindent Adem\'{a}s, se tiene la igualdad si y s\'{o}lo si $b=\frac{1}{2}.$
An\'{a}logamente,
$$
\frac{b}{a}+\frac{b}{1-a}\geq 4b.
$$
\noindent Por lo que,
$$
\frac{a}{b}+\frac{b}{a}+\frac{a}{1-b}+\frac{b}{1-a}\geq 4a+4b\geq 2%
\sqrt{4^{2}ab}=8\sqrt{k}.
$$
\noindent Con igualdad si y s\'{o}lo si $a=b.$ As\'{\i},
$$
\frac{a}{b}+\frac{b}{a}+\frac{a}{1-b}+\frac{b}{1-a}\geq 8\sqrt{k}\geq 4
$$
si y s\'{o}lo si $k\geq \frac{1}{4},$ por lo que el menor n'umero $k$ es $\frac{1}{4}.$
}

%%%%%%%%% 40
\ejercpreliminares{Sean $a$, $b$, $c$ n\'{u}meros reales no
negativos, muestre que
\begin{equation*}
(a+b)(b+c)(c+a)\geq \frac{8}{9}(a+b+c)(ab+bc+ca).
\end{equation*}
}


\solcpreliminares{Vea que, $(a+b)(b+c)(c+a)=(a+b+c)(ab+bc+ca)-abc=
\frac{8}{9}(a+b+c)(ab+bc+ca)+\frac{1}{9}(a+b+c)(ab+bc+ca)-abc$ y, 
por la desigualdad entre la media geom'etrica y la media aritm'etica, 
$(a+b+c)(ab+bc+ca)\geq \left (3\sqrt[3]{abc}\right)\left (3\sqrt[3]{(ab)(bc)(ca)}\right)=9abc$.
}

%%%%%%%%%%41
\ejercpreliminares{Sean $a$, $b$, $c$ n\'{u}meros reales positivos
que satisfacen la siguiente igualdad $(a+b)(b+c)(c+a)=1.$ Muestre que 
\begin{equation*}
ab+bc+ca\leq \frac{3}{4}.
\end{equation*}
}

\solcpreliminares{Por la desigualdad entre la media geom'etrica y la media aritm'e\-tica,  y la condici'on $(a+b)(b+c)(c+a)=1$, se tiene 
\begin{eqnarray*}
a+b+c & \geq & 3\sqrt[3]{\left(\frac{a+b}{2}\right)\left(\frac{b+c}{2}\right)\left(\frac{%
c+a}{2}\right)}=\frac{3}{2},\\
abc &  = & \sqrt{ab}\sqrt{bc}\sqrt{ca}\leq \left(\frac{a+b}{2}\right)%
\left(\frac{b+c}{2}\right)\left(\frac{c+a}{2}\right)=\frac{1}{8}.
\end{eqnarray*}
Ahora bien,  
$1=(a+b)(b+c)(c+a)=(a+b+c)(ab+bc+ca)-abc\geq \frac{3}{2}(ab+bc+ca)-\frac{1}{8}$, vea el ejercicio 1.\ref{ejerciciotresdocetres} $(iii)$.
}


%%%%%%%%%%%%% 43

\ejercpreliminares{Sean $a$, $b$, $c$ n\'{u}meros reales positivos
que satisfacen $abc=1$. Muestre que
$(a+b)(b+c)(c+a)\geq 4(a+b+c-1)$.
}

\solcpreliminares{Por el ejercicio 1.\ref{ejerciciotresdocetres} $(iii)$, basta ver que 
$ab+bc+ca+\frac{3}{a+b+c}\geq 4$. Pero 
\begin{align*}
ab+bc+ca+\frac{3}{a+b+c}& =3\left(\frac{ab+bc+ca}{3}\right)+\frac{3}{a+b+c}\\
            &\geq 4\sqrt[4]{\left( \frac{ab+bc+ca}{3}\right)
^{3}\left( \frac{3}{a+b+c}\right) }.
\end{align*}
Ahora use que, $(ab+bc+ca)^{2}\geq
3(ab\cdot bc+bc\cdot ca+ca\cdot ab)=3(a+b+c)$, 
y que $ab+bc+ca\geq 3 \sqrt[3]{a^2b^2c^2}=3$.
}

%%%%%%%%%%%%% 44

\ejercpreliminares{(APMO, 2011) Sean $a$, $b$, $c$ n\'{u}meros enteros positivos.  Muestre que es imposible
que los tres n'umeros $a^2+b+c$, $b^2+c+a$ y  $c^2+a+b$ sean cuadrados perfectos.
}

\solcpreliminares{Sin p'erdida de generalidad podemos suponer $a\leq b\leq c$. Luego, 
$ c^2<c^2+a+b\leq c^2+ 2c< ( c+1)^2$,
esto muestra que $c^2+a+b$ no puede ser un cuadrado perfecto.
}




\section{Factorizaci'on}
\label{cap1sec3}

\index{Factorizaci'on}
\noindent Una de las formas m'as importantes de manipulaci'on algebraica  
es la que se conoce como factorizaci'on. En esta secci'on estudiamos algunos 
ejemplos y problemas cuya soluci'on
depende del conocimiento de  f'ormulas  de factorizaci'on.
Muchos de los problemas ol'impicos que involucran expresiones algebraicas se 
resuelven f'acilmente haciendo uso de transformaciones algebraicas que 
utilizan factorizaciones apropiadas.
Empecemos con  algunas  f'ormulas  elementales de factorizaci'on, donde $x$, 
$y$ son  n'umeros reales:
\begin{description}
\item[$(a)$] $x^2 - y^2 =(x+y)(x-y)$.
\item[$(b)$]  $x^2 +2xy + y^2 = (x+y)^2$ y $x^2-2xy+y^2 = (x-y)^2$.
\item[$(c)$]  $x^2+y^2+z^2+2xy+2yz+2zx =(x+y+z)^2$.
\end{description}

\noindent Estas identidades algebraicas se catalogan como identidades de grado $2$. De hecho, estas cuatro identidades fueron ya estudiadas en la secci'on de productos notables, sin embargo, lo que se desea hacer ahora es, dada una expresi'on algebraica, reducirla a un producto de expresiones algebraicas  m\'as simples.

\vei

\begin{ejemplo} Para n\'{u}meros reales $a$, $b$, $x$, $y$, con $x$ y $y$ 
distintos de cero,  se tiene que  
$$
            \frac{a^{2}}{x}+\frac{b^{2}}{y}-\frac{(a+b)^{2}}{x+y}=\frac{(ay-bx)^{2}}{xy(x+y)}.
$$
\end{ejemplo}
Para obtener la igualdad que se pide, empecemos realizando la suma del lado izquierdo de la  identidad,
\begin{eqnarray*}
\frac{a^{2}}{x}+\frac{b^{2}}{y}-\frac{(a+b)^{2}}{x+y} & = & \frac{a^2 y (x+y)+ b^2 x (x+y) -
xy (a+b)^2}{xy(x+y)} \\
& = & \frac{a^2 y^2 + b^2 x^2 - 2xyab}{xy(x+y)}\\
& = & \frac{(ay-bx)^{2}}{xy(x+y)}.
\end{eqnarray*}

\vei

\noindent Una aplicaci\'on de la identidad anterior nos lleva a una 
demostraci'on inmediata de la {\bf desigualdad 'util}\footnote{Ver 
\cite{bulajich1}, p'ag. 40  o \cite{bulajich} p'ag. 34.} de grado $2$. 'Esta asegura que 
para n\'{u}meros reales $a$, $b$ y n\'{u}meros reales positivos $x$, $y$, 
se cumple que
\index{Desigualdad!'util}
$$
\frac{a^{2}}{x}+\frac{b^{2}}{y}\geq \frac{(a+b)^{2}}{x+y}.
$$

\vei

\noindent Las siguientes identidades son de grado $3$, con $x$, $y$, $z \in \mathbb{R}$:
\begin{description}
\item[$(a)$] $x^{3}-y^{3}=(x-y)\left( x^{2}+xy+y^{2}\right)$.
\item[$(b)$] $x^{3}-y^{3}=(x-y)^{3}+3xy(x-y)$.
\item[$(c)$] $(x+y)^{3}-(x^{3}+y^{3})=3xy(x+y)$.
\item[$(d)$] $x^{3}-xy^{2}+x^{2}y-y^{3}=(x+y)(x^{2}-y^{2})$.
\item[$(e)$] $x^{3}+xy^{2}-x^{2}y-y^{3}=(x-y)(x^{2}+y^{2})$.
\end{description}

\noindent Para comprobar la validez de las identidades anteriores basta desarrollar alguno de los lados o utilizar el teorema del binomio de Newton, el cual se estudiar'a en la secci'on \ref{coeficientesbinomiales}.

\vei

\noindent Otra identidad  de grado $3$ muy importante y que ya mencionamos en la identidad (\ref{abccubicas}) es
$$
x^{3}+y^{3}+z^{3}-3xyz=(x+y+z)(x^{2}+y^{2}+z^{2}-xy-yz-zx),
$$
para cualesquiera n'umeros reales $x$, $y$, $z$.  Una demostraci'on de 'esta se obtiene simplemente desarrollando el lado derecho de la identidad.  A lo largo del libro veremos otras demostraciones de esta igualdad. 

Una forma equivalente de la identidad anterior es
$$
x^{3}+y^{3}+z^{3}-3xyz=\frac{1}{2}(x+y+z)\left[ \left( x-y\right)
^{2}+(y-z)^{2}+(z-x)^{2}\right].
$$

\vei


\noindent Las identidades $x^2 - y^2 =(x+y)(x-y)$ y $x^{3}-y^{3}=(x-y)\left( x^{2}+xy+y^{2}\right)$
son casos particulares de la identidad de grado $n$,
\begin{equation}
      x^n-y^n=(x-y)(x^{n-1} + x^{n-2} y + \cdots + x y^{n-2} + y^{n-1}),
\label{antesdesophie}
\end{equation}
para cualesquiera n'umeros reales $x$, $y$.

\noindent Si $n$ es impar, podemos reemplazar $y$ por $-y$ en la 'ultima f'ormula para obtener
la f'ormula de factorizaci'on para la suma de dos potencias $n$-'esimas impares,
\begin{equation}
x^n+y^n=(x+y)(x^{n-1} - x^{n-2} y + \cdots - x y^{n-2} + y^{n-1}).
\label{identidadxalanyalan}
\end{equation}

\noindent En general, la suma de potencias $n$-\'esimas pares no es factorizable, aunque existen algunas
excepciones cuando es posible completar cuadrados, veamos el siguiente ejemplo.

\begin{ejemplo}(Identidad de Sophie Germain) Para cualesquiera n'umeros reales $x$, $y$ se tiene que
\label{sophiegermain}\index{Identidad de Sophie Germain}
$$
x^{4}+4y^{4} =(x^{2}+2y^{2}+2xy)(x^{2}+2y^{2}-2xy).
$$
\end{ejemplo}

\noindent Completando cuadrados, tenemos
\begin{eqnarray*}
     x^{4}+4y^{4} & = & x^{4}+4x^2y^2+4y^{4}- 4x^2y^2=(x^{2}+2y^{2})^2-(2xy)^2\\
            & = &(x^{2}+2y^{2}+2xy)(x^{2}+2y^{2}-2xy).
\end{eqnarray*}

\noindent Otro ejemplo, con potencias pares es el siguiente.

\begin{ejemplo} Para cualesquiera n'umeros reales $x$, $y$, se tiene que
$$
x^{2n}-y^{2n} =(x+y)(x^{2n-1}-x^{2n-2}y+x^{2n-3}y^{2}-\cdots+xy^{2n-2}-
y^{2n-1}).
$$
\end{ejemplo}
\noindent Para comprobar esto simplemente tenemos que hacer la divisi'on de $x^{2n}-y^{2n}$ entre $x+y$ o bien realizar el producto de la derecha y simplificar.

\begin{ejemplo} Veamos que  $n^4 - 22n^2+9$ es un n'umero compuesto para cualquier entero $n$.
\end{ejemplo}
La idea para mostrar lo que se pide es tratar de factorizar la expresi'on. Inten\-temos completar cuadrados, la forma m'as com'un de hacerlo es la siguiente
$$
         n^4-22n^2+9= (n^4 - 22n^2 + 121)-112=(n^2 - 11)^2-112,
$$
el problema que tenemos es que $112$ no es un cuadrado perfecto, por lo que no es inmediato factorizar. Sin embargo, podemos utilizar la siguiente forma, menos usual, de completar cuadrados
\begin{eqnarray*}
n^4-22n^2+9 & = & (n^4 - 6n^2 + 9)- 16n^2=(n^2-3)^2-16n^2\\
            & = & (n^2-3)^2-(4n)^2=(n^2-3+4n)(n^2-3-4n)\\
            & = & ((n+2)^2-7)((n-2)^2-7),
\end{eqnarray*}
y observemos que ninguno de los dos 'ultimos factores es igual a $\pm 1$.
\vei

\noindent El siguiente es otro ejemplo de c'omo utilizando formas b'asicas de 
factorizaci'on podemos  resolver problemas.

\begin{ejemplo}
Encontremos todas las parejas $(m,n)$ de n'umeros enteros positivos 
tales que $|3^m - 2^n|=1$.
\end{ejemplo}
Cuando $m=1$ o $m=2$, es f'acil encontrar las soluciones $(m,n)=(1,1)$, 
$(1,2)$, $(2,3)$. Ahora
mostremos que no hay otras soluciones. Supongamos que $(m,n)$ es una 
soluci'on de $|3^m - 2^n|=1$, con $m > 2$, y por lo tanto $n > 3$. Analicemos 
los dos casos: $3^m - 2^n=1$ y $3^m - 2^n=-1$.

\noindent Supongamos que $3^m - 2^n=-1$ con $n > 3$, entonces $3^m+1$ es 
divisible entre $8$, sin embargo
al dividir $3^m$ entre $8$ obtenemos como residuo $1$ o $3$, dependiendo de 
si $n$ es par o impar, por lo que
en este caso no hay soluci'on.

\noindent Supongamos que $3^m - 2^n=1$ con $m\geq  3$, por lo que $n\geq 5$, 
ya que $2^n+1=3^m\geq 27$. Entonces $3^m-1$ es divisible entre $8$, por lo
cual $m$ es par, digamos $m=2k$, con $k >1$. Entonces 
$2^n = 3^{2k}-1=(3^k+1)(3^k-1)$. 

Luego,
$3^k+1=2^r$, para alguna $r >3$, pero por el caso anterior sabemos que esto 
es imposible, luego en este caso tampoco hay soluciones. 


\ve

\noindent Otras  f'ormulas  'utiles de factorizaci'on son las siguientes.
Para n\'{u}meros reales $x$, $y$, $z$ se cumplen las siguientes igualdades:
\begin{equation}
\label{productodetres}
(x+y)(y+z)(z+x)+xyz=(x+y+z)(xy+yz+zx)
\end{equation}
\begin{equation}
(x+y+z)^{3}=x^{3}+y^{3}+z^{3}+3(x+y)(y+z)(z+x).
\end{equation}
Para convencerse, basta desarrollar ambos lados de cada igualdad.
De estas identidades tenemos la siguiente observaci'on. 

\begin{observacion}
$(a)$\; Si $x$, $y$, $z$ son n\'{u}meros reales, con $xyz=1$, entonces
\begin{equation}
(x+y)(y+z)(z+x)+1=(x+y+z)(xy+yz+zx).
\end{equation}
\label{observacionabcigual1}
$(b)$\; Si $x$, $y$, $z$ son n\'{u}meros reales con $xy+yz+zx=1$, entonces
\begin{equation}
(x+y)(y+z)(z+x)+xyz=x+y+z.
\end{equation}
\end{observacion}

\ve
\vei

%%%%%% 45
\ejercpreliminares{Para todos los n'umeros reales $x$, $y$ y $z$, 
se tienen las siguientes identidades:

\noindent $(i)$\;
$(x+y+z)^{3}-(y+z-x)^{3}-(z+x-y)^{3}-(x+y-z)^{3}=24xyz$.\\
\noindent $(ii)$\;
$(x-y)^{3}+(y-z)^{3}+(z-x)^{3}=3(x-y)(y-z)(z-x)$.\\
$(iii)$ \, $(x-y)(y+z)(z+x)+(y-z)(z+x)  (x+y)+(z-x)(x+y)(y+z)$\\
\phantom{.}\hspace{1.8in}        $ = - (x-y)(y-z)(z-x)$.
}

\solcpreliminares{Para demostrar todos los incisos de este ejercicio, 
'unicamente realice las operaciones y simplifique.
}

%%%%%% 46
\ejercpreliminares{Para todos los n'umeros reales $x$, $y$ y $z$, muestre lo 
siguiente:
%\noindent $(i)$\;
%$x^{3}+y^{3}+z^{3}=3xyz$  si y s\'{o}lo si $x+y+z=0$ o bien $x=y=z$.

\noindent $(i)$\; Si $f(x,y,z)=x^{3}+y^{3}+z^{3}-3xyz$, entonces
$$
        f(x,y,z)=\frac{1}{2} f(x+y, y+z, z+x)=\frac{1}{4} f(-x+y+z,x-y+z,x+y-z).
$$
\noindent $(ii)$\; Si $f(x,y,z)=x^{3}+y^{3}+z^{3} - 3xyz$, entonces
$f(x,y,z)\geq 0$ si y s\'{o}lo si $x+y+z\geq 0$ y  $f(x,y,z)\leq 0$
si y s\'{o}lo si $x+y+z\leq 0$.
}

\solcpreliminares{Para demostrar todos los incisos de este ejercicio utilice 
la identidad  (\ref{abccubicas}).
}

%%%%%% 46
\ejercpreliminares{Muestre que para n'umeros reales $x$, $y$, se tienen las 
siguientes identidades:

\noindent $(i)$\;
$(x+y)^{5}-(x^{5}+y^{5})=5xy(x+y)(x^{2}+xy+y^{2})$.

\noindent $(ii)$\; $(x+y)^{7}-(x^{7}+y^{7})=7xy(x+y)(x^{2}+xy+y^{2})^{2}$.
}

\solcpreliminares{Desarrolle ambos lados de las identidades.
}

%%%%%% 47

\ejercpreliminares{Sean $x$, $y$ y $z$ n\'umeros reales tales que $x\neq y$ y
$$x^2(y+z)=y^2(x+z)=2.$$
Determine el valor de $z^2(x+y)$.
}

\solcpreliminares{Se tiene que
\begin{eqnarray*}
0 & = & x^2(y+z)-y^2(x+z)= xy(x-y)+(x^2-y^2)z\\
  & = & (x-y)(xy+xz+yz).
\end{eqnarray*}
Como $x\neq y$, se tiene que $xy+xz+yz=0$. Multiplicando por $x-z$ se obtiene
\begin{eqnarray*}
0 &=& (x-z)(xy+xz+yz)= xz(x-z)+(x^2-z^2)y\\
&=& x^2(y+z)-z^2(x+y),
\end{eqnarray*}
de donde $z^2(x+y)=x^2(y+z)=2$.
}


%%%%%%%%%%1.48
\ejercpreliminares{Encuentre las soluciones reales $x$, $y$, $z$ y $w$
del sistema de ecuaciones
\begin{eqnarray*}
x  + y +z  & = & w \\
\frac{1}{x}+ \frac{1}{y}+\frac{1}{z} & = &\frac{1}{w}.
\end{eqnarray*}
}

\solcpreliminares{Vea que $(x+y+z)(xy+yz+zx)=xyz$ y por la ecuaci'on (\ref{productodetres}) tenemos que
$(x+y)(y+z)(z+x)=0.$ Luego, las soluciones $(x,y,z,w)$ son de la forma: $(x,-x,z,z)$,
 $(x,y,-y,x)$ y $(x,y,-x,y)$, con $x$, $y$ y $z$ n'umeros reales diferentes de cero.
}

%%%%%%%%%% 49

\ejercpreliminares{Sean $x$, $y$ y $z$ n\'{u}meros reales diferentes
de cero que cumplen las condiciones $x+y+z\neq 0$ y 
$\frac{1}{x}+\frac{1}{y}+\frac{1}{z}=\frac{1}{x+y+z}$. Muestre que, para 
cualquier n'umero entero impar $n$, se cumple que
\begin{equation*}
\dfrac{1}{x^{n}}+\dfrac{1}{y^{n}}+\dfrac{1}{z^{n}}=\dfrac{1}{x^{n}+y^{n}+z^{n}}.
\end{equation*}
}

\solcpreliminares{Por la ecuaci'on 
(\ref{productodetres}), se tiene que la condici'on es equivalente a  $(x+y)(y+z)(z+x)=0$.
Luego, un factor es cero, digamos $x+y=0$. Entonces, como $n$ es impar,
$x^{n}+y^{n}=0$, lo mismo que $\frac{1}{x^{n}}+\frac{1}{y^{n}}=0$.
}






% \input{progresiones.tex}
% \input{induccion.tex}
% \input{polinomioscuadycub.tex}
% \input{complejos.tex}
% \input{funciones.tex}
% \input{sucesiones.tex}
% \input{polinomios.tex}
% \input{problemas.tex}
 \chapter{Soluciones a los ejercicios y problemas}
       
\noindent Las primeras ocho secciones de este cap'itulo contienen todas las 
soluciones de  los ejercicios que aparecen en los primeros ocho cap'itulos.  
En la secci'on 9  se encuentran las soluciones de los problemas del cap'itulo 9.  La dificultad de los ejercicios que est'an en la teor'ia es menor que la de los problemas del  cap'itulo 9.  Sin embargo, resolver los problemas del 'ultimo cap'itulo ser'a un excelente entrenamiento para prepararse para los ex'amenes internacionales en los cuales M'exico participa.
 
Le recomendamos al lector que no consulte este cap'itulo sin antes haber
intentado resolver los ejercicios y problemas 'el mismo.


\renewcommand{\sectionmark}[1]{\markright{\thesection\, Soluciones del cap\'itulo \ref{preliminares}}}
\section{Soluciones a los ejercicios del cap\'itulo \ref{preliminares}}
\label{sugcapuno}
\setcounter{sucprel}{0}
\setcounter{surefcprel}{0}
\def\xxeleccion{11}
%%%1
\ejercpreliminares{Muestre las siguientes afirmaciones:
\label{maspormas}
\begin{description}
\item[$ (i)$] Si $a<0$, $b<0$, entonces $ab>0$.

\ven

\item[$ (ii)$] Si $a<0$, $b>0$, entonces $ab<0$.

\ven

\item[$ (iii)$] Si $a<b$, $b<c$, entonces $a<c$.

\ven

\item[$ (iv)$] Si $a<b$, $c<d$, entonces $a+c<b+d$.

\ven


\item[$ (v)$] Si $a>0$, entonces $a^{-1}>0$.

\ven

\item[$ (vi)$]  Si $a<0$, entonces $a^{-1} <0$.
\end{description}
}

\solcpreliminares{$(i)$ Si $a<0$, entonces $-a>0$. Use
tambi\'en que $(-a)(-b)=ab$. 
$(ii)$ $(-a)b>0$. 
$(iii)$ $a<b\Leftrightarrow b-a>0$, use ahora la propiedad \ref{propsuma}. 
$(iv)$ Use la propiedad \ref{propsuma}. 
$(v)$ $aa^{-1}=1>0$. 
$(vi)$ Si $a<0$, entonces $-a>0$.
}


%%%%%%% 2
\ejercpreliminares{Sean $a$, $b$ n\'{u}meros reales.  Muestre que, si 
$a+b$, $a^2+b$ y $a+b^2$ son n'umeros racionales y $a+b\neq 1$, entonces 
$a$ y $b$ son n'umeros racionales.
}

\solcpreliminares{Observe que si $a^2+b-(a+b^2)\in \qq$, entonces
$(a-b)(a+b-1)\in \qq$ y como $a+b-1\in \qq\setminus\{0\}$, entonces 
$(a-b)\in \qq$. Luego, si $a+b\in\qq$ y $a-b\in \qq$, entonces $2a$ y 
$2b$ est'an en $\qq$. Por lo tanto, $a$ y $b$ son n'umeros racionales.
}

%%%%%%%%% 3
\ejercpreliminares{Sean $a, b$ n\'{u}meros reales tales que 
$a^2+b^2$, $a^3+b^3$ y $a^4+b^4$ son  n'umeros racionales.  Muestre que 
$a+b$, $ab$ son tambi'en  n'umeros racionales.

}

\solcpreliminares{Si $a=0$ o $ b=0$ el resultado es claro. 
Suponga entonces que $ab\neq 0$. Como $(a^2+b^2)^2-(a^4+b^4)=2a^2b^2$, 
se tiene que $a^2b^2\in\qq$.  Note que $a^6+b^6=(a^2+b^2)^3-
3a^2b^2(a^2+b^2)\in\qq$, por lo que $(a^3+b^3)^2-(a^6+b^6)=2a^3b^3\in\qq$. 
Luego, 
$$
  ab= \frac{a^3b^3}{a^2b^2}\in \qq\quad \text{y}\quad 
 a+b= \frac{a^3+b^3}{a^2+b^2-ab}\in\qq.
$$
}

%%%%%%%%% 4
\ejercpreliminares{$(i)$\, Demuestre que si $p$ es un n'umero primo, 
entonces $\sqrt{p}$ es un n'umero irracional.

$(ii)$\, Demuestre que si $m$ es un n'umero entero positivo que no 
es cuadrado perfecto, entonces $\sqrt{m}$ es un n'umero irracional.
}

\solcpreliminares{$(i)$ Suponga que $\sqrt{p}$ no es un n'umero irracional, 
es decir, \linebreak $\sqrt{p}=\frac{m}{n}$, donde $m$, $n$ son n'umeros enteros 
con $(m,n)=1$, es decir, $m$ y $n$ primos relativos. Elevando al cuadrado, 
se tiene $p n^2=m^2$, esto es, $p$ divide a $m^2$, entonces $p$ divide a $m$.  
Por lo que  $m=ps$ y  $pn^2 = p^2 s^2$ implican que   $n^2 = p s^2$, lo cual 
garantiza que $p$ divide a $n^2$ y entonces divide a  $n$. Luego, $p$ divide 
a $m$ y a $n$ contradiciendo el hecho de que  $m$ y $n$ son primos relativos. 

$(ii)$ Suponga que $\sqrt{m}$ no es un n'umero irracional, 
es decir, $\sqrt{m}=\frac{r}{s}$, donde $r$, $s$ son n'umeros enteros 
con $(r,s)=1$. Elevando al cuadrado 
se tiene $m s^2=r^2$. Como $m$ no es un cuadrado perfecto, tiene un factor 
de la forma $p^{\alpha}$, donde $p$ es un n'umero primo y $\alpha$ es un 
entero positivo impar. Entonces, $p^{\alpha}$ divide a $r^2$ lo que implica que 
el primo $p$ aparece un n'umero par de veces en la descomposici'on de factores 
de $r^2$. Como $r$ y $s$ son primos relativos, $p$ no divide a $s$, de donde
$p$ aparece un n'umero impar de veces como factor de $m s^2$, lo cual es una 
contradicci'on.
}

%%%%5
\ejercpreliminares{Demuestre que existen una infinidad de parejas de n'umeros 
irracionales $a$, $b$ tales que $a+b=ab$ y adem'as este n'umero es entero.
}

\solcpreliminares{Si $a+b=ab=n$, entonces $b=n-a$ y $n=a(n-a)$.  
La 'ultima ecuaci'on es equivalente a $a^2-na+n=0$ y resolviendo se obtiene que
$$
 a=\frac{n\pm\sqrt{n^2-4n}}{2},
\quad\text{de donde }\quad b=\frac{n\mp\sqrt{n^2-4n}}{2}.
$$
Para $n\geq 5$, se tiene que 
$ (n-3)^2 <n^2-4n <(n-2)^2,$
por lo que $\sqrt{n^2-4n}$ es un  n'umero irracional, y entonces $a$ y 
$b$ son n'umeros irracionales.
}

%%%%% 6
\ejercpreliminares{Si los coeficientes de 
$$
   a x^2+b x+c =0
$$
son n'umeros enteros impares, entonces las ra'ices de la ecuaci'on 
no pueden ser n'umeros racionales.
}

\solcpreliminares{Suponga que $\frac{m}{n}$ es  ra'iz,  con $(m,n)=1$. 
Entonces $m$ y $n$ no pueden ser  ambos pares.  Por otro lado, como 
$a \left (\frac{m}{n}\right )^2+b \left (\frac{m}{n}\right )+c =0$,
se tiene que $ a m^2+b mn+c n^2 =0$. El lado derecho de la  'ultima 
ecuaci'on  es par y el izquierdo siempre es impar. Si  $m$ y $n$ son 
impares, los tres sumandos del lado izquierdo son impares. Ahora bien,  
si uno de ellos es par y el otro impar, entonces dos sumandos son pares, 
el tercero impar y la suma es impar nuevamente. Esta contradicci'on 
implica que la ecuaci'on no puede tener ra'ices racionales. 

\vei 

\ssolucion{El discriminante $b^2 - 4 ac$ deber'a ser un cuadrado 
perfecto. Pero como $a$, $b$ y $c$ son impares, se puede mostrar que
$b^2 - 4 ac\equiv 5$ $\mod 8$. Sin embargo, los cuadrados de 
n'umeros impares s'olo dejan residuo 1 m'odulo 8.}
}

%%%%%%%%%% 7
\ejercpreliminares{Muestre que para n'umeros reales positivos 
$a$ y $b$, con $\sqrt{b} < a$, se tiene que
$$
\sqrt{a+\sqrt{b}}=\sqrt{\frac{a+\sqrt{a^{2}-b}}{2}}+
\sqrt{\frac{a-\sqrt{a^{2}-b}}{2}}.
$$
}

\solcpreliminares{Sea $u=a+\sqrt{b}$ y $v=a-\sqrt{b}$, entonces
\begin{eqnarray*}
 \sqrt{a+\sqrt{b}}& = & \sqrt{u}=\frac{\sqrt{u}+\sqrt{v}}{2} + 
\frac{\sqrt{u}-\sqrt{v}}{2}\\
& = & \sqrt {\frac{(\sqrt{u}+\sqrt{v})^2}{4}} + 
\sqrt {\frac{(\sqrt{u}-\sqrt{v})^2}{4}}\\
& = & \sqrt {\frac{\frac{u+v}{2}+\sqrt{uv}}{2}} 
+\sqrt {\frac{\frac{u+v}{2}-\sqrt{uv}}{2}}\\
& = & \sqrt {\frac{\frac{a+\sqrt{b}+a-\sqrt{b}}{2}+
\sqrt{a^2-b}}{2}} +\sqrt {\frac{\frac{a+\sqrt{b}+a-
\sqrt{b}}{2}-\sqrt{a^2-b}}{2}}\\
& = & \sqrt {\frac{a+\sqrt{a^2-b}}{2}} 
+\sqrt {\frac{a-\sqrt{a^2-b}}{2}},
\end{eqnarray*}
como se quer'ia probar. 
}

%%%%%%% 8
\ejercpreliminares{Para n\'{u}meros positivos $a$ y $b$ 
encuentre el valor de:

\vei

\noindent $(i)$ $\sqrt{a\sqrt{a\sqrt{a\sqrt{a \dots}}}}.$ 
\qquad \qquad \qquad 
$(ii)$ $\sqrt{a\sqrt{b\sqrt{a\sqrt{b\dots}}}}.$
}

\solcpreliminares{$(i)$ Sea $x=\sqrt{a\sqrt{a\sqrt{a\sqrt{a \dots}}}}$, 
entonces $x^2=a\sqrt{a\sqrt{a\sqrt{a\sqrt{a \dots}}}}$, de donde $x^2=ax$. 
Factorizando, $x(x-a)=0$.  Por lo tanto, como $a$ es positivo la soluci'on 
es $x=a$.

\vei

\ssolucion{Podemos dar otra soluci'on utilizando series.  Tenemos que 
$$
     x=a^{\frac{1}{2}}a^{\frac{1}{4}}a^{\frac{1}{8}}\ldots=a^{\frac{1}{2}+\frac{1}{4}+\frac{1}{8}
+\cdots}= a,
$$
ya que $\sum_{j=1}^{\infty} \frac{1}{2^j}=1$, ver la secci'on 
\ref{seriesdepotencia}.
}

$(ii)$ Sea $x=\sqrt{a\sqrt{b\sqrt{a\sqrt{b\dots}}}}$, entonces 
$x^2=a\sqrt{b\sqrt{a\sqrt{b\sqrt{a\dots}}}}$, de donde 
$x^4=a^2 b x$. Como  $x\neq 0$, $x^3=a^2 b$. Entonces $x=\sqrt[3]{a^2b}$.

\vei

\ssolucion{Podemos tambi'en hacer otra soluci'on utilizando series.  
Tenemos que 
$$
     x=a^{\frac{1}{2}+\frac{1}{8}+\frac{1}{32}+\cdots}\, 
b^{\frac{1}{4}+\frac{1}{16}+\frac{1}{64}+\cdots}=a^{\frac{2}{3}}\,b^{\frac{1}{3}},
$$
ya que $\sum_{j=1}^{\infty} \frac{1}{2^{2j}}=\frac{1}{3}$ y 
$\sum_{j=0}^{\infty} \frac{1}{2^{2j+1}}=\frac{2}{3}$, ver la secci'on 
\ref{seriesdepotencia}. 
}
}

%%%%% 9
\ejercpreliminares{(Rumania, 2001) Sean  $x$, $y$ y $z$ n'umeros 
reales distintos de cero tales que
$xy$, $yz$ y $zx$ son n'umeros racionales. Muestre que:

$(i)$ $x^2+y^2+z^2$ es un n'umero racional.

$(ii)$ Si $x^3+y^3+z^3$ es un n'umero racional distinto de cero, 
entonces 
$x$, $y$ y $z$ son n'umeros racionales.
}

\solcpreliminares{$(i)$ Si $xy$, $yz$  y $zx$ est'an en $\qq$, 
entonces $\frac{(xy)(zx)}{yz}=x^2\in\qq$.  An'alogamente, 
$y^2$, $z^2$ $\in \qq$. Por lo tanto, $x^2+y^2+z^2\in\qq$.

$(ii)$ Por  $(i)$ se tiene que  $(x^2)^2+(xy)y^2+(xz) 
z^2=x(x^3+y^3+z^3)\in \qq$, luego  $x\in \qq$.  
An'alogamente, $y$, $z$ $\in \qq$. 
}

%%%%%%% 10
\ejercpreliminares{(Rumania, 2011) Sean $a$, $b$ n\'{u}meros reales 
positivos y distintos, tales que $a-\sqrt{ab}$ y $b-\sqrt{ab}$ 
son ambos n'umeros racionales. Muestre que $a$ y $b$ son  
n'umeros racionales.
}

\solcpreliminares{Como $a-\sqrt{ab}=a\left( 1-\frac{\sqrt{b}}{\sqrt{a}}\right)$,
bastar'a ver que  $1-\frac{\sqrt{b}}{\sqrt{a}}$ es un  n'umero 
racional distinto de cero para asegurar que $a$ es un  n'umero racional. 

\noindent Pero $ \frac{b-\sqrt{ab}}{a-\sqrt{ab}}=
\frac{\sqrt{b}(\sqrt{b}-\sqrt{a})}{\sqrt{a}(\sqrt{a}-\sqrt{b})}= 
-\frac{\sqrt{b}}{\sqrt{a}}$ 
es un  n'umero  racional diferente de $-1$ (ya que $a\neq b$), 
luego $1-\frac{\sqrt{b}}{\sqrt{a}}$ es un  n'umero racional 
distinto de 0. An'alogamente, $b$ es un  n'umero racional.
}










%%%%%%%%%%11

\ejercpreliminares{Escriba en la forma $\frac{m}{n}$, con $n$ y $m$ 
n'umeros enteros positivos, a los siguiente n'umeros reales:

$(i)$  $0.11111\dots$.  

$(ii)$ $1.14141414\dots$.
}

\solcpreliminares{Para resolver $(i)$, defina $x=0.111\dots$, 
entonces $10x=1.11\dots$. Restando la primera ecuaci'on de la 
segunda se tiene que $9x=1$, luego,  $x=\frac{1}{9}$.

\noindent $(ii)$ Sea $x=1.141414\ldots$, entonces 
$100x=114.141414\ldots$. Restando la pri\-mera ecuaci'on de la 
segunda se tiene que $99x=113$, de donde $x=\frac{113}{99}$. }

\ejercpreliminares{$(i)$\; Muestre que $121_b$ es un cuadrado 
perfecto en cualquier base $b\geq 2$.

$(ii)$\; Determine el menor valor de $b$ para el cual  $232_b$ 
es un cuadrado perfecto.
}

\solcpreliminares{$(i)$\; Primero observe que 
$121_b=(1\times b^2)+ (2\times b) +1 =(b+1)^2$ entonces 
$121_b$ es un cuadrado perfecto en cualquier base $b\geq 2$.

$(ii)$\; Como $232_b=2b^2+3b+2$ debe ser cuadrado y como 3 es 
uno de sus d'igitos, $b\geq 4$.

Para $b=4$, $232_4=46$, para $b=5$, $232_5=67$, para $b=6$, 
$232_6=92$ y para $b=7$, $232_7=121$. Luego, $b=7$ es el menor 
entero positivo tal que $232_b$ es un cuadrado perfecto. 
}

%%%%%% 12
%\ejercpreliminares{Sea $b \geq 2$ un entero positivo.
%(a) Muestre que para que un entero $N$, escrito en base $b$, sea igual a la suma del cuadrado %de sus d'igitos, es necesario que  $N = 1$ o que 
%$N$ tenga solamente dos d'igitos.
%(b) Give a complete list of all integers not exceeding 50 that, relative to
%some base $b$, are equal to the sum of the squares of their digits.
%(c) Show that for any base $b$ the number of two-digit integers that are
%equal to the sum of the squares of their digits is even.
%(d) Show that for any odd base $b$ there is an integer other than 1 that is
%equal to the sum of the squares of its digits.
%}

%%%%%%%% 13
\ejercpreliminares{(IMO, 1970) Sean $a$, $b$ y $n$ n'umeros enteros mayores 
que 1. Sean $A_{n-1}$ y $A_n$ dos n'umeros escritos en el sistema num'erico en 
base  $a$ y, $B_{n-1}$ y $B_n$ dos n'umeros escritos en el sistema n'umerico 
en base  $b$. Estos n'umeros 
est'an relacionados de la siguiente forma,
\begin{eqnarray*}
        A_n = x_nx_{n-1} \dots x_0, & & A_{n-1} = x_{n-1}x_{n-2}\dots x_0,\\
        B_n = x_nx_{n-1} \dots x_0, & & B_{n-1} = x_{n-1}x_{n-2}\dots x_0,
\end{eqnarray*}
con 
$x_n\neq 0$ y $x_{n-1}\neq 0$.  Muestre que  $a > b$ si y s'olo si
$$
                  \frac{A_{n-1}}{A_{n}} <\frac{B_{n-1}}{B_{n}}.
$$
}

\solcpreliminares{Suponga que $a>b$. Entonces para todos los 
enteros $0\leq k\leq n$, $x_nx_ka^nb^k\geq x_nx_kb^na^k$, con 
igualdad solamente cuando $k=n$ o $x_k=0$. En particular, se 
tiene una desigualdad estricta para $k=n-1$. En resumen, esto 
se convierte en
$$
              x_n a^n\sum_{k=0}^n x_kb^k > x_nb^n \sum_{k=0}^n x_k a^k
$$
o 
$$
             \frac{ x_n a^n}{A_n}> \frac{x_nb^n}{B_n}.
$$
Esto implica que
$$
             \frac{ A_{n-1}}{A_n}= 1-\frac{ x_n a^n}{A_n} < 
1-\frac{ x_n b^n}{B_n} =\frac{B_{n-1}}{B_n}.
$$ 
Por otro lado, si $a=b$, entonces evidentemente 
$\frac{ A_{n-1}}{A_n}= \frac{ B_{n-1}}{B_n}$ y si 
$a<b$, por lo que se demostr'o antes, se tiene que, 
$ \frac{ A_{n-1}}{A_n}> \frac{ B_{n-1}}{B_n}$. Por lo 
tanto, $\frac{A_{n-1}}{A_{n}} <\frac{B_{n-1}}{B_{n}}$ si y 
s'olo si $a>b$.
}




%%% 13
\ejercpreliminares{Si $a$ y $b$ son n'umeros reales 
cualesquiera, demuestre que
$$
             ||a|-|b||\leq |a-b|.
$$
}

\solcpreliminares{Observe que $|a|=|a-b+b|\leq |a-b|+|b|$, 
despejando se tiene que $|a|-|b|\leq |a-b|.$  An'alogamente, 
siguiendo los mismos pasos, se tiene que $|b| -|a|\leq |b-a|$. 
De estas dos desigualdades se sigue que $||a|-|b||\leq |a-b|$.
}



%%% 14
\ejercpreliminares{En cada caso encuentre los n'umeros reales 
$x$ que satisfacen:

$(i)$\;$|x-1|- |x+1|=0$.

$(ii)$\; $|x-1||x+1|=1$.

$(iii)$\; $|x-1|+ |x+1|=2$.
}

\solcpreliminares{$(i)$\;  $|x-1|- |x+1|=0$ es equivalente a 
$|x-1|=|x+1|$. Elevando al cuadrado y resolviendo la ecuaci'on 
$(x-1)^2 = (x+1)^2$ tenemos que $4x =0$,
luego, la 'unica soluci'on es $x=0$.

$(ii)$\; $|x-1||x+1|=1$  es equivalente a $|x^2-1|=1$, de donde
%$$
%\begin{array}{lcccl}
%          x^2 -1 =1 & &\text{o} & & -(x^2 -1) =1\\
%         x^2  =2 & &\text{o} & & x^2  =0\\
%         x = \pm\sqrt{2} & &\text{o} & & x  =0,
%\end{array}
%$$
las soluciones son $x = \pm\sqrt{2}$  y $ x  =0$.

$(iii)$\;  Si $x>1$ se cumple que  $|x+1|=x+1 >2$, luego no hay soluci'on.

Si $x<-1$ se cumple que  $|x-1|=-x+1 >2$ y tampoco hay soluci'on.

Si $-1\leq x \leq 1$,  entonces $x-1 \leq 0\leq x+1$,   luego 
$$
      |x-1|+|x+1 |=(1-x)+(x+1)=2.
$$
Por lo que los 'unicos valores de  $x$ que cumplen  la igualdad son 
 $-1\leq x\leq 1$.
}

%%% 15
\ejercpreliminares{Encuentre las ternas $(x,y,z)$  de n'umeros 
reales que satisfacen 
\begin{eqnarray*}
              |x+y| &\geq & 1\\
             2xy -z^2 & \geq & 1\\
            z-|x+y|  & \geq & -1.
\end{eqnarray*}
}

\solcpreliminares{De la primera y  tercera desigualdades 
se tiene que \linebreak $z \geq |x+y| -1\geq 0$. Por lo que, $z^2\geq 
(|x+y|-1)^2$. Ahora,   $2xy \geq z^2+1\geq (|x+y|-1)^2 + 1\geq 0$, 
entonces
$$
  2xy \geq  x^2+2xy+y^2-2|x+y|+2 \geq  |x|^2+2xy+|y|^2 - 2|x|- 2 |y| +2,
$$
cancelando $0\geq  |x|^2+|y|^2 - 2|x|- 2 |y|+2 = (|x|-1)^2 +(|y|-1)^2.$
Por lo que $|x|=1$ y $|y|=1$.
Luego,  $x$ y $y$ tienen que ser   $-1$ o 1.  Pero como $xy\geq 0$,  
los dos tienen que tener el mismo signo. Para  $x=y=1$ o $x=y=-1$ se 
tiene, sustituyendo en las ecuaciones originales,   
que $2-z^2\geq 1$ y $z-2\geq -1$. Luego, $z^2\leq 1$ y $z\geq 1$. El 
'unico valor de $z$ que satisface las dos desigualdades es $z=1$. 
Por lo tanto, hay dos soluciones al problema $x=y=z=1$ y $x=y=-1$, $z=1$. 
}


%%%%%16
\ejercpreliminares{(OMM, 2004) ?`Cu\'al es la mayor cantidad de 
n'umeros enteros positivos que se pueden encontrar de manera que 
cualesquiera dos de ellos, $a$ y $b$ (con $a\neq b$), cumplan que:
$$|a-b|\geq \frac{ab}{100}?$$
}

\solcpreliminares{Suponga que
$a_{1}<a_{2}< \dots<a_{n}$ es una colecci\'{o}n con la mayor cantidad de
n'umeros enteros con la propiedad.  Es claro que $a_{i}\geq i$, para 
toda $i=1, \ldots, n$.

\noindent Si $a$ y $b$ son dos n'umeros enteros de la colecci\'{o}n 
con $a>b$, como $%
\left\vert a-b\right\vert =a-b\geq \frac{ab}{100}$, se tiene que 
$a \left(1-\frac{b}{100} \right) \geq b$, por lo que si $100-b>0$, 
entonces $a\geq \frac{100b}{100-b}$.

\noindent Note que no existen dos n'umeros enteros $a$ y $b$ en la 
colecci\'{o}n
mayores que $100$, en efecto si $a>b>100$, entonces $a-b=\left\vert
a-b\right\vert \geq \frac{ab}{100}>a$, lo cual es falso.

\noindent Tambi\'{e}n se tiene que para n'umeros enteros $a$ y $b$ 
menores que $100$,
se cumple que $\frac{100a}{100-a}\geq \frac{100b}{100-b}$ si y s'olo si 
$100a-ab\geq 100b-ab$ si y s'olo si $a\geq b$.

\vei 

\noindent Es claro que $\left\{ 1,2,3,4,5,6,7,8,9,10\right\} $ es una 
colecci\'{o}n con la propiedad.

\noindent Ahora, $a_{11}\geq \frac{100a_{10}}{100-a_{10}}\geq \frac{100\cdot
10}{100-10}=\frac{100}{9}>11$, lo que implica que $a_{11}\geq 12$.

\ve

$a_{12}\geq \frac{100a_{11}}{100-a_{11}}\geq \frac{100\cdot 12}{100-12}=%
\frac{1200}{88}>13$, de donde $a_{12}\geq 14$.

\ve

$a_{13}\geq \frac{100a_{12}}{100-a_{12}}\geq \frac{100\cdot 14}{100-14}=%
\frac{1400}{86}>16$, de donde $a_{13}\geq 17$.

\ve

$a_{14}\geq \frac{100a_{13}}{100-a_{13}}\geq \frac{100\cdot 17}{100-17}=%
\frac{1700}{83}>20$, de donde $a_{14}\geq 21$.

\ve

$a_{15}\geq \frac{100a_{14}}{100-a_{14}}\geq \frac{100\cdot 21}{100-21}=%
\frac{2100}{79}>26$, de donde $a_{15}\geq 27$.

\ve

$a_{16}\geq \frac{100a_{15}}{100-a_{15}}\geq \frac{100\cdot 27}{100-27}=%
\frac{2700}{73}>36$, de donde  $a_{16}\geq 37$.

\ve

$a_{17}\geq \frac{100a_{16}}{100-a_{16}}\geq \frac{100\cdot 37}{100-37}=%
\frac{3700}{63}>58$, de donde  $a_{17}\geq 59$.

\ve

$a_{18}\geq \frac{100a_{17}}{100-a_{17}}\geq \frac{100\cdot 59}{100-59}=%
\frac{5900}{41}>143$, de donde  $a_{18}\geq 144$.

\ve

\noindent Adem'as, como ya se ha observado que no hay dos  
n'umeros enteros de la colecci\'{o}n mayores que $100$, 
la mayor cantidad es $18$.
La colecci\'{o}n de $18$ n'umeros enteros siguiente 
$\left\{ 1,2,3,4,5,6,7,8,9,10,12,14,17,21,27,37,59,144\right\}$ 
cumple la condici\'{o}n.
}











%%% 18
\ejercpreliminares{Para cualesquiera n'umeros reales $a$, $b>0$, 
se tiene que
$$
 \lfloor 2a \rfloor + \lfloor 2b \rfloor \geq \lfloor a \rfloor 
+\lfloor b \rfloor +\lfloor a+b\rfloor.
$$
}

\solcpreliminares{Por el ejemplo \ref{exismasunmediomenosdosexis}, 
$\lfloor 2a\rfloor = \lfloor a\rfloor + \lfloor a+\frac{1}{2}\rfloor$ 
y $\lfloor 2b\rfloor = \lfloor b\rfloor + \lfloor b+\frac{1}{2}\rfloor$, 
luego la desigualdad a demostrar es equivalente a
$$
  \lfloor a\rfloor + \left\lfloor a+\frac{1}{2}\right\rfloor+ 
\lfloor b\rfloor + \left\lfloor b+\frac{1}{2}\right\rfloor \geq 
\lfloor a\rfloor + \lfloor b\rfloor+ \lfloor a+b\rfloor,
$$
de donde bastar'a mostrar que $ \left\lfloor a+\frac{1}{2}\right\rfloor+
\left\lfloor b+\frac{1}{2}\right\rfloor \geq  \lfloor a+b\rfloor$.

Sean $a = n+y$, $b = m+x$, con $n, m\in \zz$ y $0\leq x, y < 1$. Entonces 
$0\leq x+ y <2$ y $a+b=n+m+x+y$.  Se tienen dos casos:

$(i)$ Si $1\leq x + y < 2$, entonces  $\lfloor a+ b\rfloor= n+m+1$ 
y al menos uno de los n'umeros $x$ o $y$ es mayor o igual que 
$\frac{1}{2}$.  Suponga que $x\geq \frac{1}{2}$. Entonces  
$\lfloor b+\frac{1}{2}\rfloor= \lfloor m+x+\frac{1}{2}\rfloor = m+1$, 
por lo que $ \lfloor a+\frac{1}{2}\rfloor+\lfloor b+
\frac{1}{2}\rfloor\geq m+n+1=\lfloor a+b\rfloor$.  


$(ii)$ Si $0\leq x + y < 1$, entonces  $\lfloor a+ b\rfloor= n+m$ y  
$ \lfloor a+\frac{1}{2}\rfloor+\lfloor b+\frac{1}{2}\rfloor\geq m+n 
=\lfloor a+b\rfloor$.
}


%%%%%%% 19
\ejercpreliminares{Encuentre  los valores de $x$ que cumplen 
la siguiente ecuaci'on:

$(i)$\,   $\lfloor x \lfloor x \rfloor  \rfloor =1$.

$(ii)$\,   $||x|-\lfloor x \rfloor| = \lfloor |x|- \lfloor x\rfloor\rfloor$.
}

\solcpreliminares{$(i)$\,   Se tiene que $\lfloor x \lfloor x \rfloor  
\rfloor =1$ si y s'olo si
$1\leq x\lfloor x \rfloor < 2$.  Si $x=m+y$, con $m\in\zz$ y $0\leq y <1$, 
entonces $1\leq m^2+my < 2$. Observe que   $m=0$ es imposible, al igual que 
$m\geq 2$ o $m\leq -2$.  Luego, resta ver qu'e sucede si $m=1$ o $m=-1$. 

Si $m=1$, entonces $1\leq 1+y<2$, de donde $0\leq y <1$ y entonces cualquier 
$x$ en el intervalo $[1,2)$ cumple la ecuaci'on.  Si $m=-1$, entonces, como
\linebreak 
$1\leq m^2+my < 2$, se  tiene que  $1\leq 1- y<2$, de donde $0\leq - y < 1$ 
y entonces  $y=0$ y $x=-1$.  
Por lo tanto, los n'umeros que cumplen la ecuaci'on son $x=-1$ y  
$x\in [1,2)$.

\vei 

$(ii)$\,   Como $\lfloor x \rfloor \leq x \leq |x|$, se tiene que,
$ |x|- \lfloor x \rfloor \geq 0$, por lo que\linebreak  $||x|-\lfloor x\rfloor |= 
|x| -\lfloor x\rfloor$.  Por otro lado,  por la propiedad $(c)$ en 
\ref{parteentera} se\linebreak  tiene que, $\lfloor |x|-\lfloor x\rfloor\rfloor 
= \lfloor |x|\rfloor -\lfloor x\rfloor$. Utilizando las 'ultimas igualdades 
la ecuaci'on se convierte en $|x| - \lfloor x\rfloor = \lfloor |x|\rfloor - 
\lfloor x\rfloor$ que es equivalente a $|x| = \lfloor |x| \rfloor$, luego 
$|x|$ es un n'umero entero y los valores de $x$ que cumplen la ecuaci'on 
son todos los n'umeros enteros.
}



%%% 20
\ejercpreliminares{Encuentre las soluciones del sistema de ecuaciones
\begin{eqnarray*}
x+\lfloor y \rfloor + \{z\} & = & 1.1,\\
\lfloor x \rfloor + \{y\}+z & = & 2.2,\\
\{x\} + y + \lfloor z \rfloor & = & 3.3.
\end{eqnarray*}
}

\solcpreliminares{Sume las tres ecuaciones para obtener que 
$2x+2y+2z=6.6$, luego $x+y+z=3.3$. Reste a esta 'ultima igualdad las ecuaciones
originales, para obtener $\{y\} + \lfloor z \rfloor =2.2$, 
$\{x\} + \lfloor y \rfloor =1.1$,
$\{z\}+\lfloor x \rfloor =0$.
La primera ecuaci'on da $\lfloor z \rfloor = 2$, $\{y\}=0.2$, la segunda 
$\lfloor y \rfloor=1$, $\{x\}=0.1$ y, la tercera $\lfloor x \rfloor=0$ y
$\{z\}=0$. Por lo tanto, la soluci'on es $x=0.1$, $y=1.2$ y $z=2$.
}


%%%%%%% 21
\ejercpreliminares{(Canad'a, 1987) Para cada n'umero natural $n$, 
muestre que
$$
         \lfloor \sqrt{n} +\sqrt{n+1} \rfloor = \lfloor \sqrt{4n+1} \rfloor=\lfloor \sqrt{4n+2} \rfloor=\lfloor \sqrt{4n+3} \rfloor. 
$$
}

\solcpreliminares{Se tiene que  $\sqrt{n} +\sqrt{n+1} < \sqrt{4n+2} $ si y 
s'olo si $2n+1+\sqrt{4n^2+4n} < 4n+2$, que es equivalente a $\sqrt{4n^2+4n} < 
2n+1$. Elevando al cuadrado nuevamente, la 'ultima desigualdad es equivalente 
a $4n^2+4n < 4n^2+4n+1$.  Esto prueba que $\sqrt{n} +\sqrt{n+1} < \sqrt{4n+2}$,
entonces $\lfloor \sqrt{n} +\sqrt{n+1}\rfloor 
\leq \lfloor \sqrt{4n+2}\rfloor$.  

\vei

Suponga que, para alg'un n'umero entero positivo $n$, 
$\lfloor \sqrt{n} +\sqrt{n+1}\rfloor \neq \lfloor \sqrt{4n+2}\rfloor$. Sea 
$q= \lfloor \sqrt{4n+2}\rfloor$, entonces $\sqrt{n} +\sqrt{n+1}< q \leq 
\sqrt{4n+2}$.  Elevando al cuadrado, se  obtiene que $2n+1+ \sqrt{4n^2+4n} < 
q^2\leq 4n+2$ o lo que es equivalente $\sqrt{4n^2+4n} < q^2-2n - 1 \leq 2n+1$.
Elevando al cuadrado nuevamente se\linebreak obtiene que  
$4n^2+4n < (q^2-2n - 1)^2 
\leq 4n^2+4n+1= (2n+1)^2$. Como no \linebreak existe un cuadrado entre dos 
enteros 
consecutivos, se tiene que $q^2-2n - 1 = 2n+1$ o que
$q^2 = 4n+2$, que es equivalente a decir que $q^2\equiv 2 \mod 4$.  Pero esto 
'ultimo es una contradicci'on, ya que todo cuadrado es congruente a 0 o a 1 
m'odulo $4$. Por lo tanto,  se tiene la igualdad.

\vei

Muestre ahora que, $\lfloor \sqrt{4n+1} \rfloor=\lfloor \sqrt{4n+2} 
\rfloor=\lfloor \sqrt{4n+3} \rfloor$. 

Para la  primera igualdad, suponga que 
existe una $n$ tal que $m=\lfloor \sqrt{4n+1} \rfloor < m+1=
\lfloor \sqrt{4n+2} \rfloor$, luego $m \leq \sqrt{4n+1} < m+1 \leq 
\sqrt{4n+2}$, por lo que 
$m^2 \leq 4n+1 < (m+1)^2\leq 4n+2$. 

Entonces, como $4n+1$ y 
$4n+2$ son dos n'umeros enteros consecutivos y, como $(m+1)^2 > 4n+1$, se 
tiene que $(m+1)^2 = 4n+2$ y nuevamente se ha encontrado un cuadrado que 
tiene residuo $2$ al dividirlo entre $4$, lo cual es imposible. Para la 
segunda igualdad, proceda de la misma forma. 
}




%%%%  22
\ejercpreliminares{Para todos los n'umeros reales $x$, $y$, 
se tienen las siguientes identidades de segundo grado:

\noindent $(i)$ $x^{2}+y^{2}=(x+y)^{2}-2xy=(x-y)^{2}+2xy$.

\noindent $(ii)$ $(x+y)^{2}+(x-y)^{2}=2(x^{2}+y^{2})$.

\noindent $(iii)$ $(x+y)^{2}-(x-y)^{2}=4xy$.

\noindent $(iv)$ $x^{2}+y^{2}+xy=\dfrac{x^{2}+y^{2}+(x+y)^{2}}{2}$.

\noindent $(v)$ $x^{2}+y^{2}-xy=\dfrac{x^{2}+y^{2}+(x-y)^{2}}{2}$.

\noindent $(vi)$ Muestre que  $x^{2}+y^{2}+xy\geq 0$ y 
$x^{2}+y^{2}-xy\geq 0$.
}

\solcpreliminares{Para los primeros cinco incisos utilice las ecuaciones 
(\ref{ecuac1.1.1}), (\ref{ecuac1.1.2}) y (\ref{ecuac1.1.4}).  Para el inciso $(vi)$ utilice $(iv)$ y $(v)$.}

%%%%%%%% 23
\ejercpreliminares{Para todos los n'umeros reales $x$, $y$, $z$, se tiene: 

\noindent $(i)$ $x^{2}+y^{2}+z^{2}+xy+yz+zx=\dfrac{(x+y)^{2}+(y+z)^{2}+(z+x)^{2}}{2}$.

\noindent $(ii)$ $x^{2}+y^{2}+z^{2}-xy-yz-zx=\dfrac{(x-y)^{2}+(y-z)^{2}+(z-x)^{2}}{2}.$
\label{equisyeyzetaalcuadrado}

\noindent $(iii)$ Muestre que
$x^{2}+y^{2}+z^{2}+xy+yz+zx\geq 0$
y $x^{2}+y^{2}+x^{2}-xy-yz-zx\geq 0.$
}

\solcpreliminares{Para los incisos $(i)$ y $(ii)$ utilice las ecuaciones (\ref{ecuac1.1.1}) y (\ref{ecuac1.1.2}). Para demostrar $(iii)$ use $(i)$ y $(ii)$.

}

% %%%  24
\ejercpreliminares{Para todos los n'umeros reales $x$, $y$, $z$ se tienen las siguientes identidades:

\noindent $(i)$ $(xy+yz+zx)(x+y+z)=(x^{2}y+y^{2}z+z^{2}x)+(xy^{2}+yz^{2}+zx^{2})+3xyz$. 

\noindent $(ii)$
$(x+y)(y+z)(z+x)=(x^{2}y+y^{2}z+z^{2}x)+(xy^{2}+yz^{2}+zx^{2})+2xyz$. 

\noindent $(iii)$
$(xy+yz+zx)(x+y+z)=(x+y)(y+z)(z+x)+xyz$.
\label{ejerciciotresdocetres} 


\noindent $(iv)$
$(x-y)(y-z)(z-x)=(xy^{2}+yz^{2}+zx^{2})-(x^{2}y+y^{2}z+z^{2}x)$. 


\noindent $(v)$
$(x+y)(y+z)(z+x)-8xyz=2z(x-y)^{2}+(x+y)(x-z)(y-z)$. 

\noindent $(vi)$
$xy^{2}+yz^{2}+zx^{2}-3xyz=z(x-y)^{2}+y(x-z)(y-z)$. 
}

\solcpreliminares{Para demostrar los incisos $(i)$ y $(ii)$ realice las operaciones del lado izquierdo de la ecuaci'on y reacomode. 

\noindent Para demostrar los incisos $(iii)$, $(iv)$, $(v)$ y $(vi)$ realice las operaciones de ambos lados de la ecuaci'on y vea que son iguales.
}

%%%%%%%% 25
\ejercpreliminares{Para todos los n'umeros reales $x$, $y$, $z$ se tiene:

\noindent $(i)$\; $x^{2}+y^{2}+z^{2}+3(xy+yz+zx) =(x+y)(y+z)+(y+z)(z+x)+(z+x)(x+y)$.

\noindent $(ii)$\; $ xy+yz+zx-\left(x^{2}+y^{2}+z^{2}\right) =(x-y)(y-z)+(y-z)(z-x)$\\
$ \ \ +(z-x)(x-y).$
}

\solcpreliminares{Para demostrar los incisos $(i)$ y $(ii)$ realice las operaciones del lado derecho de las ecuaciones y simplifique.
}

%%%% 22
\ejercpreliminares{Para todos 
los n'umeros reales $x$, $y$, $z$ se tiene,
\begin{eqnarray*}
  (x-y)^{2}+(y-z)^{2}+(z-x)^{2} & = & 2\left[ (x-y)(x-z)\right. \\
                                       &  &\left . +(y-z)(y-x)+(z-x)(z-y)\right].
\end{eqnarray*}

}

\solcpreliminares{Utilice las ecuaciones (\ref{ecuac1.1.1}) y (\ref{ecuac1.1.2}), haga las operaciones de ambos lados de la ecuaci'on.
}




%%%27
\ejercpreliminares{Muestre que  $\sqrt[3]{2+\sqrt{5}}+\sqrt[3]{2-\sqrt{5}}$ 
es un n'umero racional.
}

\solcpreliminares{Sea  $x=\sqrt[3]{2+\sqrt{5}}+\sqrt[3]{2-\sqrt{5}}$ entonces 
$$
         x-\sqrt[3]{2+\sqrt{5}}-\sqrt[3]{2-\sqrt{5}}=0.
$$	 
Por la ecuaci'on (\ref{abccubicas}), si $a+b+c=0$, entonces 
$a^3+b^3+c^3=3abc$, luego
$$
  x^3-\left (2+\sqrt{5}\right )-\left (2-\sqrt{5}\right )=
3x\sqrt[3]{\left (2+\sqrt{5}\right )\left (2-\sqrt{5}\right)},
$$	 
simplificando se tiene que $x^3+3x-4=0.$	 
Claramente una ra'iz de la ecuaci'on es $x=1$ y las otras dos ra'ices satisfacen
la ecuaci'on $x^2+x+4=0$ que no tiene soluciones reales. Como 
$\sqrt[3]{2+\sqrt{5}}+\sqrt[3]{2-\sqrt{5}}$ es un n'umero real, se sigue que
$\sqrt[3]{2+\sqrt{5}}+\sqrt[3]{2-\sqrt{5}}=1$, 
el cual es un n'umero racional.
}


%%% 28
\ejercpreliminares{Factorice $(x-y)^3+(y-z)^3+(z-x)^3$.
}

\solcpreliminares{Observe que si $x+y+z=0$, entonces se sigue de la ecuaci'on
(\ref{abccubicas}) que $x^3+y^3+z^3=3xyz$. Como $(x-y)+(y-z)+(z-x)=0$,
se obtiene la factorizaci'on      
$$
         (x-y)^3+(y-z)^3+(z-x)^3=3(x-y)(y-z)(z-x).
$$	 
}

%%%29
\ejercpreliminares{Factorice $(x+2y-3z)^3+(y+2z-3x)^3+(z+2x-3y)^3$.
}

\solcpreliminares{Observe que $(x+2y-3z)+(y+2z-3x)+(z+2x-3y)=0$, entonces se sigue de la 
ecuaci'on (\ref{abccubicas}) que 
$(x+2y-3z)^3+(y+2z-3x)^3+(z+2x-3y)^3=3(x+2y-3z)(y+2z-3x)(z+2x-3y)$. 
}

%%%%%30
\ejercpreliminares{Muestre que si $x$, $y$, $z$ son n'umeros reales diferentes, entonces 
$$
\sqrt[3]{x-y}+\sqrt[3]{y-z}+\sqrt[3]{z-x}\neq 0.
$$
}

\solcpreliminares{Sean $a=\sqrt[3]{x-y}$, 
$b=\sqrt[3]{y-z}$, $c=\sqrt[3]{z-x}$, y suponga que $a+b+c=0$, luego, por la ecuaci'on (\ref{abccubicas}), 
$a^{3}+b^{3}+c^{3}=3abc$, pero entonces 
$0=(x-y)+(y-z)+(z-x)=a^{3}+b^{3}+c^{3}=3abc=3\sqrt[3]{x-y}\sqrt[3]{y-z}
\sqrt[3]{z-x}\neq 0$, lo cual es un absurdo.
}

%%%% 31
\ejercpreliminares{Sea $r$ un n\'{u}mero real tal que 
$\sqrt[3]{r}-\frac{1}{\sqrt[3]{r}}=1,$ encuentre los valores de $r-\frac{1}{r}$ y de $r^{3}-\frac{1}{r^{3}}.$
}

\solcpreliminares{Al tomar $a=\sqrt[3]{r}$, $b=-\frac{1}{\sqrt[3]{r}}$ y 
$c=-1$, se tiene $a+b+c=0$, luego, $r-\frac{1}{r}-1=3\sqrt[3]{r}\left( -\frac{1%
}{\sqrt[3]{r}}\right) \left( -1\right) =3$, por lo que $r-\frac{1}{r}=4.$ An'alogamente,
$r^{3}-\frac{1}{r^{3}}-4^{3}=3r\left( -\frac{1}{r}\right) \left(
-4\right) =12,$ por lo que $r^{3}-\frac{1}{r^{3}}=76.$
}

%%%%%%%%%32
\ejercpreliminares{Sean $a$, $b$, $c$ d\'{\i}gitos diferentes de
cero. Muestre que si los n'umeros enteros (escritos en notaci\'{o}n decimal) $abc$, 
$bca$ y $cab$ son divisibles entre $n$, entonces tambi\'{e}n 
$a^{3}+b^{3}+c^{3}-3abc$ es divisible entre $n$.
}

\solcpreliminares{Se sigue de  
$$a^{3}+b^{3}+c^{3}-3abc=%
\begin{vmatrix}
a & b & c \\ 
c & a & b \\ 
b & c & a%
\end{vmatrix}%
=%
\begin{vmatrix}
100b+10c+a & b & c \\ 
100a+10b+c & a & b \\ 
100c+10a+b & c & a%
\end{vmatrix}%
=%
\begin{vmatrix}
bca & b & c \\ 
abc & a & b \\ 
cab & c & a%
\end{vmatrix}.
$$
}

%%%%%%%%%%33
\ejercpreliminares{?`Cu\'{a}ntas parejas
ordenadas de n'umeros enteros $(m,n)$ hay que cumplan las siguientes condiciones: 
$mn\geq 0$ y $m^{3}+99mn+n^{3}=33^{3}$?
}

\solcpreliminares{Escriba  la ecuaci\'{o}n como 
$m^{3}+n^{3}+(-33)^{3}-3mn(-33)=0,$ y usando la ecuaci'on 
(\ref{a3masb3masc3matrices}), se tiene 
$$
(m+n-33)\left[(m-n)^2+(m+33)^2+(n+33)^2\right] =0.
$$ 
La ecuaci\'{o}n $m+n=33$ tiene 
$34$ soluciones con $mn\geq 0$ que son $(k,33-k)$, con $k=0,1, \dots, 33$, y el
segundo factor es $0$ solamente cuando $m=n=-33$, luego hay $35$ soluciones.
}

%%%%%%%%%% 34
\ejercpreliminares{Encuentre el lugar geom\'{e}trico de los
puntos $(x,y)$ tales que $x^{3}+y^{3}+3xy=1$.
}

\solcpreliminares{Al reescribir la ecuaci\'{o}n como
$x^{3}+y^{3}+(-1)^{3}-3xy(-1)=0$ y, utilizando la ecuaci'on 
(\ref{a3masb3masc3matrices}), se tiene 
$$
(x+y-1)\left[(x-y)^{2}+(y+1)^{2}+(-1-x)^{2}\right] =0.
$$ 
Luego, los puntos $(x,y)$ deben cumplir con $x+y=1$ o bien $x=y=-1.$
}

%%%%%%%% 35
\ejercpreliminares{Encuentre las soluciones reales $x$, $y$, $z$ de
la ecuaci\'{o}n,
$$
           x^{3}+y^{3}+z^{3}=(x+y+z)^{3}.
$$
}

\solcpreliminares{Sustituyendo la ecuaci'on de la hip'otesis en  la ecuaci'on (\ref{abccubicas}) se obtiene que
\begin{eqnarray*}
(x+y+z)^{3} -3xyz & = &x^{3}+y^{3}+z^{3}-3xyz\\
                               & = & (x+y+z)(x^2+y^2+z^2- xy-yz-zx)\\
                                & = & (x+y+z)((x+y+z)^2-3xy-3yz-3zx)\\
                                & = & (x+y+z)^3-3(x+y+z)(xy+yz+zx).
\end{eqnarray*}
Ahora es claro que $xyz=(x+y+z)(xy+yz+zx)$, de ah'i que $(x+y)(y+z)(z+x)=0$. O bien use que $(x+y+z)^3=x^3+y^3+z^3+3(x+y)(y+z)(z+x)$. 

Luego, las soluciones son
$(x,-x,z)$, $(x,y,-y)$, $(x,y,-x)$, con $x$, $y$, $z$ cualesquiera n'umeros reales.
}








%%%% 35
\ejercpreliminares{Sean $a$, $b$ n\'{u}meros reales con $0\leq a\leq b\leq 1$,
muestre que:

$(i)$ $0\leq \dfrac{b-a}{1-ab}\leq 1$.

$(ii)$ $0\leq \dfrac{a}{1+b}+\dfrac{b}{1+a}\leq 1$.
}

\solcpreliminares{$(i)$ Como $0\leq b\leq 1$ y $1+a>0$, pasa que $b(1+a)\leq 1+a$, luego 
$0\leq b-a\leq 1-ab$, por lo que $0\leq \dfrac{b-a}{1-ab}\leq 1.$

\noindent $(ii)$ La desigualdad de la izquierda es clara. Como $1+a\leq 1+b,$
se tiene que $\frac{1}{1+b}\leq \frac{1}{1+a},$ luego, $\dfrac{a}{1+b}+\dfrac{b}{1+a}\leq \dfrac{a}{1+a}+\dfrac{b}{1+a}=\dfrac{a+b}{1+a}\leq 1.$
}



%%%% 36
\ejercpreliminares{(Desigualdad de Nesbitt) Si $a$, $b$,  $c\geq 0$, muestre que  
$$
         \frac{a}{b+c}+\frac{b}{a+c}+\frac{c}{a+b}\geq \frac{3}{2}.
$$
\index{Desigualdad! de Nesbitt}
}

\solcpreliminares{Al hacer $X=\frac{a}{b+c}+\frac{b}{a+c}+\frac{c}{a+b}$ y
sumando y restando tres veces la unidad se tiene
\begin{align*}
    X &=\frac{a}{b+c}+\frac{b+c}{b+c}+\frac{b}{a+c}
        +\frac{a+c}{a+c}+\frac{c}{a+b}+\frac{a+b}{a+b}-3\\[.3cm]
			&=\frac{a+b+c}{b+c}+\frac{a+b+c}{a+c}+\frac{a+b+c}{a+b} -3\\
         &=(a+b+c)\left(\frac{1}{b+c}+\frac{1}{a+c}+\frac{1}{a+b}\right)-3\\
     &=\frac{1}{2}((a+b)+(b+c)+(a+c))\left(\frac{1}{b+c}+\frac{1}{a+c}+\frac{1}{a+b}\right)-3.
\end{align*} 
Ahora, por la desigualdad entre la media geom'etrica y la media aritm'etica, $x+y+z\geq 3\sqrt[3]{xyz}$ y $\frac{1}{x}+\frac{1}{y}+\frac{1}{z}\geq 3\sqrt[3]{\frac{1}{x}\frac{1}{y}\frac{1}{z}}$. Luego, $X\geq \frac{1}{2}\cdot 3\cdot 3-3=\frac{3}{2}$.  
}

%%%% 37
\ejercpreliminares{Si $a$, $b$, $c$ son las longitudes de los lados
de un tri\'{a}ngulo, muestre que
$$
\sqrt[3]{\dfrac{a^{3}+b^{3}+c^{3}+3abc}{2}}\geq \max \left\{ a,b,c\right\}. 
$$
}

\solcpreliminares{Sin p'erdida de generalidad, se puede suponer que $a\geq b\geq c$, la desigualdad es equivalente a 
$-a^{3}+b^{3}+c^{3}+3abc\geq 0$. Pero, por la ecuaci'on (\ref{a3masb3masc3matrices}), 
$-a^3+b^3+c^3+3abc=\frac{1}{2}(-a+b+c)\left[
(a+b)^{2}+(a+c)^{2}+(b-c)^{2}\right]\geq 0$, ya que, por la desigualdad del tri'angulo,  $a<b+c$.}


%%%% 38


\ejercpreliminares{Sean $p$ y $q$ n'umeros reales positivos con $\frac{1}{p}+\frac{1}{q}=1$. Muestre que:

\noindent $(i)$ $\dfrac{1}{3}\leq \dfrac{1}{p(p+1)}+\dfrac{1}{q(q+1)}\leq \dfrac{1}{2}$.\\

\noindent $(ii)$ $\dfrac{1}{p(p-1)}+\dfrac{1}{q(q-1)}\geq 1$. 
}

\solcpreliminares{Observe que $\frac{1}{p}+ \frac{1}{q}=1$ implica que $p+q=pq=s$.  Ahora bien, $(p+q)^2\geq 4pq$ implica que $s\geq 4$.

\noindent  Para probar $(i)$, vea que
 \begin{align*}
\frac{1}{p(p+1)}+\frac{1}{q(q+1)} & = \frac{1}{p} - \frac{1}{p+1}+\frac{1}{q} - \frac{1}{q+1}= 1 - \frac{p+q+2}{(p+1)(q+1)}\\
& = 1 - \frac{s+2}{2s+1}=\frac{s-1}{2s+1}.
\end{align*}
Luego, hay  que mostrar que
$$
    \frac{1}{3}\leq \frac{s-1}{2s+1}\leq \frac{1}{2},
$$
pero $2s+1\leq 3s-3 \Leftrightarrow 4 \leq s$ y
$2s-2\leq 2s+1 \Leftrightarrow -2 \leq 1$. 

\vei

\noindent  Para probar $(ii)$, vea que
 \begin{align*}
\frac{1}{p(p-1)}+\frac{1}{q(q-1)} &= \frac{1}{p-1} - \frac{1}{p}+\frac{1}{q-1} - \frac{1}{q}= \frac{p+q-2}{(p-1)(q-1)} - 1 \\
& = \frac{s-2}{s-s+1}-1=s-3\geq 1.
\end{align*}
}

%%%% 39

\ejercpreliminares{Encuentre el menor n\'{u}mero positivo $k$ tal que, para cualesquiera $0<a$, $b<1$, con $ab=k$, se cumpla que
$$
\frac{a}{b}+\frac{b}{a}+\frac{a}{1-b}+\frac{b}{1-a}\geq 4.
$$
}

\solcpreliminares{Note primero que,
$$
\frac{a}{b}+\frac{a}{1-b}=\frac{a}{b(1-b)}\geq 4a,
$$
\noindent ya que
$$b(1-b)\leq \left( \frac{b+(1-b)}{2}\right) ^{2}=\frac{1}{4}.
$$
\noindent Adem\'{a}s, se tiene la igualdad si y s\'{o}lo si $b=\frac{1}{2}.$
An\'{a}logamente,
$$
\frac{b}{a}+\frac{b}{1-a}\geq 4b.
$$
\noindent Por lo que,
$$
\frac{a}{b}+\frac{b}{a}+\frac{a}{1-b}+\frac{b}{1-a}\geq 4a+4b\geq 2%
\sqrt{4^{2}ab}=8\sqrt{k}.
$$
\noindent Con igualdad si y s\'{o}lo si $a=b.$ As\'{\i},
$$
\frac{a}{b}+\frac{b}{a}+\frac{a}{1-b}+\frac{b}{1-a}\geq 8\sqrt{k}\geq 4
$$
si y s\'{o}lo si $k\geq \frac{1}{4},$ por lo que el menor n'umero $k$ es $\frac{1}{4}.$
}

%%%%%%%%% 40
\ejercpreliminares{Sean $a$, $b$, $c$ n\'{u}meros reales no
negativos, muestre que
\begin{equation*}
(a+b)(b+c)(c+a)\geq \frac{8}{9}(a+b+c)(ab+bc+ca).
\end{equation*}
}


\solcpreliminares{Vea que, $(a+b)(b+c)(c+a)=(a+b+c)(ab+bc+ca)-abc=
\frac{8}{9}(a+b+c)(ab+bc+ca)+\frac{1}{9}(a+b+c)(ab+bc+ca)-abc$ y, 
por la desigualdad entre la media geom'etrica y la media aritm'etica, 
$(a+b+c)(ab+bc+ca)\geq \left (3\sqrt[3]{abc}\right)\left (3\sqrt[3]{(ab)(bc)(ca)}\right)=9abc$.
}

%%%%%%%%%%41
\ejercpreliminares{Sean $a$, $b$, $c$ n\'{u}meros reales positivos
que satisfacen la siguiente igualdad $(a+b)(b+c)(c+a)=1.$ Muestre que 
\begin{equation*}
ab+bc+ca\leq \frac{3}{4}.
\end{equation*}
}

\solcpreliminares{Por la desigualdad entre la media geom'etrica y la media aritm'e\-tica,  y la condici'on $(a+b)(b+c)(c+a)=1$, se tiene 
\begin{eqnarray*}
a+b+c & \geq & 3\sqrt[3]{\left(\frac{a+b}{2}\right)\left(\frac{b+c}{2}\right)\left(\frac{%
c+a}{2}\right)}=\frac{3}{2},\\
abc &  = & \sqrt{ab}\sqrt{bc}\sqrt{ca}\leq \left(\frac{a+b}{2}\right)%
\left(\frac{b+c}{2}\right)\left(\frac{c+a}{2}\right)=\frac{1}{8}.
\end{eqnarray*}
Ahora bien,  
$1=(a+b)(b+c)(c+a)=(a+b+c)(ab+bc+ca)-abc\geq \frac{3}{2}(ab+bc+ca)-\frac{1}{8}$, vea el ejercicio 1.\ref{ejerciciotresdocetres} $(iii)$.
}


%%%%%%%%%%%%% 43

\ejercpreliminares{Sean $a$, $b$, $c$ n\'{u}meros reales positivos
que satisfacen $abc=1$. Muestre que
$(a+b)(b+c)(c+a)\geq 4(a+b+c-1)$.
}

\solcpreliminares{Por el ejercicio 1.\ref{ejerciciotresdocetres} $(iii)$, basta ver que 
$ab+bc+ca+\frac{3}{a+b+c}\geq 4$. Pero 
\begin{align*}
ab+bc+ca+\frac{3}{a+b+c}& =3\left(\frac{ab+bc+ca}{3}\right)+\frac{3}{a+b+c}\\
            &\geq 4\sqrt[4]{\left( \frac{ab+bc+ca}{3}\right)
^{3}\left( \frac{3}{a+b+c}\right) }.
\end{align*}
Ahora use que, $(ab+bc+ca)^{2}\geq
3(ab\cdot bc+bc\cdot ca+ca\cdot ab)=3(a+b+c)$, 
y que $ab+bc+ca\geq 3 \sqrt[3]{a^2b^2c^2}=3$.
}

%%%%%%%%%%%%% 44

\ejercpreliminares{(APMO, 2011) Sean $a$, $b$, $c$ n\'{u}meros enteros positivos.  Muestre que es imposible
que los tres n'umeros $a^2+b+c$, $b^2+c+a$ y  $c^2+a+b$ sean cuadrados perfectos.
}

\solcpreliminares{Sin p'erdida de generalidad podemos suponer $a\leq b\leq c$. Luego, 
$ c^2<c^2+a+b\leq c^2+ 2c< ( c+1)^2$,
esto muestra que $c^2+a+b$ no puede ser un cuadrado perfecto.
}


%%%%%% 45
\ejercpreliminares{Para todos los n'umeros reales $x$, $y$ y $z$, 
se tienen las siguientes identidades:

\noindent $(i)$\;
$(x+y+z)^{3}-(y+z-x)^{3}-(z+x-y)^{3}-(x+y-z)^{3}=24xyz$.\\
\noindent $(ii)$\;
$(x-y)^{3}+(y-z)^{3}+(z-x)^{3}=3(x-y)(y-z)(z-x)$.\\
$(iii)$ \, $(x-y)(y+z)(z+x)+(y-z)(z+x)  (x+y)+(z-x)(x+y)(y+z)$\\
\phantom{.}\hspace{1.8in}        $ = - (x-y)(y-z)(z-x)$.
}

\solcpreliminares{Para demostrar todos los incisos de este ejercicio, 
'unicamente realice las operaciones y simplifique.
}

%%%%%% 46
\ejercpreliminares{Para todos los n'umeros reales $x$, $y$ y $z$, muestre lo 
siguiente:
%\noindent $(i)$\;
%$x^{3}+y^{3}+z^{3}=3xyz$  si y s\'{o}lo si $x+y+z=0$ o bien $x=y=z$.

\noindent $(i)$\; Si $f(x,y,z)=x^{3}+y^{3}+z^{3}-3xyz$, entonces
$$
        f(x,y,z)=\frac{1}{2} f(x+y, y+z, z+x)=\frac{1}{4} f(-x+y+z,x-y+z,x+y-z).
$$
\noindent $(ii)$\; Si $f(x,y,z)=x^{3}+y^{3}+z^{3} - 3xyz$, entonces
$f(x,y,z)\geq 0$ si y s\'{o}lo si $x+y+z\geq 0$ y  $f(x,y,z)\leq 0$
si y s\'{o}lo si $x+y+z\leq 0$.
}

\solcpreliminares{Para demostrar todos los incisos de este ejercicio utilice 
la identidad  (\ref{abccubicas}).
}

%%%%%% 46
\ejercpreliminares{Muestre que para n'umeros reales $x$, $y$, se tienen las 
siguientes identidades:

\noindent $(i)$\;
$(x+y)^{5}-(x^{5}+y^{5})=5xy(x+y)(x^{2}+xy+y^{2})$.

\noindent $(ii)$\; $(x+y)^{7}-(x^{7}+y^{7})=7xy(x+y)(x^{2}+xy+y^{2})^{2}$.
}

\solcpreliminares{Desarrolle ambos lados de las identidades.
}

%%%%%% 47

\ejercpreliminares{Sean $x$, $y$ y $z$ n\'umeros reales tales que $x\neq y$ y
$$x^2(y+z)=y^2(x+z)=2.$$
Determine el valor de $z^2(x+y)$.
}

\solcpreliminares{Se tiene que
\begin{eqnarray*}
0 & = & x^2(y+z)-y^2(x+z)= xy(x-y)+(x^2-y^2)z\\
  & = & (x-y)(xy+xz+yz).
\end{eqnarray*}
Como $x\neq y$, se tiene que $xy+xz+yz=0$. Multiplicando por $x-z$ se obtiene
\begin{eqnarray*}
0 &=& (x-z)(xy+xz+yz)= xz(x-z)+(x^2-z^2)y\\
&=& x^2(y+z)-z^2(x+y),
\end{eqnarray*}
de donde $z^2(x+y)=x^2(y+z)=2$.
}


%%%%%%%%%%1.48
\ejercpreliminares{Encuentre las soluciones reales $x$, $y$, $z$ y $w$
del sistema de ecuaciones
\begin{eqnarray*}
x  + y +z  & = & w \\
\frac{1}{x}+ \frac{1}{y}+\frac{1}{z} & = &\frac{1}{w}.
\end{eqnarray*}
}

\solcpreliminares{Vea que $(x+y+z)(xy+yz+zx)=xyz$ y por la ecuaci'on (\ref{productodetres}) tenemos que
$(x+y)(y+z)(z+x)=0.$ Luego, las soluciones $(x,y,z,w)$ son de la forma: $(x,-x,z,z)$,
 $(x,y,-y,x)$ y $(x,y,-x,y)$, con $x$, $y$ y $z$ n'umeros reales diferentes de cero.
}

%%%%%%%%%% 49

\ejercpreliminares{Sean $x$, $y$ y $z$ n\'{u}meros reales diferentes
de cero que cumplen las condiciones $x+y+z\neq 0$ y 
$\frac{1}{x}+\frac{1}{y}+\frac{1}{z}=\frac{1}{x+y+z}$. Muestre que, para 
cualquier n'umero entero impar $n$, se cumple que
\begin{equation*}
\dfrac{1}{x^{n}}+\dfrac{1}{y^{n}}+\dfrac{1}{z^{n}}=\dfrac{1}{x^{n}+y^{n}+z^{n}}.
\end{equation*}
}

\solcpreliminares{Por la ecuaci'on 
(\ref{productodetres}), se tiene que la condici'on es equivalente a  $(x+y)(y+z)(z+x)=0$.
Luego, un factor es cero, digamos $x+y=0$. Entonces, como $n$ es impar,
$x^{n}+y^{n}=0$, lo mismo que $\frac{1}{x^{n}}+\frac{1}{y^{n}}=0$.
}



\cleardoublepage
\rhead[\fancyplain{}{\bf Notaci\'on}]
      {\fancyplain{}{\bfseries\thepage}}
\thispagestyle{empty}
        \markboth{}{}
	 \Hrule
        \begin{flushright}	
           \huge\bf  Notaci'on
        \end{flushright}
       \Hrule
	\addcontentsline{toc}{chapter}{\protect\numberline{}{\bf Notaci\'on}}
	\markboth{}{Notaci\'on}
	 
\ve

\noindent Utilizamos la siguiente notaci'on est'andar:
$$
\begin{array}{lcl}
\mathbb{N} & \hspace{1in} & \mbox{los n'umeros enteros positivos o n'umeros
naturales} \\
\mathbb{Z} & \hspace{1in} & \mbox{los n'umeros enteros} \\
\mathbb{Q} & \hspace{1in} & \mbox{los n'umeros racionales} \\
\mathbb{Q}^+ & \hspace{1in} & \mbox{los n'umeros racionales positivos} \\
\mathbb{R}  & \hspace{1in} & \mbox{los n'umeros reales}\\
\mathbb{R}^+  & \hspace{1in} & \mbox{los n'umeros reales positivos}\\
\mathbb{I} & \hspace{1in} & \mbox{los n'umeros irracionales} \\
\mathbb{C} & \hspace{1in} & \mbox{los n'umeros complejos} \\
\mathbb{Z}_p & \hspace{1in} & \mbox{es \{0,1,\dots,p-1\} con la suma y producto m'odulo $p$.} \\
\Leftrightarrow & \hspace{1in} & \mbox{si y s\'olo si}\\
\Rightarrow & \hspace{1in} & \mbox{implica}\\
a \in A & \hspace{1in} & \mbox{el elemento} \ a \ \mbox{pertenece al conjunto}
\ A\\
A \subset B & \hspace{1in} & A \ \mbox{es un subconjunto de} \ B\\
|x| & \hspace{1in} & \mbox{valor absoluto del n'umero real} \ x\\
|z| & \hspace{1in} & \mbox{m'odulo del n'umero complejo} \ z\\
\{ x\} & \hspace{1in} & \mbox{la parte fraccionaria de un n'umero real} \ x\\
\lfloor x \rfloor & \hspace{1in} & \mbox{la parte entera de un n'umero real} \ x\\
\left[ a,b \right] & \hspace{1in} & \mbox{el conjunto de n'umeros reales} \ x \
\mbox{tal que} \ a \leq x \leq b\\
(a,b) & \hspace{1in} & \mbox{el conjunto de n'umeros reales} \ x \ 
\mbox{tal que} \ a < x < b\\
P(x) & \hspace{1in} & \mbox{el polinomio $P$ en la variable}\ x \\
\text{grad} (P) & \hspace{1in} & \mbox{grado del polinomio $P(x)$} \\
f:\left[a, b\right] \to \mathbb{R} & \hspace{1in} & \mbox{la funci'on} \ f \ 
\mbox{definida en} \
\left[a,b\right] \ \mbox{con valores en} \ \mathbb{R}\\
f'(x) & \hspace{1in} & \mbox{la derivada de la funci'on} \ f(x)\\
f''(x) & \hspace{1in} & \mbox{la segunda derivada de la funci'on} \ f(x)\\
f^{(n)}(x) & \hspace{1in} & \mbox{la $n$-'esima derivada de la funci'on} \ f(x)\\
f(x)^n & \hspace{1in} & \mbox{la potencia $n$-'esima de la funci'on} \ f(x)\\
f^n(x) & \hspace{1in} & \mbox{la $n$-'esima iteraci'on de la funci'on} \ f(x)\\
\Delta f(x) & \hspace{1in} & \mbox{el operador diferencia de} \ f(x)\\
\mbox{det} \ A & \hspace{1in} & \mbox{el determinante de la matriz} \ A\\
\sum_{i=1}^n a_i & \hspace{1in} & \mbox{la suma} \ a_1+a_2+\cdots+a_n\\
\prod_{i=1}^n a_i & \hspace{1in} & \mbox{el producto} \ a_1 \cdot a_2 \cdots a_n\\
\prod_{i\neq j} a_i & \hspace{1in} & \mbox{el producto de todos los} \ a_1, a_2, \dots, a_n \ \mbox{excepto} \ a_j\\
\end{array}
$$

$$
\begin{array}{lcl}
\max \{a, b, \dots \} & \hspace{1in} & \mbox{el m'aximo valor
entre} \ a, b, \dots \\
\min \{ a, b, \dots \} & \hspace{1in} & \mbox{el m'inimo valor
entre} \ a, b, \dots \\
\sqrt{x} & \hspace{1in} & \mbox{la ra'iz cuadrada del n'umero real positivo} \ x\\
\sqrt[n]{x} & \hspace{1in} & \mbox{la} \ n-\mbox{'esima ra'iz 
del n'umero real positivo} \ x\\
\exp{x}=e^x & \hspace{1in} & \mbox{la funci'on exponencial}\\
\displaystyle{\sum_{\text{c\'iclica}}} f(a,b, \dots) & \hspace{1in} & \mbox{representa la suma de la 
 funci'on} \ f \ \mbox{evaluada en todas las}\\
& & \mbox{permutaciones c'iclicas de las variables} \ a, b, \dots
\end{array}
$$


\vei

Utilizamos la siguiente notaci'on referente a los problemas:

\ven

$$
\begin{array}{lcl}
\mbox{AMC} & \hspace{1in} & \mbox{Competencia Americana de 
Matem'aticas (por sus siglas en}\\ 
& \hspace{1in} &  \mbox{ingl'es)}\\
\mbox{APMO} & \hspace{1in} & \mbox{Olimpiada de la Cuenca del Pac'ifico (por
sus siglas en ingl'es)}\\
\mbox{IMO} & \hspace{1in} & \mbox{Olimpiada Internacional de 
Matem'aticas (por sus siglas en}\\ 
& \hspace{1in} & \mbox{ingl'es)}\\
\mbox{MEMO} & \hspace{1in} & \mbox{Olimpiada Matem'atica de 
Europa Central (por sus siglas en}\\ 
& \hspace{1in} & \mbox{ingl'es) }\\
\mbox{OMCC} & \hspace{1in} & \mbox{Olimpiada Matem'atica de Centroam'erica y El Caribe }\\
\mbox{OIM} & \hspace{1in} & \mbox{Olimpiada Iberoamericana de Matem'aticas }\\
\mbox{OMM} & \hspace{1in} & \mbox{Olimpiada Mexicana de Matem'aticas }\\
\mbox{(pa'is, a\~no)} & \hspace{1in} & \mbox{problema que corresponde a la 
olimpiada de matem'aticas}\\ 
& \hspace{1in} & \mbox{celebrada en ese pa'is, en ese a\~no, en alguna de las
etapas} 
\end{array}
$$
 	

\begin{thebibliography}{99}
\addcontentsline{toc}{chapter}{\protect\numberline{}{\bf Bibliograf\'ia}}
\rhead[\fancyplain{}{\bf Bibliograf\'ia}]
      {\fancyplain{}{\bfseries\thepage}}

\bibitem{complex} Andreescu T., Andrica D., \textit{Complex numbers from A to
 $\dots$ Z}, Birkh\"auser, 2005.

\bibitem{challenges} Andreescu T., Gelca R., \textit{Mathematical Olympiad 
Challenges}, Birkh\"auser, 2000.

\bibitem{treasures} Andreescu T., Enescu B., \textit{Mathematical Olympiad 
Treasures}, Birkh\"auser, 2006.

\bibitem{barbeau} Barbeau E. J., 
\textit{Polynomials},  Springer-Verlag, 1989.

\bibitem{bulajichgomez}  Bulajich Manfrino R., G'omez Ortega J.A.,
\textit{Geometr'ia}, Cuadernos de Olimpiadas, Instituto de Matem'aticas de la 
Universidad Nacional Aut'onoma de M'exico, Sociedad Matem'atica Mexicana, 2012.

\bibitem{bulajich1}  Bulajich Manfrino R., G'omez Ortega J.A., Valdez Delgado 
R., \textit{Desigualdades}, Cuadernos de Olimpiadas, Instituto de Matem'aticas 
de la Universidad Nacional Aut'onoma de M'exico, Sociedad Matem'atica Mexicana,
2010.

\bibitem{bulajich} Bulajich Manfrino R., G'omez Ortega J.A., Valdez Delgado R., 
\textit{Inequalities: A Mathematical Olympiad Approach }, Birkh\"auser, 2009.


\bibitem{cardenas}  C'ardenas H., Lluis E., Raggi F., Tom'as F. 
\textit{'Algebra Superior}, Editorial Trillas, 1973.


\bibitem{IMOcompendium} Djuki\'c D., Jankovi\'c V., Mati\'c I., Petrovi\'c N.,
\textit{The IMO Compendium}, Springer, 2006. 

\bibitem{engel} Engel A., \textit{Problem-Solving Strategies}, Springer,
1998.

\bibitem{fine} Fine B., Resenberger G., \textit{The Fundamental Theorem of
 Algebra}, Springer, 1997.
 
\bibitem{goldberg}  Goldberg S.,
\textit{Introduction to Differece Equations}, Dover Publications, 1958.

\bibitem{casillas}  G'omez Ortega J.A., Valdez Delgado R., V'azquez Padilla,
\textit{Principio de las Casillas}, Cuadernos de Olimpiadas, 
Instituto de Matem'aticas de la Universidad Nacional Aut'onoma de 
M'exico, Sociedad Matem'atica Mexicana, 2014.

\bibitem{honsberger} Honsberger R.,
\textit{Ingenuity in Mathematics}, vol. 23 in New Mathematical Library 
series,  1962.

%\bibitem{marsden} Marsden J.E., Tromba A.J., \textit{Vector Calculus}, W.H. Freeman and %Company, New York, Third Edition, 1988.

\bibitem{niven} Niven I., Montgomery H. \& Zuckerman H.,
\textit{An Introduction to the Theory of Numbers}, Wiley,  5 edition,  1991.


\bibitem{remmert} Remmert R., \textit{Theory  of Complex Functions}, Springer,
 1999.
  
\bibitem{rudin} Rudin W., \textit{Principles of Mathematical Analysis},
McGraw-Hill, 1976.

\bibitem{miniatures} Savchev S., Andreescu T., \textit{Mathematical Miniatures},
The Mathematical Association of America, 2003.

\bibitem{small} Small C.G., 
\textit{Functional Equations and How to Solve Them}, Springer, 2007.

\bibitem{pablo} Sober'on P., 
\textit{Combinatoria para Olimpiadas Internacionales}, Cuadernos de la Olimpiada, Instituto de Matem'aticas de la Universidad Aut'onoma de M'exico, Sociedad Matem'atica Mexicana,  2010.

\bibitem{spivak} Spivak M., 
\textit{Calculus}, Editorial Revert'e, 3era edici\'on, 2012.

\bibitem{serge} Tabachnikov, S. (Editor), {\it Kvant Selecta: Algebra and 
Analysis II}, American Mathematical Society, 1999.

\bibitem{functional} Venkatachala B. J., \textit{Functional Equations. A
Problem Solving Approach}, Prism Books Pvt Ltd, 2002.

\end{thebibliography}

\rhead[\fancyplain{}{\bf \'Indice}]
      {\fancyplain{}{\bfseries\thepage}}
\addcontentsline{toc}{chapter}{\protect\numberline{}{\bf \'Indice}}
\printindex

\end{document}

