

%%%%%%%%%%11

\ejercpreliminares{Escriba en la forma $\frac{m}{n}$, con $n$ y $m$ 
n'umeros enteros positivos, a los siguiente n'umeros reales:

$(i)$  $0.11111\dots$.  

$(ii)$ $1.14141414\dots$.
}

\solcpreliminares{Para resolver $(i)$, defina $x=0.111\dots$, 
entonces $10x=1.11\dots$. Restando la primera ecuaci'on de la 
segunda se tiene que $9x=1$, luego,  $x=\frac{1}{9}$.

\noindent $(ii)$ Sea $x=1.141414\ldots$, entonces 
$100x=114.141414\ldots$. Restando la pri\-mera ecuaci'on de la 
segunda se tiene que $99x=113$, de donde $x=\frac{113}{99}$. }

\ejercpreliminares{$(i)$\; Muestre que $121_b$ es un cuadrado 
perfecto en cualquier base $b\geq 2$.

$(ii)$\; Determine el menor valor de $b$ para el cual  $232_b$ 
es un cuadrado perfecto.
}

\solcpreliminares{$(i)$\; Primero observe que 
$121_b=(1\times b^2)+ (2\times b) +1 =(b+1)^2$ entonces 
$121_b$ es un cuadrado perfecto en cualquier base $b\geq 2$.

$(ii)$\; Como $232_b=2b^2+3b+2$ debe ser cuadrado y como 3 es 
uno de sus d'igitos, $b\geq 4$.

Para $b=4$, $232_4=46$, para $b=5$, $232_5=67$, para $b=6$, 
$232_6=92$ y para $b=7$, $232_7=121$. Luego, $b=7$ es el menor 
entero positivo tal que $232_b$ es un cuadrado perfecto. 
}

%%%%%% 12
%\ejercpreliminares{Sea $b \geq 2$ un entero positivo.
%(a) Muestre que para que un entero $N$, escrito en base $b$, sea igual a la suma del cuadrado %de sus d'igitos, es necesario que  $N = 1$ o que 
%$N$ tenga solamente dos d'igitos.
%(b) Give a complete list of all integers not exceeding 50 that, relative to
%some base $b$, are equal to the sum of the squares of their digits.
%(c) Show that for any base $b$ the number of two-digit integers that are
%equal to the sum of the squares of their digits is even.
%(d) Show that for any odd base $b$ there is an integer other than 1 that is
%equal to the sum of the squares of its digits.
%}

%%%%%%%% 13
\ejercpreliminares{(IMO, 1970) Sean $a$, $b$ y $n$ n'umeros enteros mayores 
que 1. Sean $A_{n-1}$ y $A_n$ dos n'umeros escritos en el sistema num'erico en 
base  $a$ y, $B_{n-1}$ y $B_n$ dos n'umeros escritos en el sistema n'umerico 
en base  $b$. Estos n'umeros 
est'an relacionados de la siguiente forma,
\begin{eqnarray*}
        A_n = x_nx_{n-1} \dots x_0, & & A_{n-1} = x_{n-1}x_{n-2}\dots x_0,\\
        B_n = x_nx_{n-1} \dots x_0, & & B_{n-1} = x_{n-1}x_{n-2}\dots x_0,
\end{eqnarray*}
con 
$x_n\neq 0$ y $x_{n-1}\neq 0$.  Muestre que  $a > b$ si y s'olo si
$$
                  \frac{A_{n-1}}{A_{n}} <\frac{B_{n-1}}{B_{n}}.
$$
}

\solcpreliminares{Suponga que $a>b$. Entonces para todos los 
enteros $0\leq k\leq n$, $x_nx_ka^nb^k\geq x_nx_kb^na^k$, con 
igualdad solamente cuando $k=n$ o $x_k=0$. En particular, se 
tiene una desigualdad estricta para $k=n-1$. En resumen, esto 
se convierte en
$$
              x_n a^n\sum_{k=0}^n x_kb^k > x_nb^n \sum_{k=0}^n x_k a^k
$$
o 
$$
             \frac{ x_n a^n}{A_n}> \frac{x_nb^n}{B_n}.
$$
Esto implica que
$$
             \frac{ A_{n-1}}{A_n}= 1-\frac{ x_n a^n}{A_n} < 
1-\frac{ x_n b^n}{B_n} =\frac{B_{n-1}}{B_n}.
$$ 
Por otro lado, si $a=b$, entonces evidentemente 
$\frac{ A_{n-1}}{A_n}= \frac{ B_{n-1}}{B_n}$ y si 
$a<b$, por lo que se demostr'o antes, se tiene que, 
$ \frac{ A_{n-1}}{A_n}> \frac{ B_{n-1}}{B_n}$. Por lo 
tanto, $\frac{A_{n-1}}{A_{n}} <\frac{B_{n-1}}{B_{n}}$ si y 
s'olo si $a>b$.
}



