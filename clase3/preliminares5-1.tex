
%%%27
\ejercpreliminares{Muestre que  $\sqrt[3]{2+\sqrt{5}}+\sqrt[3]{2-\sqrt{5}}$ 
es un n'umero racional.
}

\solcpreliminares{Sea  $x=\sqrt[3]{2+\sqrt{5}}+\sqrt[3]{2-\sqrt{5}}$ entonces 
$$
         x-\sqrt[3]{2+\sqrt{5}}-\sqrt[3]{2-\sqrt{5}}=0.
$$	 
Por la ecuaci'on (\ref{abccubicas}), si $a+b+c=0$, entonces 
$a^3+b^3+c^3=3abc$, luego
$$
  x^3-\left (2+\sqrt{5}\right )-\left (2-\sqrt{5}\right )=
3x\sqrt[3]{\left (2+\sqrt{5}\right )\left (2-\sqrt{5}\right)},
$$	 
simplificando se tiene que $x^3+3x-4=0.$	 
Claramente una ra'iz de la ecuaci'on es $x=1$ y las otras dos ra'ices satisfacen
la ecuaci'on $x^2+x+4=0$ que no tiene soluciones reales. Como 
$\sqrt[3]{2+\sqrt{5}}+\sqrt[3]{2-\sqrt{5}}$ es un n'umero real, se sigue que
$\sqrt[3]{2+\sqrt{5}}+\sqrt[3]{2-\sqrt{5}}=1$, 
el cual es un n'umero racional.
}


%%% 28
\ejercpreliminares{Factorice $(x-y)^3+(y-z)^3+(z-x)^3$.
}

\solcpreliminares{Observe que si $x+y+z=0$, entonces se sigue de la ecuaci'on
(\ref{abccubicas}) que $x^3+y^3+z^3=3xyz$. Como $(x-y)+(y-z)+(z-x)=0$,
se obtiene la factorizaci'on      
$$
         (x-y)^3+(y-z)^3+(z-x)^3=3(x-y)(y-z)(z-x).
$$	 
}

%%%29
\ejercpreliminares{Factorice $(x+2y-3z)^3+(y+2z-3x)^3+(z+2x-3y)^3$.
}

\solcpreliminares{Observe que $(x+2y-3z)+(y+2z-3x)+(z+2x-3y)=0$, entonces se sigue de la 
ecuaci'on (\ref{abccubicas}) que 
$(x+2y-3z)^3+(y+2z-3x)^3+(z+2x-3y)^3=3(x+2y-3z)(y+2z-3x)(z+2x-3y)$. 
}

%%%%%30
\ejercpreliminares{Muestre que si $x$, $y$, $z$ son n'umeros reales diferentes, entonces 
$$
\sqrt[3]{x-y}+\sqrt[3]{y-z}+\sqrt[3]{z-x}\neq 0.
$$
}

\solcpreliminares{Sean $a=\sqrt[3]{x-y}$, 
$b=\sqrt[3]{y-z}$, $c=\sqrt[3]{z-x}$, y suponga que $a+b+c=0$, luego, por la ecuaci'on (\ref{abccubicas}), 
$a^{3}+b^{3}+c^{3}=3abc$, pero entonces 
$0=(x-y)+(y-z)+(z-x)=a^{3}+b^{3}+c^{3}=3abc=3\sqrt[3]{x-y}\sqrt[3]{y-z}
\sqrt[3]{z-x}\neq 0$, lo cual es un absurdo.
}

%%%% 31
\ejercpreliminares{Sea $r$ un n\'{u}mero real tal que 
$\sqrt[3]{r}-\frac{1}{\sqrt[3]{r}}=1,$ encuentre los valores de $r-\frac{1}{r}$ y de $r^{3}-\frac{1}{r^{3}}.$
}

\solcpreliminares{Al tomar $a=\sqrt[3]{r}$, $b=-\frac{1}{\sqrt[3]{r}}$ y 
$c=-1$, se tiene $a+b+c=0$, luego, $r-\frac{1}{r}-1=3\sqrt[3]{r}\left( -\frac{1%
}{\sqrt[3]{r}}\right) \left( -1\right) =3$, por lo que $r-\frac{1}{r}=4.$ An'alogamente,
$r^{3}-\frac{1}{r^{3}}-4^{3}=3r\left( -\frac{1}{r}\right) \left(
-4\right) =12,$ por lo que $r^{3}-\frac{1}{r^{3}}=76.$
}

%%%%%%%%%32
\ejercpreliminares{Sean $a$, $b$, $c$ d\'{\i}gitos diferentes de
cero. Muestre que si los n'umeros enteros (escritos en notaci\'{o}n decimal) $abc$, 
$bca$ y $cab$ son divisibles entre $n$, entonces tambi\'{e}n 
$a^{3}+b^{3}+c^{3}-3abc$ es divisible entre $n$.
}

\solcpreliminares{Se sigue de  
$$a^{3}+b^{3}+c^{3}-3abc=%
\begin{vmatrix}
a & b & c \\ 
c & a & b \\ 
b & c & a%
\end{vmatrix}%
=%
\begin{vmatrix}
100b+10c+a & b & c \\ 
100a+10b+c & a & b \\ 
100c+10a+b & c & a%
\end{vmatrix}%
=%
\begin{vmatrix}
bca & b & c \\ 
abc & a & b \\ 
cab & c & a%
\end{vmatrix}.
$$
}

%%%%%%%%%%33
\ejercpreliminares{?`Cu\'{a}ntas parejas
ordenadas de n'umeros enteros $(m,n)$ hay que cumplan las siguientes condiciones: 
$mn\geq 0$ y $m^{3}+99mn+n^{3}=33^{3}$?
}

\solcpreliminares{Escriba  la ecuaci\'{o}n como 
$m^{3}+n^{3}+(-33)^{3}-3mn(-33)=0,$ y usando la ecuaci'on 
(\ref{a3masb3masc3matrices}), se tiene 
$$
(m+n-33)\left[(m-n)^2+(m+33)^2+(n+33)^2\right] =0.
$$ 
La ecuaci\'{o}n $m+n=33$ tiene 
$34$ soluciones con $mn\geq 0$ que son $(k,33-k)$, con $k=0,1, \dots, 33$, y el
segundo factor es $0$ solamente cuando $m=n=-33$, luego hay $35$ soluciones.
}

%%%%%%%%%% 34
\ejercpreliminares{Encuentre el lugar geom\'{e}trico de los
puntos $(x,y)$ tales que $x^{3}+y^{3}+3xy=1$.
}

\solcpreliminares{Al reescribir la ecuaci\'{o}n como
$x^{3}+y^{3}+(-1)^{3}-3xy(-1)=0$ y, utilizando la ecuaci'on 
(\ref{a3masb3masc3matrices}), se tiene 
$$
(x+y-1)\left[(x-y)^{2}+(y+1)^{2}+(-1-x)^{2}\right] =0.
$$ 
Luego, los puntos $(x,y)$ deben cumplir con $x+y=1$ o bien $x=y=-1.$
}

%%%%%%%% 35
\ejercpreliminares{Encuentre las soluciones reales $x$, $y$, $z$ de
la ecuaci\'{o}n,
$$
           x^{3}+y^{3}+z^{3}=(x+y+z)^{3}.
$$
}

\solcpreliminares{Sustituyendo la ecuaci'on de la hip'otesis en  la ecuaci'on (\ref{abccubicas}) se obtiene que
\begin{eqnarray*}
(x+y+z)^{3} -3xyz & = &x^{3}+y^{3}+z^{3}-3xyz\\
                               & = & (x+y+z)(x^2+y^2+z^2- xy-yz-zx)\\
                                & = & (x+y+z)((x+y+z)^2-3xy-3yz-3zx)\\
                                & = & (x+y+z)^3-3(x+y+z)(xy+yz+zx).
\end{eqnarray*}
Ahora es claro que $xyz=(x+y+z)(xy+yz+zx)$, de ah'i que $(x+y)(y+z)(z+x)=0$. O bien use que $(x+y+z)^3=x^3+y^3+z^3+3(x+y)(y+z)(z+x)$. 

Luego, las soluciones son
$(x,-x,z)$, $(x,y,-y)$, $(x,y,-x)$, con $x$, $y$, $z$ cualesquiera n'umeros reales.
}




