'Algebra se ha convertido en un 'area fundamental en las olimpiadas de 
mate\-m'a\-ticas. Son frecuentes los problemas de este tema que aparecen en los concursos,
y son tambi'en  frecuentes los problemas de otras 'areas que hacen uso del 'algebra para 
su soluci'on. En este libro queremos se\~nalar las principales herramientas de 'algebra 
que un alumno deber'a asimilar paso a paso en su preparaci'on para los concursos y olimpiadas 
de matem'aticas. 

Algunos de los t'opicos que tratamos en el libro son parte de los temarios de matem'aticas 
del bachillerato, pero hay otros que son de nivel universitario. Esto permite que el libro pueda 
ser utilizado como un texto de consulta para los alumnos del primer a\~no de la 
universidad  que gusten de enfrentar  problemas de 'algebra y tengan inter'es en aprender t'ecnicas para resolverlos.

El libro se ha dividido en diez cap'itulos. Los primeros cuatro corresponden a temas del 
bachillerato y son b'asicos para los alumnos que se entrenan para las olimpiadas de matem'aticas  
a nivel estatal y nacional. Los siguientes cuatro cap'itulos  usualmente se estudian 
en cursos del primer a\~no de una carrera universitaria, pero se han convertido en t'opicos y 
herramientas  que los alumnos que participan en 
competencias internacionales deben conocer y dominar.  Los 'ultimos dos cap'itulos contienen problemas y soluciones del material tratado a lo largo del libro.  



El primer cap'itulo,  cubre material b'asico de 'algebra, como son  sistemas n'umericos,  valor 
absoluto, productos notables, factorizaci'on, entre otros. Buscamos que el lector  adquiera 
destreza en la manipulaci'on de ecuaciones y f'ormulas algebraicas para llevarlas a formas 
equivalentes m'as f'aciles de entender y trabajar.

En el cap'itulo 2 se presenta el estudio de las sumas finitas de n'umeros, como por ejemplo, la 
suma de los cuadrados de los primeros $n$ naturales. Se analizan  sumas
telesc'opicas, progresiones aritm'eticas y geom'etricas, as'i como  varias de sus 
propiedades.

El cap'itulo 3  trata sobre la t'ecnica matem'atica para demostrar 
proposiciones, conocida como 
el principio de inducci'on matem'atica, adem'as se ejemplifica su uso con 
varios problemas. Tambi'en  se presentan  formulaciones equivalentes del 
principio de inducci'on.

\newpage

Para completar la primera parte del libro, en el cap'itulo 4 se estudian
polinomios cuadr'aticos y c'ubicos, haciendo 'enfasis en el estudio del 
discriminante de los cuadr'aticos y de las f'ormulas de Vieta para 
estos dos tipos de polinomios.

%Para completar esta primera parte del libro, en el cap'itulo 4 se estudian
%los polinomios cuadr'aticos y c'ubicos, haciendo 'enfasis en el estudio del 
%discriminante de un polinomio cuadr'atico y de las f'ormulas de Vieta para 
%estos dos tipos de polinomios.

La segunda parte del texto, inicia en el cap'itulo 5 donde se estudian los n'umeros complejos, 
sus propiedades y algunas aplicaciones. Todo esto siempre ejemplificado con problemas de las olimpiadas
de matem'aticas. Se incluye tambi'en una demostraci'on elemental del teorema fundamental del 'algebra.

En el cap'itulo 6 se estudian las propiedades principales de las funciones. Tambi'en se presenta una 
introducci'on a las ecuaciones funcionales, sus propiedades y  se dan algunas 
recomendaciones para resolver problemas donde aparecen ecua\-ciones de este tipo.

El cap'itulo 7 habla  de la noci'on de sucesi'on y serie. Se estudian  sucesiones especiales como las acotadas, 
las  periodicas, las mon'otonas, las recursivas entre otras.
Se introduce tambi'en el concepto de  convergencia para  sucesiones y series.

En el cap'itulo 8 se generaliza el estudio de los polinomios que se trat'o en la primera parte. 
Se presenta la teor'ia de polinomios de grado arbitrario y di\-versas t'ecnicas para analizar 
propiedades de los mismos. Al final del 
cap'itulo se  introducen los polinomios de varias variables.

La mayor'ia de las secciones de   estos primeros ocho  cap'itulos tienen al final una 
lista de ejercicios para el lector, seleccionados y  adecuados para practicar los temas que se abordan en ellas. 
La dificultad de los ejercicios var'ia desde ser una aplicaci'on directa de un resultado 
visto en la secci'on hasta ser un problema de un concurso, que con la t'ecnica tratada es factible resolver.  

El cap'itulo 9 es una recopilaci'on de problemas, cada uno de ellos  cercano a uno o m'as de los 
temas tratados en el libro. Estos problemas tienen un \linebreak grado de dificultad mayor a los ejercicios.  
La mayor'ia de ellos han aparecido en alg'un concurso u olimpiada de matem'aticas. En la soluci'on  de cada problema 
est'a impl'icito el conocimiento y destreza que se debe adquirir para la  manipulaci'on de expresiones algebraicas.  


Finalmente,  el cap'itulo 10 contiene las soluciones de todos los ejercicios y   
problemas planteados en el libro.

El lector podr'a notar que al final del t'itulo de algunas secciones aparece el s'imbolo $\star$, esto indica  
que el nivel de  tal secci'on  es m'as d'ificil.  En una primera lectura, el lector puede 
omitir estas secciones, sin embargo recomendamos que las tenga presentes por las t'ecnicas que se utilizan en ellas. 


Agradecemos infinitamente a Leonardo Ignacio Mart'inez Sandoval por sus \linebreak siempre 'utiles  comentarios y 
sugerencias, los cuales contribuyeron al mejoramiento del material presentado en este  libro.   

\vei

Radmila Bulajich \hspace{.75in} Jos'e Antonio G'omez \hspace{.75in}
Rogelio Valdez



