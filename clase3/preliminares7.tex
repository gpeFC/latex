
\section{Factorizaci'on}
\label{cap1sec3}

\index{Factorizaci'on}
\noindent Una de las formas m'as importantes de manipulaci'on algebraica  
es la que se conoce como factorizaci'on. En esta secci'on estudiamos algunos 
ejemplos y problemas cuya soluci'on
depende del conocimiento de  f'ormulas  de factorizaci'on.
Muchos de los problemas ol'impicos que involucran expresiones algebraicas se 
resuelven f'acilmente haciendo uso de transformaciones algebraicas que 
utilizan factorizaciones apropiadas.
Empecemos con  algunas  f'ormulas  elementales de factorizaci'on, donde $x$, 
$y$ son  n'umeros reales:
\begin{description}
\item[$(a)$] $x^2 - y^2 =(x+y)(x-y)$.
\item[$(b)$]  $x^2 +2xy + y^2 = (x+y)^2$ y $x^2-2xy+y^2 = (x-y)^2$.
\item[$(c)$]  $x^2+y^2+z^2+2xy+2yz+2zx =(x+y+z)^2$.
\end{description}

\noindent Estas identidades algebraicas se catalogan como identidades de grado $2$. De hecho, estas cuatro identidades fueron ya estudiadas en la secci'on de productos notables, sin embargo, lo que se desea hacer ahora es, dada una expresi'on algebraica, reducirla a un producto de expresiones algebraicas  m\'as simples.

\vei

\begin{ejemplo} Para n\'{u}meros reales $a$, $b$, $x$, $y$, con $x$ y $y$ 
distintos de cero,  se tiene que  
$$
            \frac{a^{2}}{x}+\frac{b^{2}}{y}-\frac{(a+b)^{2}}{x+y}=\frac{(ay-bx)^{2}}{xy(x+y)}.
$$
\end{ejemplo}
Para obtener la igualdad que se pide, empecemos realizando la suma del lado izquierdo de la  identidad,
\begin{eqnarray*}
\frac{a^{2}}{x}+\frac{b^{2}}{y}-\frac{(a+b)^{2}}{x+y} & = & \frac{a^2 y (x+y)+ b^2 x (x+y) -
xy (a+b)^2}{xy(x+y)} \\
& = & \frac{a^2 y^2 + b^2 x^2 - 2xyab}{xy(x+y)}\\
& = & \frac{(ay-bx)^{2}}{xy(x+y)}.
\end{eqnarray*}

\vei

\noindent Una aplicaci\'on de la identidad anterior nos lleva a una 
demostraci'on inmediata de la {\bf desigualdad 'util}\footnote{Ver 
\cite{bulajich1}, p'ag. 40  o \cite{bulajich} p'ag. 34.} de grado $2$. 'Esta asegura que 
para n\'{u}meros reales $a$, $b$ y n\'{u}meros reales positivos $x$, $y$, 
se cumple que
\index{Desigualdad!'util}
$$
\frac{a^{2}}{x}+\frac{b^{2}}{y}\geq \frac{(a+b)^{2}}{x+y}.
$$

\vei

\noindent Las siguientes identidades son de grado $3$, con $x$, $y$, $z \in \mathbb{R}$:
\begin{description}
\item[$(a)$] $x^{3}-y^{3}=(x-y)\left( x^{2}+xy+y^{2}\right)$.
\item[$(b)$] $x^{3}-y^{3}=(x-y)^{3}+3xy(x-y)$.
\item[$(c)$] $(x+y)^{3}-(x^{3}+y^{3})=3xy(x+y)$.
\item[$(d)$] $x^{3}-xy^{2}+x^{2}y-y^{3}=(x+y)(x^{2}-y^{2})$.
\item[$(e)$] $x^{3}+xy^{2}-x^{2}y-y^{3}=(x-y)(x^{2}+y^{2})$.
\end{description}

\noindent Para comprobar la validez de las identidades anteriores basta desarrollar alguno de los lados o utilizar el teorema del binomio de Newton, el cual se estudiar'a en la secci'on \ref{coeficientesbinomiales}.

\vei

\noindent Otra identidad  de grado $3$ muy importante y que ya mencionamos en la identidad (\ref{abccubicas}) es
$$
x^{3}+y^{3}+z^{3}-3xyz=(x+y+z)(x^{2}+y^{2}+z^{2}-xy-yz-zx),
$$
para cualesquiera n'umeros reales $x$, $y$, $z$.  Una demostraci'on de 'esta se obtiene simplemente desarrollando el lado derecho de la identidad.  A lo largo del libro veremos otras demostraciones de esta igualdad. 

Una forma equivalente de la identidad anterior es
$$
x^{3}+y^{3}+z^{3}-3xyz=\frac{1}{2}(x+y+z)\left[ \left( x-y\right)
^{2}+(y-z)^{2}+(z-x)^{2}\right].
$$

\vei


\noindent Las identidades $x^2 - y^2 =(x+y)(x-y)$ y $x^{3}-y^{3}=(x-y)\left( x^{2}+xy+y^{2}\right)$
son casos particulares de la identidad de grado $n$,
\begin{equation}
      x^n-y^n=(x-y)(x^{n-1} + x^{n-2} y + \cdots + x y^{n-2} + y^{n-1}),
\label{antesdesophie}
\end{equation}
para cualesquiera n'umeros reales $x$, $y$.

\noindent Si $n$ es impar, podemos reemplazar $y$ por $-y$ en la 'ultima f'ormula para obtener
la f'ormula de factorizaci'on para la suma de dos potencias $n$-'esimas impares,
\begin{equation}
x^n+y^n=(x+y)(x^{n-1} - x^{n-2} y + \cdots - x y^{n-2} + y^{n-1}).
\label{identidadxalanyalan}
\end{equation}

\noindent En general, la suma de potencias $n$-\'esimas pares no es factorizable, aunque existen algunas
excepciones cuando es posible completar cuadrados, veamos el siguiente ejemplo.

\begin{ejemplo}(Identidad de Sophie Germain) Para cualesquiera n'umeros reales $x$, $y$ se tiene que
\label{sophiegermain}\index{Identidad de Sophie Germain}
$$
x^{4}+4y^{4} =(x^{2}+2y^{2}+2xy)(x^{2}+2y^{2}-2xy).
$$
\end{ejemplo}

\noindent Completando cuadrados, tenemos
\begin{eqnarray*}
     x^{4}+4y^{4} & = & x^{4}+4x^2y^2+4y^{4}- 4x^2y^2=(x^{2}+2y^{2})^2-(2xy)^2\\
            & = &(x^{2}+2y^{2}+2xy)(x^{2}+2y^{2}-2xy).
\end{eqnarray*}

\noindent Otro ejemplo, con potencias pares es el siguiente.

\begin{ejemplo} Para cualesquiera n'umeros reales $x$, $y$, se tiene que
$$
x^{2n}-y^{2n} =(x+y)(x^{2n-1}-x^{2n-2}y+x^{2n-3}y^{2}-\cdots+xy^{2n-2}-
y^{2n-1}).
$$
\end{ejemplo}
\noindent Para comprobar esto simplemente tenemos que hacer la divisi'on de $x^{2n}-y^{2n}$ entre $x+y$ o bien realizar el producto de la derecha y simplificar.

\begin{ejemplo} Veamos que  $n^4 - 22n^2+9$ es un n'umero compuesto para cualquier entero $n$.
\end{ejemplo}
La idea para mostrar lo que se pide es tratar de factorizar la expresi'on. Inten\-temos completar cuadrados, la forma m'as com'un de hacerlo es la siguiente
$$
         n^4-22n^2+9= (n^4 - 22n^2 + 121)-112=(n^2 - 11)^2-112,
$$
el problema que tenemos es que $112$ no es un cuadrado perfecto, por lo que no es inmediato factorizar. Sin embargo, podemos utilizar la siguiente forma, menos usual, de completar cuadrados
\begin{eqnarray*}
n^4-22n^2+9 & = & (n^4 - 6n^2 + 9)- 16n^2=(n^2-3)^2-16n^2\\
            & = & (n^2-3)^2-(4n)^2=(n^2-3+4n)(n^2-3-4n)\\
            & = & ((n+2)^2-7)((n-2)^2-7),
\end{eqnarray*}
y observemos que ninguno de los dos 'ultimos factores es igual a $\pm 1$.
\vei

\noindent El siguiente es otro ejemplo de c'omo utilizando formas b'asicas de 
factorizaci'on podemos  resolver problemas.

\begin{ejemplo}
Encontremos todas las parejas $(m,n)$ de n'umeros enteros positivos 
tales que $|3^m - 2^n|=1$.
\end{ejemplo}
Cuando $m=1$ o $m=2$, es f'acil encontrar las soluciones $(m,n)=(1,1)$, 
$(1,2)$, $(2,3)$. Ahora
mostremos que no hay otras soluciones. Supongamos que $(m,n)$ es una 
soluci'on de $|3^m - 2^n|=1$, con $m > 2$, y por lo tanto $n > 3$. Analicemos 
los dos casos: $3^m - 2^n=1$ y $3^m - 2^n=-1$.

\noindent Supongamos que $3^m - 2^n=-1$ con $n > 3$, entonces $3^m+1$ es 
divisible entre $8$, sin embargo
al dividir $3^m$ entre $8$ obtenemos como residuo $1$ o $3$, dependiendo de 
si $n$ es par o impar, por lo que
en este caso no hay soluci'on.

\noindent Supongamos que $3^m - 2^n=1$ con $m\geq  3$, por lo que $n\geq 5$, 
ya que $2^n+1=3^m\geq 27$. Entonces $3^m-1$ es divisible entre $8$, por lo
cual $m$ es par, digamos $m=2k$, con $k >1$. Entonces 
$2^n = 3^{2k}-1=(3^k+1)(3^k-1)$. 

Luego,
$3^k+1=2^r$, para alguna $r >3$, pero por el caso anterior sabemos que esto 
es imposible, luego en este caso tampoco hay soluciones. 


\ve

\noindent Otras  f'ormulas  'utiles de factorizaci'on son las siguientes.
Para n\'{u}meros reales $x$, $y$, $z$ se cumplen las siguientes igualdades:
\begin{equation}
\label{productodetres}
(x+y)(y+z)(z+x)+xyz=(x+y+z)(xy+yz+zx)
\end{equation}
\begin{equation}
(x+y+z)^{3}=x^{3}+y^{3}+z^{3}+3(x+y)(y+z)(z+x).
\end{equation}
Para convencerse, basta desarrollar ambos lados de cada igualdad.
De estas identidades tenemos la siguiente observaci'on. 

\begin{observacion}
$(a)$\; Si $x$, $y$, $z$ son n\'{u}meros reales, con $xyz=1$, entonces
\begin{equation}
(x+y)(y+z)(z+x)+1=(x+y+z)(xy+yz+zx).
\end{equation}
\label{observacionabcigual1}
$(b)$\; Si $x$, $y$, $z$ son n\'{u}meros reales con $xy+yz+zx=1$, entonces
\begin{equation}
(x+y)(y+z)(z+x)+xyz=x+y+z.
\end{equation}
\end{observacion}

\ve
\vei

%%%%%% 45
\ejercpreliminares{Para todos los n'umeros reales $x$, $y$ y $z$, 
se tienen las siguientes identidades:

\noindent $(i)$\;
$(x+y+z)^{3}-(y+z-x)^{3}-(z+x-y)^{3}-(x+y-z)^{3}=24xyz$.\\
\noindent $(ii)$\;
$(x-y)^{3}+(y-z)^{3}+(z-x)^{3}=3(x-y)(y-z)(z-x)$.\\
$(iii)$ \, $(x-y)(y+z)(z+x)+(y-z)(z+x)  (x+y)+(z-x)(x+y)(y+z)$\\
\phantom{.}\hspace{1.8in}        $ = - (x-y)(y-z)(z-x)$.
}

\solcpreliminares{Para demostrar todos los incisos de este ejercicio, 
'unicamente realice las operaciones y simplifique.
}

%%%%%% 46
\ejercpreliminares{Para todos los n'umeros reales $x$, $y$ y $z$, muestre lo 
siguiente:
%\noindent $(i)$\;
%$x^{3}+y^{3}+z^{3}=3xyz$  si y s\'{o}lo si $x+y+z=0$ o bien $x=y=z$.

\noindent $(i)$\; Si $f(x,y,z)=x^{3}+y^{3}+z^{3}-3xyz$, entonces
$$
        f(x,y,z)=\frac{1}{2} f(x+y, y+z, z+x)=\frac{1}{4} f(-x+y+z,x-y+z,x+y-z).
$$
\noindent $(ii)$\; Si $f(x,y,z)=x^{3}+y^{3}+z^{3} - 3xyz$, entonces
$f(x,y,z)\geq 0$ si y s\'{o}lo si $x+y+z\geq 0$ y  $f(x,y,z)\leq 0$
si y s\'{o}lo si $x+y+z\leq 0$.
}

\solcpreliminares{Para demostrar todos los incisos de este ejercicio utilice 
la identidad  (\ref{abccubicas}).
}

%%%%%% 46
\ejercpreliminares{Muestre que para n'umeros reales $x$, $y$, se tienen las 
siguientes identidades:

\noindent $(i)$\;
$(x+y)^{5}-(x^{5}+y^{5})=5xy(x+y)(x^{2}+xy+y^{2})$.

\noindent $(ii)$\; $(x+y)^{7}-(x^{7}+y^{7})=7xy(x+y)(x^{2}+xy+y^{2})^{2}$.
}

\solcpreliminares{Desarrolle ambos lados de las identidades.
}

%%%%%% 47

\ejercpreliminares{Sean $x$, $y$ y $z$ n\'umeros reales tales que $x\neq y$ y
$$x^2(y+z)=y^2(x+z)=2.$$
Determine el valor de $z^2(x+y)$.
}

\solcpreliminares{Se tiene que
\begin{eqnarray*}
0 & = & x^2(y+z)-y^2(x+z)= xy(x-y)+(x^2-y^2)z\\
  & = & (x-y)(xy+xz+yz).
\end{eqnarray*}
Como $x\neq y$, se tiene que $xy+xz+yz=0$. Multiplicando por $x-z$ se obtiene
\begin{eqnarray*}
0 &=& (x-z)(xy+xz+yz)= xz(x-z)+(x^2-z^2)y\\
&=& x^2(y+z)-z^2(x+y),
\end{eqnarray*}
de donde $z^2(x+y)=x^2(y+z)=2$.
}


%%%%%%%%%%1.48
\ejercpreliminares{Encuentre las soluciones reales $x$, $y$, $z$ y $w$
del sistema de ecuaciones
\begin{eqnarray*}
x  + y +z  & = & w \\
\frac{1}{x}+ \frac{1}{y}+\frac{1}{z} & = &\frac{1}{w}.
\end{eqnarray*}
}

\solcpreliminares{Vea que $(x+y+z)(xy+yz+zx)=xyz$ y por la ecuaci'on (\ref{productodetres}) tenemos que
$(x+y)(y+z)(z+x)=0.$ Luego, las soluciones $(x,y,z,w)$ son de la forma: $(x,-x,z,z)$,
 $(x,y,-y,x)$ y $(x,y,-x,y)$, con $x$, $y$ y $z$ n'umeros reales diferentes de cero.
}

%%%%%%%%%% 49

\ejercpreliminares{Sean $x$, $y$ y $z$ n\'{u}meros reales diferentes
de cero que cumplen las condiciones $x+y+z\neq 0$ y 
$\frac{1}{x}+\frac{1}{y}+\frac{1}{z}=\frac{1}{x+y+z}$. Muestre que, para 
cualquier n'umero entero impar $n$, se cumple que
\begin{equation*}
\dfrac{1}{x^{n}}+\dfrac{1}{y^{n}}+\dfrac{1}{z^{n}}=\dfrac{1}{x^{n}+y^{n}+z^{n}}.
\end{equation*}
}

\solcpreliminares{Por la ecuaci'on 
(\ref{productodetres}), se tiene que la condici'on es equivalente a  $(x+y)(y+z)(z+x)=0$.
Luego, un factor es cero, digamos $x+y=0$. Entonces, como $n$ es impar,
$x^{n}+y^{n}=0$, lo mismo que $\frac{1}{x^{n}}+\frac{1}{y^{n}}=0$.
}



