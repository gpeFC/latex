\section{N'umeros}
\label{numeros}

\noindent Consideramos que el lector est'a familiarizado con el conjunto de 
n'umeros que se utilizan para contar. A este conjunto se le conoce como el 
conjunto de  n'umeros naturales y se denota por
$\nn$, es decir, \index{N'umeros naturales}
$$
     \nn = \{1,2,3,\dots\}.
$$
En este conjunto estamos acostumbrados a realizar dos operaciones, la suma y la multiplicaci'on, 
entendiendo con esto que si sumamos o multiplicamos dos n'u\-meros del conjunto obtenemos otro n'umero natural. 
A estas operaciones las conocemos como la suma  (o adici'on) y la multiplicaci'on (o producto). En \linebreak algunos libros el 0 se considera tambi'en como un n'umero natural, sin embargo, en este libro no, pero convenimos que 0 es tal que $n+0=n$, para todo n'umero natural $n$. 

Ahora, supongamos que deseamos resolver la ecuaci'on $x+a = 0$, con $a\in \nn$, es decir, encontrar una $x$ para la cual la igualdad anterior se cumpla.  
Esta ecuaci'on no tiene soluci'on en el conjunto de los n'umeros naturales $\nn$, por lo cual necesitamos definir un conjunto de n'umeros que incluya al 
conjunto de n'umeros $\nn$ y a sus negativos.  Es decir, necesitamos extender el conjunto de los n'umeros
$\nn$ para que este tipo de ecuaciones tengan soluci'on en el nuevo conjunto.   A este  conjunto lo llamamos el conjunto de los  n'umeros enteros y lo  denotamos por $\zz$, es decir,\index{N'umeros enteros}
$$
     \zz = \{\dots, -3 ,-2, -1, 0, 1, 2, 3,\dots\}.
$$
En este conjunto tambi'en hay dos operaciones, la suma y la
multiplicaci'on,  que satisfacen las siguientes propiedades.


\begin{propiedades}
\begin{description}
\item[$(a)$] La suma y la multiplicaci'on de n'umeros enteros son operaciones 
conmutativas. 
Esto es, si $a,\,b\in \zz$, entonces
$$
       a+b= b+a\quad \text{y}\quad ab = b a.
$$
\item[$(b)$] La suma y el producto de n'umeros enteros son operaciones 
asociativas. Esto es, 
si $a,\,b\,
\text{y}\; c\in \zz$, entonces
$$
       (a+b)+c= a+(b+c) \quad \text{y}\quad (ab)c= a(bc).
$$
\item[$(c)$] Existe en $\zz$ un elemento neutro para la suma, el n'umero 0. 
Es decir,
si $a\,\in \zz$, entonces
$$
       a+0= 0+a=a.
$$
\item[$(d)$] Existe en $\zz$  un elemento neutro para la multiplicaci'on, el 
n'umero 1.
Es decir,
si $a\,\in \zz$, entonces
$$
       a1= 1a=a.
$$
\item[$(e)$] Para cada $a\,\in\zz$ existe  su inverso aditivo que se
denota por $-a$. Esto es,
$$
       a+(-a)= (-a)+a=0.
$$
\item[$(f)$] En $\zz$, el producto se  distribuye con respecto a la suma.  
Es decir, si
$a$, $b$ y $c \in \zz$, entonces
\begin{equation*}
       a(b+c) =   ab+ac.
\end{equation*}
\end{description}
\end{propiedades}


\vei

Notemos que la existencia del inverso aditivo nos permite resolver cualquier 
ecuaci'on del tipo mencionado, es decir, $x+a= b$, donde $a$ y $b$ son 
n'umeros enteros. Sin embargo,
no existe necesariamente un n'umero entero $x$ que resuelva la ecuaci'on 
$qx=p$, con
$p$ y $q$ n'umeros enteros, por lo que nuevamente surge la necesidad de 
extender el  conjunto de n'umeros.  Consideramos ahora el  conjunto de los 
n'umeros 
racionales, que denotamos  como $\qq$, es decir,  \index{N'umeros racionales}
$$
     \qq = \left \{\frac{p}{q} \ | \ p\in \zz \ \text{y} \ q\in 
\zz\backslash \{0\}  \right \}.
$$
En general, para trabajar con los n'umeros racionales $\frac{p}{q}$ pedimos 
que $p$ y $q$ no tengan factores primos comunes, es decir, que sean primos 
relativos, esto lo denotamos como $(p,q)=1$.
En el conjunto de n'umeros racionales  tambi'en existen las operaciones de 
suma y  producto, las cuales cumplen las mismas propie\-dades que los n'umeros 
enteros.  Adem'as, en el producto existe otra  propiedad: la existencia del inverso multiplicativo.

\begin{propiedad}
Si $\frac{p}{q}\in \qq$, con $p\neq 0$ y $(p,q)=1$, entonces existe un 
'unico n'umero, $\frac{q}{p}\in \qq$, llamado el inverso multiplicativo 
de $\frac{p}{q}$ tal que
$$
               \frac{p}{q}\cdot \frac{q}{p} = 1.
$$
\end{propiedad}
\index{N'umeros racionales}


\noindent Con esta nueva propiedad tenemos garant'ia de poder resolver cualquier ecuaci'on de la forma $q x = p$. Sin embargo, existen n'umeros que no podemos escribir como cociente de dos n'umeros enteros, por ejemplo, si queremos resolver la ecuaci'on  $x^2-2=0$, 'esta no tiene soluci'on en el conjunto de los 
n'umeros $\qq$. Las soluciones de la ecuaci'on son 
$x=\pm\sqrt{2}$ y mostramos que $\sqrt{2}$ no est'a en $\qq$.

\begin{proposicion}
El n'umero $\sqrt{2}$ no es un n'umero racional.
\end{proposicion}

\demostracion{Supongamos lo contrario, es decir, que $\sqrt{2}$ es un n'umero racional, entonces lo podemos escribir como $\sqrt{2}=\frac{p}{q}$, donde $p$ y $q$ no tienen factores comunes. Elevando al cuadrado de ambos lados tenemos que
$2=\frac{p^2}{q^2}$, es decir, $2q^2= p^2$. Esto quiere decir, que $p^2$ es un 
n'umero par, pero entonces el mismo $p$ es par.  Pero si $p$ es par, digamos de la forma $p=2m$,  entonces $2q^2= (2m)^2= 4m^2$. Dividiendo entre 2 ambos lados de  la ecuaci'on tenemos que $q^2=2m^2$, esto es, $q^2$ es par y entonces $q$ es tambi'en par. As'i, $p$ y $q$ son pares,  contradiciendo el hecho de que $p$ y $q$ no tienen factores comunes.  Por lo tanto, $\sqrt{2}$ no es un n'umero 
racional.
}

Podemos dar una representaci'on geom'etrica de los n'umeros racionales como puntos sobre una recta, que llamamos la 
{\bf recta num'erica}\index{Recta!num'erica}. Una recta la podemos recorrer en dos sentidos, a uno de ellos le llamamos  
{\bf sentido positivo} y al otro {\bf sentido negativo}. Una vez convenido cual es el sentido positivo decimos que tenemos 
una {\bf recta orientada}\index{Recta!orientada}. Por ejemplo, podemos convenir que el sentido positivo es el que va de 
izquierda a derecha. Si consideramos dos puntos $O$ y $U$ en la recta, le daremos la misma orientaci'on al segmento que a 
la recta que lo contiene. Esto es, tomando el punto $O$ como el 0 y 
$U$  a la derecha de 'el, decimos que el segmento $OU$ se recorre en el sentido positivo. Si el punto $U$ representa al 1, 
llamamos a $OU$ un segmento orientado y unitario. As'i, podemos ir colocando, hacia la derecha, todos los n'umeros enteros 
positivos a lo largo de la recta separados, cada dos consecutivos,  una distancia $OU$. Para representar los n'umeros enteros 
negativos basta que hagamos lo mismo pero ahora iniciando en $O$ y recorriendo la recta en sentido negativo.

El n'umero racional de la forma $\frac{p}{q}$, lo definimos como el segmento orientado
$\frac{p}{q}  OU$
que es el segmento que se obtiene al sumar $p$ veces la $q$-'esima parte del segmento $OU$.  Con m'as precisi'on, hacemos lo siguiente:

$(a)$ Dividimos en $q$ partes iguales el segmento $OU$. Para esto, trazamos una
recta auxiliar por $O$  y sobre ella tomamos $q$ puntos $W_1$, $\dots$, $W_q$, 
donde dos consecutivos est'an separados una distancia $OW_1$. Ahora, se une 
$W_q$ con $U$ y por cada uno de los puntos $W_j$ se traza una recta paralela a
$UW_q$, los puntos de corte de las paralelas con $OU$ ser'an los puntos de 
divisi'on de $OU$ en $q$ partes iguales. Si $V$ es el punto de corte de la
paralela $UW_q$ por $W_1$, se tendr'a que $V$ es el punto que representa al 
n'umero $\frac{1}{q}$ (note que $OV$ tiene la misma orientaci'on de $OU$).  
Consideramos  tambi'en $V^\prime$ el punto sim'etrico, con respecto a $O$, de 
$V$.  En la siguiente figura, hemos tomado $q=4$

\centerline{
\psset{unit=1.5cm}
\begin{pspicture}(0,-2)(0,2.5)
       \psline(-4,0)(4,0)
        \psline(-2,-2)(2,2)
       	\psline(.3,.28)(.5,0)
        \psline(.58,.58)(1,0)
        \psline(.88,.88)(1.5,0)
         \psline(1.18,1.18)(2,0)
	\psline(-.5,0)(-.3,-.28)
       	\psline(-1,0)(-.58,-.58)
        \psline(-1.5,0)(-.88,-.88)
        \psline(-2,0)(-1.18,-1.18)
        \psdot(2,0)
        \psdot(4,0)
        \psdot(.5,0)
       \psdot(-.5,0)	
       \psdot(1,0)
        \psdot(1.5,0)
       \psdot(.58,.58)
       \psdot(.88,.88)
       \psdot(1.18,1.18)
       \psdot(-2,0)
       \psdot(-1.18,-1.18)
        \psdot(.3,.3)
        \psdot(-4,0)
         \psdot(-1,0)
        \psdot(-1.5,0)
        \psdot(-1,0)
         \psdot(-.3,-.3)
        \psdot(-.58,-.6)
       \psdot(-.88,-.9)
        \psdot(-1.5,0)
        \rput(0,-.3){$O$}
        \rput(0,.2){$0$}
	\rput(2,-.2){$U$}
	\rput(2.1,.2){$1$}
        \rput(4,.2){$2$}
       \rput(.5,-.2){$V$}
       \rput(-.5,.2){$V^\prime$}
       \rput(.2,.5){$W_1$}
       \rput(1.05,1.4){$W_q$}
     \end{pspicture}
}


$(b)$ Si $p$ es un n'umero entero no negativo, tomamos
$$
         OP=\underbrace{OV+OV+\cdots+OV}_{p\,\text{ veces}} = p \cdot OV.
$$
El segmento $OP$ es, por definici'on, $\frac{p}{q} OU$.  En la siguiente 
figura marcamos al punto $P$, con $p=6$ y $q=4$

\centerline{
\psset{unit=1.5cm}
\begin{pspicture}(0,-1)(0,1)
       \psline(-4,0)(4,0)
        \psdot(2,0)
        \psdot(4,0)
        \psdot(.5,0)	
       \psdot(1,0)
        \psdot(1.5,0)
       \psdot(-2,0)
       \psdot(2.5,0)
        \psdot(3,0)
       \psdot(0,0)
        \rput(0,-.3){$O$}
        \rput(0,.3){$0$}
	\rput(2,-.3){$U$}
        \rput(-2,-.3){$U^\prime$}
	\rput(2,.3){$1$}
        \rput(4,.3){$2$}
       \rput(3,-.3){$P$}
     \end{pspicture}
}

$(c)$ Si $p$ es negativo, sea $p^\prime$ el n'umero entero positivo tal que $p=-p^\prime$. Tomamos entonces
$$
         OP=\underbrace{OV^\prime+OV^\prime+\cdots+OV^\prime}_{p^\prime\,\text{ veces}} = p^\prime OV^\prime=(-p^\prime) OV=p \cdot OV.
$$
El segmento $OP$ es, por definici'on, $\frac{p}{q} OU$.  Como $OU$ es el segmento unitario, a este punto lo denotamos simplemente como  $\frac{p}{q}$.

\vei

Con esta representaci'on de los n'umeros racionales, tenemos que todo n'umero racional est'a representado por un punto en la recta num'erica, pero hay puntos en la recta num'erica que no representan ning'un  n'umero racional.  Por ejemplo, determinemos en la recta num'erica el n'umero $\sqrt{2}$, que ya vimos que no es un n'umero racional.  Si tomamos un tri'angulo rect'angulo con catetos de longitud 1, entonces, por el teorema de Pit'agoras, la hipotenusa de este tri'angulo mide $\sqrt{2}$.  Si tomamos un comp'as y trazamos una circunferencia de radio $\sqrt{2}$ y centro en $0$, el punto donde la circunferencia intersecta a la parte 
positiva de la recta num'erica es el punto en la recta num'erica que 
corresponde a $\sqrt{2}$.

\centerline{
\psset{unit=1.5cm}
\begin{pspicture}(0,-.5)(3,2.5)
       \psline(-1,0)(4,0)
        \psdot(2,0)
        \psdot(2,2)
        \psline(2,0)(2,2)(0,0)	
       \psarc(0,0){2.83}{0}{45}
        \rput(0,-.3){$O$}
        \rput(0,.3){$0$}
	\rput(2,-.3){$U$}
        \rput(1,-.2){$1$}
	\rput(2.3,1){$1$}
        \rput(.8,1.1){$\sqrt{2}$}
       \rput(2.83,-.2){$\sqrt{2}$}
     \end{pspicture}
}
\noindent Un punto de la recta num'erica que no corresponda 
a un n'umero racional representar'a a un n'umero irracional y al conjunto de 
los n'umeros irracionales lo denotamos por $\ii$.
\index{N'umeros irracionales}

A la uni'on de estos dos conjuntos, le llamamos el conjunto de los n'umeros 
reales y lo denotamos como $\rr$, es decir, 
$\rr=\qq\cup\ii$.\index{N'umeros reales}

\noindent El conjunto de los n'umeros $\rr$ contiene al conjunto de los n'umeros naturales, al  de los n'umeros enteros y al de los n'umeros racionales.  De hecho, tenemos las siguientes contenciones $\nn\subset\zz\subset\qq\subset\rr$.

\vei

Dados dos puntos en la recta num'erica que sabemos representan a dos n'umeros reales, podemos localizar al punto que es la suma de ellos  de la siguiente manera:  si $P$ y $Q$ son dos puntos sobre la recta y $O$ es el origen,  la suma ser'a la suma de  los segmentos dirigidos $OP$ y $OQ$, como se muestra en la siguiente figura.


\centerline{
\psset{unit=1.5cm}
\begin{pspicture}(0,-1)(-1,2)
       \psline(-4,0)(2,0)
        \psdot(-2,0)
        \psdot(-3,0)	
        \psdot(0,0)
        \psdot(1,0)
        \psline(-3,.7)(-2.9,.8)(-2.6,.8)(-2.5,.9)(-2.4,.8)(-2.1,.8)(-2,.7)		
      \psline(-3,.5)(-2.9,.6)(-1.6,.6)(-1.5,.7)(-1.4,.6)(-.1,.6)(0,.5)
	\psline(-3,1.3)(-2.9,1.4)(-1.1,1.4)(-1,1.5)(-.9,1.4)(.9,1.4)(1,1.3)
      \psline(0,.5)(.1,.6)(.4,.6)(.5,.7)(.6,.6)(.9,.6)(1,.5)		
        \rput(-3,-.3){$O$}
        \rput(-3,.3){$0$}
	\rput(-2,-.3){$P$}
        \rput(0,-.3){$Q$}
        	\rput(-2.5,1.15){$OP$}
	\rput(.5,.9){$OP$}
        	\rput(-1.5,.9){$OQ$}
	\rput(1,-.3){$P+Q$}
	\rput(-1,1.7){$OP+OQ$}
     \end{pspicture}
}

Asimismo, podemos encontrar el punto que representa el producto de dos puntos $P$ y $Q$  sobre la recta num'erica como sigue.  Consideremos una recta auxiliar que ser'a una  copia de la recta real con el mismo origen $O$.  Localizamos en la recta auxiliar la unidad $U$ y el punto $Q$. Por $Q$ trazamos la recta paralela a $UP$ la cual intersecta a la recta real en $R$.


Como los tri'angulos  $ORQ$ y $OPU$ son semejantes tenemos que
$\frac{OR}{OP}=\frac{OQ}{OU}$ por lo que $OR\cdot OU=OP\cdot OQ$,
de donde $OR$ representa al producto de $P$ y $Q$.

\centerline{
\psset{unit=1.5cm}
\begin{pspicture}(0,-1.3)(2,2.5)
       \psline(-1.5,0)(3,0)
       \psline(-1,-1)(2,2)       
       \psline(.58,.58)(1,0)
       \psline(1.18,1.18)(2,0)
       \psdot(1,0)
       \psdot(.58,.58)
       \psdot(2,0)
       \psdot(1.18,1.18)
       \rput(1,1.35){$Q$}
       \rput(0,-.2){$O$}
       \rput(2,-.2){$R$}
       \rput(.5,.8){$U$}
       \rput(1,-.2){$P$}
     \end{pspicture}
}


Con esto podemos localizar la suma y el producto de cualesquiera dos n'u\-meros  
reales sobre la recta num'erica, sin importar si son n'umeros racionales o 
irracionales.  

Al igual que en el conjunto de los n'umeros enteros,  
las operaciones en el conjunto de n'umeros reales cumplen  todas las  
propiedades mencionadas.

\begin{propiedades}
\begin{description}
\item[$(a)$] La suma de dos n'umeros reales es un n'umero real.
\item[$(b)$] La suma de dos n'umeros reales es conmutativa.
\item[$(c)$] La suma es asociativa.
\item[$(d)$] El n'umero $0$ es el neutro aditivo. Es decir, $x+0=x$, para todo $x \in \rr$.
\item[$(e)$] Todo n'umero real $x$ tiene un inverso aditivo. Es decir, existe un n'umero real que se denota por $-x$ y cumple que  $x+(-x)= 0$.
\item[$(f)$] El producto de dos n'umeros reales es un n'umero real.
\item[$(g)$] El producto  es conmutativo.
\item[$(h)$] El producto  es asociativo.
\item[$(i)$] El n'umero $1$ es el neutro multiplicativo. Es decir, $x\cdot 1=x$, para todo $x \in \rr$.
\item[$(j)$] Todo n'umero real $x$ distinto de $0$, tiene inverso multiplicativo.  Es decir, existe  un n'umero real que se denota por $x^{-1}$, tal que
$x\cdot x^{-1} =1$.
\item[$(k)$] El producto distribuye a la suma, es decir, si
$x$, $y$, $z\in \rr$, entonces
$$
      x \left(y+z\right ) = x\cdot y+ x\cdot z.
$$
\end{description}
\end{propiedades}



\noindent En los n'umeros enteros tenemos un orden. Con esto queremos se\~nalar que  dados dos n'umeros enteros $a$ y $b$ podemos decir cual es el mayor de ellos.  Decimos que $a$ {\bf es mayor que} $b$ si $a-b$ es un n'umero natural, en s'imbolos tenemos\index{N'umeros!mayor que}
$$
       a > b \;\;\text{si y s'olo si}\;\; a-b \in\nn.
$$
Esto es equivalente a decir que $a-b>0$.

\noindent En general, la notaci'on $a>b$ es equivalente a $b<a$.  La expresi'on $a\geq b$ significa que $a>b$ o $a=b$. An'alogamente, $a\leq b$ significa que $a < b$ o $a=b$.


\begin{propiedades}
Si $a$ es un n'umero entero, se cumple una y solamente una de las condiciones siguientes:
\begin{description}
\item[$(a)$] $a>0$,
\item[$(b)$] $a=0$,
\item[$(c)$] $a<0$.
\end{description}
\end{propiedades}

En los n'umeros racionales y en los n'umeros reales tambi'en hay un orden.   El orden en los n\'umeros reales
nos permitir\'a comparar dos n\'umeros y
decidir cual de ellos es mayor o bien si son iguales. A fin de evitar
justificaciones tediosas, asumiremos que en los n\'umeros reales hay un
conjunto $P$ que llamamos el conjunto de n\'umeros positivos, y
simb\'olicamente escribimos $x>0$, para decir que un n\'umero $x$
est\'a en $P$. En la representaci'on geom'etrica de los n'umeros reales, el conjunto $P$ en la 
recta num'erica es, de las dos partes en que $O$ ha dividido a la recta, la parte que contiene a $U$ (el $1$). Resaltamos que se cumplen las   propiedades siguientes.

\ve

\begin{propiedad}
Cada n\'umero real $x$ tiene una y s\'olo una de
las siguientes caracter\'{\i}sticas:
\label{tricotomia}
\begin{description}
\item[$ (a)$] $x=0$.
\item[$(b)$] $x \in P$ (esto es $x>0$).
\item[$(c)$]$-x\in P$ (esto es $-x>0$).
\end{description}
\end{propiedad}

\begin{propiedad} 
Sean $x$, $y$ n'umeros reales.
\begin{description}
\item[$(a)$] Si $x$, $y \in P$, entonces $x+y\in P$
\noindent (en s\'{\i}mbolos $x>0$, $y>0\Rightarrow x+y>0$).
\label{propsuma}
\item[$(b)$] Si $x$, $y\in P$, entonces $xy\in P$
(en s\'{\i}mbolos $x>0$, $y>0\Rightarrow xy>0$).
\label{propproducto}
\end{description}
\end{propiedad}

\noindent Ahora podemos definir la relaci\'on $x$ {\bf es mayor que}
\index{N'umeros!mayor que} $y$,
si $x-y\in P$ (en s\'{\i}mbolos $x>y$). An\'alogamente, $x$ {\bf es menor
que}\index{N'umeros!menor que} $y$, si $y-x\in P$ (en s\'{\i}mbolos $x<y$).
Observemos que
$x<y$ es equivalente a $y>x$. Definimos tambi\'en, $x$ {\bf es menor o igual
que}\index{N'umeros!menor o igual que} $y$, si $x<y$ o $x=y$,
(en s\'{\i}mbolos $x\leq y$).

\noindent Denotamos al conjunto $P$ de
n\'umeros reales positivos por $\rr^{+}$.

\begin{ejemplo}  Sean $x$, $y$, $z$ n'umeros reales.
\begin{description}
\item[$(a)$] Si $x<y$, entonces $x+z<y+z$.
\item[$(b)$] Si $x<y$ y $z>0$, entonces $xz<yz$.
\label{ordenenlosreales}
\end{description}
\end{ejemplo}

\noindent En efecto, para mostrar $(a)$ tenemos que $x+z<y+z$ si y s'olo si
$(y+z)-(x+z)>0$ si y s'olo si $y-x>0$ si y s'olo si $x<y$. Para ver $(b)$, tenemos
que $x<y$ implica $y-x>0$ y como $z>0$, resulta que $(y-x)z>0$, luego
$yz-xz>0$ y entonces $xz<yz$.

\ve 


%%%1
\ejercpreliminares{Muestre las siguientes afirmaciones:
\label{maspormas}
\begin{description}
\item[$ (i)$] Si $a<0$, $b<0$, entonces $ab>0$.

\ven

\item[$ (ii)$] Si $a<0$, $b>0$, entonces $ab<0$.

\ven

\item[$ (iii)$] Si $a<b$, $b<c$, entonces $a<c$.

\ven

\item[$ (iv)$] Si $a<b$, $c<d$, entonces $a+c<b+d$.

\ven


\item[$ (v)$] Si $a>0$, entonces $a^{-1}>0$.

\ven

\item[$ (vi)$]  Si $a<0$, entonces $a^{-1} <0$.
\end{description}
}

\solcpreliminares{$(i)$ Si $a<0$, entonces $-a>0$. Use
tambi\'en que $(-a)(-b)=ab$. 
$(ii)$ $(-a)b>0$. 
$(iii)$ $a<b\Leftrightarrow b-a>0$, use ahora la propiedad \ref{propsuma}. 
$(iv)$ Use la propiedad \ref{propsuma}. 
$(v)$ $aa^{-1}=1>0$. 
$(vi)$ Si $a<0$, entonces $-a>0$.
}


%%%%%%% 2
\ejercpreliminares{Sean $a$, $b$ n\'{u}meros reales.  Muestre que, si 
$a+b$, $a^2+b$ y $a+b^2$ son n'umeros racionales y $a+b\neq 1$, entonces 
$a$ y $b$ son n'umeros racionales.
}

\solcpreliminares{Observe que si $a^2+b-(a+b^2)\in \qq$, entonces
$(a-b)(a+b-1)\in \qq$ y como $a+b-1\in \qq\setminus\{0\}$, entonces 
$(a-b)\in \qq$. Luego, si $a+b\in\qq$ y $a-b\in \qq$, entonces $2a$ y 
$2b$ est'an en $\qq$. Por lo tanto, $a$ y $b$ son n'umeros racionales.
}

%%%%%%%%% 3
\ejercpreliminares{Sean $a, b$ n\'{u}meros reales tales que 
$a^2+b^2$, $a^3+b^3$ y $a^4+b^4$ son  n'umeros racionales.  Muestre que 
$a+b$, $ab$ son tambi'en  n'umeros racionales.

}

\solcpreliminares{Si $a=0$ o $ b=0$ el resultado es claro. 
Suponga entonces que $ab\neq 0$. Como $(a^2+b^2)^2-(a^4+b^4)=2a^2b^2$, 
se tiene que $a^2b^2\in\qq$.  Note que $a^6+b^6=(a^2+b^2)^3-
3a^2b^2(a^2+b^2)\in\qq$, por lo que $(a^3+b^3)^2-(a^6+b^6)=2a^3b^3\in\qq$. 
Luego, 
$$
  ab= \frac{a^3b^3}{a^2b^2}\in \qq\quad \text{y}\quad 
 a+b= \frac{a^3+b^3}{a^2+b^2-ab}\in\qq.
$$
}

%%%%%%%%% 4
\ejercpreliminares{$(i)$\, Demuestre que si $p$ es un n'umero primo, 
entonces $\sqrt{p}$ es un n'umero irracional.

$(ii)$\, Demuestre que si $m$ es un n'umero entero positivo que no 
es cuadrado perfecto, entonces $\sqrt{m}$ es un n'umero irracional.
}

\solcpreliminares{$(i)$ Suponga que $\sqrt{p}$ no es un n'umero irracional, 
es decir, \linebreak $\sqrt{p}=\frac{m}{n}$, donde $m$, $n$ son n'umeros enteros 
con $(m,n)=1$, es decir, $m$ y $n$ primos relativos. Elevando al cuadrado, 
se tiene $p n^2=m^2$, esto es, $p$ divide a $m^2$, entonces $p$ divide a $m$.  
Por lo que  $m=ps$ y  $pn^2 = p^2 s^2$ implican que   $n^2 = p s^2$, lo cual 
garantiza que $p$ divide a $n^2$ y entonces divide a  $n$. Luego, $p$ divide 
a $m$ y a $n$ contradiciendo el hecho de que  $m$ y $n$ son primos relativos. 

$(ii)$ Suponga que $\sqrt{m}$ no es un n'umero irracional, 
es decir, $\sqrt{m}=\frac{r}{s}$, donde $r$, $s$ son n'umeros enteros 
con $(r,s)=1$. Elevando al cuadrado 
se tiene $m s^2=r^2$. Como $m$ no es un cuadrado perfecto, tiene un factor 
de la forma $p^{\alpha}$, donde $p$ es un n'umero primo y $\alpha$ es un 
entero positivo impar. Entonces, $p^{\alpha}$ divide a $r^2$ lo que implica que 
el primo $p$ aparece un n'umero par de veces en la descomposici'on de factores 
de $r^2$. Como $r$ y $s$ son primos relativos, $p$ no divide a $s$, de donde
$p$ aparece un n'umero impar de veces como factor de $m s^2$, lo cual es una 
contradicci'on.
}

%%%%5
\ejercpreliminares{Demuestre que existen una infinidad de parejas de n'umeros 
irracionales $a$, $b$ tales que $a+b=ab$ y adem'as este n'umero es entero.
}

\solcpreliminares{Si $a+b=ab=n$, entonces $b=n-a$ y $n=a(n-a)$.  
La 'ultima ecuaci'on es equivalente a $a^2-na+n=0$ y resolviendo se obtiene que
$$
 a=\frac{n\pm\sqrt{n^2-4n}}{2},
\quad\text{de donde }\quad b=\frac{n\mp\sqrt{n^2-4n}}{2}.
$$
Para $n\geq 5$, se tiene que 
$ (n-3)^2 <n^2-4n <(n-2)^2,$
por lo que $\sqrt{n^2-4n}$ es un  n'umero irracional, y entonces $a$ y 
$b$ son n'umeros irracionales.
}

%%%%% 6
\ejercpreliminares{Si los coeficientes de 
$$
   a x^2+b x+c =0
$$
son n'umeros enteros impares, entonces las ra'ices de la ecuaci'on 
no pueden ser n'umeros racionales.
}

\solcpreliminares{Suponga que $\frac{m}{n}$ es  ra'iz,  con $(m,n)=1$. 
Entonces $m$ y $n$ no pueden ser  ambos pares.  Por otro lado, como 
$a \left (\frac{m}{n}\right )^2+b \left (\frac{m}{n}\right )+c =0$,
se tiene que $ a m^2+b mn+c n^2 =0$. El lado derecho de la  'ultima 
ecuaci'on  es par y el izquierdo siempre es impar. Si  $m$ y $n$ son 
impares, los tres sumandos del lado izquierdo son impares. Ahora bien,  
si uno de ellos es par y el otro impar, entonces dos sumandos son pares, 
el tercero impar y la suma es impar nuevamente. Esta contradicci'on 
implica que la ecuaci'on no puede tener ra'ices racionales. 

\vei 

\ssolucion{El discriminante $b^2 - 4 ac$ deber'a ser un cuadrado 
perfecto. Pero como $a$, $b$ y $c$ son impares, se puede mostrar que
$b^2 - 4 ac\equiv 5$ $\mod 8$. Sin embargo, los cuadrados de 
n'umeros impares s'olo dejan residuo 1 m'odulo 8.}
}

%%%%%%%%%% 7
\ejercpreliminares{Muestre que para n'umeros reales positivos 
$a$ y $b$, con $\sqrt{b} < a$, se tiene que
$$
\sqrt{a+\sqrt{b}}=\sqrt{\frac{a+\sqrt{a^{2}-b}}{2}}+
\sqrt{\frac{a-\sqrt{a^{2}-b}}{2}}.
$$
}

\solcpreliminares{Sea $u=a+\sqrt{b}$ y $v=a-\sqrt{b}$, entonces
\begin{eqnarray*}
 \sqrt{a+\sqrt{b}}& = & \sqrt{u}=\frac{\sqrt{u}+\sqrt{v}}{2} + 
\frac{\sqrt{u}-\sqrt{v}}{2}\\
& = & \sqrt {\frac{(\sqrt{u}+\sqrt{v})^2}{4}} + 
\sqrt {\frac{(\sqrt{u}-\sqrt{v})^2}{4}}\\
& = & \sqrt {\frac{\frac{u+v}{2}+\sqrt{uv}}{2}} 
+\sqrt {\frac{\frac{u+v}{2}-\sqrt{uv}}{2}}\\
& = & \sqrt {\frac{\frac{a+\sqrt{b}+a-\sqrt{b}}{2}+
\sqrt{a^2-b}}{2}} +\sqrt {\frac{\frac{a+\sqrt{b}+a-
\sqrt{b}}{2}-\sqrt{a^2-b}}{2}}\\
& = & \sqrt {\frac{a+\sqrt{a^2-b}}{2}} 
+\sqrt {\frac{a-\sqrt{a^2-b}}{2}},
\end{eqnarray*}
como se quer'ia probar. 
}

%%%%%%% 8
\ejercpreliminares{Para n\'{u}meros positivos $a$ y $b$ 
encuentre el valor de:

\vei

\noindent $(i)$ $\sqrt{a\sqrt{a\sqrt{a\sqrt{a \dots}}}}.$ 
\qquad \qquad \qquad 
$(ii)$ $\sqrt{a\sqrt{b\sqrt{a\sqrt{b\dots}}}}.$
}

\solcpreliminares{$(i)$ Sea $x=\sqrt{a\sqrt{a\sqrt{a\sqrt{a \dots}}}}$, 
entonces $x^2=a\sqrt{a\sqrt{a\sqrt{a\sqrt{a \dots}}}}$, de donde $x^2=ax$. 
Factorizando, $x(x-a)=0$.  Por lo tanto, como $a$ es positivo la soluci'on 
es $x=a$.

\vei

\ssolucion{Podemos dar otra soluci'on utilizando series.  Tenemos que 
$$
     x=a^{\frac{1}{2}}a^{\frac{1}{4}}a^{\frac{1}{8}}\ldots=a^{\frac{1}{2}+\frac{1}{4}+\frac{1}{8}
+\cdots}= a,
$$
ya que $\sum_{j=1}^{\infty} \frac{1}{2^j}=1$, ver la secci'on 
\ref{seriesdepotencia}.
}

$(ii)$ Sea $x=\sqrt{a\sqrt{b\sqrt{a\sqrt{b\dots}}}}$, entonces 
$x^2=a\sqrt{b\sqrt{a\sqrt{b\sqrt{a\dots}}}}$, de donde 
$x^4=a^2 b x$. Como  $x\neq 0$, $x^3=a^2 b$. Entonces $x=\sqrt[3]{a^2b}$.

\vei

\ssolucion{Podemos tambi'en hacer otra soluci'on utilizando series.  
Tenemos que 
$$
     x=a^{\frac{1}{2}+\frac{1}{8}+\frac{1}{32}+\cdots}\, 
b^{\frac{1}{4}+\frac{1}{16}+\frac{1}{64}+\cdots}=a^{\frac{2}{3}}\,b^{\frac{1}{3}},
$$
ya que $\sum_{j=1}^{\infty} \frac{1}{2^{2j}}=\frac{1}{3}$ y 
$\sum_{j=0}^{\infty} \frac{1}{2^{2j+1}}=\frac{2}{3}$, ver la secci'on 
\ref{seriesdepotencia}. 
}
}

%%%%% 9
\ejercpreliminares{(Rumania, 2001) Sean  $x$, $y$ y $z$ n'umeros 
reales distintos de cero tales que
$xy$, $yz$ y $zx$ son n'umeros racionales. Muestre que:

$(i)$ $x^2+y^2+z^2$ es un n'umero racional.

$(ii)$ Si $x^3+y^3+z^3$ es un n'umero racional distinto de cero, 
entonces 
$x$, $y$ y $z$ son n'umeros racionales.
}

\solcpreliminares{$(i)$ Si $xy$, $yz$  y $zx$ est'an en $\qq$, 
entonces $\frac{(xy)(zx)}{yz}=x^2\in\qq$.  An'alogamente, 
$y^2$, $z^2$ $\in \qq$. Por lo tanto, $x^2+y^2+z^2\in\qq$.

$(ii)$ Por  $(i)$ se tiene que  $(x^2)^2+(xy)y^2+(xz) 
z^2=x(x^3+y^3+z^3)\in \qq$, luego  $x\in \qq$.  
An'alogamente, $y$, $z$ $\in \qq$. 
}

%%%%%%% 10
\ejercpreliminares{(Rumania, 2011) Sean $a$, $b$ n\'{u}meros reales 
positivos y distintos, tales que $a-\sqrt{ab}$ y $b-\sqrt{ab}$ 
son ambos n'umeros racionales. Muestre que $a$ y $b$ son  
n'umeros racionales.
}

\solcpreliminares{Como $a-\sqrt{ab}=a\left( 1-\frac{\sqrt{b}}{\sqrt{a}}\right)$,
bastar'a ver que  $1-\frac{\sqrt{b}}{\sqrt{a}}$ es un  n'umero 
racional distinto de cero para asegurar que $a$ es un  n'umero racional. 

\noindent Pero $ \frac{b-\sqrt{ab}}{a-\sqrt{ab}}=
\frac{\sqrt{b}(\sqrt{b}-\sqrt{a})}{\sqrt{a}(\sqrt{a}-\sqrt{b})}= 
-\frac{\sqrt{b}}{\sqrt{a}}$ 
es un  n'umero  racional diferente de $-1$ (ya que $a\neq b$), 
luego $1-\frac{\sqrt{b}}{\sqrt{a}}$ es un  n'umero racional 
distinto de 0. An'alogamente, $b$ es un  n'umero racional.
}











\index{Sistema decimal}
\noindent El sistema decimal es un sistema posicional en el que
cada d'igito toma un valor de acuerdo a su posici'on con relaci'on al punto
decimal. Esto es, el d'igito se multiplica por una potencia de 10.  
Para el d'igito de las unidades, o sea, el
d'igito que est'a inmediatamente a la izquierda del punto decimal, lo  tenemos que multiplicar por $10^n$, con $n=0$.  El d'igito de las decenas lo multiplicamos por $10^1=10$. El exponente
aumenta de uno en uno conforme nos movemos a la izquierda y disminuye de uno en uno conforme nos movemos a la derecha. Por ejemplo,
$$
    87325.31= 8\cdot 10^4+7 \cdot 10^3+3\cdot 10^2+2\cdot 10^1+5\cdot 10^0+3\cdot 10^{-1}+1\cdot 10^{-2}.
$$

\vei

En general, todo n'umero real puede escribirse como una expansi'on decimal 
infinita de la siguiente manera
$$
                 b_m\dots b_1b_0.a_1a_2a_3\dots,
$$
donde los $b_i$ y los $a_i$ est'an en $\{0,1,\dots,9\}$. Los puntos suspensivos
de la derecha significan que despu'es del punto decimal podemos tener una 
infinidad de d'igitos, as'i el n'umero $ b_m\dots b_1b_0.a_1a_2a_3\dots$, 
representa al n'umero real 
$$
b_m\cdot 10^m+\cdots+b_1 \cdot 10^1+b_0\cdot 10^0+a_1\cdot 
10^{-1}+a_2\cdot 10^{-2}+\cdots.
$$
Por ejemplo,
$$
\begin{array}{lcccccl}
    \frac{1}{3} & = & 0.3333\dots, & & \frac{3}{7} & = & 
0.428571428571\dots,\\[.2cm]
    \frac{1}{2} & = & 0.50000\dots, & & \sqrt{2} & = & 1.4142135\dots.
\end{array}
$$
\noindent Con esta notaci'on podemos tambi'en distinguir entre los n'umeros racionales y los irracionales. Los n'umeros racionales son aquellos para los cuales la expansi'on decimal es finita o bien infinita pero en alg'un momento se hace peri'odica, como por ejemplo en $\frac{34}{275}=0.123636 \dots$, que se hace peri'odica de periodo 2 a partir del tercer d'igito. En cambio, para los n'umeros irracionales, la expansi'on decimal es infinita, pero no s'olo eso, sino que adem'as nunca se hace peri'odica.

\vei

\noindent Al igual  que en la representaci'on decimal en base 10, podemos representar a los n'umeros enteros en cualquier base. Si $m$ es un n'umero entero positivo, para encontrar su representaci'on en base $b$ lo
escribimos como suma de potencias de $b$, es decir, $m=a_r b^r+\cdots+a_1 b+a_0$. Los n'umeros
enteros que aparecen como coeficientes de las potencias de $b$ en la
representaci'on deben  ser menores que $b$.

\begin{observacion}
Cuando escribimos un n'umero en una base distinta de 10, ponemos como sub'indice
la base en la que est'a escrito el n'umero, por ejemplo, $1204_7$ significa que el
n'umero 1204 es un n'umero en base 7.
\end{observacion}


Veamos el siguiente ejemplo.
\begin{ejemplo}
?`En qu'e base el n'umero $221$ es un factor de $1215$?
\end{ejemplo}

El n'umero 1215 en base $a$ se escribe como $a^3 + 2a^2 + a + 5$ y el
n'umero 221 en base $a$ es $ 2a^2 + 2a + 1$.  Ahora bien, si dividimos 
$a^3 + 2a^2 + a + 5$ entre $ 2a^2 + 2a + 1$ obtenemos que
$$
   a^3 + 2a^2 + a + 5 = (2a^2 + 2a + 1)\left (\frac{1}{2}a+\frac{1}{2}\right)+
                        \left(-\frac{1}{2}a + \frac{9}{2}\right).
$$
Como $1215_a$ tiene que ser un m'ultiplo de $221_a$, el residuo
$\left(-\frac{1}{2}a + \frac{9}{2}\right)$ tiene que ser 0 y
$\left(\frac{1}{2}a + \frac{1}{2}\right )$ tiene que ser un n'umero entero.  Por lo
tanto, $a=9$.


\ve 
\ve



%%%%%%%%%%11

\ejercpreliminares{Escriba en la forma $\frac{m}{n}$, con $n$ y $m$ 
n'umeros enteros positivos, a los siguiente n'umeros reales:

$(i)$  $0.11111\dots$.  

$(ii)$ $1.14141414\dots$.
}

\solcpreliminares{Para resolver $(i)$, defina $x=0.111\dots$, 
entonces $10x=1.11\dots$. Restando la primera ecuaci'on de la 
segunda se tiene que $9x=1$, luego,  $x=\frac{1}{9}$.

\noindent $(ii)$ Sea $x=1.141414\ldots$, entonces 
$100x=114.141414\ldots$. Restando la pri\-mera ecuaci'on de la 
segunda se tiene que $99x=113$, de donde $x=\frac{113}{99}$. }

\ejercpreliminares{$(i)$\; Muestre que $121_b$ es un cuadrado 
perfecto en cualquier base $b\geq 2$.

$(ii)$\; Determine el menor valor de $b$ para el cual  $232_b$ 
es un cuadrado perfecto.
}

\solcpreliminares{$(i)$\; Primero observe que 
$121_b=(1\times b^2)+ (2\times b) +1 =(b+1)^2$ entonces 
$121_b$ es un cuadrado perfecto en cualquier base $b\geq 2$.

$(ii)$\; Como $232_b=2b^2+3b+2$ debe ser cuadrado y como 3 es 
uno de sus d'igitos, $b\geq 4$.

Para $b=4$, $232_4=46$, para $b=5$, $232_5=67$, para $b=6$, 
$232_6=92$ y para $b=7$, $232_7=121$. Luego, $b=7$ es el menor 
entero positivo tal que $232_b$ es un cuadrado perfecto. 
}

%%%%%% 12
%\ejercpreliminares{Sea $b \geq 2$ un entero positivo.
%(a) Muestre que para que un entero $N$, escrito en base $b$, sea igual a la suma del cuadrado %de sus d'igitos, es necesario que  $N = 1$ o que 
%$N$ tenga solamente dos d'igitos.
%(b) Give a complete list of all integers not exceeding 50 that, relative to
%some base $b$, are equal to the sum of the squares of their digits.
%(c) Show that for any base $b$ the number of two-digit integers that are
%equal to the sum of the squares of their digits is even.
%(d) Show that for any odd base $b$ there is an integer other than 1 that is
%equal to the sum of the squares of its digits.
%}

%%%%%%%% 13
\ejercpreliminares{(IMO, 1970) Sean $a$, $b$ y $n$ n'umeros enteros mayores 
que 1. Sean $A_{n-1}$ y $A_n$ dos n'umeros escritos en el sistema num'erico en 
base  $a$ y, $B_{n-1}$ y $B_n$ dos n'umeros escritos en el sistema n'umerico 
en base  $b$. Estos n'umeros 
est'an relacionados de la siguiente forma,
\begin{eqnarray*}
        A_n = x_nx_{n-1} \dots x_0, & & A_{n-1} = x_{n-1}x_{n-2}\dots x_0,\\
        B_n = x_nx_{n-1} \dots x_0, & & B_{n-1} = x_{n-1}x_{n-2}\dots x_0,
\end{eqnarray*}
con 
$x_n\neq 0$ y $x_{n-1}\neq 0$.  Muestre que  $a > b$ si y s'olo si
$$
                  \frac{A_{n-1}}{A_{n}} <\frac{B_{n-1}}{B_{n}}.
$$
}

\solcpreliminares{Suponga que $a>b$. Entonces para todos los 
enteros $0\leq k\leq n$, $x_nx_ka^nb^k\geq x_nx_kb^na^k$, con 
igualdad solamente cuando $k=n$ o $x_k=0$. En particular, se 
tiene una desigualdad estricta para $k=n-1$. En resumen, esto 
se convierte en
$$
              x_n a^n\sum_{k=0}^n x_kb^k > x_nb^n \sum_{k=0}^n x_k a^k
$$
o 
$$
             \frac{ x_n a^n}{A_n}> \frac{x_nb^n}{B_n}.
$$
Esto implica que
$$
             \frac{ A_{n-1}}{A_n}= 1-\frac{ x_n a^n}{A_n} < 
1-\frac{ x_n b^n}{B_n} =\frac{B_{n-1}}{B_n}.
$$ 
Por otro lado, si $a=b$, entonces evidentemente 
$\frac{ A_{n-1}}{A_n}= \frac{ B_{n-1}}{B_n}$ y si 
$a<b$, por lo que se demostr'o antes, se tiene que, 
$ \frac{ A_{n-1}}{A_n}> \frac{ B_{n-1}}{B_n}$. Por lo 
tanto, $\frac{A_{n-1}}{A_{n}} <\frac{B_{n-1}}{B_{n}}$ si y 
s'olo si $a>b$.
}




