\section{Parte entera y parte fraccionaria}
\label{parteentera}

Dado cualquier n'umero $x \in \rr$, algunas veces es 'util considerar
el n'umero entero m'ax $\{ k \in \zz \ | \ k \leq x\}$, es decir, 
el mayor entero menor o igual que $x$. A este n'umero lo denotamos por 
$\lfloor x \rfloor$ y se le conoce como 
{\bf la parte entera de $x$}.\index{N'umero!parte entera}

De la definici'on anterior tenemos las siguientes propiedades.

\begin{propiedades} Sean $x, y \in \rr$, $n \in \nn$ y $m \in \zz$. 
Entonces se tiene que:
\begin{description}\label{parteentera}
\item[$(a)$] $x-1<\lfloor x \rfloor \leq x < \lfloor x \rfloor +1$.
\item[$(b)$]  $x$ es entero si y s'olo si $\lfloor x \rfloor=x$.
\item[$(c)$] $\lfloor x +m \rfloor= \lfloor x \rfloor +m$.
\item[$(d)$] $\left\lfloor \frac{\lfloor x \rfloor}{n}\right\rfloor =
 \left\lfloor \frac{x}{n}\right\rfloor$.
\item[$(e)$]  $\lfloor x \rfloor + \lfloor y \rfloor \leq
\lfloor x +y \rfloor \leq \lfloor x \rfloor + \lfloor y \rfloor +1$.
\end{description}
\end{propiedades}

\demostracion{Las primeras tres propiedades son inmediatas.  

$(d)$ Al dividir $\lfloor x\rfloor$ entre $n$ tenemos que $\lfloor x\rfloor= 
a n +b$, para un n'umero entero $a$ y para un n'umero entero $b$ tal que 
$0\leq b<n$.

Por un lado, tenemos que 
 $\left\lfloor \frac{\lfloor x \rfloor}{n}\right\rfloor =
 \left\lfloor \frac{an+b}{n}\right\rfloor= a+\left\lfloor 
\frac{b}{n}\right\rfloor= a$. Por otro lado, como $x=\lfloor x\rfloor+c$, 
con $0 \leq c < 1$, tenemos que
$\left\lfloor \frac{ x}{n}\right\rfloor =
 \left\lfloor \frac{an+b+c}{n}\right\rfloor= 
a+\left\lfloor \frac{b+c}{n}\right\rfloor= a$, ya que $b+c <n-1+1=n$.
Luego, la igualdad es v'alida. 

$(e)$ Como  $x=\lfloor x\rfloor+a$  y $y=\lfloor y\rfloor+b$ con $0\leq a, b < 1$, entonces
 $\lfloor x +y \rfloor =\lfloor x\rfloor + \lfloor  y\rfloor+\lfloor a+b \rfloor$ por la propiedad $(c)$. Las desigualdades se siguen de observar que si $0\leq a, b <1$ entonces $0\leq \lfloor a+b\rfloor \leq 1$.
}


\begin{ejemplo}
Para todo n'umero real $x$ se cumple que 
$$\lfloor x \rfloor + \left\lfloor x + \frac{1}{2}\right\rfloor
- \lfloor 2x \rfloor=0.$$
\label{exismasunmediomenosdosexis}
\end{ejemplo}

Si se hace $n=\lfloor x \rfloor$, $x$ se puede expresar de la forma 
$x=n+a$ con $0 \leq a < 1$, luego se tiene que
\begin{eqnarray*}  
\lfloor x \rfloor + \left\lfloor x + \frac{1}{2}\right\rfloor
- \lfloor 2x \rfloor & = & n + \left\lfloor n+a + \frac{1}{2}
\right\rfloor - \lfloor 2(n+a) \rfloor\\
& = & n + n + 
\left\lfloor a + \frac{1}{2}\right\rfloor -2n - \left\lfloor 2a \right\rfloor\\
& = & \left\lfloor a + \frac{1}{2}\right\rfloor - \left\lfloor 2a \right\rfloor,
\end{eqnarray*}
donde la segunda igualdad se sigue por la propiedad $(c)$. Ahora, si 
$0 \leq a < \frac{1}{2}$, entonces $\left\lfloor a + \frac{1}{2}\right\rfloor=
\lfloor 2a \rfloor=0$, mientras que en el caso $\frac{1}{2} \leq a < 1$, se 
tiene que $\left\lfloor a + \frac{1}{2}\right\rfloor=
\lfloor 2a \rfloor=1$.


\begin{ejemplo}
Si $n$ y $m$ son enteros positivos sin factores
comunes, entonces
$$
\left\lfloor \dfrac{n}{m}\right\rfloor +\left \lfloor \dfrac{2n}{m}\right 
\rfloor + \left\lfloor\dfrac{3n}{m}\right \rfloor +\cdots +
\left\lfloor\dfrac{(m-1)n}{m}\right \rfloor =\dfrac{(m-1)(n-1)}{2}.
$$
\label{sumadepartesenteras}
\end{ejemplo}

Consideramos en el plano cartesiano la recta que pasa por el origen y el punto
$(m,n)$. Como $m$ y $n$ son primos relativos, sobre el segmento de recta que 
une los puntos $(0,0)$ y $(m,n)$ no hay otro punto de coordenadas enteras. 

\centerline{
	\psset{unit=.7cm}
	\begin{pspicture}(0,-1)(6,6.5)
	\psaxes[labels=none]{->}(8.5,5.5)
	\psline(1,1)(7,1)(7,4)(1,4)(1,1)
	\psline(0,0)(8,5)
	\psdot(8,5)
         \pscircle[fillstyle=solid,fillcolor=black](8,0){.1}
         \pscircle[fillstyle=solid,fillcolor=black](0,5){.1}
	\rput{0}(1,-.5){$1$}
	\rput{0}(-.5,1){$1$}
	\rput{0}(2,-.5){$2$}
         \rput{0}(-.5,2){$\vdots$}
	\rput{0}(3.5,-.5){$\cdots$}
	\rput{0}(7,-.5){$m-1$}
	\rput{0}(-1,4){$n-1$}
        \rput{0}(8.6,.5){$A=(m,0)$}
	\rput{0}(8,5.5){$B=(m,n)$}
	\rput{0}(-1.7,5){$C=(0,n)$}
	\rput{0}(-.3,-.3){$O$}
        \end{pspicture}
}

La ecuaci'on de la recta es $y=\frac{n}{m} x$ y pasa por los puntos 
$(j, \frac{n}{m} j)$, con $j=1,\ldots, (m-1)$, y adem'as $\frac{n}{m} j$ 
no es entero. 
El n'umero $\left \lfloor \frac{n}{m} j\right \rfloor$ es igual al n'umero de 
puntos de coordenadas enteras que est'an sobre la recta $x=j$ y, entre las 
rectas $y=\frac{n}{m} x$ y $y=1$ inclusive. La suma es igual al n'umero 
de puntos de coordenadas enteras  en el interior del tri'angulo $OAB$, 
por simetr'ia es 
igual a la mitad de los puntos de coordenadas enteras dentro del rect'angulo 
$OABC$. Como 
la cantidad de  puntos de coordenadas enteras dentro del rect'angulo 
es $(n-1)(m-1)$, tenemos que
$$
\left\lfloor \dfrac{n}{m}\right\rfloor +\left \lfloor \dfrac{2n}{m}\right 
\rfloor + \left\lfloor\dfrac{3n}{m}\right \rfloor +\cdots +
\left\lfloor\dfrac{(m-1)n}{m}\right \rfloor =\dfrac{(m-1)(n-1)}{2}.
$$

\vei

\begin{observacion}
Como el lado derecho de la 'ultima igualdad es sim'etrico en $m$ y $n$, 
entonces
$$\left\lfloor \dfrac{n}{m}\right\rfloor +\left \lfloor \dfrac{2n}{m}\right 
\rfloor + \cdots +
\left\lfloor\dfrac{(m-1)n}{m}\right \rfloor =
\left\lfloor \dfrac{m}{n}\right\rfloor +\left \lfloor \dfrac{2m}{n}\right 
\rfloor + \cdots +
\left\lfloor\dfrac{(n-1)m}{n}\right \rfloor.$$
\end{observacion}

\ve

Para un n'umero real $x$, consideremos tambi'en el n'umero
$\{x\}=x - \lfloor x \rfloor$, al cual  llamamos la 
{\bf parte fraccionaria de $x$}\index{N'umero!parte fraccionaria},
y  cumple las siguientes propiedades. 

\begin{propiedades} Sean $x, y \in \rr$ y $n \in \zz$. Entonces se tiene que:
\begin{description}
\item[$(a)$] $0 \leq \{x\} <1$.

\item[$(b)$] $x =  \lfloor x \rfloor  + \{x\}$.

\item[$(c)$] $\{x+ y\} \leq \{x\} + \{y\} \leq  \{x +y\} +1$.

\item[$(d)$] $\{ x +n \}= \{x\}$.

\end{description}
\end{propiedades}


\ve

%%% 18
\ejercpreliminares{Para cualesquiera n'umeros reales $a$, $b>0$, 
se tiene que
$$
 \lfloor 2a \rfloor + \lfloor 2b \rfloor \geq \lfloor a \rfloor 
+\lfloor b \rfloor +\lfloor a+b\rfloor.
$$
}

\solcpreliminares{Por el ejemplo \ref{exismasunmediomenosdosexis}, 
$\lfloor 2a\rfloor = \lfloor a\rfloor + \lfloor a+\frac{1}{2}\rfloor$ 
y $\lfloor 2b\rfloor = \lfloor b\rfloor + \lfloor b+\frac{1}{2}\rfloor$, 
luego la desigualdad a demostrar es equivalente a
$$
  \lfloor a\rfloor + \left\lfloor a+\frac{1}{2}\right\rfloor+ 
\lfloor b\rfloor + \left\lfloor b+\frac{1}{2}\right\rfloor \geq 
\lfloor a\rfloor + \lfloor b\rfloor+ \lfloor a+b\rfloor,
$$
de donde bastar'a mostrar que $ \left\lfloor a+\frac{1}{2}\right\rfloor+
\left\lfloor b+\frac{1}{2}\right\rfloor \geq  \lfloor a+b\rfloor$.

Sean $a = n+y$, $b = m+x$, con $n, m\in \zz$ y $0\leq x, y < 1$. Entonces 
$0\leq x+ y <2$ y $a+b=n+m+x+y$.  Se tienen dos casos:

$(i)$ Si $1\leq x + y < 2$, entonces  $\lfloor a+ b\rfloor= n+m+1$ 
y al menos uno de los n'umeros $x$ o $y$ es mayor o igual que 
$\frac{1}{2}$.  Suponga que $x\geq \frac{1}{2}$. Entonces  
$\lfloor b+\frac{1}{2}\rfloor= \lfloor m+x+\frac{1}{2}\rfloor = m+1$, 
por lo que $ \lfloor a+\frac{1}{2}\rfloor+\lfloor b+
\frac{1}{2}\rfloor\geq m+n+1=\lfloor a+b\rfloor$.  


$(ii)$ Si $0\leq x + y < 1$, entonces  $\lfloor a+ b\rfloor= n+m$ y  
$ \lfloor a+\frac{1}{2}\rfloor+\lfloor b+\frac{1}{2}\rfloor\geq m+n 
=\lfloor a+b\rfloor$.
}


%%%%%%% 19
\ejercpreliminares{Encuentre  los valores de $x$ que cumplen 
la siguiente ecuaci'on:

$(i)$\,   $\lfloor x \lfloor x \rfloor  \rfloor =1$.

$(ii)$\,   $||x|-\lfloor x \rfloor| = \lfloor |x|- \lfloor x\rfloor\rfloor$.
}

\solcpreliminares{$(i)$\,   Se tiene que $\lfloor x \lfloor x \rfloor  
\rfloor =1$ si y s'olo si
$1\leq x\lfloor x \rfloor < 2$.  Si $x=m+y$, con $m\in\zz$ y $0\leq y <1$, 
entonces $1\leq m^2+my < 2$. Observe que   $m=0$ es imposible, al igual que 
$m\geq 2$ o $m\leq -2$.  Luego, resta ver qu'e sucede si $m=1$ o $m=-1$. 

Si $m=1$, entonces $1\leq 1+y<2$, de donde $0\leq y <1$ y entonces cualquier 
$x$ en el intervalo $[1,2)$ cumple la ecuaci'on.  Si $m=-1$, entonces, como
\linebreak 
$1\leq m^2+my < 2$, se  tiene que  $1\leq 1- y<2$, de donde $0\leq - y < 1$ 
y entonces  $y=0$ y $x=-1$.  
Por lo tanto, los n'umeros que cumplen la ecuaci'on son $x=-1$ y  
$x\in [1,2)$.

\vei 

$(ii)$\,   Como $\lfloor x \rfloor \leq x \leq |x|$, se tiene que,
$ |x|- \lfloor x \rfloor \geq 0$, por lo que\linebreak  $||x|-\lfloor x\rfloor |= 
|x| -\lfloor x\rfloor$.  Por otro lado,  por la propiedad $(c)$ en 
\ref{parteentera} se\linebreak  tiene que, $\lfloor |x|-\lfloor x\rfloor\rfloor 
= \lfloor |x|\rfloor -\lfloor x\rfloor$. Utilizando las 'ultimas igualdades 
la ecuaci'on se convierte en $|x| - \lfloor x\rfloor = \lfloor |x|\rfloor - 
\lfloor x\rfloor$ que es equivalente a $|x| = \lfloor |x| \rfloor$, luego 
$|x|$ es un n'umero entero y los valores de $x$ que cumplen la ecuaci'on 
son todos los n'umeros enteros.
}



%%% 20
\ejercpreliminares{Encuentre las soluciones del sistema de ecuaciones
\begin{eqnarray*}
x+\lfloor y \rfloor + \{z\} & = & 1.1,\\
\lfloor x \rfloor + \{y\}+z & = & 2.2,\\
\{x\} + y + \lfloor z \rfloor & = & 3.3.
\end{eqnarray*}
}

\solcpreliminares{Sume las tres ecuaciones para obtener que 
$2x+2y+2z=6.6$, luego $x+y+z=3.3$. Reste a esta 'ultima igualdad las ecuaciones
originales, para obtener $\{y\} + \lfloor z \rfloor =2.2$, 
$\{x\} + \lfloor y \rfloor =1.1$,
$\{z\}+\lfloor x \rfloor =0$.
La primera ecuaci'on da $\lfloor z \rfloor = 2$, $\{y\}=0.2$, la segunda 
$\lfloor y \rfloor=1$, $\{x\}=0.1$ y, la tercera $\lfloor x \rfloor=0$ y
$\{z\}=0$. Por lo tanto, la soluci'on es $x=0.1$, $y=1.2$ y $z=2$.
}


%%%%%%% 21
\ejercpreliminares{(Canad'a, 1987) Para cada n'umero natural $n$, 
muestre que
$$
         \lfloor \sqrt{n} +\sqrt{n+1} \rfloor = \lfloor \sqrt{4n+1} \rfloor=\lfloor \sqrt{4n+2} \rfloor=\lfloor \sqrt{4n+3} \rfloor. 
$$
}

\solcpreliminares{Se tiene que  $\sqrt{n} +\sqrt{n+1} < \sqrt{4n+2} $ si y 
s'olo si $2n+1+\sqrt{4n^2+4n} < 4n+2$, que es equivalente a $\sqrt{4n^2+4n} < 
2n+1$. Elevando al cuadrado nuevamente, la 'ultima desigualdad es equivalente 
a $4n^2+4n < 4n^2+4n+1$.  Esto prueba que $\sqrt{n} +\sqrt{n+1} < \sqrt{4n+2}$,
entonces $\lfloor \sqrt{n} +\sqrt{n+1}\rfloor 
\leq \lfloor \sqrt{4n+2}\rfloor$.  

\vei

Suponga que, para alg'un n'umero entero positivo $n$, 
$\lfloor \sqrt{n} +\sqrt{n+1}\rfloor \neq \lfloor \sqrt{4n+2}\rfloor$. Sea 
$q= \lfloor \sqrt{4n+2}\rfloor$, entonces $\sqrt{n} +\sqrt{n+1}< q \leq 
\sqrt{4n+2}$.  Elevando al cuadrado, se  obtiene que $2n+1+ \sqrt{4n^2+4n} < 
q^2\leq 4n+2$ o lo que es equivalente $\sqrt{4n^2+4n} < q^2-2n - 1 \leq 2n+1$.
Elevando al cuadrado nuevamente se\linebreak obtiene que  
$4n^2+4n < (q^2-2n - 1)^2 
\leq 4n^2+4n+1= (2n+1)^2$. Como no \linebreak existe un cuadrado entre dos 
enteros 
consecutivos, se tiene que $q^2-2n - 1 = 2n+1$ o que
$q^2 = 4n+2$, que es equivalente a decir que $q^2\equiv 2 \mod 4$.  Pero esto 
'ultimo es una contradicci'on, ya que todo cuadrado es congruente a 0 o a 1 
m'odulo $4$. Por lo tanto,  se tiene la igualdad.

\vei

Muestre ahora que, $\lfloor \sqrt{4n+1} \rfloor=\lfloor \sqrt{4n+2} 
\rfloor=\lfloor \sqrt{4n+3} \rfloor$. 

Para la  primera igualdad, suponga que 
existe una $n$ tal que $m=\lfloor \sqrt{4n+1} \rfloor < m+1=
\lfloor \sqrt{4n+2} \rfloor$, luego $m \leq \sqrt{4n+1} < m+1 \leq 
\sqrt{4n+2}$, por lo que 
$m^2 \leq 4n+1 < (m+1)^2\leq 4n+2$. 

Entonces, como $4n+1$ y 
$4n+2$ son dos n'umeros enteros consecutivos y, como $(m+1)^2 > 4n+1$, se 
tiene que $(m+1)^2 = 4n+2$ y nuevamente se ha encontrado un cuadrado que 
tiene residuo $2$ al dividirlo entre $4$, lo cual es imposible. Para la 
segunda igualdad, proceda de la misma forma. 
}




