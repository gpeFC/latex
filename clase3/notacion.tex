
\ve

\noindent Utilizamos la siguiente notaci'on est'andar:
$$
\begin{array}{lcl}
\mathbb{N} & \hspace{1in} & \mbox{los n'umeros enteros positivos o n'umeros
naturales} \\
\mathbb{Z} & \hspace{1in} & \mbox{los n'umeros enteros} \\
\mathbb{Q} & \hspace{1in} & \mbox{los n'umeros racionales} \\
\mathbb{Q}^+ & \hspace{1in} & \mbox{los n'umeros racionales positivos} \\
\mathbb{R}  & \hspace{1in} & \mbox{los n'umeros reales}\\
\mathbb{R}^+  & \hspace{1in} & \mbox{los n'umeros reales positivos}\\
\mathbb{I} & \hspace{1in} & \mbox{los n'umeros irracionales} \\
\mathbb{C} & \hspace{1in} & \mbox{los n'umeros complejos} \\
\mathbb{Z}_p & \hspace{1in} & \mbox{es \{0,1,\dots,p-1\} con la suma y producto m'odulo $p$.} \\
\Leftrightarrow & \hspace{1in} & \mbox{si y s\'olo si}\\
\Rightarrow & \hspace{1in} & \mbox{implica}\\
a \in A & \hspace{1in} & \mbox{el elemento} \ a \ \mbox{pertenece al conjunto}
\ A\\
A \subset B & \hspace{1in} & A \ \mbox{es un subconjunto de} \ B\\
|x| & \hspace{1in} & \mbox{valor absoluto del n'umero real} \ x\\
|z| & \hspace{1in} & \mbox{m'odulo del n'umero complejo} \ z\\
\{ x\} & \hspace{1in} & \mbox{la parte fraccionaria de un n'umero real} \ x\\
\lfloor x \rfloor & \hspace{1in} & \mbox{la parte entera de un n'umero real} \ x\\
\left[ a,b \right] & \hspace{1in} & \mbox{el conjunto de n'umeros reales} \ x \
\mbox{tal que} \ a \leq x \leq b\\
(a,b) & \hspace{1in} & \mbox{el conjunto de n'umeros reales} \ x \ 
\mbox{tal que} \ a < x < b\\
P(x) & \hspace{1in} & \mbox{el polinomio $P$ en la variable}\ x \\
\text{grad} (P) & \hspace{1in} & \mbox{grado del polinomio $P(x)$} \\
f:\left[a, b\right] \to \mathbb{R} & \hspace{1in} & \mbox{la funci'on} \ f \ 
\mbox{definida en} \
\left[a,b\right] \ \mbox{con valores en} \ \mathbb{R}\\
f'(x) & \hspace{1in} & \mbox{la derivada de la funci'on} \ f(x)\\
f''(x) & \hspace{1in} & \mbox{la segunda derivada de la funci'on} \ f(x)\\
f^{(n)}(x) & \hspace{1in} & \mbox{la $n$-'esima derivada de la funci'on} \ f(x)\\
f(x)^n & \hspace{1in} & \mbox{la potencia $n$-'esima de la funci'on} \ f(x)\\
f^n(x) & \hspace{1in} & \mbox{la $n$-'esima iteraci'on de la funci'on} \ f(x)\\
\Delta f(x) & \hspace{1in} & \mbox{el operador diferencia de} \ f(x)\\
\mbox{det} \ A & \hspace{1in} & \mbox{el determinante de la matriz} \ A\\
\sum_{i=1}^n a_i & \hspace{1in} & \mbox{la suma} \ a_1+a_2+\cdots+a_n\\
\prod_{i=1}^n a_i & \hspace{1in} & \mbox{el producto} \ a_1 \cdot a_2 \cdots a_n\\
\prod_{i\neq j} a_i & \hspace{1in} & \mbox{el producto de todos los} \ a_1, a_2, \dots, a_n \ \mbox{excepto} \ a_j\\
\end{array}
$$

$$
\begin{array}{lcl}
\max \{a, b, \dots \} & \hspace{1in} & \mbox{el m'aximo valor
entre} \ a, b, \dots \\
\min \{ a, b, \dots \} & \hspace{1in} & \mbox{el m'inimo valor
entre} \ a, b, \dots \\
\sqrt{x} & \hspace{1in} & \mbox{la ra'iz cuadrada del n'umero real positivo} \ x\\
\sqrt[n]{x} & \hspace{1in} & \mbox{la} \ n-\mbox{'esima ra'iz 
del n'umero real positivo} \ x\\
\exp{x}=e^x & \hspace{1in} & \mbox{la funci'on exponencial}\\
\displaystyle{\sum_{\text{c\'iclica}}} f(a,b, \dots) & \hspace{1in} & \mbox{representa la suma de la 
 funci'on} \ f \ \mbox{evaluada en todas las}\\
& & \mbox{permutaciones c'iclicas de las variables} \ a, b, \dots
\end{array}
$$


\vei

Utilizamos la siguiente notaci'on referente a los problemas:

\ven

$$
\begin{array}{lcl}
\mbox{AMC} & \hspace{1in} & \mbox{Competencia Americana de 
Matem'aticas (por sus siglas en}\\ 
& \hspace{1in} &  \mbox{ingl'es)}\\
\mbox{APMO} & \hspace{1in} & \mbox{Olimpiada de la Cuenca del Pac'ifico (por
sus siglas en ingl'es)}\\
\mbox{IMO} & \hspace{1in} & \mbox{Olimpiada Internacional de 
Matem'aticas (por sus siglas en}\\ 
& \hspace{1in} & \mbox{ingl'es)}\\
\mbox{MEMO} & \hspace{1in} & \mbox{Olimpiada Matem'atica de 
Europa Central (por sus siglas en}\\ 
& \hspace{1in} & \mbox{ingl'es) }\\
\mbox{OMCC} & \hspace{1in} & \mbox{Olimpiada Matem'atica de Centroam'erica y El Caribe }\\
\mbox{OIM} & \hspace{1in} & \mbox{Olimpiada Iberoamericana de Matem'aticas }\\
\mbox{OMM} & \hspace{1in} & \mbox{Olimpiada Mexicana de Matem'aticas }\\
\mbox{(pa'is, a\~no)} & \hspace{1in} & \mbox{problema que corresponde a la 
olimpiada de matem'aticas}\\ 
& \hspace{1in} & \mbox{celebrada en ese pa'is, en ese a\~no, en alguna de las
etapas} 
\end{array}
$$
