
\section{Productos notables}
\label{cap1sec2}

\index{Producto notable!dos variables}

\noindent El 'area de un cuadrado es el cuadrado de la  longitud de su
lado.  Si sus lados miden $a+b$ entonces el 'area es $(a+b)^2$, pero el 'area de este cuadrado la podemos dividir en cuatro rect'angulos como se muestra en la figura. 

\index{Binomio!cuadrado}

\centerline{
       \psset{unit=1cm}
	\begin{pspicture}(0,0)(6,4.5)
	\psline(1,1)(1,4)(4,4)(4,1)(1,1)
	\psline(1,2)(4,2)
	\psline(3,1)(3,4)
	\psline(1,.9)(1.1,.8)(1.9,.8)(2,.7)(2.1,.8)(2.9,.8)(3,.9)
	\psline(3,.9)(3.1,.8)(3.4,.8)(3.5,.7)(3.6,.8)(3.9,.8)(4,.9)
	\psline(.9,1)(.8,1.1)(.8,1.4)(.7,1.5)(.8,1.6)(.8,1.9)(.9,2)
	\psline(.9,2)(.8,2.1)(.8,2.9)(.7,3)(.8,3.1)(.8,3.9)(.9,4)
	\rput{0}(2,.4){\large $a$}
	\rput{0}(3.5,.4){\large $b$}
	\rput{0}(2,1.5){\large $ab$}
	\rput{0}(3.5,3){\large $ab$}
	\rput{0}(2,3){\large $a^2$}
	\rput{0}(3.5,1.5){\large $b^2$}
	\rput{0}(.4,1.5){\large $b$}
	\rput{0}(.4,3){\large $a$}
	\end{pspicture}
}

Luego, la suma de las 'areas de  los cuatro rect'angulos ser'a igual al 'area del cuadrado, es decir, 
\begin{equation}
\label{ecuac1.1.1}
                         (a+b)^2=a^2+ab+ab+b^2=a^2+2ab+b^2.
\end{equation}

\noindent Veamos ahora c'omo obtener geom'etricamente el cuadrado de la diferencia $a-b$, donde $b\leq a$.   El problema es ahora
encontrar el 'area de un cuadrado de lado $a-b$.  

\centerline{
        \psset{unit=1cm}
	\begin{pspicture}(0,0)(6,4.5)
	\psline(1,1)(1,4)(4,4)(4,1)(1,1)
\psline[fillstyle=solid,fillcolor=lightgray,linewidth=1pt](1,1)(1,2)(3,2)(3,4)(4,4)(4,1)(1,1)
	\psline[linewidth=1pt](1,2)(4,2)
	\psline[linewidth=1pt](3,1)(3,4)
	\psline(4.1,1)(4.2,1.1)(4.2,1.4)(4.3,1.5)(4.2,1.6)(4.2,1.9)(4.1,2)
	\psline(.9,1)(.8,1.1)(.8,2.4)(.7,2.5)(.8,2.6)(.8,3.9)(.9,4)
	\rput{0}(.4,2.5){\large $a$}
	\rput{0}(4.5,1.5){\large $b$}
	\rput{0}(2,1.5){$(a-b)b$}
	\rput{90}(3.5,3){$(a-b)b$}
	\rput{0}(2,3){\large $(a-b)^2$}
	\rput{0}(3.5,1.5){\large $b^2$}
	\end{pspicture}
}

\noindent En la figura observamos que
el 'area de un cuadrado de lado $a$ es igual a la suma de las 'areas de los cuadrados de lados $(a-b)$ y $b$, m'as el 'area de dos rect'angulos iguales de lados $b$ y $(a-b)$. Esto es, $a^2=(a-b)^2+b^2+(a-b)b+b(a-b)$, de donde 				      
\begin{equation}
\label{ecuac1.1.2}
(a-b)^2=a^2-2ab+b^2.
\end{equation}

\noindent Para encontrar el 'area de la parte sombreada de la siguiente figura, 

\centerline{
      \psset{unit=1cm}
	\begin{pspicture}(0,0)(6,4.5)
	\psline(1,1)(1,4)(4,4)(4,1)(1,1)
  \psline[fillstyle=solid,fillcolor=lightgray,linewidth=1pt](1,1)(3,1)(3,2)(4,2)
   (4,4)(1,4)(1,1)
	\psline[linewidth=1pt](1,2)(4,2)
	\psline[linewidth=1pt](3,1)(3,4)
	\psline(4.1,1)(4.2,1.1)(4.2,1.4)(4.3,1.5)(4.2,1.6)(4.2,1.9)(4.1,2)
	\psline(4.1,2)(4.2,2.1)(4.2,2.9)(4.3,3)(4.2,3.1)(4.2,3.9)(4.1,4)
	\psline(.9,1)(.8,1.1)(.8,2.4)(.7,2.5)(.8,2.6)(.8,3.9)(.9,4)
	\psline(1,.9)(1.1,.8)(1.9,.8)(2,.7)(2.1,.8)(2.9,.8)(3,.9)
	\psline(3,.9)(3.1,.8)(3.4,.8)(3.5,.7)(3.6,.8)(3.9,.8)(4,.9)
	\rput{0}(.4,2.5){\large $a$}
	\rput{0}(4.5,1.5){\large $b$}
	\rput{0}(5,3){\large $a - b$}
	\rput{0}(2,.4){\large $a - b$}
	\rput{0}(3.5,.4){\large $b$}
	\rput{0}(2,1.5){ $(a-b)b$}
	\rput{90}(3.5,3){$(a-b)b$}
	\rput{0}(2,3){\large $(a-b)^2$}
	\rput{0}(3.5,1.5){\large $b^2$}
	\end{pspicture}
}
\noindent observamos que la suma de las 'areas de los rect'angulos 
que la forman es \linebreak $a(a-b)+b(a-b)$ y si 
factorizamos esta  suma tenemos que
\begin{equation}
\label{ecuac1.1.3}
 a(a-b)+b(a-b)=(a+b)(a-b),
\end{equation}
pero es equivalente al 'area del cuadrado grande menos el 'area del cuadrado
chico, es decir, 
\begin{equation}
\label{ecuac1.1.4}
(a+b)(a-b)= a^2-b^2.
\end{equation}

\vei

\index{Producto notable!tres variables}
 Otro producto notable, pero ahora  de tres variables, est'a dado por 
\begin{equation}
    (a+b+c)^2 =  a^2 +b^2+c^2+2ab+2ac+2bc.
\label{productonotabledetres}
\end{equation}
La representaci'on geom'etrica de este producto est'a dada por la igualdad entre el  'area del cuadrado con lados de longitud $a+b+c$ y la suma de las 'areas de los nueve rect'angulos en que se ha dividido el cuadrado, esto es,	      
$$
         (a+b+c)^2 =  a^2 +b^2+c^2+ab+ac+ba+bc+ca+cb= a^2 +b^2+c^2+2ab+2ac+2bc.
$$


\centerline{
      \psset{unit=.7cm}
      \begin{pspicture}(0,-.7)(6,6.5)
	\psline(0,1)(5,1)(5,6)(0,6)(0,1)
	\psline(1,1)(1,6)
	\psline(3.3,1)(3.3,6)
	\psline(0,5)(5,5)
	\psline(0,2.7)(5,2.7)
        \psline(5.2,1)(5.3,1.1)(5.3,1.75)(5.4,1.85)(5.3,1.95)(5.3,2.6)(5.2,2.7)
	\psline(5.2,2.7)(5.3,2.8)(5.3,3.75)(5.4,3.85)(5.3,3.95)(5.3,4.9)(5.2,5)
	\psline(5.2,5)(5.3,5.1)(5.3,5.4)(5.4,5.5)(5.3,5.6)(5.3,5.9)(5.2,6)
	\psline(0,.9)(.1,.8)(.4,.8)(.5,.7)(.6,.8)(.9,.8)(1,.9)
	\psline(1,.9)(1.1,.8)(2.05,.8)(2.15,.7)(2.25,.8)(3.2,.8)(3.3,.9)
	\psline(3.3,.9)(3.4,.8)(4.05,.8)(4.15,.7)(4.25,.8)(4.9,.8)(5,.9)
	\rput{0}(.5,.3){$a$}
	\rput{0}(2.15,.3){$b$}
	\rput{0}(4.15,.3){$c$}	
        \rput{0}(5.8,1.85){$c$}
	\rput{0}(5.8,3.85){$b$}
	\rput{0}(5.8,5.5){$a$}
        \rput{0}(.5,1.85){$ac$}		
	\rput{0}(2.15,1.85){$bc$}		
	\rput{0}(4.15,1.85){$c^2$}	
         \rput{0}(.5,3.85){$ab$}		
	\rput{0}(2.15,3.85){$b^2$}		
	\rput{0}(4.15,3.85){$cb$}		
         \rput{0}(.5,5.5){$a^2$}		
	\rput{0}(2.15,5.5){$ba$}		
	\rput{0}(4.15,5.5){$ca$}		
	      \end{pspicture}    
}

\noindent A continuaci'on damos una serie de identidades, algunas de ellas muy conocidas y otras no tanto, 'utiles para resolver varios problemas.

\ve

%%%%  22
\ejercpreliminares{Para todos los n'umeros reales $x$, $y$, 
se tienen las siguientes identidades de segundo grado:

\noindent $(i)$ $x^{2}+y^{2}=(x+y)^{2}-2xy=(x-y)^{2}+2xy$.

\noindent $(ii)$ $(x+y)^{2}+(x-y)^{2}=2(x^{2}+y^{2})$.

\noindent $(iii)$ $(x+y)^{2}-(x-y)^{2}=4xy$.

\noindent $(iv)$ $x^{2}+y^{2}+xy=\dfrac{x^{2}+y^{2}+(x+y)^{2}}{2}$.

\noindent $(v)$ $x^{2}+y^{2}-xy=\dfrac{x^{2}+y^{2}+(x-y)^{2}}{2}$.

\noindent $(vi)$ Muestre que  $x^{2}+y^{2}+xy\geq 0$ y 
$x^{2}+y^{2}-xy\geq 0$.
}

\solcpreliminares{Para los primeros cinco incisos utilice las ecuaciones 
(\ref{ecuac1.1.1}), (\ref{ecuac1.1.2}) y (\ref{ecuac1.1.4}).  Para el inciso $(vi)$ utilice $(iv)$ y $(v)$.}

%%%%%%%% 23
\ejercpreliminares{Para todos los n'umeros reales $x$, $y$, $z$, se tiene: 

\noindent $(i)$ $x^{2}+y^{2}+z^{2}+xy+yz+zx=\dfrac{(x+y)^{2}+(y+z)^{2}+(z+x)^{2}}{2}$.

\noindent $(ii)$ $x^{2}+y^{2}+z^{2}-xy-yz-zx=\dfrac{(x-y)^{2}+(y-z)^{2}+(z-x)^{2}}{2}.$
\label{equisyeyzetaalcuadrado}

\noindent $(iii)$ Muestre que
$x^{2}+y^{2}+z^{2}+xy+yz+zx\geq 0$
y $x^{2}+y^{2}+x^{2}-xy-yz-zx\geq 0.$
}

\solcpreliminares{Para los incisos $(i)$ y $(ii)$ utilice las ecuaciones (\ref{ecuac1.1.1}) y (\ref{ecuac1.1.2}). Para demostrar $(iii)$ use $(i)$ y $(ii)$.

}

% %%%  24
\ejercpreliminares{Para todos los n'umeros reales $x$, $y$, $z$ se tienen las siguientes identidades:

\noindent $(i)$ $(xy+yz+zx)(x+y+z)=(x^{2}y+y^{2}z+z^{2}x)+(xy^{2}+yz^{2}+zx^{2})+3xyz$. 

\noindent $(ii)$
$(x+y)(y+z)(z+x)=(x^{2}y+y^{2}z+z^{2}x)+(xy^{2}+yz^{2}+zx^{2})+2xyz$. 

\noindent $(iii)$
$(xy+yz+zx)(x+y+z)=(x+y)(y+z)(z+x)+xyz$.
\label{ejerciciotresdocetres} 


\noindent $(iv)$
$(x-y)(y-z)(z-x)=(xy^{2}+yz^{2}+zx^{2})-(x^{2}y+y^{2}z+z^{2}x)$. 


\noindent $(v)$
$(x+y)(y+z)(z+x)-8xyz=2z(x-y)^{2}+(x+y)(x-z)(y-z)$. 

\noindent $(vi)$
$xy^{2}+yz^{2}+zx^{2}-3xyz=z(x-y)^{2}+y(x-z)(y-z)$. 
}

\solcpreliminares{Para demostrar los incisos $(i)$ y $(ii)$ realice las operaciones del lado izquierdo de la ecuaci'on y reacomode. 

\noindent Para demostrar los incisos $(iii)$, $(iv)$, $(v)$ y $(vi)$ realice las operaciones de ambos lados de la ecuaci'on y vea que son iguales.
}

%%%%%%%% 25
\ejercpreliminares{Para todos los n'umeros reales $x$, $y$, $z$ se tiene:

\noindent $(i)$\; $x^{2}+y^{2}+z^{2}+3(xy+yz+zx) =(x+y)(y+z)+(y+z)(z+x)+(z+x)(x+y)$.

\noindent $(ii)$\; $ xy+yz+zx-\left(x^{2}+y^{2}+z^{2}\right) =(x-y)(y-z)+(y-z)(z-x)$\\
$ \ \ +(z-x)(x-y).$
}

\solcpreliminares{Para demostrar los incisos $(i)$ y $(ii)$ realice las operaciones del lado derecho de las ecuaciones y simplifique.
}

%%%% 22
\ejercpreliminares{Para todos 
los n'umeros reales $x$, $y$, $z$ se tiene,
\begin{eqnarray*}
  (x-y)^{2}+(y-z)^{2}+(z-x)^{2} & = & 2\left[ (x-y)(x-z)\right. \\
                                       &  &\left . +(y-z)(y-x)+(z-x)(z-y)\right].
\end{eqnarray*}

}

\solcpreliminares{Utilice las ecuaciones (\ref{ecuac1.1.1}) y (\ref{ecuac1.1.2}), haga las operaciones de ambos lados de la ecuaci'on.
}




 

