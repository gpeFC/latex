%%% 13
\ejercpreliminares{Si $a$ y $b$ son n'umeros reales 
cualesquiera, demuestre que
$$
             ||a|-|b||\leq |a-b|.
$$
}

\solcpreliminares{Observe que $|a|=|a-b+b|\leq |a-b|+|b|$, 
despejando se tiene que $|a|-|b|\leq |a-b|.$  An'alogamente, 
siguiendo los mismos pasos, se tiene que $|b| -|a|\leq |b-a|$. 
De estas dos desigualdades se sigue que $||a|-|b||\leq |a-b|$.
}



%%% 14
\ejercpreliminares{En cada caso encuentre los n'umeros reales 
$x$ que satisfacen:

$(i)$\;$|x-1|- |x+1|=0$.

$(ii)$\; $|x-1||x+1|=1$.

$(iii)$\; $|x-1|+ |x+1|=2$.
}

\solcpreliminares{$(i)$\;  $|x-1|- |x+1|=0$ es equivalente a 
$|x-1|=|x+1|$. Elevando al cuadrado y resolviendo la ecuaci'on 
$(x-1)^2 = (x+1)^2$ tenemos que $4x =0$,
luego, la 'unica soluci'on es $x=0$.

$(ii)$\; $|x-1||x+1|=1$  es equivalente a $|x^2-1|=1$, de donde
%$$
%\begin{array}{lcccl}
%          x^2 -1 =1 & &\text{o} & & -(x^2 -1) =1\\
%         x^2  =2 & &\text{o} & & x^2  =0\\
%         x = \pm\sqrt{2} & &\text{o} & & x  =0,
%\end{array}
%$$
las soluciones son $x = \pm\sqrt{2}$  y $ x  =0$.

$(iii)$\;  Si $x>1$ se cumple que  $|x+1|=x+1 >2$, luego no hay soluci'on.

Si $x<-1$ se cumple que  $|x-1|=-x+1 >2$ y tampoco hay soluci'on.

Si $-1\leq x \leq 1$,  entonces $x-1 \leq 0\leq x+1$,   luego 
$$
      |x-1|+|x+1 |=(1-x)+(x+1)=2.
$$
Por lo que los 'unicos valores de  $x$ que cumplen  la igualdad son 
 $-1\leq x\leq 1$.
}

%%% 15
\ejercpreliminares{Encuentre las ternas $(x,y,z)$  de n'umeros 
reales que satisfacen 
\begin{eqnarray*}
              |x+y| &\geq & 1\\
             2xy -z^2 & \geq & 1\\
            z-|x+y|  & \geq & -1.
\end{eqnarray*}
}

\solcpreliminares{De la primera y  tercera desigualdades 
se tiene que \linebreak $z \geq |x+y| -1\geq 0$. Por lo que, $z^2\geq 
(|x+y|-1)^2$. Ahora,   $2xy \geq z^2+1\geq (|x+y|-1)^2 + 1\geq 0$, 
entonces
$$
  2xy \geq  x^2+2xy+y^2-2|x+y|+2 \geq  |x|^2+2xy+|y|^2 - 2|x|- 2 |y| +2,
$$
cancelando $0\geq  |x|^2+|y|^2 - 2|x|- 2 |y|+2 = (|x|-1)^2 +(|y|-1)^2.$
Por lo que $|x|=1$ y $|y|=1$.
Luego,  $x$ y $y$ tienen que ser   $-1$ o 1.  Pero como $xy\geq 0$,  
los dos tienen que tener el mismo signo. Para  $x=y=1$ o $x=y=-1$ se 
tiene, sustituyendo en las ecuaciones originales,   
que $2-z^2\geq 1$ y $z-2\geq -1$. Luego, $z^2\leq 1$ y $z\geq 1$. El 
'unico valor de $z$ que satisface las dos desigualdades es $z=1$. 
Por lo tanto, hay dos soluciones al problema $x=y=z=1$ y $x=y=-1$, $z=1$. 
}


%%%%%16
\ejercpreliminares{(OMM, 2004) ?`Cu\'al es la mayor cantidad de 
n'umeros enteros positivos que se pueden encontrar de manera que 
cualesquiera dos de ellos, $a$ y $b$ (con $a\neq b$), cumplan que:
$$|a-b|\geq \frac{ab}{100}?$$
}

\solcpreliminares{Suponga que
$a_{1}<a_{2}< \dots<a_{n}$ es una colecci\'{o}n con la mayor cantidad de
n'umeros enteros con la propiedad.  Es claro que $a_{i}\geq i$, para 
toda $i=1, \ldots, n$.

\noindent Si $a$ y $b$ son dos n'umeros enteros de la colecci\'{o}n 
con $a>b$, como $%
\left\vert a-b\right\vert =a-b\geq \frac{ab}{100}$, se tiene que 
$a \left(1-\frac{b}{100} \right) \geq b$, por lo que si $100-b>0$, 
entonces $a\geq \frac{100b}{100-b}$.

\noindent Note que no existen dos n'umeros enteros $a$ y $b$ en la 
colecci\'{o}n
mayores que $100$, en efecto si $a>b>100$, entonces $a-b=\left\vert
a-b\right\vert \geq \frac{ab}{100}>a$, lo cual es falso.

\noindent Tambi\'{e}n se tiene que para n'umeros enteros $a$ y $b$ 
menores que $100$,
se cumple que $\frac{100a}{100-a}\geq \frac{100b}{100-b}$ si y s'olo si 
$100a-ab\geq 100b-ab$ si y s'olo si $a\geq b$.

\vei 

\noindent Es claro que $\left\{ 1,2,3,4,5,6,7,8,9,10\right\} $ es una 
colecci\'{o}n con la propiedad.

\noindent Ahora, $a_{11}\geq \frac{100a_{10}}{100-a_{10}}\geq \frac{100\cdot
10}{100-10}=\frac{100}{9}>11$, lo que implica que $a_{11}\geq 12$.

\ve

$a_{12}\geq \frac{100a_{11}}{100-a_{11}}\geq \frac{100\cdot 12}{100-12}=%
\frac{1200}{88}>13$, de donde $a_{12}\geq 14$.

\ve

$a_{13}\geq \frac{100a_{12}}{100-a_{12}}\geq \frac{100\cdot 14}{100-14}=%
\frac{1400}{86}>16$, de donde $a_{13}\geq 17$.

\ve

$a_{14}\geq \frac{100a_{13}}{100-a_{13}}\geq \frac{100\cdot 17}{100-17}=%
\frac{1700}{83}>20$, de donde $a_{14}\geq 21$.

\ve

$a_{15}\geq \frac{100a_{14}}{100-a_{14}}\geq \frac{100\cdot 21}{100-21}=%
\frac{2100}{79}>26$, de donde $a_{15}\geq 27$.

\ve

$a_{16}\geq \frac{100a_{15}}{100-a_{15}}\geq \frac{100\cdot 27}{100-27}=%
\frac{2700}{73}>36$, de donde  $a_{16}\geq 37$.

\ve

$a_{17}\geq \frac{100a_{16}}{100-a_{16}}\geq \frac{100\cdot 37}{100-37}=%
\frac{3700}{63}>58$, de donde  $a_{17}\geq 59$.

\ve

$a_{18}\geq \frac{100a_{17}}{100-a_{17}}\geq \frac{100\cdot 59}{100-59}=%
\frac{5900}{41}>143$, de donde  $a_{18}\geq 144$.

\ve

\noindent Adem'as, como ya se ha observado que no hay dos  
n'umeros enteros de la colecci\'{o}n mayores que $100$, 
la mayor cantidad es $18$.
La colecci\'{o}n de $18$ n'umeros enteros siguiente 
$\left\{ 1,2,3,4,5,6,7,8,9,10,12,14,17,21,27,37,59,144\right\}$ 
cumple la condici\'{o}n.
}










