


%%%% 35
\ejercpreliminares{Sean $a$, $b$ n\'{u}meros reales con $0\leq a\leq b\leq 1$,
muestre que:

$(i)$ $0\leq \dfrac{b-a}{1-ab}\leq 1$.

$(ii)$ $0\leq \dfrac{a}{1+b}+\dfrac{b}{1+a}\leq 1$.
}

\solcpreliminares{$(i)$ Como $0\leq b\leq 1$ y $1+a>0$, pasa que $b(1+a)\leq 1+a$, luego 
$0\leq b-a\leq 1-ab$, por lo que $0\leq \dfrac{b-a}{1-ab}\leq 1.$

\noindent $(ii)$ La desigualdad de la izquierda es clara. Como $1+a\leq 1+b,$
se tiene que $\frac{1}{1+b}\leq \frac{1}{1+a},$ luego, $\dfrac{a}{1+b}+\dfrac{b}{1+a}\leq \dfrac{a}{1+a}+\dfrac{b}{1+a}=\dfrac{a+b}{1+a}\leq 1.$
}



%%%% 36
\ejercpreliminares{(Desigualdad de Nesbitt) Si $a$, $b$,  $c\geq 0$, muestre que  
$$
         \frac{a}{b+c}+\frac{b}{a+c}+\frac{c}{a+b}\geq \frac{3}{2}.
$$
\index{Desigualdad! de Nesbitt}
}

\solcpreliminares{Al hacer $X=\frac{a}{b+c}+\frac{b}{a+c}+\frac{c}{a+b}$ y
sumando y restando tres veces la unidad se tiene
\begin{align*}
    X &=\frac{a}{b+c}+\frac{b+c}{b+c}+\frac{b}{a+c}
        +\frac{a+c}{a+c}+\frac{c}{a+b}+\frac{a+b}{a+b}-3\\[.3cm]
			&=\frac{a+b+c}{b+c}+\frac{a+b+c}{a+c}+\frac{a+b+c}{a+b} -3\\
         &=(a+b+c)\left(\frac{1}{b+c}+\frac{1}{a+c}+\frac{1}{a+b}\right)-3\\
     &=\frac{1}{2}((a+b)+(b+c)+(a+c))\left(\frac{1}{b+c}+\frac{1}{a+c}+\frac{1}{a+b}\right)-3.
\end{align*} 
Ahora, por la desigualdad entre la media geom'etrica y la media aritm'etica, $x+y+z\geq 3\sqrt[3]{xyz}$ y $\frac{1}{x}+\frac{1}{y}+\frac{1}{z}\geq 3\sqrt[3]{\frac{1}{x}\frac{1}{y}\frac{1}{z}}$. Luego, $X\geq \frac{1}{2}\cdot 3\cdot 3-3=\frac{3}{2}$.  
}

%%%% 37
\ejercpreliminares{Si $a$, $b$, $c$ son las longitudes de los lados
de un tri\'{a}ngulo, muestre que
$$
\sqrt[3]{\dfrac{a^{3}+b^{3}+c^{3}+3abc}{2}}\geq \max \left\{ a,b,c\right\}. 
$$
}

\solcpreliminares{Sin p'erdida de generalidad, se puede suponer que $a\geq b\geq c$, la desigualdad es equivalente a 
$-a^{3}+b^{3}+c^{3}+3abc\geq 0$. Pero, por la ecuaci'on (\ref{a3masb3masc3matrices}), 
$-a^3+b^3+c^3+3abc=\frac{1}{2}(-a+b+c)\left[
(a+b)^{2}+(a+c)^{2}+(b-c)^{2}\right]\geq 0$, ya que, por la desigualdad del tri'angulo,  $a<b+c$.}


%%%% 38


\ejercpreliminares{Sean $p$ y $q$ n'umeros reales positivos con $\frac{1}{p}+\frac{1}{q}=1$. Muestre que:

\noindent $(i)$ $\dfrac{1}{3}\leq \dfrac{1}{p(p+1)}+\dfrac{1}{q(q+1)}\leq \dfrac{1}{2}$.\\

\noindent $(ii)$ $\dfrac{1}{p(p-1)}+\dfrac{1}{q(q-1)}\geq 1$. 
}

\solcpreliminares{Observe que $\frac{1}{p}+ \frac{1}{q}=1$ implica que $p+q=pq=s$.  Ahora bien, $(p+q)^2\geq 4pq$ implica que $s\geq 4$.

\noindent  Para probar $(i)$, vea que
 \begin{align*}
\frac{1}{p(p+1)}+\frac{1}{q(q+1)} & = \frac{1}{p} - \frac{1}{p+1}+\frac{1}{q} - \frac{1}{q+1}= 1 - \frac{p+q+2}{(p+1)(q+1)}\\
& = 1 - \frac{s+2}{2s+1}=\frac{s-1}{2s+1}.
\end{align*}
Luego, hay  que mostrar que
$$
    \frac{1}{3}\leq \frac{s-1}{2s+1}\leq \frac{1}{2},
$$
pero $2s+1\leq 3s-3 \Leftrightarrow 4 \leq s$ y
$2s-2\leq 2s+1 \Leftrightarrow -2 \leq 1$. 

\vei

\noindent  Para probar $(ii)$, vea que
 \begin{align*}
\frac{1}{p(p-1)}+\frac{1}{q(q-1)} &= \frac{1}{p-1} - \frac{1}{p}+\frac{1}{q-1} - \frac{1}{q}= \frac{p+q-2}{(p-1)(q-1)} - 1 \\
& = \frac{s-2}{s-s+1}-1=s-3\geq 1.
\end{align*}
}

%%%% 39

\ejercpreliminares{Encuentre el menor n\'{u}mero positivo $k$ tal que, para cualesquiera $0<a$, $b<1$, con $ab=k$, se cumpla que
$$
\frac{a}{b}+\frac{b}{a}+\frac{a}{1-b}+\frac{b}{1-a}\geq 4.
$$
}

\solcpreliminares{Note primero que,
$$
\frac{a}{b}+\frac{a}{1-b}=\frac{a}{b(1-b)}\geq 4a,
$$
\noindent ya que
$$b(1-b)\leq \left( \frac{b+(1-b)}{2}\right) ^{2}=\frac{1}{4}.
$$
\noindent Adem\'{a}s, se tiene la igualdad si y s\'{o}lo si $b=\frac{1}{2}.$
An\'{a}logamente,
$$
\frac{b}{a}+\frac{b}{1-a}\geq 4b.
$$
\noindent Por lo que,
$$
\frac{a}{b}+\frac{b}{a}+\frac{a}{1-b}+\frac{b}{1-a}\geq 4a+4b\geq 2%
\sqrt{4^{2}ab}=8\sqrt{k}.
$$
\noindent Con igualdad si y s\'{o}lo si $a=b.$ As\'{\i},
$$
\frac{a}{b}+\frac{b}{a}+\frac{a}{1-b}+\frac{b}{1-a}\geq 8\sqrt{k}\geq 4
$$
si y s\'{o}lo si $k\geq \frac{1}{4},$ por lo que el menor n'umero $k$ es $\frac{1}{4}.$
}

%%%%%%%%% 40
\ejercpreliminares{Sean $a$, $b$, $c$ n\'{u}meros reales no
negativos, muestre que
\begin{equation*}
(a+b)(b+c)(c+a)\geq \frac{8}{9}(a+b+c)(ab+bc+ca).
\end{equation*}
}


\solcpreliminares{Vea que, $(a+b)(b+c)(c+a)=(a+b+c)(ab+bc+ca)-abc=
\frac{8}{9}(a+b+c)(ab+bc+ca)+\frac{1}{9}(a+b+c)(ab+bc+ca)-abc$ y, 
por la desigualdad entre la media geom'etrica y la media aritm'etica, 
$(a+b+c)(ab+bc+ca)\geq \left (3\sqrt[3]{abc}\right)\left (3\sqrt[3]{(ab)(bc)(ca)}\right)=9abc$.
}

%%%%%%%%%%41
\ejercpreliminares{Sean $a$, $b$, $c$ n\'{u}meros reales positivos
que satisfacen la siguiente igualdad $(a+b)(b+c)(c+a)=1.$ Muestre que 
\begin{equation*}
ab+bc+ca\leq \frac{3}{4}.
\end{equation*}
}

\solcpreliminares{Por la desigualdad entre la media geom'etrica y la media aritm'e\-tica,  y la condici'on $(a+b)(b+c)(c+a)=1$, se tiene 
\begin{eqnarray*}
a+b+c & \geq & 3\sqrt[3]{\left(\frac{a+b}{2}\right)\left(\frac{b+c}{2}\right)\left(\frac{%
c+a}{2}\right)}=\frac{3}{2},\\
abc &  = & \sqrt{ab}\sqrt{bc}\sqrt{ca}\leq \left(\frac{a+b}{2}\right)%
\left(\frac{b+c}{2}\right)\left(\frac{c+a}{2}\right)=\frac{1}{8}.
\end{eqnarray*}
Ahora bien,  
$1=(a+b)(b+c)(c+a)=(a+b+c)(ab+bc+ca)-abc\geq \frac{3}{2}(ab+bc+ca)-\frac{1}{8}$, vea el ejercicio 1.\ref{ejerciciotresdocetres} $(iii)$.
}


%%%%%%%%%%%%% 43

\ejercpreliminares{Sean $a$, $b$, $c$ n\'{u}meros reales positivos
que satisfacen $abc=1$. Muestre que
$(a+b)(b+c)(c+a)\geq 4(a+b+c-1)$.
}

\solcpreliminares{Por el ejercicio 1.\ref{ejerciciotresdocetres} $(iii)$, basta ver que 
$ab+bc+ca+\frac{3}{a+b+c}\geq 4$. Pero 
\begin{align*}
ab+bc+ca+\frac{3}{a+b+c}& =3\left(\frac{ab+bc+ca}{3}\right)+\frac{3}{a+b+c}\\
            &\geq 4\sqrt[4]{\left( \frac{ab+bc+ca}{3}\right)
^{3}\left( \frac{3}{a+b+c}\right) }.
\end{align*}
Ahora use que, $(ab+bc+ca)^{2}\geq
3(ab\cdot bc+bc\cdot ca+ca\cdot ab)=3(a+b+c)$, 
y que $ab+bc+ca\geq 3 \sqrt[3]{a^2b^2c^2}=3$.
}

%%%%%%%%%%%%% 44

\ejercpreliminares{(APMO, 2011) Sean $a$, $b$, $c$ n\'{u}meros enteros positivos.  Muestre que es imposible
que los tres n'umeros $a^2+b+c$, $b^2+c+a$ y  $c^2+a+b$ sean cuadrados perfectos.
}

\solcpreliminares{Sin p'erdida de generalidad podemos suponer $a\leq b\leq c$. Luego, 
$ c^2<c^2+a+b\leq c^2+ 2c< ( c+1)^2$,
esto muestra que $c^2+a+b$ no puede ser un cuadrado perfecto.
}

