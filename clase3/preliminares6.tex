
\section{Desigualdades}
\label{sec-desigualdades}
\index{Desigualdades}
\noindent Iniciamos esta secci'on con una de las  desigualdades m'as importantes.
Para cualquier n'umero real $x$, tenemos que
\begin{equation}
\label{ecuac1.2.1}
                  x^2  \geq  0.
\end{equation}
Esto se sigue de la igualdad  $x^2=|x|^2 \geq 0$.

\noindent A partir de este resultado podemos deducir  que la suma de $n$  
n'umeros cuadrados es no negativa,
\begin{equation}
\label{ecuac1.2.2}
                      x_1^2+x_2^2+\cdots+ x_n^2  \geq  0
\end{equation}
y ser'a cero  si y solamente si todos los $x_i$ son cero.

\noindent Si en la ecuaci'on (\ref{ecuac1.2.1}) sustituimos $x=a-b$, donde
$a$ y $b$ son n'umeros reales no negativos, tenemos que
$$
     (a-b)^2\geq 0.
$$
Desarrollando el binomio, la desigualdad anterior toma la forma,
\begin{equation}
        a^2+b^2 \geq 2ab.
\label{a2masb2mayoroiguala2ab}
\end{equation}
Como 
$$
   a^2+b^2\geq 2ab \;\;\text{si y s'olo si}\;\; 2a^2+2b^2\geq a^2+2ab+b^2=
	 (a+b)^2,
$$ 
tenemos tambi'en la siguiente desigualdad 
\begin{equation}
\label{ecuac1.2.3}\sqrt{ \frac{a^2+b^2}{2}}  \geq  \frac{(a+b)}{2}.
\end{equation}				
En caso de que $a$ y $b$ sean positivos, la  desigualdad (\ref{a2masb2mayoroiguala2ab}) garantiza que 
\begin{equation}
\label{numeroyreciproco}
        \frac{a}{b}+\frac{b}{a}\geq 2.
\end{equation}
Si en la desigualdad anterior tomamos $b=1$, entonces tenemos que  
$a+\frac{1}{a}\geq 2$, es decir, la suma de $a>0$ y su rec'iproco es 
mayor o igual que 2, y es 2 si y s'olo si  $a=1$.

\vei

\noindent Remplazando $a$, $b$ por $\sqrt{a}$, $\sqrt{b}$ en (\ref{a2masb2mayoroiguala2ab}) obtenemos
\begin{equation}
\label{ecuac1.2.5}
   a+b\geq 2\sqrt{ab}\;\; \text{si y s'olo si}\;\;\frac{a+b}{2}\geq \sqrt{ab}.
\end{equation}
Multiplicando la 'ultima desigualdad por $\sqrt{ab}$ y reacomodando,  tenemos
\begin{equation}
\label{ecuac1.2.4e}
 \sqrt{ab} \geq \frac{2ab}{a+b}.
\end{equation}
Juntando  las desigualdades (\ref{ecuac1.2.3}), (\ref{ecuac1.2.5})   y (\ref{ecuac1.2.4e}),  hemos demostrado que
\begin{equation}
    \frac{2ab}{a+b}\leq \sqrt{ab}\leq \frac{a+b}{2}\leq \sqrt{\frac{a^2+b^2}{2}}.
\label{todaslasmedias}
\end{equation}
La primera expresi'on  se conoce como la {\bf media arm'onica}\index{Media!arm'onica} ($MH$), la segunda es la
{\bf media geom'etrica} \index{Media!geom'etrica} ($MG$), la tercera es la
{\bf media aritm'etica} \index{Media!aritm'etica} ($MA$) y la 'ultima es la
{\bf media cuadr'atica}\index{Media!cuadr'atica} ($MQ$). 

Estas desigualdades tambi'en  se pueden demostrar
geom'etricamente como sigue. Consideremos un semicircunferencia con centro $O$, radio $\frac{a+b}{2}$ y los tri'angulos
rect'an\-gulos $ABC$, $DBA$
y $DAC$, como se muestra en la figura

\centerline{
	\psset{unit=1cm}
	\begin{pspicture}(0,0)(6,4.5)
	\psset{linewidth=0.5pt}
	\psline(0,1)(6,1)
	\psline(0,.6)(.1,.4)(1.3,.4)(1.5,.6)
	\psline(1.5,.6)(1.6,.4)(5.9,.4)(6,.6)
	\psarc[linewidth=.5pt](3,1){3}{0}{180}
	\psline(1.5,1)(1.5,3.61)
	\psline(1.5,3.61)(3,1)
	\psline(0,1)(1.5,3.61)(6,1)
	\psline(1.5,1)(2.68,1.58)
          \psline(1.3,1)(1.3,1.2)(1.5,1.2)
	\psline(2.4,1.44)(2.3,1.63)(2.57,1.77)
	\rput{0}(.75,.1){$ a$}
	\rput{0}(4,.1){$b$}
	\rput{0}(2,1.4){$e$}
	\rput{0}(1.3,2){$h$}
	\rput{0}(2.5,2.5){$y$}
	\rput{0}(3.2,1.3){$z$}
	\rput{0}(1.5,3.9){$A$}
	\rput{0}(-.3,1){$ B$}
	\rput{0}(6.3,1){$ C$}
	\rput{0}(1.5,.8){$ D$}
	\rput{0}(2.9,1.7){$E$}
	\rput{0}(3,.8){$ O$}
	\end{pspicture}
}
\noindent Estos tri'angulos son semejantes por lo que tenemos que
\begin{eqnarray*}
	     \frac{AD}{DB} & = &\frac{DC}{DA}\\
             \frac{h}{a} & = &\frac{b}{h}\\
                    h^2  & = & ab,
\end{eqnarray*}
es decir, que la altura com'un de los tri'angulos es $h=\sqrt{ab}$, que claramente
es menor que el radio de la semicircunferencia.  Luego,
$\sqrt{ab}\leq \frac{a+b}{2}$.

\noindent Para demostrar la primera desigualdad de (\ref{todaslasmedias}), observemos que los tri'angulos
$DAE$ y $OAD$ son semejantes, entonces
\begin{eqnarray*}
             \frac{AD}{AE} & = &\frac{AO}{AD}\\
                    h^2  & = & y(y+z)\\
             \frac{2ab}{a+b} & = & y,
\end{eqnarray*}
es decir, $y$ representa la media arm'onica. Claramente tenemos que $y\leq h$,
luego $\frac{2ab}{a+b}\leq \sqrt{ab}$.

\noindent Para demostrar geom'etricamente, la 'ultima desigualdad de (\ref{todaslasmedias}), conside\-remos
la siguiente figura

\centerline{\psset{unit=.9cm}
	\begin{pspicture}(0,0)(5,4.7)
	\psset{linewidth=0.5pt}
	\psline(0,1)(6,1)
	\psarc[linewidth=.5pt](3,1){3}{0}{180}
	\psline(1.5,1)(1.5,3.61)
	\psline(1.5,1)(3,4)
	\psline(3,1)(3,4)
	\rput{0}(1.5,3.9){$ A$}
	\rput{0}(3,4.3){$ L$}
	\rput{0}(1.5,.7){$ D$}
	\rput{0}(3,.7){$ O$}
	\end{pspicture}
}

\noindent Tenemos que $OD=\frac{a+b}{2}-a=\frac{b-a}{2}$ y utilizando el teorema
de Pit'agoras tenemos que
$$     
 DL^2  =  OD^2 + OL^2=\Big(\frac{b-a}{2}\Big )^2 +\Big(\frac{a+b}{2}\Big )^2 =  \frac{a^2+b^2}{2},
$$			
es decir, $DL =\sqrt{\frac{a^2+b^2}{2}}$ que claramente es mayor que
$\frac{a+b}{2}$.

\ve

\noindent Utilizando el ejemplo \ref{ejemplocubicas} podemos dar una
demostraci'on de la desigualdad entre la media geom'etrica y la media aritm'etica
 para tres n'umeros reales no negativos. En efecto, por la identidad
$$
a^3+b^3+c^3-3abc=\frac{1}{2}(a+b+c)\left [ (a-b)^2+(b-c)^2 +(c-a)^2\right ],	
$$
es claro que si $a$, $b$ y $c$ son no negativos, entonces
$a^3+b^3+c^3-3abc\geq 0$, es decir, $a^3+b^3+c^3 \geq 3abc$.
Adem'as, tenemos la igualdad si $a+b+c=0$ o $(a-b)^2+(b-c)^2 +(c-a)^2=0$, esto es solamente cuando $a=b=c$.
 Ahora si $x$, $y$ y $z$ son n'umeros no negativos, definiendo $a=\sqrt[3]{x}$,
$b=\sqrt[3]{y}$ y $c=\sqrt[3]{z}$, tenemos que
\begin{equation}
    \frac{x+y+z}{3}\geq \sqrt[3]{xyz},
\label{mediaaritgeomtresvariables}
\end{equation}
con igualdad si y s'olo si $x=y=z$.


\begin{ejemplo}
Para todo n'umero real $x$, sucede que
$\frac{x^2+2}{\sqrt{x^2+1}}\geq 2$.
\end{ejemplo}

En efecto,
$$
    \frac{x^2+2}{\sqrt{x^2+1}}  =  \frac{x^2+1}{\sqrt{x^2+1}}+\frac{1}{\sqrt{x^2+1}}
                                =  \sqrt{x^2+1}+\frac{1}{\sqrt{x^2+1}} \geq 2.
$$
La desigualdad se sigue de aplicar la desigualdad (\ref{numeroyreciproco}).

\begin{ejemplo}
Si $a$, $b$, $c$ son n'umeros no negativos, entonces
$$
          (a+b)(b+c)(a+c)\geq 8abc.
$$
\end{ejemplo}
\noindent Como hemos visto, $\frac{(a+b)}{2}\geq \sqrt{ab}$, $\frac{(b+c)}{2}\geq \sqrt{bc}$\; y $\frac{(a+c)}{2}\geq \sqrt{ac}$, de donde
$$
    \left (\frac{a+b}{2}\right )\left (\frac{b+c}{2}\right )\left (\frac{a+c}{2}\right ) \geq\sqrt{a^2b^2c^2}=abc.
$$

\begin{ejemplo}
\label{ejemploreacomodo}
Si $x_1>x_2>x_3$ y $y_1>y_2>y_3$, ?`cu'al de las siguientes sumas es mayor?
\begin{eqnarray*}
     S & = & x_1y_1+x_2y_2+x_3y_3\\
		 S^\prime & = & x_1y_2+x_2y_1+x_3y_3.
\end{eqnarray*}		
\end{ejemplo}
Consideremos la diferencia,
\begin{eqnarray*}
     S^\prime- S & = & x_1y_2-x_1y_1+x_2y_1-x_2y_2\\
               	 & = & x_1(y_2-y_1)+x_2(y_1-y_2)\\
		  & = & -x_1(y_1-y_2)+x_2(y_1-y_2)\\
		 & = & (x_2-x_1)(y_1-y_2) < 0,
\end{eqnarray*}		
por lo tanto, $S^\prime < S$.

\noindent M'as generalmente, para cualquier permutaci'on $\{y^\prime_1,y^\prime_2,y^\prime_3\}$ de
$\{y_1,y_2,y_3\}$ tenemos que,
\begin{equation}
         S\geq x_1y^\prime_1+x_2y^\prime_2+x_3y^\prime_3,
\label{desigualdadreacomodo}
\end{equation} que se conoce como la 
{\bf desigualdad del reacomodo}\footnote{Para una versi'on general de la 
desigualdad del reacomodo, vea el ejemplo \ref{desigualdaddelreacomodo}.}. 
\index{Desigualdad!del reacomodo}

\vei		




%%%% 35
\ejercpreliminares{Sean $a$, $b$ n\'{u}meros reales con $0\leq a\leq b\leq 1$,
muestre que:

$(i)$ $0\leq \dfrac{b-a}{1-ab}\leq 1$.

$(ii)$ $0\leq \dfrac{a}{1+b}+\dfrac{b}{1+a}\leq 1$.
}

\solcpreliminares{$(i)$ Como $0\leq b\leq 1$ y $1+a>0$, pasa que $b(1+a)\leq 1+a$, luego 
$0\leq b-a\leq 1-ab$, por lo que $0\leq \dfrac{b-a}{1-ab}\leq 1.$

\noindent $(ii)$ La desigualdad de la izquierda es clara. Como $1+a\leq 1+b,$
se tiene que $\frac{1}{1+b}\leq \frac{1}{1+a},$ luego, $\dfrac{a}{1+b}+\dfrac{b}{1+a}\leq \dfrac{a}{1+a}+\dfrac{b}{1+a}=\dfrac{a+b}{1+a}\leq 1.$
}



%%%% 36
\ejercpreliminares{(Desigualdad de Nesbitt) Si $a$, $b$,  $c\geq 0$, muestre que  
$$
         \frac{a}{b+c}+\frac{b}{a+c}+\frac{c}{a+b}\geq \frac{3}{2}.
$$
\index{Desigualdad! de Nesbitt}
}

\solcpreliminares{Al hacer $X=\frac{a}{b+c}+\frac{b}{a+c}+\frac{c}{a+b}$ y
sumando y restando tres veces la unidad se tiene
\begin{align*}
    X &=\frac{a}{b+c}+\frac{b+c}{b+c}+\frac{b}{a+c}
        +\frac{a+c}{a+c}+\frac{c}{a+b}+\frac{a+b}{a+b}-3\\[.3cm]
			&=\frac{a+b+c}{b+c}+\frac{a+b+c}{a+c}+\frac{a+b+c}{a+b} -3\\
         &=(a+b+c)\left(\frac{1}{b+c}+\frac{1}{a+c}+\frac{1}{a+b}\right)-3\\
     &=\frac{1}{2}((a+b)+(b+c)+(a+c))\left(\frac{1}{b+c}+\frac{1}{a+c}+\frac{1}{a+b}\right)-3.
\end{align*} 
Ahora, por la desigualdad entre la media geom'etrica y la media aritm'etica, $x+y+z\geq 3\sqrt[3]{xyz}$ y $\frac{1}{x}+\frac{1}{y}+\frac{1}{z}\geq 3\sqrt[3]{\frac{1}{x}\frac{1}{y}\frac{1}{z}}$. Luego, $X\geq \frac{1}{2}\cdot 3\cdot 3-3=\frac{3}{2}$.  
}

%%%% 37
\ejercpreliminares{Si $a$, $b$, $c$ son las longitudes de los lados
de un tri\'{a}ngulo, muestre que
$$
\sqrt[3]{\dfrac{a^{3}+b^{3}+c^{3}+3abc}{2}}\geq \max \left\{ a,b,c\right\}. 
$$
}

\solcpreliminares{Sin p'erdida de generalidad, se puede suponer que $a\geq b\geq c$, la desigualdad es equivalente a 
$-a^{3}+b^{3}+c^{3}+3abc\geq 0$. Pero, por la ecuaci'on (\ref{a3masb3masc3matrices}), 
$-a^3+b^3+c^3+3abc=\frac{1}{2}(-a+b+c)\left[
(a+b)^{2}+(a+c)^{2}+(b-c)^{2}\right]\geq 0$, ya que, por la desigualdad del tri'angulo,  $a<b+c$.}


%%%% 38


\ejercpreliminares{Sean $p$ y $q$ n'umeros reales positivos con $\frac{1}{p}+\frac{1}{q}=1$. Muestre que:

\noindent $(i)$ $\dfrac{1}{3}\leq \dfrac{1}{p(p+1)}+\dfrac{1}{q(q+1)}\leq \dfrac{1}{2}$.\\

\noindent $(ii)$ $\dfrac{1}{p(p-1)}+\dfrac{1}{q(q-1)}\geq 1$. 
}

\solcpreliminares{Observe que $\frac{1}{p}+ \frac{1}{q}=1$ implica que $p+q=pq=s$.  Ahora bien, $(p+q)^2\geq 4pq$ implica que $s\geq 4$.

\noindent  Para probar $(i)$, vea que
 \begin{align*}
\frac{1}{p(p+1)}+\frac{1}{q(q+1)} & = \frac{1}{p} - \frac{1}{p+1}+\frac{1}{q} - \frac{1}{q+1}= 1 - \frac{p+q+2}{(p+1)(q+1)}\\
& = 1 - \frac{s+2}{2s+1}=\frac{s-1}{2s+1}.
\end{align*}
Luego, hay  que mostrar que
$$
    \frac{1}{3}\leq \frac{s-1}{2s+1}\leq \frac{1}{2},
$$
pero $2s+1\leq 3s-3 \Leftrightarrow 4 \leq s$ y
$2s-2\leq 2s+1 \Leftrightarrow -2 \leq 1$. 

\vei

\noindent  Para probar $(ii)$, vea que
 \begin{align*}
\frac{1}{p(p-1)}+\frac{1}{q(q-1)} &= \frac{1}{p-1} - \frac{1}{p}+\frac{1}{q-1} - \frac{1}{q}= \frac{p+q-2}{(p-1)(q-1)} - 1 \\
& = \frac{s-2}{s-s+1}-1=s-3\geq 1.
\end{align*}
}

%%%% 39

\ejercpreliminares{Encuentre el menor n\'{u}mero positivo $k$ tal que, para cualesquiera $0<a$, $b<1$, con $ab=k$, se cumpla que
$$
\frac{a}{b}+\frac{b}{a}+\frac{a}{1-b}+\frac{b}{1-a}\geq 4.
$$
}

\solcpreliminares{Note primero que,
$$
\frac{a}{b}+\frac{a}{1-b}=\frac{a}{b(1-b)}\geq 4a,
$$
\noindent ya que
$$b(1-b)\leq \left( \frac{b+(1-b)}{2}\right) ^{2}=\frac{1}{4}.
$$
\noindent Adem\'{a}s, se tiene la igualdad si y s\'{o}lo si $b=\frac{1}{2}.$
An\'{a}logamente,
$$
\frac{b}{a}+\frac{b}{1-a}\geq 4b.
$$
\noindent Por lo que,
$$
\frac{a}{b}+\frac{b}{a}+\frac{a}{1-b}+\frac{b}{1-a}\geq 4a+4b\geq 2%
\sqrt{4^{2}ab}=8\sqrt{k}.
$$
\noindent Con igualdad si y s\'{o}lo si $a=b.$ As\'{\i},
$$
\frac{a}{b}+\frac{b}{a}+\frac{a}{1-b}+\frac{b}{1-a}\geq 8\sqrt{k}\geq 4
$$
si y s\'{o}lo si $k\geq \frac{1}{4},$ por lo que el menor n'umero $k$ es $\frac{1}{4}.$
}

%%%%%%%%% 40
\ejercpreliminares{Sean $a$, $b$, $c$ n\'{u}meros reales no
negativos, muestre que
\begin{equation*}
(a+b)(b+c)(c+a)\geq \frac{8}{9}(a+b+c)(ab+bc+ca).
\end{equation*}
}


\solcpreliminares{Vea que, $(a+b)(b+c)(c+a)=(a+b+c)(ab+bc+ca)-abc=
\frac{8}{9}(a+b+c)(ab+bc+ca)+\frac{1}{9}(a+b+c)(ab+bc+ca)-abc$ y, 
por la desigualdad entre la media geom'etrica y la media aritm'etica, 
$(a+b+c)(ab+bc+ca)\geq \left (3\sqrt[3]{abc}\right)\left (3\sqrt[3]{(ab)(bc)(ca)}\right)=9abc$.
}

%%%%%%%%%%41
\ejercpreliminares{Sean $a$, $b$, $c$ n\'{u}meros reales positivos
que satisfacen la siguiente igualdad $(a+b)(b+c)(c+a)=1.$ Muestre que 
\begin{equation*}
ab+bc+ca\leq \frac{3}{4}.
\end{equation*}
}

\solcpreliminares{Por la desigualdad entre la media geom'etrica y la media aritm'e\-tica,  y la condici'on $(a+b)(b+c)(c+a)=1$, se tiene 
\begin{eqnarray*}
a+b+c & \geq & 3\sqrt[3]{\left(\frac{a+b}{2}\right)\left(\frac{b+c}{2}\right)\left(\frac{%
c+a}{2}\right)}=\frac{3}{2},\\
abc &  = & \sqrt{ab}\sqrt{bc}\sqrt{ca}\leq \left(\frac{a+b}{2}\right)%
\left(\frac{b+c}{2}\right)\left(\frac{c+a}{2}\right)=\frac{1}{8}.
\end{eqnarray*}
Ahora bien,  
$1=(a+b)(b+c)(c+a)=(a+b+c)(ab+bc+ca)-abc\geq \frac{3}{2}(ab+bc+ca)-\frac{1}{8}$, vea el ejercicio 1.\ref{ejerciciotresdocetres} $(iii)$.
}


%%%%%%%%%%%%% 43

\ejercpreliminares{Sean $a$, $b$, $c$ n\'{u}meros reales positivos
que satisfacen $abc=1$. Muestre que
$(a+b)(b+c)(c+a)\geq 4(a+b+c-1)$.
}

\solcpreliminares{Por el ejercicio 1.\ref{ejerciciotresdocetres} $(iii)$, basta ver que 
$ab+bc+ca+\frac{3}{a+b+c}\geq 4$. Pero 
\begin{align*}
ab+bc+ca+\frac{3}{a+b+c}& =3\left(\frac{ab+bc+ca}{3}\right)+\frac{3}{a+b+c}\\
            &\geq 4\sqrt[4]{\left( \frac{ab+bc+ca}{3}\right)
^{3}\left( \frac{3}{a+b+c}\right) }.
\end{align*}
Ahora use que, $(ab+bc+ca)^{2}\geq
3(ab\cdot bc+bc\cdot ca+ca\cdot ab)=3(a+b+c)$, 
y que $ab+bc+ca\geq 3 \sqrt[3]{a^2b^2c^2}=3$.
}

%%%%%%%%%%%%% 44

\ejercpreliminares{(APMO, 2011) Sean $a$, $b$, $c$ n\'{u}meros enteros positivos.  Muestre que es imposible
que los tres n'umeros $a^2+b+c$, $b^2+c+a$ y  $c^2+a+b$ sean cuadrados perfectos.
}

\solcpreliminares{Sin p'erdida de generalidad podemos suponer $a\leq b\leq c$. Luego, 
$ c^2<c^2+a+b\leq c^2+ 2c< ( c+1)^2$,
esto muestra que $c^2+a+b$ no puede ser un cuadrado perfecto.
}


