%%%%  22
\ejercpreliminares{Para todos los n'umeros reales $x$, $y$, 
se tienen las siguientes identidades de segundo grado:

\noindent $(i)$ $x^{2}+y^{2}=(x+y)^{2}-2xy=(x-y)^{2}+2xy$.

\noindent $(ii)$ $(x+y)^{2}+(x-y)^{2}=2(x^{2}+y^{2})$.

\noindent $(iii)$ $(x+y)^{2}-(x-y)^{2}=4xy$.

\noindent $(iv)$ $x^{2}+y^{2}+xy=\dfrac{x^{2}+y^{2}+(x+y)^{2}}{2}$.

\noindent $(v)$ $x^{2}+y^{2}-xy=\dfrac{x^{2}+y^{2}+(x-y)^{2}}{2}$.

\noindent $(vi)$ Muestre que  $x^{2}+y^{2}+xy\geq 0$ y 
$x^{2}+y^{2}-xy\geq 0$.
}

\solcpreliminares{Para los primeros cinco incisos utilice las ecuaciones 
(\ref{ecuac1.1.1}), (\ref{ecuac1.1.2}) y (\ref{ecuac1.1.4}).  Para el inciso $(vi)$ utilice $(iv)$ y $(v)$.}

%%%%%%%% 23
\ejercpreliminares{Para todos los n'umeros reales $x$, $y$, $z$, se tiene: 

\noindent $(i)$ $x^{2}+y^{2}+z^{2}+xy+yz+zx=\dfrac{(x+y)^{2}+(y+z)^{2}+(z+x)^{2}}{2}$.

\noindent $(ii)$ $x^{2}+y^{2}+z^{2}-xy-yz-zx=\dfrac{(x-y)^{2}+(y-z)^{2}+(z-x)^{2}}{2}.$
\label{equisyeyzetaalcuadrado}

\noindent $(iii)$ Muestre que
$x^{2}+y^{2}+z^{2}+xy+yz+zx\geq 0$
y $x^{2}+y^{2}+x^{2}-xy-yz-zx\geq 0.$
}

\solcpreliminares{Para los incisos $(i)$ y $(ii)$ utilice las ecuaciones (\ref{ecuac1.1.1}) y (\ref{ecuac1.1.2}). Para demostrar $(iii)$ use $(i)$ y $(ii)$.

}

% %%%  24
\ejercpreliminares{Para todos los n'umeros reales $x$, $y$, $z$ se tienen las siguientes identidades:

\noindent $(i)$ $(xy+yz+zx)(x+y+z)=(x^{2}y+y^{2}z+z^{2}x)+(xy^{2}+yz^{2}+zx^{2})+3xyz$. 

\noindent $(ii)$
$(x+y)(y+z)(z+x)=(x^{2}y+y^{2}z+z^{2}x)+(xy^{2}+yz^{2}+zx^{2})+2xyz$. 

\noindent $(iii)$
$(xy+yz+zx)(x+y+z)=(x+y)(y+z)(z+x)+xyz$.
\label{ejerciciotresdocetres} 


\noindent $(iv)$
$(x-y)(y-z)(z-x)=(xy^{2}+yz^{2}+zx^{2})-(x^{2}y+y^{2}z+z^{2}x)$. 


\noindent $(v)$
$(x+y)(y+z)(z+x)-8xyz=2z(x-y)^{2}+(x+y)(x-z)(y-z)$. 

\noindent $(vi)$
$xy^{2}+yz^{2}+zx^{2}-3xyz=z(x-y)^{2}+y(x-z)(y-z)$. 
}

\solcpreliminares{Para demostrar los incisos $(i)$ y $(ii)$ realice las operaciones del lado izquierdo de la ecuaci'on y reacomode. 

\noindent Para demostrar los incisos $(iii)$, $(iv)$, $(v)$ y $(vi)$ realice las operaciones de ambos lados de la ecuaci'on y vea que son iguales.
}

%%%%%%%% 25
\ejercpreliminares{Para todos los n'umeros reales $x$, $y$, $z$ se tiene:

\noindent $(i)$\; $x^{2}+y^{2}+z^{2}+3(xy+yz+zx) =(x+y)(y+z)+(y+z)(z+x)+(z+x)(x+y)$.

\noindent $(ii)$\; $ xy+yz+zx-\left(x^{2}+y^{2}+z^{2}\right) =(x-y)(y-z)+(y-z)(z-x)$\\
$ \ \ +(z-x)(x-y).$
}

\solcpreliminares{Para demostrar los incisos $(i)$ y $(ii)$ realice las operaciones del lado derecho de las ecuaciones y simplifique.
}

%%%% 22
\ejercpreliminares{Para todos 
los n'umeros reales $x$, $y$, $z$ se tiene,
\begin{eqnarray*}
  (x-y)^{2}+(y-z)^{2}+(z-x)^{2} & = & 2\left[ (x-y)(x-z)\right. \\
                                       &  &\left . +(y-z)(y-x)+(z-x)(z-y)\right].
\end{eqnarray*}

}

\solcpreliminares{Utilice las ecuaciones (\ref{ecuac1.1.1}) y (\ref{ecuac1.1.2}), haga las operaciones de ambos lados de la ecuaci'on.
}


