\section{Valor absoluto}
\label{valorabsoluto}

\noindent Definimos el {\bf  valor absoluto}\index{Valor absoluto} de un n'umero real $x$ como
\begin{equation}
\label{ecuac1.3.1}
     |x|=\left \{\begin{array}{lc}
                 x, & \makebox[1cm]{si} x \geq 0,\\[2mm]
                 -x, & \makebox[1cm]{si} x < 0.
              \end{array}\right.
\end{equation}


\noindent Para $k$ un n'umero real no negativo, la identidad $|x|=k$ s'olo la satisfacen los n'umeros $x=k$ y $x=-k$.

\noindent La desigualdad $|x|\leq k$ es equivalente a
$-k\leq x\leq k$, lo cual podemos ver de la siguiente manera. Si $x\geq 0$, entonces
$0\leq x=|x|\leq k$. Por otro lado, si $x\leq 0$, entonces $-x=|x|\leq k$, de donde $x\geq -k$.  Como consecuencia de lo anterior observemos que $x\leq |x|$.  En la figura siguiente se muestran los valores de $x$ que
satisfacen la desigualdad,  'estos son los que se encuentran entre $-k$ y $k$, incluy'endolos.  Al conjunto $[-k,k]=\{x\in\rr\, |\, -k\leq x\leq k\}$ le llamamos un {\bf intervalo cerrado}, ya que contiene a $k$ y $-k$.\index{Intervalo!cerrado} A $-k$ y $k$ les llamamos los {\bf puntos extremos} del intervalo. \index{Puntos extremos}

\centerline{
 \begin{pspicture}(6,0)(7,4)
 \psset{unit=1cm}
	\psline(1,1)(2,1)
	\psline(4,1)(5,1)
	\psset{linewidth=2.5pt}
	\psline(2,1)(4,1)
	\rput{0}(2,.5){\large $-k$}
	\rput{0}(4,.5){\large $k$}
	\rput{0}(3,.5){ $O$}
	\rput{0}(3,1){\large $|$}
	\rput{0}(2,1){\large [}
	\rput{0}(4,1){\large ]}
 \end{pspicture}
}

\noindent An'alogamente, la desigualdad $|x|\geq k$ es equivalente a
$ x\geq k$ o $ -x\geq k$.
En la figura siguiente los valores de $x$
 que satisfacen
las desigualdades son los que se encuentran antes,  o son iguales,  a $-k$ o despu'es, o son iguales, a $k$. El conjunto $(-k,k)=\{x\in\rr\, |\, -k < x < k\}$ le llamamos un {\bf intervalo abierto}, ya que no contiene a $k$ y $-k$, es decir, un intervalo abierto es aquel que no contiene sus puntos extremos.\index{Intervalo!abierto}  Con esta definici'on vemos que el conjunto de las $x$ que cumplen que $|x|\geq k$, son los valores de $x \notin (-k,k)$. 


\centerline{
 \begin{pspicture}(6,0)(7,4)
 \psset{unit=1cm}
 \psline(2,1)(4,1)
 \psset{linewidth=2.5pt}
 \psline(1,1)(2,1)
 \psline(4,1)(5,1)
 \rput{0}(2,.5){\large $-k$}
 \rput{0}(4,.5){\large $k$}
 \rput{0}(3,.5){ $O$}
 \rput{0}(3,1){$|$}
 \rput{0}(2,1){\large ]}
 \rput{0}(4,1){\large [}
 \end{pspicture}
}

\index{Plano Cartesiano}

\begin{ejemplo}
\noindent  Encontremos, en el plano cartesiano\footnote{El {\bf plano 
cartesiano} se define como $\rr^2=\rr \times \rr=\{(x,y) \,|\,x\in\rr, y \in\rr\}$.}, el 'area
encerrada por la gr'afica de la relaci'on $|x|+|y|=1$.
\end{ejemplo}
\noindent Para $|x|+|y|=1$ tenemos que considerar cuatro casos:
\begin{description}
\item[$(a)$] $x\geq 0$ y $y\geq 0$ lo que implica que $x+y=1$, es decir, $y=1-x$.
\item[$(b)$] $x\geq 0$ y $y < 0$ lo que implica que $x-y=1$, es decir, $y=x-1$.
\item[$(c)$] $x < 0$ y $y \geq 0$ lo que implica que $-x+y=1$, es decir, $y=x+1$.
\item[$(d)$] $x< 0$ y $y < 0$ lo que implica que $-x-y=1$, es decir, $y=-x-1$.
\end{description}

\noindent Podemos ahora dibujar la gr\'afica.

\centerline{
      \psset{unit=1cm}
      \begin{pspicture}(0,0)(4,4.5)
     % \psgrid(4,4)
       \psline(0,2.5)(3,2.5)
       \psline(1.5,1)(1.5,4)	
       \psline(0,2)(2,4)
	\psline(1,4)(3,2)
 	\psline(3,3)(1,1)
	\psline(2,1)(0,3)
	 \rput{0}(-.4,2.7){\scriptsize{$(-1,0)$}}
       \rput{0}(2.1,3.6){\scriptsize{$(0,1)$}}
       \rput{0}(3.3,2.7){\scriptsize{$(1,0)$}}
       \rput{0}(2.2,1.4){\scriptsize{$(0,-1)$}}	
       \end{pspicture}
}

\noindent El 'area encerrada por las cuatro rectas est\'a formada por
cuatro tri\'angulos rect\'angulos is'osceles, que tienen cada uno, dos lados iguales a $1$. Como el \'area de
cada uno de estos tri\'angulos es $\frac{1\times 1}{2}=\frac{1}{2}$,  el \'area del cuadrado es $4\left (\frac{1}{2}\right )=2$.

\begin{ejemplo}
Resolvamos la ecuaci'on $|2x-4|=|x+5|$.
\end{ejemplo}
\noindent Tenemos que
\begin{equation*}
    |2x-4|=\left \{\begin{array}{lr}
    		 2x-4, & \makebox[1cm]{si} x\geq 2,\\[2mm]	
                 -2x+4, &\makebox[1cm]{si}  x< 2.
                \end{array}\right.
\end{equation*}
Adem'as, tenemos que
\begin{equation*}
    |x+5|=\left \{\begin{array}{lr}
                 x+5, &\makebox[1cm]{si}  x\geq -5,\\[2mm]
                 -x-5, & \makebox[1cm]{si} x< -5.
                \end{array}\right.
\end{equation*}
Si $x\geq 2$, entonces $2x-4=x+5$, es decir, $x=9$. Si $x<-5$, entonces
$-2x+4=-x-5$, de donde $x=9$, lo cual es imposible ya que $x<-5$. El 'ultimo caso que nos falta considerar es
 $-5\leq x<2$, entonces la ecuaci'on que tenemos que resolver es $-2x+4=x+5$,
despejando $x$, tenemos que $x=-\frac{1}{3}$. Por lo tanto, los n'umeros que resuelven la ecuaci'on son $x=9$ y $x=-\frac{1}{3}$.

\vei

\noindent Muchas veces es m'as f'acil resolver estas ecuaciones sin
utilizar la forma expl'icita del valor absoluto,
si observamos que $|a|=|b|$ si y s'olo si $a=\pm b$ y utilizamos las propiedades del valor absoluto.

\vei


\begin{observacion}
Si $x$ es un n'umero real cualquiera, entonces la
relaci'on entre la ra'iz cuadrada y el valor absoluto est'a dada por
$\sqrt{x^2}=|x|$,
la identidad se sigue de que  $|x|^2=x^2$ y $|x|\geq 0$.	
\end{observacion}

\index{Valor absoluto!propiedades del}

\begin{propiedades}
Si $x$ y $y$ son n'umeros reales, se cumple lo siguiente:
\begin{description}
\item[$(a)$] $|xy|=|x||y|$. De aqu'i se sigue  tambi'en que $\left |\frac{x}{y}\right |=
\frac{|x|}{|y|}$, si $y\neq 0$.
\item [$(b)$] $|x+y|\leq |x|+|y|$, donde la igualdad se da si y s'olo si $xy\geq 0$.
\end{description}
\label{desigualdadesabsolut}
\end{propiedades}


\demostracion{$(a)$ La demostraci'on es directa de $|xy|^2=(xy)^2=x^2y^2=|x|^2|y|^2$, 
y ahora sacando ra'iz obtenemos el resultado.

$(b)$ Como ambos lados de la desigualdad son n\'umeros positivos,
bastar\'a entonces verificar que
$\left|x+y\right| ^{2}\leq \left( \left| x\right| +\left| y\right|
\right)^{2}$.
\begin{eqnarray*}
\left| x+y\right| ^{2} & = & (x+y)^{2}=x^{2}+2xy+y^{2}=\left| x\right|
^{2}+2xy+\left| y\right| ^{2}\\
& \leq & \left| x\right| ^{2}+2\left| xy\right|
+\left| y\right| ^{2}
 =  \left| x\right| ^{2}+2\left| x\right| \left| y\right| +\left| y\right|
^{2}=\left( \left| x\right| +\left| y\right| \right) ^{2}.
\end{eqnarray*}
\noindent En las relaciones anteriores hay una sola desigualdad y \'esta es
inmediata ya que  $xy\leq \left| xy\right|$. Adem'as, obtenemos la igualdad si y s'olo si $xy=|xy|$ que sucede 'unicamente cuando  $xy\geq 0$.
}

\noindent  La desigualdad $(b)$ en \ref{desigualdadesabsolut} se puede extender en una forma general
como,
$$
|\pm x_1\pm x_2\pm\cdots\pm x_n|\leq |x_1|+|x_2|+\cdots+|x_n|,
$$
para n'umeros reales $x_1$, $x_2$, $\dots$, $x_n$.
La igualdad se tiene cuando todos los $\pm x_i$
tienen el mismo signo.
'Esta se demuestra de manera similar, 
o bien por inducci\'on\footnote{Ver secci'on \ref{cap3sec1}, para ver
demostraciones por inducci\'on.}.

\vei



%%% 13
\ejercpreliminares{Si $a$ y $b$ son n'umeros reales 
cualesquiera, demuestre que
$$
             ||a|-|b||\leq |a-b|.
$$
}

\solcpreliminares{Observe que $|a|=|a-b+b|\leq |a-b|+|b|$, 
despejando se tiene que $|a|-|b|\leq |a-b|.$  An'alogamente, 
siguiendo los mismos pasos, se tiene que $|b| -|a|\leq |b-a|$. 
De estas dos desigualdades se sigue que $||a|-|b||\leq |a-b|$.
}



%%% 14
\ejercpreliminares{En cada caso encuentre los n'umeros reales 
$x$ que satisfacen:

$(i)$\;$|x-1|- |x+1|=0$.

$(ii)$\; $|x-1||x+1|=1$.

$(iii)$\; $|x-1|+ |x+1|=2$.
}

\solcpreliminares{$(i)$\;  $|x-1|- |x+1|=0$ es equivalente a 
$|x-1|=|x+1|$. Elevando al cuadrado y resolviendo la ecuaci'on 
$(x-1)^2 = (x+1)^2$ tenemos que $4x =0$,
luego, la 'unica soluci'on es $x=0$.

$(ii)$\; $|x-1||x+1|=1$  es equivalente a $|x^2-1|=1$, de donde
%$$
%\begin{array}{lcccl}
%          x^2 -1 =1 & &\text{o} & & -(x^2 -1) =1\\
%         x^2  =2 & &\text{o} & & x^2  =0\\
%         x = \pm\sqrt{2} & &\text{o} & & x  =0,
%\end{array}
%$$
las soluciones son $x = \pm\sqrt{2}$  y $ x  =0$.

$(iii)$\;  Si $x>1$ se cumple que  $|x+1|=x+1 >2$, luego no hay soluci'on.

Si $x<-1$ se cumple que  $|x-1|=-x+1 >2$ y tampoco hay soluci'on.

Si $-1\leq x \leq 1$,  entonces $x-1 \leq 0\leq x+1$,   luego 
$$
      |x-1|+|x+1 |=(1-x)+(x+1)=2.
$$
Por lo que los 'unicos valores de  $x$ que cumplen  la igualdad son 
 $-1\leq x\leq 1$.
}

%%% 15
\ejercpreliminares{Encuentre las ternas $(x,y,z)$  de n'umeros 
reales que satisfacen 
\begin{eqnarray*}
              |x+y| &\geq & 1\\
             2xy -z^2 & \geq & 1\\
            z-|x+y|  & \geq & -1.
\end{eqnarray*}
}

\solcpreliminares{De la primera y  tercera desigualdades 
se tiene que \linebreak $z \geq |x+y| -1\geq 0$. Por lo que, $z^2\geq 
(|x+y|-1)^2$. Ahora,   $2xy \geq z^2+1\geq (|x+y|-1)^2 + 1\geq 0$, 
entonces
$$
  2xy \geq  x^2+2xy+y^2-2|x+y|+2 \geq  |x|^2+2xy+|y|^2 - 2|x|- 2 |y| +2,
$$
cancelando $0\geq  |x|^2+|y|^2 - 2|x|- 2 |y|+2 = (|x|-1)^2 +(|y|-1)^2.$
Por lo que $|x|=1$ y $|y|=1$.
Luego,  $x$ y $y$ tienen que ser   $-1$ o 1.  Pero como $xy\geq 0$,  
los dos tienen que tener el mismo signo. Para  $x=y=1$ o $x=y=-1$ se 
tiene, sustituyendo en las ecuaciones originales,   
que $2-z^2\geq 1$ y $z-2\geq -1$. Luego, $z^2\leq 1$ y $z\geq 1$. El 
'unico valor de $z$ que satisface las dos desigualdades es $z=1$. 
Por lo tanto, hay dos soluciones al problema $x=y=z=1$ y $x=y=-1$, $z=1$. 
}


%%%%%16
\ejercpreliminares{(OMM, 2004) ?`Cu\'al es la mayor cantidad de 
n'umeros enteros positivos que se pueden encontrar de manera que 
cualesquiera dos de ellos, $a$ y $b$ (con $a\neq b$), cumplan que:
$$|a-b|\geq \frac{ab}{100}?$$
}

\solcpreliminares{Suponga que
$a_{1}<a_{2}< \dots<a_{n}$ es una colecci\'{o}n con la mayor cantidad de
n'umeros enteros con la propiedad.  Es claro que $a_{i}\geq i$, para 
toda $i=1, \ldots, n$.

\noindent Si $a$ y $b$ son dos n'umeros enteros de la colecci\'{o}n 
con $a>b$, como $%
\left\vert a-b\right\vert =a-b\geq \frac{ab}{100}$, se tiene que 
$a \left(1-\frac{b}{100} \right) \geq b$, por lo que si $100-b>0$, 
entonces $a\geq \frac{100b}{100-b}$.

\noindent Note que no existen dos n'umeros enteros $a$ y $b$ en la 
colecci\'{o}n
mayores que $100$, en efecto si $a>b>100$, entonces $a-b=\left\vert
a-b\right\vert \geq \frac{ab}{100}>a$, lo cual es falso.

\noindent Tambi\'{e}n se tiene que para n'umeros enteros $a$ y $b$ 
menores que $100$,
se cumple que $\frac{100a}{100-a}\geq \frac{100b}{100-b}$ si y s'olo si 
$100a-ab\geq 100b-ab$ si y s'olo si $a\geq b$.

\vei 

\noindent Es claro que $\left\{ 1,2,3,4,5,6,7,8,9,10\right\} $ es una 
colecci\'{o}n con la propiedad.

\noindent Ahora, $a_{11}\geq \frac{100a_{10}}{100-a_{10}}\geq \frac{100\cdot
10}{100-10}=\frac{100}{9}>11$, lo que implica que $a_{11}\geq 12$.

\ve

$a_{12}\geq \frac{100a_{11}}{100-a_{11}}\geq \frac{100\cdot 12}{100-12}=%
\frac{1200}{88}>13$, de donde $a_{12}\geq 14$.

\ve

$a_{13}\geq \frac{100a_{12}}{100-a_{12}}\geq \frac{100\cdot 14}{100-14}=%
\frac{1400}{86}>16$, de donde $a_{13}\geq 17$.

\ve

$a_{14}\geq \frac{100a_{13}}{100-a_{13}}\geq \frac{100\cdot 17}{100-17}=%
\frac{1700}{83}>20$, de donde $a_{14}\geq 21$.

\ve

$a_{15}\geq \frac{100a_{14}}{100-a_{14}}\geq \frac{100\cdot 21}{100-21}=%
\frac{2100}{79}>26$, de donde $a_{15}\geq 27$.

\ve

$a_{16}\geq \frac{100a_{15}}{100-a_{15}}\geq \frac{100\cdot 27}{100-27}=%
\frac{2700}{73}>36$, de donde  $a_{16}\geq 37$.

\ve

$a_{17}\geq \frac{100a_{16}}{100-a_{16}}\geq \frac{100\cdot 37}{100-37}=%
\frac{3700}{63}>58$, de donde  $a_{17}\geq 59$.

\ve

$a_{18}\geq \frac{100a_{17}}{100-a_{17}}\geq \frac{100\cdot 59}{100-59}=%
\frac{5900}{41}>143$, de donde  $a_{18}\geq 144$.

\ve

\noindent Adem'as, como ya se ha observado que no hay dos  
n'umeros enteros de la colecci\'{o}n mayores que $100$, 
la mayor cantidad es $18$.
La colecci\'{o}n de $18$ n'umeros enteros siguiente 
$\left\{ 1,2,3,4,5,6,7,8,9,10,12,14,17,21,27,37,59,144\right\}$ 
cumple la condici\'{o}n.
}











